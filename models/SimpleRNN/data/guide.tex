\IfFileExists{stacks-project.cls}{%
\documentclass{stacks-project}
}{%
\documentclass{amsart}
}

% For dealing with references we use the comment environment
\usepackage{verbatim}
\newenvironment{reference}{\comment}{\endcomment}
%\newenvironment{reference}{}{}
\newenvironment{slogan}{\comment}{\endcomment}
\newenvironment{history}{\comment}{\endcomment}

% For commutative diagrams we use Xy-pic
\usepackage[all]{xy}

% We use 2cell for 2-commutative diagrams.
\xyoption{2cell}
\UseAllTwocells

% We use multicol for the list of chapters between chapters
\usepackage{multicol}

% This is generall recommended for better output
\usepackage{lmodern}
\usepackage[T1]{fontenc}

% For cross-file-references
\usepackage{xr-hyper}

% Package for hypertext links:
\usepackage{hyperref}

% For any local file, say "hello.tex" you want to link to please
% use \externaldocument[hello-]{hello}
\externaldocument[introduction-]{introduction}
\externaldocument[conventions-]{conventions}
\externaldocument[sets-]{sets}
\externaldocument[categories-]{categories}
\externaldocument[topology-]{topology}
\externaldocument[sheaves-]{sheaves}
\externaldocument[sites-]{sites}
\externaldocument[stacks-]{stacks}
\externaldocument[fields-]{fields}
\externaldocument[algebra-]{algebra}
\externaldocument[brauer-]{brauer}
\externaldocument[homology-]{homology}
\externaldocument[derived-]{derived}
\externaldocument[simplicial-]{simplicial}
\externaldocument[more-algebra-]{more-algebra}
\externaldocument[smoothing-]{smoothing}
\externaldocument[modules-]{modules}
\externaldocument[sites-modules-]{sites-modules}
\externaldocument[injectives-]{injectives}
\externaldocument[cohomology-]{cohomology}
\externaldocument[sites-cohomology-]{sites-cohomology}
\externaldocument[dga-]{dga}
\externaldocument[dpa-]{dpa}
\externaldocument[sdga-]{sdga}
\externaldocument[hypercovering-]{hypercovering}
\externaldocument[schemes-]{schemes}
\externaldocument[constructions-]{constructions}
\externaldocument[properties-]{properties}
\externaldocument[morphisms-]{morphisms}
\externaldocument[coherent-]{coherent}
\externaldocument[divisors-]{divisors}
\externaldocument[limits-]{limits}
\externaldocument[varieties-]{varieties}
\externaldocument[topologies-]{topologies}
\externaldocument[descent-]{descent}
\externaldocument[perfect-]{perfect}
\externaldocument[more-morphisms-]{more-morphisms}
\externaldocument[flat-]{flat}
\externaldocument[groupoids-]{groupoids}
\externaldocument[more-groupoids-]{more-groupoids}
\externaldocument[etale-]{etale}
\externaldocument[chow-]{chow}
\externaldocument[intersection-]{intersection}
\externaldocument[pic-]{pic}
\externaldocument[weil-]{weil}
\externaldocument[adequate-]{adequate}
\externaldocument[dualizing-]{dualizing}
\externaldocument[duality-]{duality}
\externaldocument[discriminant-]{discriminant}
\externaldocument[derham-]{derham}
\externaldocument[local-cohomology-]{local-cohomology}
\externaldocument[algebraization-]{algebraization}
\externaldocument[curves-]{curves}
\externaldocument[resolve-]{resolve}
\externaldocument[models-]{models}
\externaldocument[functors-]{functors}
\externaldocument[equiv-]{equiv}
\externaldocument[pione-]{pione}
\externaldocument[etale-cohomology-]{etale-cohomology}
\externaldocument[proetale-]{proetale}
\externaldocument[relative-cycles-]{relative-cycles}
\externaldocument[more-etale-]{more-etale}
\externaldocument[trace-]{trace}
\externaldocument[crystalline-]{crystalline}
\externaldocument[spaces-]{spaces}
\externaldocument[spaces-properties-]{spaces-properties}
\externaldocument[spaces-morphisms-]{spaces-morphisms}
\externaldocument[decent-spaces-]{decent-spaces}
\externaldocument[spaces-cohomology-]{spaces-cohomology}
\externaldocument[spaces-limits-]{spaces-limits}
\externaldocument[spaces-divisors-]{spaces-divisors}
\externaldocument[spaces-over-fields-]{spaces-over-fields}
\externaldocument[spaces-topologies-]{spaces-topologies}
\externaldocument[spaces-descent-]{spaces-descent}
\externaldocument[spaces-perfect-]{spaces-perfect}
\externaldocument[spaces-more-morphisms-]{spaces-more-morphisms}
\externaldocument[spaces-flat-]{spaces-flat}
\externaldocument[spaces-groupoids-]{spaces-groupoids}
\externaldocument[spaces-more-groupoids-]{spaces-more-groupoids}
\externaldocument[bootstrap-]{bootstrap}
\externaldocument[spaces-pushouts-]{spaces-pushouts}
\externaldocument[spaces-chow-]{spaces-chow}
\externaldocument[groupoids-quotients-]{groupoids-quotients}
\externaldocument[spaces-more-cohomology-]{spaces-more-cohomology}
\externaldocument[spaces-simplicial-]{spaces-simplicial}
\externaldocument[spaces-duality-]{spaces-duality}
\externaldocument[formal-spaces-]{formal-spaces}
\externaldocument[restricted-]{restricted}
\externaldocument[spaces-resolve-]{spaces-resolve}
\externaldocument[formal-defos-]{formal-defos}
\externaldocument[defos-]{defos}
\externaldocument[cotangent-]{cotangent}
\externaldocument[examples-defos-]{examples-defos}
\externaldocument[algebraic-]{algebraic}
\externaldocument[examples-stacks-]{examples-stacks}
\externaldocument[stacks-sheaves-]{stacks-sheaves}
\externaldocument[criteria-]{criteria}
\externaldocument[artin-]{artin}
\externaldocument[quot-]{quot}
\externaldocument[stacks-properties-]{stacks-properties}
\externaldocument[stacks-morphisms-]{stacks-morphisms}
\externaldocument[stacks-limits-]{stacks-limits}
\externaldocument[stacks-cohomology-]{stacks-cohomology}
\externaldocument[stacks-perfect-]{stacks-perfect}
\externaldocument[stacks-introduction-]{stacks-introduction}
\externaldocument[stacks-more-morphisms-]{stacks-more-morphisms}
\externaldocument[stacks-geometry-]{stacks-geometry}
\externaldocument[moduli-]{moduli}
\externaldocument[moduli-curves-]{moduli-curves}
\externaldocument[examples-]{examples}
\externaldocument[exercises-]{exercises}
\externaldocument[guide-]{guide}
\externaldocument[desirables-]{desirables}
\externaldocument[coding-]{coding}
\externaldocument[obsolete-]{obsolete}
\externaldocument[fdl-]{fdl}
\externaldocument[index-]{index}

% Theorem environments.
%
\theoremstyle{plain}
\newtheorem{theorem}[subsection]{Theorem}
\newtheorem{proposition}[subsection]{Proposition}
\newtheorem{lemma}[subsection]{Lemma}

\theoremstyle{definition}
\newtheorem{definition}[subsection]{Definition}
\newtheorem{example}[subsection]{Example}
\newtheorem{exercise}[subsection]{Exercise}
\newtheorem{situation}[subsection]{Situation}

\theoremstyle{remark}
\newtheorem{remark}[subsection]{Remark}
\newtheorem{remarks}[subsection]{Remarks}

\numberwithin{equation}{subsection}

% Macros
%
\def\lim{\mathop{\mathrm{lim}}\nolimits}
\def\colim{\mathop{\mathrm{colim}}\nolimits}
\def\Spec{\mathop{\mathrm{Spec}}}
\def\Hom{\mathop{\mathrm{Hom}}\nolimits}
\def\Ext{\mathop{\mathrm{Ext}}\nolimits}
\def\SheafHom{\mathop{\mathcal{H}\!\mathit{om}}\nolimits}
\def\SheafExt{\mathop{\mathcal{E}\!\mathit{xt}}\nolimits}
\def\Sch{\mathit{Sch}}
\def\Mor{\mathop{\mathrm{Mor}}\nolimits}
\def\Ob{\mathop{\mathrm{Ob}}\nolimits}
\def\Sh{\mathop{\mathit{Sh}}\nolimits}
\def\NL{\mathop{N\!L}\nolimits}
\def\CH{\mathop{\mathrm{CH}}\nolimits}
\def\proetale{{pro\text{-}\acute{e}tale}}
\def\etale{{\acute{e}tale}}
\def\QCoh{\mathit{QCoh}}
\def\Ker{\mathop{\mathrm{Ker}}}
\def\Im{\mathop{\mathrm{Im}}}
\def\Coker{\mathop{\mathrm{Coker}}}
\def\Coim{\mathop{\mathrm{Coim}}}

% Boxtimes
%
\DeclareMathSymbol{\boxtimes}{\mathbin}{AMSa}{"02}

%
% Macros for moduli stacks/spaces
%
\def\QCohstack{\mathcal{QC}\!\mathit{oh}}
\def\Cohstack{\mathcal{C}\!\mathit{oh}}
\def\Spacesstack{\mathcal{S}\!\mathit{paces}}
\def\Quotfunctor{\mathrm{Quot}}
\def\Hilbfunctor{\mathrm{Hilb}}
\def\Curvesstack{\mathcal{C}\!\mathit{urves}}
\def\Polarizedstack{\mathcal{P}\!\mathit{olarized}}
\def\Complexesstack{\mathcal{C}\!\mathit{omplexes}}
% \Pic is the operator that assigns to X its picard group, usage \Pic(X)
% \Picardstack_{X/B} denotes the Picard stack of X over B
% \Picardfunctor_{X/B} denotes the Picard functor of X over B
\def\Pic{\mathop{\mathrm{Pic}}\nolimits}
\def\Picardstack{\mathcal{P}\!\mathit{ic}}
\def\Picardfunctor{\mathrm{Pic}}
\def\Deformationcategory{\mathcal{D}\!\mathit{ef}}


% OK, start here.
%
\begin{document}

\title{A Guide to the Literature}

\maketitle

\phantomsection
\label{section-phantom}

\tableofcontents

\section{Short introductory articles}
\label{section-short-introductions}


\begin{itemize}
\item Barbara Fantechi: \emph{Stacks for Everybody} \cite{fantechi_stacks}
\item Dan Edidin: \emph{What is a stack?} \cite{edidin_whatis}
\item Dan Edidin: \emph{Notes on the construction of the moduli space of
curves} \cite{edidin_notes}
\item Angelo Vistoli: \emph{Intersection theory on algebraic
stacks and on their moduli spaces}, and especially the appendix.
\cite{vistoli_intersection}
\end{itemize}


\section{Classic references}
\label{section-classics}

\begin{itemize}
\item Mumford: \emph{Picard groups of moduli problems}
\cite{mumford_picard}
\begin{quote}
Mumford never uses the term ``stack'' here but the
concept is implicit in the paper; he computes the picard group of the moduli
stack of elliptic curves.
\end{quote}
\item Deligne, Mumford: \emph{The irreducibility of the space of curves of
given genus} \cite{DM}
\begin{quote}
This influential paper introduces ``algebraic stacks'' in the
sense which are now universally called Deligne-Mumford stacks (stacks with
representable diagonal which admit \'etale presentations by schemes). There
are many foundational results \emph{without proof}. The paper uses stacks to
give two proofs of the irreducibility of the moduli space of curves of genus
$g$.
\end{quote}
\item Artin: \emph{Versal deformations and algebraic stacks}
\cite{ArtinVersal}
\begin{quote}
This paper introduces ``algebraic stacks'' which generalize Deligne-Mumford
stacks and are now commonly referred to as \emph{Artin stacks}, stacks with
representable diagonal which admit smooth presentations by schemes.
This paper gives deformation-theoretic criterion known as Artin's criterion
which allows one to prove that a given moduli stack is an Artin stack without
explicitly exhibiting a presentation.
\end{quote}
\end{itemize}




\section{Books and online notes}
\label{section-books}


\begin{itemize}
\item Laumon, Moret-Bailly: \emph{Champs Alg\'ebriques} \cite{LM-B}
\begin{quote}
This book is currently the most exhaustive reference on stacks containing
many foundational results. It assumes the reader is familiar with algebraic
spaces and frequently references Knutson's book \cite{Kn}. There is an
error in chapter 12 concerning the functoriality of the lisse-\'etale site of
an algebraic stack. One doesn't need to worry about this as the error has been
patched by Martin Olsson (see \cite{olsson_sheaves}) and the results in the
remaining chapters (after perhaps slight modification) are correct.
\end{quote}
\item The Stacks Project Authors: \emph{Stacks Project}
\cite{stacks-project}.
\begin{quote}
You are reading it!
\end{quote}
\item Anton Geraschenko:
\emph{Lecture notes for Martin Olsson's class on stacks} \cite{olsson_stacks}
\begin{quote}
This course systematically develops the theory of algebraic
spaces before introducing algebraic stacks (first defined in Lecture 27!). In
addition to basic properties, the course covers the equivalence between being
Deligne-Mumford and having unramified diagonal, the lisse-\'etale site on an
Artin stack, the theory of quasi-coherent sheaves, the Keel-Mori theorem,
cohomological descent, and gerbes (and their relation to the Brauer group).
There are also some exercises.
\end{quote}
\item
Behrend, Conrad, Edidin, Fantechi, Fulton, G\"ottsche, and Kresch:
\emph{Algebraic stacks}, online notes for a book being currently written
\cite{stacks_book}
\begin{quote}
The aim of this book is to give a friendly introduction to stacks without
assuming a sophisticated background with a focus on examples and applications.
Unlike \cite{LM-B}, it is not assumed that the reader has digested the theory of
algebraic spaces. Instead, Deligne-Mumford stacks are introduced with
algebraic spaces being a special case with part of the goal being to develop
enough theory to prove the assertions in \cite{DM}. The general
theory of Artin stacks is to be developed in the second part. Only a fraction
of the book is now available on Kresch's website.
\end{quote}
\item Olsson, Martin: \emph{Algebraic spaces and stacks}, \cite{olsson_book}
\begin{quote}
Highly recommended introduction to algebraic spaces and algebraic stacks
starting at the level of somebody who has mastered Hartshorne's book
on algebraic geometry.
\end{quote}
\end{itemize}


\section{Related references on foundations of stacks}
\label{section-related}


\begin{itemize}
\item Vistoli:
\emph{Notes on Grothendieck topologies, fibered categories and descent theory}
\cite{vistoli_fga}
\begin{quote}
Contains useful facts on fibered categories, stacks and descent theory in the
fpqc topology as well as rigorous proofs.
\end{quote}
\item Knutson: \emph{Algebraic Spaces} \cite{Kn}
\begin{quote}
This book, which evolved from his PhD thesis under Michael Artin,
contains the foundations of the theory of algebraic spaces. The book
\cite{LM-B} frequently references this text. See also Artin's papers on
algebraic spaces: \cite{Artin-Algebraic-Approximation},
\cite{ArtinI}, \cite{Artin-Implicit-Function},
\cite{ArtinII}, \cite{Artin-Construction-Techniques},
\cite{Artin-Algebraic-Spaces}, \cite{Artin-Theorem-Representability}, and
\cite{ArtinVersal}
\end{quote}
\item Grothendieck et al, \emph{Th\'eorie des Topos et Cohomologie \'Etale des
Sch\'emas I, II, III} also known as SGA4 \cite{SGA4}
\begin{quote}
Volume 1 contains many general facts on universes, sites and fibered
categories. The word ``champ'' (French for ``stack'') appears in
Deligne's Expos\'e XVIII.
\end{quote}
\item Jean Giraud: \emph{Cohomologie non ab\'elienne} \cite{giraud}
\begin{quote}
The book discusses fibered categories, stacks, torsors and gerbes over general
sites but does not discuss algebraic stacks. For instance, if $G$ is a sheaf
of abelian groups on $X$, then in the same way $H^1(X, G)$ can be identified
with $G$-torsors, $H^2(X, G)$ can be identified with an appropriately defined
set of $G$-gerbes. When $G$ is not abelian, then $H^2(X, G)$ is defined as the
set of $G$-gerbes.
\end{quote}
\item Kelly and Street: \emph{Review of the elements of 2-categories}
\cite{kelly-street}
\begin{quote}
The category of stacks form a 2-category although a simple type of 2-category
where are 2-morphisms are invertible. This is a reference on general
2-categories. I have never used this so I cannot say how useful it is. Also
note that \cite{stacks-project} contains some basics on 2-categories.
\end{quote}
\end{itemize}




\section{Papers in the literature}
\label{section-papers}

\noindent
Below is a list of research papers which contain fundamental results on stacks
and algebraic spaces. The intention of the summaries is to indicate only the
results of the paper which contribute toward stack theory; in many cases these
results are subsidiary to the main goals of the paper. We divide the papers
into categories with some papers falling into multiple categories.


\subsection{Deformation theory and algebraic stacks}
\label{subsection-deformation-theory}

\noindent
The first three papers by Artin do not contain anything on stacks but they
contain powerful results with the first two papers being essential for
\cite{ArtinVersal}.
\begin{itemize}
\item Artin: \emph{Algebraic approximation of structures over
complete local rings} \cite{Artin-Algebraic-Approximation}
\begin{quote}
It is proved that under mild hypotheses any effective formal deformation can be
approximated: if $F: (\Sch/S) \to (\textit{Sets})$
is a contravariant functor
locally of finite presentation with $S$ finite type over a field or excellent
DVR, $s \in S$, and $\hat{\xi} \in F(\hat{\mathcal{O}}_{S, s})$ is an effective
formal
deformation, then for any $n > 0$, there exists an residually trivial \'etale
neighborhood $(S', s') \to (S, s)$ and $\xi' \in F(S')$ such that $\xi'$ and
$\hat{\xi}$ agree up to order $n$ (ie. have the same restriction in
$F(\mathcal{O}_{S, s} / \mathfrak m^n)$).
\end{quote}
\item
Artin: \emph{Algebraization of formal moduli I} \cite{ArtinI}
\begin{quote}
It is proved that under mild hypotheses any effective formal versal deformation
is algebraizable. Let $F: (\Sch/S) \to (\textit{Sets})$ be a
contravariant functor
locally of finite presentation with $S$ finite type over a field or excellent
DVR, $s \in S$ be a locally closed point, $\hat A$ be a complete Noetherian
local $\mathcal{O}_S$-algebra with residue field $k'$ a finite extension of
$k(s)$,
and $\hat{\xi} \in F(\hat A)$ be an effective formal versal deformation of an
element $\xi_0 \in F(k')$. Then there is a scheme $X$ finite type over $S$ and
a closed point $x \in X$ with residue field $k(x) = k'$ and an element $\xi \in
F(X)$ such that there is an isomorphism $\hat{\mathcal{O}}_{X, x} \cong \hat{A}$
identifying the restrictions of $\xi$ and $\hat{\xi}$ in each $F(\hat A /
\mathfrak m^n)$. The algebraization is unique if $\hat{\xi}$ is a universal
deformation. Applications are given to the representability of the Hilbert
and Picard schemes.
\end{quote}
\item Artin: \emph{Algebraization of formal moduli. II}
\cite{ArtinII}
\begin{quote}
Vaguely, it is shown that if one can contract a closed subset $Y' \subset X'$
formally locally around $Y'$, then exists a global morphism $X' \to X$
contracting $Y$ with $X$ an algebraic space.
\end{quote}
\item
Artin: \emph{Versal deformations and algebraic stacks} \cite{ArtinVersal}
\begin{quote}
This momentous paper builds on his work in
\cite{Artin-Algebraic-Approximation} and \cite{ArtinI}. This paper
introduces Artin's criterion which allows one to prove algebraicity of a
stack by verifying deformation-theoretic properties. More precisely (but
not very precisely), Artin constructs a presentation of a limit preserving
stack $\mathcal{X}$
locally around a point $x \in \mathcal{X}(k)$ as follows: assuming the stack
$\mathcal{X}$
satisfies Schlessinger's criterion(\cite{Sch}), there exists a formal
versal deformation
$\hat{\xi} \in \lim \mathcal{X}(\hat A / \mathfrak m^n)$ of
$x$. Assuming that formal deformations are effective (i.e.,
$\mathcal{X}(\hat{A}) \to \lim \mathcal{X}(\hat A / \mathfrak m^n)$
is bijective), then one obtains an effective formal versal
deformation $\xi \in \mathcal{X}(\hat A)$. Using results in
\cite{ArtinI}, one produces a finite type scheme $U$ and an
element $\xi_U: U \to \mathcal{X}$ which is formally versal at a point
$u \in U$ over $x$. Then if we assume $\mathcal{X}$ admits a deformation
and obstruction theory
satisfying certain conditions (ie. compatibility with \'etale localization and
completion as well as constructibility condition), then it is shown in section
4 that formal versality is an open condition so that after shrinking $U$, $U
\to \mathcal{X}$ is smooth.
Artin also presents a proof that any stack admitting an fppf presentation by
a scheme admits a smooth presentation by a scheme so that in particular
one can form quotient stacks by flat, separated, finitely presented group
schemes.
\end{quote}
\item Conrad, de Jong: \emph{Approximation of Versal Deformations}
\cite{conrad-dejong}
\begin{quote}
This paper offers an approach to Artin's algebraization result by applying
Popescu's powerful result: if $A$ is a Noetherian ring and $B$ a Noetherian
$A$-algebra, then the map $A \to B$ is a regular morphism if and only if $B$
is a direct limit of smooth $A$-algebras. It is not hard to see that Popescu's
result implies Artin's approximation over an arbitrary excellent scheme (the
excellence hypothesis implies that for a local ring $A$, the map
$A^{\text{h}} \to \hat A$ from the henselization to the completion is regular).
The paper uses Popescu's result to give a ``groupoid'' generalization of the
main theorem in \cite{ArtinI} which is valid over arbitrary
excellent base schemes and for arbitrary points $s \in S$.
In particular, the results in \cite{ArtinVersal} hold under an arbitrary
excellent base. They discuss the \'etale-local uniqueness of the
algebraization and whether the automorphism group of the object acts naturally
on the henselization of the algebraization.
\end{quote}
\item Jason Starr: \emph{Artin's axioms, composition, and moduli spaces}
\cite{starr_artin}
\begin{quote}
The paper establishes that Artin's axioms for algebraization are compatible
with the composition of 1-morphisms.
\end{quote}
\item Martin Olsson: \emph{Deformation theory of representable
morphism of algebraic stacks} \cite{olsson_deformation}
\begin{quote}
This generalizes standard deformation theory results for morphisms of schemes
to representable morphisms of algebraic stacks in terms of the cotangent
complex. These results cannot be viewed as consequences of Illusie's general
theory as the cotangent complex of a representable morphism $X \to \mathcal{X}$
is not
defined in terms of cotangent complex of a morphism of ringed topoi (because
the lisse-\'etale site is not functorial).
\end{quote}
\end{itemize}

\subsection{Coarse moduli spaces}
\label{subsection-coarse-moduli-spaces}

\noindent
Papers discussing coarse moduli spaces.
\begin{itemize}
\item Keel, Mori: \emph{Quotients in Groupoids} \cite{K-M}
\begin{quote}
It had apparently long been ``folklore'' that separated Deligne-Mumford stacks
admitted coarse moduli spaces. A rigorous (although terse) proof of the
following theorem is presented here: if $\mathcal{X}$ is an Artin stack
locally of
finite type over a Noetherian base scheme such that the inertia stack
$I_\mathcal{X} \to \mathcal{X}$ is finite, then there exists a coarse
moduli space $\phi : \mathcal{X} \to Y$
with $\phi$ separated and $Y$ an algebraic space locally of finite type over
$S$. The hypothesis that the inertia is finite is precisely the right
condition: there exists a coarse moduli space $\phi : \mathcal{X} \to Y$ with
$\phi$
separated if and only if the inertia is finite.
\end{quote}
\item Conrad: \emph{The Keel-Mori Theorem via Stacks} \cite{conrad}
\begin{quote}
Keel and Mori's paper \cite{K-M} is written in the groupoid language and
some find it challenging to grasp. Brian Conrad presents a stack-theoretic
version of the proof which is quite transparent although it uses the
sophisticated language of stacks. Conrad also removes the Noetherian
hypothesis.
\end{quote}
\item Rydh: \emph{Existence of quotients by finite groups and coarse moduli
spaces} \cite{rydh_quotients}
\begin{quote}
Rydh removes the hypothesis from \cite{K-M} and \cite{conrad} that
$\mathcal{X}$
be finitely presented over some base.
\end{quote}
\item
Abramovich, Olsson, Vistoli: \emph{Tame stacks in positive characteristic}
\cite{tame}
\begin{quote}
They define a \emph{tame Artin stack} as an Artin stack with finite inertia
such that if $\phi : \mathcal{X} \to Y$ is the coarse moduli space,
$\phi_*$ is exact
on quasi-coherent sheaves. They prove that for an Artin stack with finite
inertia, the following are equivalent: $\mathcal{X}$ is tame if and only
if the stabilizers of $\mathcal{X}$ are linearly reductive
if and only if $\mathcal{X}$ is \'etale locally on the coarse
moduli space a quotient of an affine scheme by a linearly reductive group
scheme. For a tame Artin stack, the coarse moduli space is particularly nice.
For instance, the coarse moduli space commutes with arbitrary base change while
a general coarse moduli space for an Artin stack with finite inertia will only
commute with flat base change.
\end{quote}
\item Alper: \emph{Good moduli spaces for Artin stacks} \cite{alper_good}
\begin{quote}
For general Artin stacks with infinite affine stabilizer groups (which are
necessarily non-separated), coarse moduli spaces often do not exist. The
simplest example is $[\mathbf{A}^1 / \mathbf{G}_m]$. It is defined here that a
quasi-compact
morphism $\phi : \mathcal{X} \to Y$ is a \emph{good moduli space} if
$\mathcal{O}_Y \to \phi_*
\mathcal{O}_\mathcal{X}$ is an isomorphism and $\phi_*$ is exact on
quasi-coherent sheaves.
This notion generalizes a tame Artin stack in \cite{tame} as well as
encapsulates Mumford's geometric invariant theory: if $G$ is a reductive group
acting linearly on $X \subset \mathbf{P}^n$, then the morphism from the
quotient
stack of the semi-stable locus to the GIT quotient $[X^{ss}/G] \to X//G$ is a
good moduli space. The notion of a good moduli space has many nice geometric
properties: (1) $\phi$ is surjective, universally closed, and universally
submersive, (2) $\phi$ identifies points in $Y$ with points in $\mathcal{X}$ up
to
closure equivalence, (3) $\phi$ is universal for maps to algebraic spaces, (4)
good moduli spaces are stable under arbitrary base change, and (5) a vector
bundle on an Artin stack descends to the good moduli space if and only if the
representations are trivial at closed points.
\end{quote}
\end{itemize}


\subsection{Intersection theory}
\label{subsection-intersection-theory}

\noindent
Papers discussing intersection theory on algebraic stacks.
\begin{itemize}
\item
Vistoli: \emph{Intersection theory on algebraic stacks and on their moduli
spaces} \cite{vistoli_intersection}
\begin{quote}
This paper develops the foundations for intersection theory with rational
coefficients for Deligne-Mumford stacks. If $\mathcal{X}$ is a separated
Deligne-Mumford stack, the chow group $\CH_*(\mathcal{X})$ with rational
coefficients is
defined as the free abelian group of integral closed substacks of dimension $k$
up to rational equivalence. There is a flat pullback, a proper push-forward
and a generalized Gysin homomorphism for regular local embeddings. If
$\phi : \mathcal{X} \to Y$ is a moduli space (ie. a proper morphism with
is bijective on
geometric points), there is an induced push-forward $\CH_*(\mathcal{X}) \to
\CH_k(Y)$
which is an isomorphism.
\end{quote}
\item Edidin, Graham: \emph{Equivariant Intersection Theory}
\cite{edidin-graham}
\begin{quote}
The purpose of this article is to develop intersection theory with integral
coefficients for a quotient stack $[X/G]$ of an action of an algebraic group
$G$ on an algebraic space $X$ or, in other words, to develop a $G$-equivariant
intersection theory of $X$. Equivariant chow groups defined using only
invariant cycles does not produce a theory with nice properties. Instead,
generalizing Totaro's definition in the case of $BG$ and motivated by the fact
that if $V \to X$ is a vector bundle then $\CH_i(X) \cong \CH_i(V)$
naturally, the authors define $\CH_i^G(X)$ as follows:
Let $\dim(X) = n$ and $\dim(G) = g$. For each $i$, choose a $l$-dimensional
$G$-representation $V$ where $G$ acts freely on an open subset $U \subset V$
whose complement as codimension $d > n - i$. So $X_G = [X \times U / G]$ is an
algebraic space (it can even be chosen to be a scheme). Then they define
$\CH_i^G(X) = \CH_{i + l - g}(X_G)$. For the quotient stack, one defines
$\CH_i( [X/G]) = \CH_{i + g}^G(X) = \CH_{i + l}(X_G)$.
In particular, $\CH_i([X/G]) = 0$ for $i > \dim [X/G] = n - g$
but can be non-zero for $i \ll 0$. For example
$\CH_i(B \mathbf{G}_m) = \mathbf{Z}$ for $i \le 0$.
They establish that these equivariant Chow groups enjoy the same functorial
properties as ordinary Chow groups. Furthermore, they establish that if
$[X / G] \cong [Y / H]$ that $\CH_i([X/G]) = \CH_i([Y/H])$
so that the definition is
independent on how the stack is presented as a quotient stack.
\end{quote}
\item
Kresch: \emph{Cycle Groups for Artin Stacks} \cite{kresch_cycle}
\begin{quote}
Kresch defines Chow groups for arbitrary Artin stacks agreeing with Edidin and
Graham's definition in \cite{edidin-graham} in the case of quotient stack. For
algebraic stacks with affine stabilizer groups, the theory satisfies the usual
properties.
\end{quote}
\item Behrend and Fantechi: \emph{The intrinsic normal cone}
\cite{behrend-fantechi}
\begin{quote}
Generalizing a construction due to Li and Tian, Behrend and Fantechi construct
a virtual fundamental class for a Deligne-Mumford stack.
\end{quote}
\end{itemize}


\subsection{Quotient stacks}
\label{subsection-quotient-stacks}

\noindent
Quotient stacks\footnote{In the literature,
\emph{quotient stack} often means a stack of the
form $[X/G]$ with $X$ an algebraic space and $G$ a subgroup scheme
of $\text{GL}_n$ rather than an arbitrary flat group scheme.}
form a very important subclass of Artin stacks which include almost all moduli
stacks studied by algebraic geometers. The geometry of a quotient stack
$[X/G]$ is the $G$-equivariant geometry of $X$. It is often easier to show
properties are true for quotient stacks and some results are only known to be
true for quotient stacks. The following papers address: When is an algebraic
stack a global quotient stack? Is an algebraic stack ``locally'' a quotient
stack?
\begin{itemize}
\item Laumon, Moret-Bailly: \cite[Chapter 6]{LM-B}
\begin{quote}
Chapter 6 contains several facts about the local and global structure of
algebraic stacks. It is proved that an algebraic stack $\mathcal{X}$ over $S$
is a
quotient stack $[Y/G]$ with $Y$ an algebraic space (resp. scheme, resp. affine
scheme) and $G$ a finite group if and only if there exists an algebraic space
(resp. scheme, resp. affine scheme) $Y'$ and an finite \'etale morphism $Y'
\to \mathcal{X}$. It is shown that any Deligne-Mumford stack over $S$ and
$x : \Spec(K) \to \mathcal{X}$ admits an representable, \'etale and
separated morphism $\phi : [X/G] \to \mathcal{X}$ where $G$ is a finite group
acting on an affine scheme over $S$ such
that $\Spec(K) = [X/G] \times_\mathcal{X} \Spec(K)$.
The existence of presentations
with geometrically connected fibers is also discussed in detail.
\end{quote}
\item
Edidin, Hassett, Kresch, Vistoli:
\emph{Brauer Groups and Quotient stacks} \cite{ehkv}
\begin{quote}
First, they establish some fundamental (although not very difficult)
facts concerning
when a given algebraic stack (always assumed finite type over a Noetherian
scheme in this paper) is a quotient stack. For an algebraic stack
$\mathcal{X}$ : $\mathcal{X}$ is a quotient stack if and only if
there exists a vector bundle $V \to \mathcal{X}$ such that for every
geometric point, the stabilizer acts faithfully on the fiber
if and only if there exists a vector bundle $V \to \mathcal{X}$ and
a locally closed substack
$V^0 \subset V$ such that $V^0$ is representable and surjects onto
$\mathcal{X}$. They
establish that an algebraic stack is a quotient stack if there exists finite
flat cover by an algebraic space. Any smooth Deligne-Mumford stack with
generically trivial stabilizer is a quotient stack.
They show that a $\mathbf{G}_m$-gerbe over a Noetherian scheme $X$
corresponding to
$\beta \in H^2(X, \mathbf{G}_m)$ is a quotient stack if and only if $\beta$ is
in the
image of the Brauer map $\text{Br}(X) \to \text{Br}'(X)$. They use this to
produce a
non-separated Deligne-Mumford stack that is not a quotient stack.
\end{quote}
\item Totaro: \emph{The resolution property for schemes and stacks}
\cite{totaro_resolution}
\begin{quote}
A stack has the resolution property if every coherent sheaf is the quotient of
a vector bundle. The first main theorem is that if $\mathcal{X}$ is a normal
Noetherian algebraic stack with affine stabilizer groups at closed points, then
the following are equivalent: (1) $\mathcal{X}$ has the resolution property and
(2)
$\mathcal{X} = [Y/\text{GL}_n]$ with $Y$ quasi-affine. In the case
$\mathcal{X}$ is finite type over
a field, then (1) and (2) are equivalent to: (3)
$\mathcal{X} = [\Spec(A)/G]$ with $G$
an affine group scheme finite type over $k$. The implication that quotient
stacks have the resolution property was proven by Thomason.
The second main theorem is that if $\mathcal{X}$ is a smooth Deligne-Mumford
stack over
a field which has a finite and generically trivial stabilizer group
$I_\mathcal{X}
\to \mathcal{X}$ and whose coarse moduli space is a scheme with affine
diagonal, then
$\mathcal{X}$ has the resolution property. Another cool result states that if
$\mathcal{X}$ is
a Noetherian algebraic stack satisfying the resolution property, then
$\mathcal{X}$ has
affine diagonal if and only if the closed points have affine stabilizer.
\end{quote}
\item Kresch: \emph{On the Geometry of Deligne-Mumford Stacks}
\cite{kresch_geometry}
\begin{quote}
This article summarizes general structure results of Deligne-Mumford
stacks (of finite type over a field) and contains some interesting results
concerning quotient stacks. It is shown that any smooth, separated,
generically tame Deligne-Mumford stack with quasi-projective coarse moduli
space is a quotient stack $[Y/G]$ with $Y$ quasi-projective and $G$ an
algebraic group. If $\mathcal{X}$ is a Deligne-Mumford stack whose coarse
moduli space
is a scheme, then $\mathcal{X}$ is Zariski-locally a quotient stack if and only
if it
admits a Zariski-open covering by stack quotients of schemes by finite groups.
If $\mathcal{X}$ is a Deligne-Mumford stack proper over a field of
characteristic 0
with coarse moduli space $Y$, then: $Y$ is projective and $\mathcal{X}$ is a
quotient stack if and only if $Y$ is projective and $\mathcal{X}$
possesses a generating sheaf if and only if
$\mathcal{X}$ admits a closed embedding into a smooth proper DM stack with
projective
coarse moduli space. This motivates a definition that a Deligne-Mumford stack
is \emph{projective} if there exists a closed embedding into a smooth, proper
Deligne-Mumford stack with projective coarse moduli space.
\end{quote}
\item Kresch, Vistoli \emph{On coverings of Deligne-Mumford stacks and
surjectivity of the Brauer map} \cite{kresch-vistoli}
\begin{quote}
It is shown that in characteristic 0 and for a fixed $n$, the following two
statements are equivalent: (1) every smooth Deligne-Mumford stack of dimension
$n$ is a quotient stack and (2) the Azumaya Brauer group coincides with the
cohomological Brauer group for smooth schemes of dimension $n$.
\end{quote}
\item Kresch: \emph{Cycle Groups for Artin Stacks} \cite{kresch_cycle}
\begin{quote}
It is shown that a reduced Artin stack finite type over a field with affine
stabilizer groups admits a stratification by quotient stacks.
\end{quote}
\item Abramovich-Vistoli:
\emph{Compactifying the space of stable maps} \cite{abramovich-vistoli}
\begin{quote}
Lemma 2.2.3 establishes that for any separated Deligne-Mumford stack is
\'etale-locally on the coarse moduli space a quotient stack $[U/G]$ where $U$
affine and $G$ a finite group. \cite[Theorem 2.12]{olsson_homstacks} shows in
this argument $G$ is even the stabilizer group.
\end{quote}
\item Abramovich, Olsson, Vistoli:
\emph{Tame stacks in positive characteristic} \cite{tame}
\begin{quote}
This paper shows that a tame Artin stack is \'etale locally on the coarse
moduli space a quotient stack of an affine by the stabilizer group.
\end{quote}
\item Alper: \emph{On the local quotient structure of Artin stacks}
\cite{alper_quotient}
\begin{quote}
It is conjectured that for an Artin stack $\mathcal{X}$ and a closed point $x
\in \mathcal{X}$
with linearly reductive stabilizer, then there is an \'etale morphism $[V/G_x]
\to \mathcal{X}$ with $V$ an algebraic space. Some evidence for this
conjecture is
given. A simple deformation theory argument (based on ideas in \cite{tame})
shows that it is true formally locally. A stack-theoretic proof of Luna's
\'etale slice theorem is presented proving that for stacks
$\mathcal{X} = [\Spec(A)/G]$
with $G$ linearly reductive, then \'etale locally on the GIT quotient
$\Spec(A^G)$, $\mathcal{X}$ is a quotient stack by the stabilizer.
\end{quote}
\end{itemize}

\subsection{Cohomology}
\label{subsection-cohomology}

\noindent
Papers discussing cohomology of sheaves on algebraic stacks.
\begin{itemize}
\item Olsson: \emph{Sheaves on Artin stacks} \cite{olsson_sheaves}
\begin{quote}
This paper develops the theory of quasi-coherent and constructible sheaves
proving basic cohomological properties. This paper corrects a mistake in
\cite{LM-B} in the functoriality of the lisse-\'etale site. The cotangent
complex is constructed. In addition, the following theorems are proved:
Grothendieck's Fundamental Theorem for proper morphisms, Grothendieck's
Existence Theorem, Zariski's Connectedness Theorem and finiteness theorem
for proper pushforwards of coherent and constructible sheaves.
\end{quote}
\item Behrend: \emph{Derived $l$-adic categories for algebraic stacks}
\cite{behrend_derived}
\begin{quote}
Proves the Lefschetz trace formula for algebraic stacks.
\end{quote}
\item Behrend: \emph{Cohomology of stacks} \cite{behrend_cohomology}
\begin{quote}
Defines the de Rham cohomology for differentiable stacks and singular
cohomology for topological stacks.
\end{quote}
\item Faltings:
\emph{Finiteness of coherent cohomology for proper fppf stacks}
\cite{faltings_finiteness}
\begin{quote}
Proves coherence for direct images of coherent sheaves for proper morphisms.
\end{quote}
\item Abramovich, Corti, Vistoli:
\emph{Twisted bundles and admissible covers} \cite{acv}
\begin{quote}
The appendix contains the proper base change theorem for \'etale cohomology
for tame Deligne-Mumford stacks.
\end{quote}
\end{itemize}


\subsection{Existence of finite covers by schemes}
\label{subsection-finite-covers}

\noindent
The existence of finite covers of Deligne-Mumford stacks by schemes
is an important result. In intersection theory on Deligne-Mumford
stacks, it is an essential ingredient in defining proper push-forward
for non-representable morphisms. There are several
results about $\overline{\mathcal{M}}_g$ relying on the existence of a finite
cover by a \emph{smooth} scheme which was proven by Looijenga. Perhaps the
first result in this direction is \cite[Theorem 6.1]{seshadri_quotients}
which treats the equivariant setting.
\begin{itemize}
\item Vistoli: \emph{Intersection theory on algebraic stacks and on their
moduli spaces} \cite{vistoli_intersection}
\begin{quote}
If $\mathcal{X}$ is a Deligne-Mumford stack with a moduli space (ie. a proper
morphism
which is bijective on geometric points), then there exists a finite morphism
$X \to \mathcal{X}$ from a scheme $X$.
\end{quote}
\item Laumon, Moret-Bailly: \cite[Chapter 16]{LM-B}
\begin{quote}
As an application of Zariski's main theorem, Theorem 16.6 establishes: if
$\mathcal{X}$ is a Deligne-Mumford stack finite type over a Noetherian scheme,
then
there exists a finite, surjective, generically \'etale morphism $Z \to
\mathcal{X}$
with $Z$ a scheme. It is also shown in Corollary 16.6.2 that any Noetherian
normal algebraic space is isomorphic to the algebraic space quotient $X'/G$
for a finite group $G$ acting a normal scheme $X$.
\end{quote}
\item Edidin, Hassett, Kresch, Vistoli:
\emph{Brauer Groups and Quotient stacks} \cite{ehkv}
\begin{quote}
Theorem 2.7 states:
if $\mathcal{X}$ is an algebraic stack of finite type over a
Noetherian ground scheme $S$, then the diagonal
$\mathcal{X} \to \mathcal{X} \times_S \mathcal{X}$ is
quasi-finite if and only if there exists a finite surjective
morphism $X \to F$ from a scheme $X$.
\end{quote}
\item Kresch, Vistoli: \emph{On coverings of Deligne-Mumford stacks and
surjectivity of the Brauer map} \cite{kresch-vistoli}
\begin{quote}
It is proved here that any smooth, separated Deligne-Mumford stack finite type
over a field with quasi-projective coarse moduli space admits a finite, flat
cover by a smooth quasi-projective scheme.
\end{quote}
\item Olsson: \emph{On proper coverings of Artin stacks} \cite{olsson_proper}
\begin{quote}
Proves that if $\mathcal{X}$ is an Artin stack separated
and finite type over $S$, then
there exists a proper surjective morphism $X \to \mathcal{X}$ from a scheme $X$
quasi-projective over $S$. As an application, Olsson proves coherence and
constructibility of direct image sheaves under proper morphisms. As an
application, he proves Grothendieck's existence theorem for proper
Artin stacks.
\end{quote}
\item Rydh: \emph{Noetherian approximation of algebraic spaces and stacks}
\cite{rydh_approx}
\begin{quote}
Theorem B of this paper is as follows.
Let $X$ be a quasi-compact algebraic stack with quasi-finite and separated
diagonal (resp.\ a quasi-compact Deligne-Mumford stack
with quasi-compact and separated diagonal). Then there exists a scheme
$Z$ and a finite, finitely presented and surjective morphism $Z \to X$
that is flat (resp.\ \'etale) over a dense quasi-compact open substack
$U \subset X$.
\end{quote}
\end{itemize}


\subsection{Rigidification}
\label{subsection-rigidification}

\noindent
Rigidification is a process for removing a flat subgroup from the inertia.
For example, if $X$ is a projective variety, the morphism from the Picard
stack to the Picard scheme is a rigidification of the group of automorphism
$\mathbf{G}_m$.
\begin{itemize}
\item Abramovich, Corti, Vistoli:
\emph{Twisted bundles and admissible covers} \cite{acv}
\begin{quote}
Let $\mathcal{X}$ be an algebraic stack over $S$ and $H$ be a flat, finitely
presented separated group scheme over $S$. Assume that for every object
$\xi \in \mathcal{X}(T)$ there is an embedding
$H(T) \hookrightarrow \text{Aut}_{\mathcal{X}(T)}(\xi)$ which is compatible
under pullbacks in the sense that for every arrow $\phi : \xi \rightarrow \xi'$
over $f: T \rightarrow T'$ and $g \in H(T')$, $g \circ \phi = \phi \circ f^*g$.
Then there exists an algebraic stack $\mathcal{X}/H$ and a
morphism $\rho : \mathcal{X} \rightarrow \mathcal{X}/H$ which is
an fppf gerbe such that for every $\xi \in \mathcal{X}(T)$, the morphism
$\text{Aut}_{\mathcal{X}(T)} (\xi)
\rightarrow \text{Aut}_{\mathcal{X}/H (T)} (\xi) $
is surjective with kernel $H(T)$.
\end{quote}
\item Romagny: \emph{Group actions on stacks and applications}
\cite{romagny_actions}
\begin{quote}
Discusses how group actions behave with respect to rigidifications.
\end{quote}
\item Abramovich, Graber, Vistoli:
\emph{Gromov-Witten theory for Deligne-Mumford stacks} \cite{agv}
\begin{quote}
The appendix gives a summary of rigidification as in \cite{acv} with two
alternative interpretations. This paper also contains constructions for
gluing algebraic stacks along closed substacks and for taking roots of line
bundles.
\end{quote}
\item
Abramovich, Olsson, Vistoli: \emph{Tame stacks in positive characteristic}
(\cite{tame})
\begin{quote}
The appendix handles the more complicated situation where the flat subgroup
stack of the inertia $H \subset I_\mathcal{X}$ is normal but not
necessarily central.
\end{quote}
\end{itemize}

\subsection{Stacky curves}
\label{subsection-stacky-curves}

\noindent
Papers discussing stacky curves.
\begin{itemize}
\item Abramovich, Vistoli: \emph{Compactifying the space of stable maps}
\cite{abramovich-vistoli}
\begin{quote}
This paper introduces \emph{twisted curves}. The moduli space of stable
maps from stable curves into an algebraic stack is typically not compact.
By using maps from twisted curves, the authors construct a moduli stack
which is proper when the target is a tame Deligne-Mumford stack whose
coarse moduli space is projective.
\end{quote}
\item Behrend, Noohi: \emph{Uniformization of Deligne-Mumford curves}
\cite{behrend-noohi}
\begin{quote}
Proves a uniformization theorem of Deligne-Mumford analytic curves.
\end{quote}
\end{itemize}

\subsection{Hilbert, Quot, Hom and branchvariety stacks}
\label{subsection-hilbert-quot-hom}

\noindent
Papers discussing Hilbert schemes and the like.
\begin{itemize}
\item Vistoli: \emph{The Hilbert stack and the theory of moduli of families}
\cite{vistoli_hilbert}
\begin{quote}
If $\mathcal{X}$ is a algebraic stack separated and locally of finite type
over a locally Noetherian and locally separated algebraic space $S$, Vistoli
defines the Hilbert stack $\mathcal{H}\text{ilb}(\mathcal{F} / S)$
parameterizing finite and unramified morphisms from proper schemes.
It is claimed without proof that $\mathcal{H}\text{ilb}(\mathcal{F} / S)$
is an algebraic stack. As a consequence, it is proved that with $\mathcal{X}$
as above, the Hom stack $\mathcal{H} \text{om}_S(T, \mathcal{X})$ is an
algebraic stack if $T$ is proper and flat over $S$.
\end{quote}
\item Olsson, Starr: \emph{Quot functors for Deligne-Mumford stacks}
\cite{olsson-starr}
\begin{quote}
If $\mathcal{X}$ is a Deligne-Mumford stack separated and locally of finite
presentation over an algebraic space $S$ and $\mathcal{F}$ is a locally
finitely-presented $\mathcal{O}_\mathcal{X}$-module, the quot functor
$\text{Quot}(\mathcal{F} / \mathcal{X} / S)$ is represented by an algebraic
space separated and locally of finite presentation over $S$. This paper
also defines generating sheaves and proves existence of a generating sheaf
for tame, separated Deligne-Mumford stacks which are global quotient stacks
of a scheme by a finite group.
\end{quote}
\item Olsson: \emph{Hom-stacks and Restrictions of Scalars}
\cite{olsson_homstacks}
\begin{quote}
Suppose $\mathcal{X}$ and $\mathcal{Y}$ are Artin stacks locally of finite
presentation over an algebraic space $S$ with finite diagonal with
$\mathcal{X}$ proper and flat over $S$ such that fppf-locally on $S$,
$\mathcal{X}$ admits a finite finitely presented flat cover by an algebraic
space (eg. $\mathcal{X}$ is Deligne-Mumford or a tame Artin stack).
Then $\Hom_S(\mathcal{X}, \mathcal{Y})$ is an Artin stack locally
of finite presentation over $S$.
\end{quote}
\item Alexeev and Knutson: \emph{Complete moduli spaces of branchvarieties}
(\cite{alexeev-knutson})
\begin{quote}
They define a branchvariety of $\mathbf{P}^n$ as a finite morphism
$X \rightarrow \mathbf{P}^n$ from a \emph{reduced} scheme $X$. They prove
that the moduli stack of branchvarieties with fixed Hilbert polynomial and
total degrees of $i$-dimensional components is a proper Artin stack with
finite stabilizer. They compare the stack of branchvarieties with the
Hilbert scheme, Chow scheme and moduli space of stable maps.
\end{quote}
\item Lieblich: \emph{Remarks on the stack of coherent algebras}
\cite{lieblich_remarks}
\begin{quote}
This paper constructs a generalization of Alexeev and Knutson's stack of
branch-varieties over a scheme $Y$ by building the stack as a stack of
algebras over the structure sheaf of $Y$. Existence proofs of $\text{Quot}$
and $\Hom$ spaces are given.
\end{quote}
\item Starr: \emph{Artin's axioms, composition, and moduli spaces}
\cite{starr_artin}
\begin{quote}
As an application of the main result, a common generalization of Vistoli's
Hilbert stack \cite{vistoli_hilbert} and Alexeev and Knutson's stack of
branchvarieties \cite{alexeev-knutson} is provided. If $\mathcal{X}$ is
an algebraic stack locally of finite type over an excellent scheme $S$
with finite diagonal, then the stack $\mathcal{H}$ parameterizing morphisms
$g: T \rightarrow \mathcal{X}$ from a proper algebraic space $T$ with a
$G$-ample line bundle $L$ is an Artin stack locally of finite type over $S$.
\end{quote}
\item Lundkvist and Skjelnes:
\emph{Non-effective deformations of Grothendieck's Hilbert functor}
\cite{lundkvist-skjelnes}
\begin{quote}
Shows that the Hilbert functor of a non-separated scheme is not represented
since there are non-effective deformations.
\end{quote}
\item Halpern-Leistner and Preygel:
\emph{Mapping stacks and categorical notions of properness}
\cite{HL-P}
\begin{quote}
This paper gives a proof that the Hom stack is algebraic under some
hypotheses on source and target which are more general than, or at
least different from, the ones in Olsson's paper.
\end{quote}
\end{itemize}



\subsection{Toric stacks}
\label{subsection-toric}

\noindent
Toric stacks provide a great class of examples and a natural testing ground
for conjectures due to the dictionary between the geometry of a toric stack
and the combinatorics of its stacky fan in a similar way that toric varieties
provide examples and counterexamples in scheme theory.
\begin{itemize}
\item Borisov, Chen and Smith: \emph{The orbifold Chow ring of toric
Deligne-Mumford stacks} \cite{bcs}
\begin{quote}
Inspired by Cox's construction for toric varieties, this paper defines
smooth toric DM stacks as explicit quotient stacks associated to a
combinatorial object called a \emph{stacky fan}.
\end{quote}
\item Iwanari: \emph{The category of toric stacks} \cite{iwanari_toric}
\begin{quote}
This paper defines a \emph{toric triple} as a smooth Deligne-Mumford stack
$\mathcal{X}$ with an open immersion $\mathbf{G}_m \hookrightarrow \mathcal{X}$
with dense image (and therefore $\mathcal{X}$ is an orbifold) and an action
$\mathcal{X} \times \mathbf{G}_m \rightarrow \mathcal{X}$. It is shown that
there is an equivalence between the 2-category of toric triples and the
1-category of stacky fans. The relationship between toric triples and the
definition of smooth toric DM stacks in \cite{bcs} is discussed.
\end{quote}
\item Iwanari: \emph{Integral Chow rings for toric stacks}
\cite{iwanari_chow}
\begin{quote}
Generalizes Cox's $\Delta$-collections for toric varieties to toric orbifolds.
\end{quote}
\item Perroni: \emph{A note on toric Deligne-Mumford stacks}
\cite{perroni}
\begin{quote}
Generalizes Cox's $\Delta$-collections and Iwanari's paper
\cite{iwanari_chow} to general smooth toric DM stacks.
\end{quote}
\item Fantechi, Mann, and Nironi: \emph{Smooth toric DM stacks}
\cite{fmn}
\begin{quote}
This paper defines a smooth toric DM stack as a smooth DM stack
$\mathcal{X}$ with the action of a DM torus $\mathcal{T}$ (ie. a Picard
stack isomorphic to $T \times BG$ with $G$ finite) having an open dense
orbit isomorphic to $\mathcal{T}$. They give a ``bottom-up description''
and prove an equivalence between smooth toric DM stacks and stacky fans.
\end{quote}
\item Geraschenko and Satriano: \emph{Toric Stacks I and II}
\cite{gs_toric1} and \cite{gs_toric2}
\begin{quote}
These papers define a toric stack as the stack quotient of a toric
variety by a subgroup of its torus. A generically stacky toric stack
is defined as a torus invariant substack of a toric stack. This definition
encompasses and extends previous definitions of toric stacks. The first
paper develops a dictionary between the combinatorics of stacky fans
and the geometry of the corresponding stacks. It also gives a moduli
interpretation of smooth toric stacks, generalizing the one in \cite{perroni}.
The second paper proves an intrinsic characterization of toric stacks.
\end{quote}
\end{itemize}



\subsection{Theorem on formal functions and Grothendieck's Existence Theorem}
\label{subsection-theorem-formal-functions}

\noindent
These papers give generalizations of the theorem on formal functions
\cite[III.4.1.5]{EGA} (sometimes referred to Grothendieck's Fundamental
Theorem for proper morphisms) and Grothendieck's Existence
Theorem \cite[III.5.1.4]{EGA}.
\begin{itemize}
\item Knutson: \emph{Algebraic spaces} \cite[Chapter V]{Kn}
\begin{quote}
Generalizes these theorems to algebraic spaces.
\end{quote}
\item Abramovich-Vistoli: \emph{Compactifying the space of stable maps}
\cite[A.1.1]{abramovich-vistoli}
\begin{quote}
Generalizes these theorems to tame Deligne-Mumford stacks
\end{quote}
\item Olsson and Starr: \emph{Quot functors for Deligne-Mumford stacks}
\cite{olsson-starr}
\begin{quote}
Generalizes these theorems to separated Deligne-Mumford stacks.
\end{quote}
\item Olsson: \emph{On proper coverings of Artin stacks}
\cite{olsson_proper}
\begin{quote}
Provides a generalization to proper Artin stacks.
\end{quote}
\item Conrad: \emph{Formal GAGA on Artin stacks} \cite{conrad_gaga}
\begin{quote}
Provides a generalization to proper Artin stacks and proves a formal
GAGA theorem.
\end{quote}
\item Olsson: \emph{Sheaves on Artin stacks} \cite{olsson_sheaves}
\begin{quote}
Provides another proof for the generalization to proper Artin stacks.
\end{quote}
\end{itemize}


\subsection{Group actions on stacks}
\label{subsection-group-actions}

\noindent
Actions of groups on algebraic stacks naturally appear.
For instance, symmetric group $S_n$ acts on $\overline{\mathcal{M}}_{g, n}$
and for an action of a group $G$ on a scheme $X$, the normalizer of $G$ in
$\text{Aut}(X)$ acts on $[X/G]$. Furthermore, torus actions on stacks
often appear in Gromov-Witten theory.
\begin{itemize}
\item Romagny: \emph{Group actions on stacks and applications}
\cite{romagny_actions}
\begin{quote}
This paper makes precise what it means for a group to act on an algebraic
stack and proves existence of fixed points as well as existence of quotients
for actions of group schemes on algebraic stacks. See also Romagny's earlier
note \cite{romagny_notes}.
\end{quote}
\end{itemize}

\subsection{Taking roots of line bundles}
\label{subsection-root-stacks}

\noindent
This useful construction was discovered independently by Cadman and by
Abramovich, Graber and Vistoli. Given a scheme $X$ with an effective Cartier
divisor $D$, the $r$th root stack is an Artin stack branched over $X$ at $D$
with a $\mu_r$ stabilizer over $D$ and scheme-like away from $D$.
\begin{itemize}
\item Charles Cadman
\emph{Using Stacks to Impose Tangency Conditions on Curves}
\cite{cadman}
\item Abramovich, Graber, Vistoli: \emph{Gromov-Witten theory for
Deligne-Mumford stacks} \cite{agv}
\end{itemize}

\subsection{Other papers}
\label{subsection-other}

\noindent
Potpourri of other papers.
\begin{itemize}
\item Lieblich: \emph{Moduli of twisted sheaves} \cite{lieblich_twisted}
\begin{quote}
This paper contains a summary of gerbes and twisted sheaves.
If $\mathcal{X} \rightarrow X$ is a $\mu_n$-gerbe with $X$ a projective
relative surface with smooth connected geometric fibers, it is shown that
the stack of semistable $\mathcal{X}$-twisted sheaves is an Artin stack
locally of finite presentation over $S$. This paper also develops the
theory of associated points and purity of sheaves on Artin stacks.
\end{quote}
\item Lieblich, Osserman:
\emph{Functorial reconstruction theorem for stacks}
\cite{lieblich-osserman}
\begin{quote}
Proves some surprising and interesting results on
when an algebraic stack can be reconstructed from its associated functor.
\end{quote}
\item David Rydh:
\emph{Noetherian approximation of algebraic spaces and stacks}
\cite{rydh_approx}
\begin{quote}
This paper shows that every quasi-compact algebraic stack
with quasi-finite diagonal can be approximated by a Noetherian stack.
There are applications to removing the Noetherian hypothesis in results
of Chevalley, Serre, Zariski and Chow.
\end{quote}
\end{itemize}



\section{Stacks in other fields}
\label{section-stacks-areas}

\begin{itemize}
\item Behrend and Noohi: \emph{Uniformization of Deligne-Mumford curves}
\cite{behrend-noohi}
\begin{quote}
Gives an overview and comparison of topological, analytic and algebraic stacks.
\end{quote}
\item Behrang Noohi: \emph{Foundations of topological stacks I} \cite{noohi}
\item David Metzler: \emph{Topological and smooth stacks} \cite{metzler}
\end{itemize}


\section{Higher stacks}
\label{section-higher-stacks}

\begin{itemize}
\item Lurie: \emph{Higher topos theory} \cite{lurie_topos}
\item Lurie: \emph{Derived Algebraic Geometry I - V}
\cite{dag1}, \cite{dag2}, \cite{dag3}, \cite{dag4}, \cite{dag5}
\item To\"en: \emph{Higher and derived stacks: a global overview}
\cite{toen_higher}
\item To\"en and Vezzosi: \emph{Homotopical algebraic geometry I, II}
\cite{hag1}, \cite{hag2}
\end{itemize}

\begin{multicols}{2}[\section{Other chapters}]
\noindent
Preliminaries
\begin{enumerate}
\item \hyperref[introduction-section-phantom]{Introduction}
\item \hyperref[conventions-section-phantom]{Conventions}
\item \hyperref[sets-section-phantom]{Set Theory}
\item \hyperref[categories-section-phantom]{Categories}
\item \hyperref[topology-section-phantom]{Topology}
\item \hyperref[sheaves-section-phantom]{Sheaves on Spaces}
\item \hyperref[sites-section-phantom]{Sites and Sheaves}
\item \hyperref[stacks-section-phantom]{Stacks}
\item \hyperref[fields-section-phantom]{Fields}
\item \hyperref[algebra-section-phantom]{Commutative Algebra}
\item \hyperref[brauer-section-phantom]{Brauer Groups}
\item \hyperref[homology-section-phantom]{Homological Algebra}
\item \hyperref[derived-section-phantom]{Derived Categories}
\item \hyperref[simplicial-section-phantom]{Simplicial Methods}
\item \hyperref[more-algebra-section-phantom]{More on Algebra}
\item \hyperref[smoothing-section-phantom]{Smoothing Ring Maps}
\item \hyperref[modules-section-phantom]{Sheaves of Modules}
\item \hyperref[sites-modules-section-phantom]{Modules on Sites}
\item \hyperref[injectives-section-phantom]{Injectives}
\item \hyperref[cohomology-section-phantom]{Cohomology of Sheaves}
\item \hyperref[sites-cohomology-section-phantom]{Cohomology on Sites}
\item \hyperref[dga-section-phantom]{Differential Graded Algebra}
\item \hyperref[dpa-section-phantom]{Divided Power Algebra}
\item \hyperref[sdga-section-phantom]{Differential Graded Sheaves}
\item \hyperref[hypercovering-section-phantom]{Hypercoverings}
\end{enumerate}
Schemes
\begin{enumerate}
\setcounter{enumi}{25}
\item \hyperref[schemes-section-phantom]{Schemes}
\item \hyperref[constructions-section-phantom]{Constructions of Schemes}
\item \hyperref[properties-section-phantom]{Properties of Schemes}
\item \hyperref[morphisms-section-phantom]{Morphisms of Schemes}
\item \hyperref[coherent-section-phantom]{Cohomology of Schemes}
\item \hyperref[divisors-section-phantom]{Divisors}
\item \hyperref[limits-section-phantom]{Limits of Schemes}
\item \hyperref[varieties-section-phantom]{Varieties}
\item \hyperref[topologies-section-phantom]{Topologies on Schemes}
\item \hyperref[descent-section-phantom]{Descent}
\item \hyperref[perfect-section-phantom]{Derived Categories of Schemes}
\item \hyperref[more-morphisms-section-phantom]{More on Morphisms}
\item \hyperref[flat-section-phantom]{More on Flatness}
\item \hyperref[groupoids-section-phantom]{Groupoid Schemes}
\item \hyperref[more-groupoids-section-phantom]{More on Groupoid Schemes}
\item \hyperref[etale-section-phantom]{\'Etale Morphisms of Schemes}
\end{enumerate}
Topics in Scheme Theory
\begin{enumerate}
\setcounter{enumi}{41}
\item \hyperref[chow-section-phantom]{Chow Homology}
\item \hyperref[intersection-section-phantom]{Intersection Theory}
\item \hyperref[pic-section-phantom]{Picard Schemes of Curves}
\item \hyperref[weil-section-phantom]{Weil Cohomology Theories}
\item \hyperref[adequate-section-phantom]{Adequate Modules}
\item \hyperref[dualizing-section-phantom]{Dualizing Complexes}
\item \hyperref[duality-section-phantom]{Duality for Schemes}
\item \hyperref[discriminant-section-phantom]{Discriminants and Differents}
\item \hyperref[derham-section-phantom]{de Rham Cohomology}
\item \hyperref[local-cohomology-section-phantom]{Local Cohomology}
\item \hyperref[algebraization-section-phantom]{Algebraic and Formal Geometry}
\item \hyperref[curves-section-phantom]{Algebraic Curves}
\item \hyperref[resolve-section-phantom]{Resolution of Surfaces}
\item \hyperref[models-section-phantom]{Semistable Reduction}
\item \hyperref[functors-section-phantom]{Functors and Morphisms}
\item \hyperref[equiv-section-phantom]{Derived Categories of Varieties}
\item \hyperref[pione-section-phantom]{Fundamental Groups of Schemes}
\item \hyperref[etale-cohomology-section-phantom]{\'Etale Cohomology}
\item \hyperref[crystalline-section-phantom]{Crystalline Cohomology}
\item \hyperref[proetale-section-phantom]{Pro-\'etale Cohomology}
\item \hyperref[relative-cycles-section-phantom]{Relative Cycles}
\item \hyperref[more-etale-section-phantom]{More \'Etale Cohomology}
\item \hyperref[trace-section-phantom]{The Trace Formula}
\end{enumerate}
Algebraic Spaces
\begin{enumerate}
\setcounter{enumi}{64}
\item \hyperref[spaces-section-phantom]{Algebraic Spaces}
\item \hyperref[spaces-properties-section-phantom]{Properties of Algebraic Spaces}
\item \hyperref[spaces-morphisms-section-phantom]{Morphisms of Algebraic Spaces}
\item \hyperref[decent-spaces-section-phantom]{Decent Algebraic Spaces}
\item \hyperref[spaces-cohomology-section-phantom]{Cohomology of Algebraic Spaces}
\item \hyperref[spaces-limits-section-phantom]{Limits of Algebraic Spaces}
\item \hyperref[spaces-divisors-section-phantom]{Divisors on Algebraic Spaces}
\item \hyperref[spaces-over-fields-section-phantom]{Algebraic Spaces over Fields}
\item \hyperref[spaces-topologies-section-phantom]{Topologies on Algebraic Spaces}
\item \hyperref[spaces-descent-section-phantom]{Descent and Algebraic Spaces}
\item \hyperref[spaces-perfect-section-phantom]{Derived Categories of Spaces}
\item \hyperref[spaces-more-morphisms-section-phantom]{More on Morphisms of Spaces}
\item \hyperref[spaces-flat-section-phantom]{Flatness on Algebraic Spaces}
\item \hyperref[spaces-groupoids-section-phantom]{Groupoids in Algebraic Spaces}
\item \hyperref[spaces-more-groupoids-section-phantom]{More on Groupoids in Spaces}
\item \hyperref[bootstrap-section-phantom]{Bootstrap}
\item \hyperref[spaces-pushouts-section-phantom]{Pushouts of Algebraic Spaces}
\end{enumerate}
Topics in Geometry
\begin{enumerate}
\setcounter{enumi}{81}
\item \hyperref[spaces-chow-section-phantom]{Chow Groups of Spaces}
\item \hyperref[groupoids-quotients-section-phantom]{Quotients of Groupoids}
\item \hyperref[spaces-more-cohomology-section-phantom]{More on Cohomology of Spaces}
\item \hyperref[spaces-simplicial-section-phantom]{Simplicial Spaces}
\item \hyperref[spaces-duality-section-phantom]{Duality for Spaces}
\item \hyperref[formal-spaces-section-phantom]{Formal Algebraic Spaces}
\item \hyperref[restricted-section-phantom]{Algebraization of Formal Spaces}
\item \hyperref[spaces-resolve-section-phantom]{Resolution of Surfaces Revisited}
\end{enumerate}
Deformation Theory
\begin{enumerate}
\setcounter{enumi}{89}
\item \hyperref[formal-defos-section-phantom]{Formal Deformation Theory}
\item \hyperref[defos-section-phantom]{Deformation Theory}
\item \hyperref[cotangent-section-phantom]{The Cotangent Complex}
\item \hyperref[examples-defos-section-phantom]{Deformation Problems}
\end{enumerate}
Algebraic Stacks
\begin{enumerate}
\setcounter{enumi}{93}
\item \hyperref[algebraic-section-phantom]{Algebraic Stacks}
\item \hyperref[examples-stacks-section-phantom]{Examples of Stacks}
\item \hyperref[stacks-sheaves-section-phantom]{Sheaves on Algebraic Stacks}
\item \hyperref[criteria-section-phantom]{Criteria for Representability}
\item \hyperref[artin-section-phantom]{Artin's Axioms}
\item \hyperref[quot-section-phantom]{Quot and Hilbert Spaces}
\item \hyperref[stacks-properties-section-phantom]{Properties of Algebraic Stacks}
\item \hyperref[stacks-morphisms-section-phantom]{Morphisms of Algebraic Stacks}
\item \hyperref[stacks-limits-section-phantom]{Limits of Algebraic Stacks}
\item \hyperref[stacks-cohomology-section-phantom]{Cohomology of Algebraic Stacks}
\item \hyperref[stacks-perfect-section-phantom]{Derived Categories of Stacks}
\item \hyperref[stacks-introduction-section-phantom]{Introducing Algebraic Stacks}
\item \hyperref[stacks-more-morphisms-section-phantom]{More on Morphisms of Stacks}
\item \hyperref[stacks-geometry-section-phantom]{The Geometry of Stacks}
\end{enumerate}
Topics in Moduli Theory
\begin{enumerate}
\setcounter{enumi}{107}
\item \hyperref[moduli-section-phantom]{Moduli Stacks}
\item \hyperref[moduli-curves-section-phantom]{Moduli of Curves}
\end{enumerate}
Miscellany
\begin{enumerate}
\setcounter{enumi}{109}
\item \hyperref[examples-section-phantom]{Examples}
\item \hyperref[exercises-section-phantom]{Exercises}
\item \hyperref[guide-section-phantom]{Guide to Literature}
\item \hyperref[desirables-section-phantom]{Desirables}
\item \hyperref[coding-section-phantom]{Coding Style}
\item \hyperref[obsolete-section-phantom]{Obsolete}
\item \hyperref[fdl-section-phantom]{GNU Free Documentation License}
\item \hyperref[index-section-phantom]{Auto Generated Index}
\end{enumerate}
\end{multicols}



\bibliography{my}
\bibliographystyle{amsalpha}

\end{document}
