\IfFileExists{stacks-project.cls}{%
\documentclass{stacks-project}
}{%
\documentclass{amsart}
}

% For dealing with references we use the comment environment
\usepackage{verbatim}
\newenvironment{reference}{\comment}{\endcomment}
%\newenvironment{reference}{}{}
\newenvironment{slogan}{\comment}{\endcomment}
\newenvironment{history}{\comment}{\endcomment}

% For commutative diagrams we use Xy-pic
\usepackage[all]{xy}

% We use 2cell for 2-commutative diagrams.
\xyoption{2cell}
\UseAllTwocells

% We use multicol for the list of chapters between chapters
\usepackage{multicol}

% This is generall recommended for better output
\usepackage{lmodern}
\usepackage[T1]{fontenc}

% For cross-file-references
\usepackage{xr-hyper}

% Package for hypertext links:
\usepackage{hyperref}

% For any local file, say "hello.tex" you want to link to please
% use \externaldocument[hello-]{hello}
\externaldocument[introduction-]{introduction}
\externaldocument[conventions-]{conventions}
\externaldocument[sets-]{sets}
\externaldocument[categories-]{categories}
\externaldocument[topology-]{topology}
\externaldocument[sheaves-]{sheaves}
\externaldocument[sites-]{sites}
\externaldocument[stacks-]{stacks}
\externaldocument[fields-]{fields}
\externaldocument[algebra-]{algebra}
\externaldocument[brauer-]{brauer}
\externaldocument[homology-]{homology}
\externaldocument[derived-]{derived}
\externaldocument[simplicial-]{simplicial}
\externaldocument[more-algebra-]{more-algebra}
\externaldocument[smoothing-]{smoothing}
\externaldocument[modules-]{modules}
\externaldocument[sites-modules-]{sites-modules}
\externaldocument[injectives-]{injectives}
\externaldocument[cohomology-]{cohomology}
\externaldocument[sites-cohomology-]{sites-cohomology}
\externaldocument[dga-]{dga}
\externaldocument[dpa-]{dpa}
\externaldocument[sdga-]{sdga}
\externaldocument[hypercovering-]{hypercovering}
\externaldocument[schemes-]{schemes}
\externaldocument[constructions-]{constructions}
\externaldocument[properties-]{properties}
\externaldocument[morphisms-]{morphisms}
\externaldocument[coherent-]{coherent}
\externaldocument[divisors-]{divisors}
\externaldocument[limits-]{limits}
\externaldocument[varieties-]{varieties}
\externaldocument[topologies-]{topologies}
\externaldocument[descent-]{descent}
\externaldocument[perfect-]{perfect}
\externaldocument[more-morphisms-]{more-morphisms}
\externaldocument[flat-]{flat}
\externaldocument[groupoids-]{groupoids}
\externaldocument[more-groupoids-]{more-groupoids}
\externaldocument[etale-]{etale}
\externaldocument[chow-]{chow}
\externaldocument[intersection-]{intersection}
\externaldocument[pic-]{pic}
\externaldocument[weil-]{weil}
\externaldocument[adequate-]{adequate}
\externaldocument[dualizing-]{dualizing}
\externaldocument[duality-]{duality}
\externaldocument[discriminant-]{discriminant}
\externaldocument[derham-]{derham}
\externaldocument[local-cohomology-]{local-cohomology}
\externaldocument[algebraization-]{algebraization}
\externaldocument[curves-]{curves}
\externaldocument[resolve-]{resolve}
\externaldocument[models-]{models}
\externaldocument[functors-]{functors}
\externaldocument[equiv-]{equiv}
\externaldocument[pione-]{pione}
\externaldocument[etale-cohomology-]{etale-cohomology}
\externaldocument[proetale-]{proetale}
\externaldocument[relative-cycles-]{relative-cycles}
\externaldocument[more-etale-]{more-etale}
\externaldocument[trace-]{trace}
\externaldocument[crystalline-]{crystalline}
\externaldocument[spaces-]{spaces}
\externaldocument[spaces-properties-]{spaces-properties}
\externaldocument[spaces-morphisms-]{spaces-morphisms}
\externaldocument[decent-spaces-]{decent-spaces}
\externaldocument[spaces-cohomology-]{spaces-cohomology}
\externaldocument[spaces-limits-]{spaces-limits}
\externaldocument[spaces-divisors-]{spaces-divisors}
\externaldocument[spaces-over-fields-]{spaces-over-fields}
\externaldocument[spaces-topologies-]{spaces-topologies}
\externaldocument[spaces-descent-]{spaces-descent}
\externaldocument[spaces-perfect-]{spaces-perfect}
\externaldocument[spaces-more-morphisms-]{spaces-more-morphisms}
\externaldocument[spaces-flat-]{spaces-flat}
\externaldocument[spaces-groupoids-]{spaces-groupoids}
\externaldocument[spaces-more-groupoids-]{spaces-more-groupoids}
\externaldocument[bootstrap-]{bootstrap}
\externaldocument[spaces-pushouts-]{spaces-pushouts}
\externaldocument[spaces-chow-]{spaces-chow}
\externaldocument[groupoids-quotients-]{groupoids-quotients}
\externaldocument[spaces-more-cohomology-]{spaces-more-cohomology}
\externaldocument[spaces-simplicial-]{spaces-simplicial}
\externaldocument[spaces-duality-]{spaces-duality}
\externaldocument[formal-spaces-]{formal-spaces}
\externaldocument[restricted-]{restricted}
\externaldocument[spaces-resolve-]{spaces-resolve}
\externaldocument[formal-defos-]{formal-defos}
\externaldocument[defos-]{defos}
\externaldocument[cotangent-]{cotangent}
\externaldocument[examples-defos-]{examples-defos}
\externaldocument[algebraic-]{algebraic}
\externaldocument[examples-stacks-]{examples-stacks}
\externaldocument[stacks-sheaves-]{stacks-sheaves}
\externaldocument[criteria-]{criteria}
\externaldocument[artin-]{artin}
\externaldocument[quot-]{quot}
\externaldocument[stacks-properties-]{stacks-properties}
\externaldocument[stacks-morphisms-]{stacks-morphisms}
\externaldocument[stacks-limits-]{stacks-limits}
\externaldocument[stacks-cohomology-]{stacks-cohomology}
\externaldocument[stacks-perfect-]{stacks-perfect}
\externaldocument[stacks-introduction-]{stacks-introduction}
\externaldocument[stacks-more-morphisms-]{stacks-more-morphisms}
\externaldocument[stacks-geometry-]{stacks-geometry}
\externaldocument[moduli-]{moduli}
\externaldocument[moduli-curves-]{moduli-curves}
\externaldocument[examples-]{examples}
\externaldocument[exercises-]{exercises}
\externaldocument[guide-]{guide}
\externaldocument[desirables-]{desirables}
\externaldocument[coding-]{coding}
\externaldocument[obsolete-]{obsolete}
\externaldocument[fdl-]{fdl}
\externaldocument[index-]{index}

% Theorem environments.
%
\theoremstyle{plain}
\newtheorem{theorem}[subsection]{Theorem}
\newtheorem{proposition}[subsection]{Proposition}
\newtheorem{lemma}[subsection]{Lemma}

\theoremstyle{definition}
\newtheorem{definition}[subsection]{Definition}
\newtheorem{example}[subsection]{Example}
\newtheorem{exercise}[subsection]{Exercise}
\newtheorem{situation}[subsection]{Situation}

\theoremstyle{remark}
\newtheorem{remark}[subsection]{Remark}
\newtheorem{remarks}[subsection]{Remarks}

\numberwithin{equation}{subsection}

% Macros
%
\def\lim{\mathop{\mathrm{lim}}\nolimits}
\def\colim{\mathop{\mathrm{colim}}\nolimits}
\def\Spec{\mathop{\mathrm{Spec}}}
\def\Hom{\mathop{\mathrm{Hom}}\nolimits}
\def\Ext{\mathop{\mathrm{Ext}}\nolimits}
\def\SheafHom{\mathop{\mathcal{H}\!\mathit{om}}\nolimits}
\def\SheafExt{\mathop{\mathcal{E}\!\mathit{xt}}\nolimits}
\def\Sch{\mathit{Sch}}
\def\Mor{\mathop{\mathrm{Mor}}\nolimits}
\def\Ob{\mathop{\mathrm{Ob}}\nolimits}
\def\Sh{\mathop{\mathit{Sh}}\nolimits}
\def\NL{\mathop{N\!L}\nolimits}
\def\CH{\mathop{\mathrm{CH}}\nolimits}
\def\proetale{{pro\text{-}\acute{e}tale}}
\def\etale{{\acute{e}tale}}
\def\QCoh{\mathit{QCoh}}
\def\Ker{\mathop{\mathrm{Ker}}}
\def\Im{\mathop{\mathrm{Im}}}
\def\Coker{\mathop{\mathrm{Coker}}}
\def\Coim{\mathop{\mathrm{Coim}}}

% Boxtimes
%
\DeclareMathSymbol{\boxtimes}{\mathbin}{AMSa}{"02}

%
% Macros for moduli stacks/spaces
%
\def\QCohstack{\mathcal{QC}\!\mathit{oh}}
\def\Cohstack{\mathcal{C}\!\mathit{oh}}
\def\Spacesstack{\mathcal{S}\!\mathit{paces}}
\def\Quotfunctor{\mathrm{Quot}}
\def\Hilbfunctor{\mathrm{Hilb}}
\def\Curvesstack{\mathcal{C}\!\mathit{urves}}
\def\Polarizedstack{\mathcal{P}\!\mathit{olarized}}
\def\Complexesstack{\mathcal{C}\!\mathit{omplexes}}
% \Pic is the operator that assigns to X its picard group, usage \Pic(X)
% \Picardstack_{X/B} denotes the Picard stack of X over B
% \Picardfunctor_{X/B} denotes the Picard functor of X over B
\def\Pic{\mathop{\mathrm{Pic}}\nolimits}
\def\Picardstack{\mathcal{P}\!\mathit{ic}}
\def\Picardfunctor{\mathrm{Pic}}
\def\Deformationcategory{\mathcal{D}\!\mathit{ef}}


% OK, start here.
%
\begin{document}

\title{Decent Algebraic Spaces}


\maketitle

\phantomsection
\label{section-phantom}

\tableofcontents

\section{Introduction}
\label{section-introduction}

\noindent
In this chapter we study ``local'' properties of general
algebraic spaces, i.e., those algebraic spaces which aren't quasi-separated.
Quasi-separated algebraic spaces are studied in \cite{Kn}.
It turns out that essentially new phenomena happen, especially
regarding points and specializations of points, on more
general algebraic spaces. On the other hand, for most basic results
on algebraic spaces, one needn't worry about these phenomena, which is why
we have decided to have this material in a separate chapter following
the standard development of the theory.



\section{Conventions}
\label{section-conventions}

\noindent
The standing assumption is that all schemes are contained in
a big fppf site $\Sch_{fppf}$. And all rings $A$ considered
have the property that $\Spec(A)$ is (isomorphic) to an
object of this big site.

\medskip\noindent
Let $S$ be a scheme and let $X$ be an algebraic space over $S$.
In this chapter and the following we will write $X \times_S X$
for the product of $X$ with itself (in the category of algebraic
spaces over $S$), instead of $X \times X$.



\section{Universally bounded fibres}
\label{section-universally-bounded}

\noindent
We briefly discuss what it means for a morphism from a scheme to an
algebraic space to have universally bounded fibres. Please refer to
Morphisms, Section \ref{morphisms-section-universally-bounded}
for similar definitions and results on morphisms of schemes.

\begin{definition}
\label{definition-universally-bounded}
Let $S$ be a scheme. Let $X$ be an algebraic space over $S$, and
let $U$ be a scheme over $S$. Let $f : U \to X$ be a morphism over $S$.
We say the {\it fibres of $f$ are universally bounded}\footnote{This is
probably nonstandard notation.}
if there exists an integer $n$ such that for all fields
$k$ and all morphisms $\Spec(k) \to X$ the fibre
product $\Spec(k) \times_X U$ is a finite scheme over $k$
whose degree over $k$ is $\leq n$.
\end{definition}

\noindent
This definition makes sense because the fibre product
$\Spec(k) \times_Y X$ is a scheme. Moreover, if $Y$ is a scheme
we recover the notion of
Morphisms, Definition \ref{morphisms-definition-universally-bounded}
by virtue of
Morphisms, Lemma \ref{morphisms-lemma-characterize-universally-bounded}.

\begin{lemma}
\label{lemma-composition-universally-bounded}
Let $S$ be a scheme. Let $X$ be an algebraic space over $S$.
Let $V \to U$ be a morphism of schemes over $S$, and let
$U \to X$ be a morphism from $U$ to $X$. If the fibres of
$V \to U$ and $U \to X$ are universally bounded, then so
are the fibres of $V \to X$.
\end{lemma}

\begin{proof}
Let $n$ be an integer which works for $V \to U$, and let $m$ be
an integer which works for $U \to X$ in
Definition \ref{definition-universally-bounded}.
Let $\Spec(k) \to X$ be a morphism, where $k$ is a field.
Consider the morphisms
$$
\Spec(k) \times_X V
\longrightarrow
\Spec(k) \times_X U
\longrightarrow
\Spec(k).
$$
By assumption the scheme $\Spec(k) \times_X U$
is finite of degree at most $m$ over $k$, and $n$ is an integer which
bounds the degree of the fibres of the first morphism. Hence by
Morphisms, Lemma \ref{morphisms-lemma-composition-universally-bounded}
we conclude that $\Spec(k) \times_X V$ is finite over $k$
of degree at most $nm$.
\end{proof}

\begin{lemma}
\label{lemma-base-change-universally-bounded}
Let $S$ be a scheme.
Let $Y \to X$ be a representable morphism of algebraic spaces over $S$.
Let $U \to X$ be a morphism from a scheme to $X$.
If the fibres of $U \to X$ are universally bounded, then the fibres
of $U \times_X Y \to Y$ are universally bounded.
\end{lemma}

\begin{proof}
This is clear from the definition, and properties of fibre products.
(Note that $U \times_X Y$ is a scheme
as we assumed $Y \to X$ representable, so the definition applies.)
\end{proof}

\begin{lemma}
\label{lemma-descent-universally-bounded}
Let $S$ be a scheme. Let $g : Y \to X$ be a representable morphism of
algebraic spaces over $S$. Let $f : U \to X$ be a morphism from a scheme
towards $X$. Let $f' : U \times_X Y \to Y$ be the base change of $f$.
If
$$
\Im(|f| : |U| \to |X|) \subset \Im(|g| : |Y| \to |X|)
$$
and $f'$ has universally bounded fibres, then $f$ has universally
bounded fibres.
\end{lemma}

\begin{proof}
Let $n \geq 0$ be an integer bounding the degrees of the fibre
products $\Spec(k) \times_Y (U \times_X Y)$ as in
Definition \ref{definition-universally-bounded} for the morphism $f'$.
We claim that $n$ works for $f$ also. Namely, suppose that
$x : \Spec(k) \to X$ is a morphism from the spectrum of
a field. Then either $\Spec(k) \times_X U$ is empty (and there
is nothing to prove), or $x$ is in the image of $|f|$. By
Properties of Spaces,
Lemma \ref{spaces-properties-lemma-points-cartesian}
and the assumption of the lemma we see
that this means there exists a field extension $k'/k$ and a
commutative diagram
$$
\xymatrix{
\Spec(k') \ar[r] \ar[d] & Y \ar[d] \\
\Spec(k) \ar[r] & X
}
$$
Hence we see that
$$
\Spec(k') \times_Y (U \times_X Y) =
\Spec(k') \times_{\Spec(k)} (\Spec(k) \times_X U)
$$
Since the scheme $\Spec(k') \times_Y (U \times_X Y)$ is assumed finite
of degree $\leq n$ over $k'$ it follows that also $\Spec(k) \times_X U$
is finite of degree $\leq n$ over $k$ as desired. (Some details omitted.)
\end{proof}

\begin{lemma}
\label{lemma-universally-bounded-permanence}
Let $S$ be a scheme. Let $X$ be an algebraic space over $S$.
Consider a commutative diagram
$$
\xymatrix{
U \ar[rd]_g \ar[rr]_f & & V \ar[ld]^h \\
& X &
}
$$
where $U$ and $V$ are schemes. If $g$ has universally bounded fibres,
and $f$ is surjective and flat, then also $h$ has universally bounded fibres.
\end{lemma}

\begin{proof}
Assume $g$ has universally bounded fibres, and $f$ is surjective and flat.
Say $n \geq 0$ is an integer which bounds the degrees of the schemes
$\Spec(k) \times_X U$ as in
Definition \ref{definition-universally-bounded}.
We claim $n$ also works for $h$.
Let $\Spec(k) \to X$ be a morphism from the spectrum of a
field to $X$. Consider the morphism of schemes
$$
\Spec(k) \times_X V \longrightarrow \Spec(k) \times_X U
$$
It is flat and surjective. By assumption the scheme
on the left is finite of degree $\leq n$ over $\Spec(k)$.
It follows from
Morphisms, Lemma \ref{morphisms-lemma-universally-bounded-permanence}
that the degree of the scheme on the right is also bounded by $n$
as desired.
\end{proof}

\begin{lemma}
\label{lemma-universally-bounded-finite-fibres}
Let $S$ be a scheme.
Let $X$ be an algebraic space over $S$, and let $U$ be a scheme over $S$.
Let $\varphi : U \to X$ be a morphism over $S$.
If the fibres of $\varphi$ are universally bounded, then there exists an
integer $n$ such that each fibre of $|U| \to |X|$ has at most
$n$ elements.
\end{lemma}

\begin{proof}
The integer $n$ of Definition \ref{definition-universally-bounded} works.
Namely, pick $x \in |X|$. Represent $x$ by a morphism
$x : \Spec(k) \to X$. Then we get a commutative diagram
$$
\xymatrix{
\Spec(k) \times_X U \ar[r] \ar[d] & U \ar[d] \\
\Spec(k) \ar[r]^x & X
}
$$
which shows (via
Properties of Spaces,
Lemma \ref{spaces-properties-lemma-points-cartesian})
that the inverse image of $x$ in $|U|$ is the image of
the top horizontal arrow. Since $\Spec(k) \times_X U$ is finite
of degree $\leq n$ over $k$ it has at most $n$ points.
\end{proof}








\section{Finiteness conditions and points}
\label{section-points-monomorphisms}

\noindent
In this section we elaborate on the question of when points can be represented
by monomorphisms from spectra of fields into the space.

\begin{remark}
\label{remark-recall}
Before we give the proof of the next lemma let us recall some facts
about \'etale morphisms of schemes:
\begin{enumerate}
\item An \'etale morphism is flat and hence generalizations lift along
an \'etale morphism
(Morphisms, Lemmas \ref{morphisms-lemma-etale-flat}
and \ref{morphisms-lemma-generalizations-lift-flat}).
\item An \'etale morphism is unramified, an unramified morphism is locally
quasi-finite, hence fibres are discrete
(Morphisms, Lemmas \ref{morphisms-lemma-flat-unramified-etale},
\ref{morphisms-lemma-unramified-quasi-finite}, and
\ref{morphisms-lemma-quasi-finite-at-point-characterize}).
\item A quasi-compact \'etale morphism is quasi-finite and in particular
has finite fibres
(Morphisms, Lemmas \ref{morphisms-lemma-quasi-finite-locally-quasi-compact} and
\ref{morphisms-lemma-quasi-finite}).
\item An \'etale scheme over a field $k$ is a disjoint union of spectra
of finite separable field extension of $k$
(Morphisms, Lemma \ref{morphisms-lemma-etale-over-field}).
\end{enumerate}
For a general discussion of \'etale morphisms, please see
\'Etale Morphisms, Section \ref{etale-section-etale-morphisms}.
\end{remark}

\begin{lemma}
\label{lemma-U-finite-above-x}
Let $S$ be a scheme. Let $X$ be an algebraic space over $S$.
Let $x \in |X|$. The following are equivalent:
\begin{enumerate}
\item there exists a family of schemes $U_i$ and
\'etale morphisms $\varphi_i : U_i \to X$ such that
$\coprod \varphi_i : \coprod U_i \to X$ is surjective,
and such that for each $i$ the fibre of
$|U_i| \to |X|$ over $x$ is finite, and
\item for every affine scheme $U$ and \'etale morphism $\varphi : U \to X$
the fibre of $|U| \to |X|$ over $x$ is finite.
\end{enumerate}
\end{lemma}

\begin{proof}
The implication (2) $\Rightarrow$ (1) is trivial.
Let $\varphi_i : U_i \to X$ be a family of \'etale morphisms as in (1).
Let $\varphi : U \to X$ be an \'etale morphism from an affine scheme
towards $X$. Consider the fibre product diagrams
$$
\xymatrix{
U \times_X U_i \ar[r]_-{p_i} \ar[d]_{q_i} & U_i \ar[d]^{\varphi_i} \\
U \ar[r]^\varphi & X
}
\quad \quad
\xymatrix{
\coprod U \times_X U_i \ar[r]_-{\coprod p_i} \ar[d]_{\coprod q_i} &
\coprod U_i \ar[d]^{\coprod \varphi_i} \\
U \ar[r]^\varphi & X
}
$$
Since $q_i$ is \'etale it is open (see Remark \ref{remark-recall}).
Moreover, the morphism $\coprod q_i$ is surjective.
Hence there exist finitely many indices $i_1, \ldots, i_n$ and
a quasi-compact opens $W_{i_j} \subset U \times_X U_{i_j}$
which surject onto $U$.
The morphism $p_i$ is \'etale, hence locally quasi-finite (see remark on
\'etale morphisms above). Thus we may apply
Morphisms, Lemma
\ref{morphisms-lemma-locally-quasi-finite-qc-source-universally-bounded}
to see the fibres of $p_{i_j}|_{W_{i_j}} : W_{i_j} \to U_i$ are finite.
Hence by
Properties of Spaces,
Lemma \ref{spaces-properties-lemma-points-cartesian}
and the assumption on $\varphi_i$ we conclude that the fibre
of $\varphi$ over $x$ is finite. In other words (2) holds.
\end{proof}

\begin{lemma}
\label{lemma-R-finite-above-x}
Let $S$ be a scheme. Let $X$ be an algebraic space over $S$.
Let $x \in |X|$. The following are equivalent:
\begin{enumerate}
\item there exists a scheme $U$, an \'etale morphism
$\varphi : U \to X$, and points $u, u' \in U$ mapping to
$x$ such that setting $R = U \times_X U$ the fibre of
$$
|R| \to |U| \times_{|X|} |U|
$$
over $(u, u')$ is finite,
\item for every scheme $U$, \'etale morphism $\varphi : U \to X$ and
any points $u, u' \in U$ mapping to
$x$ setting $R = U \times_X U$ the fibre of
$$
|R| \to |U| \times_{|X|} |U|
$$
over $(u, u')$ is finite,
\item there exists a morphism $\Spec(k) \to X$ with $k$ a field
in the equivalence class of $x$ such that the projections
$\Spec(k) \times_X \Spec(k) \to \Spec(k)$ are
\'etale and quasi-compact, and
\item there exists a monomorphism $\Spec(k) \to X$ with $k$ a field
in the equivalence class of $x$.
\end{enumerate}
\end{lemma}

\begin{proof}
Assume (1), i.e., let $\varphi : U \to X$ be an \'etale morphism from a scheme
towards $X$, and let $u, u'$ be points of $U$ lying over $x$
such that the fibre of $|R| \to |U| \times_{|X|} |U|$ over $(u, u')$
is a finite set. In this proof we think of a point $u = \Spec(\kappa(u))$
as a scheme. Note that $u \to U$, $u' \to U$ are monomorphisms (see
Schemes, Lemma \ref{schemes-lemma-injective-points-surjective-stalks}),
hence $u \times_X u' \to R = U \times_X U$ is a monomorphism.
In this language the assumption really means that
$u \times_X u'$ is a scheme whose underlying topological space has
finitely many points.
Let $\psi : W \to X$ be an \'etale morphism from a scheme towards $X$.
Let $w, w' \in W$ be points of $W$ mapping to $x$.
We have to show that $w \times_X w'$ is a scheme whose underlying topological
space has finitely many points.
Consider the fibre product diagram
$$
\xymatrix{
W \times_X U \ar[r]_p \ar[d]_q & U \ar[d]^\varphi \\
W \ar[r]^\psi & X
}
$$
As $x$ is the image of $u$ and $u'$ we may pick points
$\tilde w, \tilde w'$ in $W \times_X U$ with $q(\tilde w) = w$,
$q(\tilde w') = w'$, $u = p(\tilde w)$ and $u' = p(\tilde w')$, see
Properties of Spaces,
Lemma \ref{spaces-properties-lemma-points-cartesian}.
As $p$, $q$ are \'etale the field extensions
$\kappa(w) \subset \kappa(\tilde w) \supset \kappa(u)$ and
$\kappa(w') \subset \kappa(\tilde w') \supset \kappa(u')$ are
finite separable, see Remark \ref{remark-recall}.
Then we get a commutative diagram
$$
\xymatrix{
w \times_X w' \ar[d] &
\tilde w \times_X \tilde w' \ar[l] \ar[d] \ar[r] &
u \times_X u' \ar[d] \\
w \times_X w' &
\tilde w \times_S \tilde w' \ar[l] \ar[r] &
u \times_S u'
}
$$
where the squares are fibre product squares. The lower horizontal
morphisms are \'etale and quasi-compact, as any scheme of the form
$\Spec(k) \times_S \Spec(k')$ is affine, and by our
observations about the field extensions above.
Thus we see that the top horizontal arrows are \'etale and quasi-compact
and hence have finite fibres.
We have seen above that $|u \times_X u'|$ is finite, so we conclude that
$|w \times_X w'|$ is finite. In other words, (2) holds.

\medskip\noindent
Assume (2). Let $U \to X$ be an \'etale morphism from a scheme $U$
such that $x$ is in the image of $|U| \to |X|$. Let $u \in U$ be
a point mapping to $x$. Then we have seen in the previous
paragraph that $u = \Spec(\kappa(u)) \to X$ has the property that
$u \times_X u$ has a finite underlying topological space. On the other
hand, the projection maps $u \times_X u \to u$ are the composition
$$
u \times_X u \longrightarrow
u \times_X U \longrightarrow
u \times_X X = u,
$$
i.e., the composition of a monomorphism (the base change of the monomorphism
$u \to U$) by an \'etale morphism (the base change of the \'etale morphism
$U \to X$). Hence $u \times_X U$ is a disjoint union of spectra of fields
finite separable over $\kappa(u)$ (see
Remark \ref{remark-recall}). Since $u \times_X u$ is finite the image
of it in $u \times_X U$ is a finite disjoint union of spectra of fields
finite separable over $\kappa(u)$. By
Schemes, Lemma \ref{schemes-lemma-mono-towards-spec-field}
we conclude that $u \times_X u$ is a finite disjoint union of spectra
of fields finite separable over $\kappa(u)$. In other words, we see that
$u \times_X u \to u$ is quasi-compact and \'etale. This means that (3) holds.

\medskip\noindent
Let us prove that (3) implies (4). Let $\Spec(k) \to X$ be a morphism
from the spectrum of a field into $X$, in the equivalence class of $x$
such that the two projections
$t, s : R = \Spec(k) \times_X \Spec(k)  \to \Spec(k)$
are quasi-compact and \'etale.
This means in particular
that $R$ is an \'etale equivalence relation on $\Spec(k)$.
By Spaces, Theorem \ref{spaces-theorem-presentation}
we know that the quotient sheaf
$X' = \Spec(k)/R$ is an algebraic space. By
Groupoids, Lemma \ref{groupoids-lemma-quotient-groupoid-restrict}
the map $X' \to X$ is a monomorphism.
Since $s, t$ are quasi-compact, we see that $R$ is quasi-compact and hence
Properties of Spaces,
Lemma \ref{spaces-properties-lemma-point-like-spaces}
applies to $X'$, and we see that
$X' = \Spec(k')$ for some field $k'$. Hence we get a factorization
$$
\Spec(k) \longrightarrow
\Spec(k') \longrightarrow X
$$
which shows that $\Spec(k') \to X$ is a monomorphism mapping
to $x \in |X|$. In other words (4) holds.

\medskip\noindent
Finally, we prove that (4) implies (1). Let $\Spec(k) \to X$
be a monomorphism with $k$ a field in the equivalence class of $x$.
Let $U \to X$ be a surjective \'etale morphism from a scheme $U$ to $X$.
Let $u \in U$ be a point over $x$. Since $\Spec(k) \times_X u$
is nonempty, and since $\Spec(k) \times_X u \to u$ is a monomorphism
we conclude that $\Spec(k) \times_X u = u$ (see
Schemes, Lemma \ref{schemes-lemma-mono-towards-spec-field}).
Hence $u \to U \to X$ factors through $\Spec(k) \to X$, here is
a picture
$$
\xymatrix{
u \ar[r] \ar[d] & U \ar[d] \\
\Spec(k) \ar[r] & X
}
$$
Since the right vertical arrow is \'etale this implies that
$\kappa(u)/k$ is a finite separable extension. Hence we conclude that
$$
u \times_X u = u \times_{\Spec(k)} u
$$
is a finite scheme, and we win by the discussion of the meaning of property
(1) in the first paragraph of this proof.
\end{proof}

\begin{lemma}
\label{lemma-weak-UR-finite-above-x}
Let $S$ be a scheme. Let $X$ be an algebraic space over $S$.
Let $x \in |X|$.
Let $U$ be a scheme and let $\varphi : U \to X$ be an \'etale morphism.
The following are equivalent:
\begin{enumerate}
\item $x$ is in the image of $|U| \to |X|$, and
setting $R = U \times_X U$ the fibres of both
$$
|U| \longrightarrow |X|
\quad\text{and}\quad
|R| \longrightarrow |X|
$$
over $x$ are finite,
\item there exists a monomorphism $\Spec(k) \to X$ with $k$ a field
in the equivalence class of $x$, and
the fibre product $\Spec(k) \times_X U$ is
a finite nonempty scheme over $k$.
\end{enumerate}
\end{lemma}

\begin{proof}
Assume (1). This clearly implies the first condition of
Lemma \ref{lemma-R-finite-above-x} and hence we obtain a monomorphism
$\Spec(k) \to X$ in the class of $x$. Taking the fibre product
we see that $\Spec(k) \times_X U \to \Spec(k)$ is a scheme
\'etale over $\Spec(k)$ with finitely many points, hence a finite
nonempty scheme over $k$, i.e., (2) holds.

\medskip\noindent
Assume (2). By assumption $x$ is in the image of
$|U| \to |X|$. The finiteness of the fibre of
$|U| \to |X|$ over $x$ is clear since this fibre is equal to
$|\Spec(k) \times_X U|$ by
Properties of Spaces,
Lemma \ref{spaces-properties-lemma-points-cartesian}.
The finiteness of the fibre of $|R| \to |X|$ above $x$ is also clear
since it is equal to the set underlying the scheme
$$
(\Spec(k) \times_X U) \times_{\Spec(k)} (\Spec(k) \times_X U)
$$
which is finite over $k$. Thus (1) holds.
\end{proof}

\begin{lemma}
\label{lemma-UR-finite-above-x}
Let $S$ be a scheme. Let $X$ be an algebraic space over $S$.
Let $x \in |X|$. The following are equivalent:
\begin{enumerate}
\item for every affine scheme $U$, any \'etale morphism
$\varphi : U \to X$ setting $R = U \times_X U$ the fibres of both
$$
|U| \longrightarrow |X|
\quad\text{and}\quad
|R| \longrightarrow |X|
$$
over $x$ are finite,
\item there exist schemes $U_i$ and \'etale morphisms
$U_i \to X$ such that $\coprod U_i \to X$ is surjective and for each
$i$, setting $R_i = U_i \times_X U_i$ the fibres of both
$$
|U_i| \longrightarrow |X|
\quad\text{and}\quad
|R_i| \longrightarrow |X|
$$
over $x$ are finite,
\item there exists a monomorphism $\Spec(k) \to X$ with $k$ a field
in the equivalence class of $x$, and for any affine scheme $U$ and \'etale
morphism $U \to X$ the fibre product $\Spec(k) \times_X U$ is
a finite scheme over $k$,
\item there exists a quasi-compact monomorphism $\Spec(k) \to X$
with $k$ a field in the equivalence class of $x$,
\item there exists a quasi-compact morphism $\Spec(k) \to X$
with $k$ a field in the equivalence class of $x$, and
\item every morphism $\Spec(k) \to X$ with $k$ a field in the
equivalence class of $x$ is quasi-compact.
\end{enumerate}
\end{lemma}

\begin{proof}
The equivalence of (1) and (3) follows on applying
Lemma \ref{lemma-weak-UR-finite-above-x}
to every \'etale morphism $U \to X$ with $U$ affine.
It is clear that (3) implies (2).
Assume $U_i \to X$ and $R_i$ are as in (2). We conclude from
Lemma \ref{lemma-U-finite-above-x}
that for any affine scheme $U$ and \'etale morphism $U \to X$
the fibre of $|U| \to |X|$ over $x$ is finite.
Say this fibre is $\{u_1, \ldots, u_n\}$. Then, as
Lemma \ref{lemma-R-finite-above-x} (1)
applies to $U_i \to X$ for some $i$ such that $x$ is in the image of
$|U_i| \to |X|$, we see that the fibre of
$|R = U \times_X U| \to |U| \times_{|X|} |U|$
is finite over $(u_a, u_b)$, $a, b \in \{1, \ldots, n\}$.
Hence the fibre of $|R| \to |X|$ over $x$ is finite.
In this way we see that (1) holds. At this point we know that
(1), (2), and (3) are equivalent.

\medskip\noindent
If (4) holds, then for any affine scheme $U$ and \'etale morphism
$U \to X$ the scheme $\Spec(k) \times_X U$ is on the one hand
\'etale over $k$ (hence a disjoint union of spectra of finite separable
extensions of $k$ by
Remark \ref{remark-recall})
and on the other hand quasi-compact over $U$ (hence quasi-compact).
Thus we see that (3) holds.
Conversely, if $U_i \to X$ is as in (2) and $\Spec(k) \to X$
is a monomorphism as in (3), then
$$
\coprod \Spec(k) \times_X U_i
\longrightarrow
\coprod U_i
$$
is quasi-compact (because over each $U_i$ we see that
$\Spec(k) \times_X U_i$ is a finite disjoint union spectra of fields).
Thus $\Spec(k) \to X$ is quasi-compact by
Morphisms of Spaces, Lemma \ref{spaces-morphisms-lemma-quasi-compact-local}.

\medskip\noindent
It is immediate that (4) implies (5). Conversely, let $\Spec(k) \to X$
be a quasi-compact morphism in the equivalence class of $x$. Let $U \to X$
be an \'etale morphism with $U$ affine. Consider the fibre product
$$
\xymatrix{
F \ar[r] \ar[d] & U \ar[d] \\
\Spec(k) \ar[r] & X
}
$$
Then $F \to U$ is quasi-compact, hence $F$ is quasi-compact.
On the other hand, $F \to \Spec(k)$ is \'etale, hence $F$ is a
finite disjoint union of spectra of finite separable extensions of $k$
(Remark \ref{remark-recall}). Since the image of $|F| \to |U|$
is the fibre of $|U| \to |X|$ over $x$ (Properties of Spaces, Lemma
\ref{spaces-properties-lemma-points-cartesian}), we conclude that
the fibre of $|U| \to |X|$ over $x$ is finite. The scheme
$F \times_{\Spec(k)} F$ is also a finite union of spectra of fields
because it is also quasi-compact and \'etale over $\Spec(k)$.
There is a monomorphism
$F \times_X F \to F \times_{\Spec(k)} F$, hence $F \times_X F$ is
a finite disjoint union of spectra of fields
(Schemes, Lemma \ref{schemes-lemma-mono-towards-spec-field}).
Thus the image of $F \times_X F \to U \times_X U = R$ is finite.
Since this image is the fibre of $|R| \to |X|$ over $x$ by
Properties of Spaces, Lemma \ref{spaces-properties-lemma-points-cartesian}
we conclude that (1) holds. At this point we know that
(1) -- (5) are equivalent.

\medskip\noindent
It is clear that (6) implies (5). Conversly, assume $\Spec(k) \to X$ is
as in (4) and let $\Spec(k') \to X$ be another morphism with $k'$ a
field in the equivalence class of $x$. By
Properties of Spaces, Lemma
\ref{spaces-properties-lemma-equivalence-class-point-monomorphism}
we have a factorization $\Spec(k') \to \Spec(k) \to X$ of the
given morphism. This is a composition of quasi-compact
morphisms and hence quasi-compact (Morphisms of Spaces,
Lemma \ref{spaces-morphisms-lemma-composition-quasi-compact}) as desired.
\end{proof}

\begin{lemma}
\label{lemma-U-universally-bounded}
Let $S$ be a scheme. Let $X$ be an algebraic space over $S$.
The following are equivalent:
\begin{enumerate}
\item there exist schemes $U_i$ and \'etale morphisms
$U_i \to X$ such that $\coprod U_i \to X$ is surjective and
each $U_i \to X$ has universally bounded fibres, and
\item for every affine scheme $U$ and \'etale morphism $\varphi : U \to X$
the fibres of $U \to X$ are universally bounded.
\end{enumerate}
\end{lemma}

\begin{proof}
The implication (2) $\Rightarrow$ (1) is trivial.
Assume (1). Let $(\varphi_i : U_i \to X)_{i \in I}$ be a collection of
\'etale morphisms from schemes towards $X$, covering $X$, such that
each $\varphi_i$ has universally bounded fibres.
Let $\psi : U \to X$ be an \'etale morphism from an affine scheme towards $X$.
For each $i$ consider the fibre product diagram
$$
\xymatrix{
U \times_X U_i \ar[r]_{p_i} \ar[d]_{q_i} & U_i \ar[d]^{\varphi_i} \\
U \ar[r]^\psi & X
}
$$
Since $q_i$ is \'etale it is open (see Remark \ref{remark-recall}).
Moreover, we have $U = \bigcup \Im(q_i)$, since the family
$(\varphi_i)_{i \in I}$ is surjective. Since $U$ is affine, hence quasi-compact
we can finite finitely many $i_1, \ldots, i_n \in I$ and quasi-compact
opens $W_j \subset U \times_X U_{i_j}$ such that
$U = \bigcup p_{i_j}(W_j)$.
The morphism $p_{i_j}$ is \'etale, hence locally quasi-finite
(see remark on \'etale morphisms above). Thus we may apply
Morphisms, Lemma
\ref{morphisms-lemma-locally-quasi-finite-qc-source-universally-bounded}
to see the fibres of $p_{i_j}|_{W_j} : W_j \to U_{i_j}$ are universally
bounded. Hence by
Lemma \ref{lemma-composition-universally-bounded}
we see that the fibres of $W_j \to X$ are universally bounded.
Thus also $\coprod_{j = 1, \ldots, n} W_j \to X$ has universally
bounded fibres. Since $\coprod_{j = 1, \ldots, n} W_j \to X$ factors
through the surjective \'etale map
$\coprod q_{i_j}|_{W_j} : \coprod_{j = 1, \ldots, n} W_j \to U$ we
see that the fibres of $U \to X$ are universally bounded by
Lemma \ref{lemma-universally-bounded-permanence}.
In other words (2) holds.
\end{proof}

\begin{lemma}
\label{lemma-characterize-very-reasonable}
Let $S$ be a scheme.
Let $X$ be an algebraic space over $S$.
The following are equivalent:
\begin{enumerate}
\item there exists a Zariski covering $X = \bigcup X_i$ and for
each $i$ a scheme $U_i$ and a quasi-compact surjective \'etale
morphism $U_i \to X_i$, and
\item there exist schemes $U_i$ and \'etale morphisms $U_i \to X$
such that the projections $U_i \times_X U_i \to U_i$ are quasi-compact
and $\coprod U_i \to X$ is surjective.
\end{enumerate}
\end{lemma}

\begin{proof}
If (1) holds then the morphisms $U_i \to X_i \to X$ are \'etale (combine
Morphisms, Lemma \ref{morphisms-lemma-composition-etale}
and
Spaces, Lemmas
\ref{spaces-lemma-composition-representable-transformations-property} and
\ref{spaces-lemma-morphism-schemes-gives-representable-transformation-property}
).
Moreover, as $U_i \times_X U_i = U_i \times_{X_i} U_i$,
both projections $U_i \times_X U_i \to U_i$ are quasi-compact.

\medskip\noindent
If (2) holds then let $X_i \subset X$ be the open subspace corresponding
to the image of the open map $|U_i| \to |X|$, see
Properties of Spaces,
Lemma \ref{spaces-properties-lemma-etale-image-open}.
The morphisms $U_i \to X_i$ are surjective.
Hence $U_i \to X_i$ is surjective \'etale, and the projections
$U_i \times_{X_i} U_i \to U_i$ are quasi-compact, because
$U_i \times_{X_i} U_i = U_i \times_X U_i$. Thus by
Spaces, Lemma \ref{spaces-lemma-representable-morphisms-spaces-property}
the morphisms $U_i \to X_i$ are quasi-compact.
\end{proof}









\section{Conditions on algebraic spaces}
\label{section-conditions}

\noindent
In this section we discuss the relationship between various natural
conditions on algebraic spaces we have seen above. Please read
Section \ref{section-reasonable-decent}
to get a feeling for the meaning of these conditions.

\begin{lemma}
\label{lemma-bounded-fibres}
Let $S$ be a scheme. Let $X$ be an algebraic space over $S$.
Consider the following conditions on $X$:
\begin{itemize}
\item[] $(\alpha)$ For every $x \in |X|$, the equivalent conditions of
Lemma \ref{lemma-U-finite-above-x}
hold.
\item[] $(\beta)$ For every $x \in |X|$, the equivalent conditions of
Lemma \ref{lemma-R-finite-above-x}
hold.
\item[] $(\gamma)$ For every $x \in |X|$, the equivalent conditions of
Lemma \ref{lemma-UR-finite-above-x}
hold.
\item[] $(\delta)$ The equivalent conditions of
Lemma \ref{lemma-U-universally-bounded}
hold.
\item[] $(\epsilon)$ The equivalent conditions of
Lemma \ref{lemma-characterize-very-reasonable}
hold.
\item[] $(\zeta)$ The space $X$ is Zariski locally quasi-separated.
\item[] $(\eta)$ The space $X$ is quasi-separated
\item[] $(\theta)$ The space $X$ is representable, i.e., $X$ is a scheme.
\item[] $(\iota)$ The space $X$ is a quasi-separated scheme.
\end{itemize}
We have
$$
\xymatrix{
& (\theta) \ar@{=>}[rd] & & & &  \\
(\iota) \ar@{=>}[ru] \ar@{=>}[rd] & &
(\zeta) \ar@{=>}[r] &
(\epsilon) \ar@{=>}[r] &
(\delta) \ar@{=>}[r] &
(\gamma) \ar@{<=>}[r] & (\alpha) + (\beta) \\
& (\eta) \ar@{=>}[ru] & & & &
}
$$
\end{lemma}

\begin{proof}
The implication $(\gamma) \Leftrightarrow (\alpha) + (\beta)$ is immediate.
The implications in the diamond on the left are clear from the
definitions.

\medskip\noindent
Assume $(\zeta)$, i.e., that $X$ is Zariski locally quasi-separated.
Then $(\epsilon)$ holds by
Properties of Spaces, Lemma
\ref{spaces-properties-lemma-quasi-separated-quasi-compact-pieces}.

\medskip\noindent
Assume $(\epsilon)$. By
Lemma \ref{lemma-characterize-very-reasonable}
there exists
a Zariski open covering $X = \bigcup X_i$ such that for each $i$
there exists a scheme $U_i$ and a quasi-compact surjective \'etale morphism
$U_i \to X_i$. Choose an $i$ and an affine open subscheme $W \subset U_i$.
It suffices to show that $W \to X$ has universally bounded fibres, since then
the family of all these morphisms $W \to X$ covers $X$.
To do this we consider the diagram
$$
\xymatrix{
W \times_X U_i \ar[r]_-p \ar[d]_q & U_i \ar[d] \\
W \ar[r] & X
}
$$
Since $W \to X$ factors through $X_i$ we see that
$W \times_X U_i = W \times_{X_i} U_i$, and hence $q$ is quasi-compact.
Since $W$ is affine this implies that the scheme $W \times_X U_i$
is quasi-compact. Thus we may apply
Morphisms, Lemma
\ref{morphisms-lemma-locally-quasi-finite-qc-source-universally-bounded}
and we conclude that $p$ has universally bounded fibres. From
Lemma \ref{lemma-descent-universally-bounded}
we conclude that $W \to X$ has universally bounded fibres as well.

\medskip\noindent
Assume $(\delta)$. Let $U$ be an affine scheme, and let $U \to X$ be an \'etale
morphism. By assumption the fibres of the morphism $U \to X$ are universally
bounded. Thus also the fibres of both projections $R = U \times_X U \to U$
are universally bounded, see
Lemma \ref{lemma-base-change-universally-bounded}.
And by
Lemma \ref{lemma-composition-universally-bounded}
also the fibres of $R \to X$ are universally bounded.
Hence for any $x \in X$ the fibres of $|U| \to |X|$ and $|R| \to |X|$
over $x$ are finite, see
Lemma \ref{lemma-universally-bounded-finite-fibres}.
In other words, the equivalent conditions of
Lemma \ref{lemma-UR-finite-above-x}
hold. This proves that $(\delta) \Rightarrow (\gamma)$.
\end{proof}

\begin{lemma}
\label{lemma-properties-local}
Let $S$ be a scheme.
Let $\mathcal{P}$ be one of the properties
$(\alpha)$, $(\beta)$, $(\gamma)$, $(\delta)$, $(\epsilon)$, $(\zeta)$, or
$(\theta)$ of algebraic spaces listed in
Lemma \ref{lemma-bounded-fibres}.
Then if $X$ is an algebraic space over $S$, and $X = \bigcup X_i$ is a
Zariski open covering such that each $X_i$ has $\mathcal{P}$,
then $X$ has $\mathcal{P}$.
\end{lemma}

\begin{proof}
Let $X$ be an algebraic space over $S$, and let $X = \bigcup X_i$ is a
Zariski open covering such that each $X_i$ has $\mathcal{P}$.

\medskip\noindent
The case $\mathcal{P} = (\alpha)$. The condition $(\alpha)$ for $X_i$
means that for every $x \in |X_i|$ and every affine scheme $U$, and
\'etale morphism $\varphi : U \to X_i$ the fibre of $\varphi : |U| \to |X_i|$
over $x$ is finite. Consider $x \in X$, an affine scheme $U$ and
an \'etale morphism $U \to X$. Since $X = \bigcup X_i$ is a
Zariski open covering there exits a finite affine open covering
$U = U_1 \cup \ldots \cup U_n$ such that each $U_j \to X$ factors through
some $X_{i_j}$. By assumption the fibres of $|U_j | \to |X_{i_j}|$
over $x$ are finite for $j = 1, \ldots, n$. Clearly this means that
the fibre of $|U| \to |X|$ over $x$ is finite.
This proves the result for $(\alpha)$.

\medskip\noindent
The case $\mathcal{P} = (\beta)$. The condition $(\beta)$ for $X_i$ means
that every $x \in |X_i|$ is represented by a monomorphism from the
spectrum of a field towards $X_i$. Hence the same follows for $X$
as $X_i \to X$ is a monomorphism and $X = \bigcup X_i$.

\medskip\noindent
The case $\mathcal{P} = (\gamma)$.
Note that $(\gamma) = (\alpha) + (\beta)$ by
Lemma \ref{lemma-bounded-fibres}
hence the lemma for $(\gamma)$ follows from the cases treated above.

\medskip\noindent
The case $\mathcal{P} = (\delta)$. The condition $(\delta)$ for $X_i$ means
there exist schemes $U_{ij}$ and \'etale morphisms $U_{ij} \to X_i$ with
universally bounded fibres which cover $X_i$. These schemes also give an
\'etale surjective morphism $\coprod U_{ij} \to X$ and $U_{ij} \to X$
still has universally bounded fibres.

\medskip\noindent
The case $\mathcal{P} = (\epsilon)$. The condition $(\epsilon)$ for $X_i$ means
we can find a set $J_i$ and morphisms
$\varphi_{ij} : U_{ij} \to X_i$ such that each $\varphi_{ij}$
is \'etale, both projections $U_{ij} \times_{X_i} U_{ij} \to U_{ij}$
are quasi-compact, and $\coprod_{j \in J_i} U_{ij} \to X_i$ is surjective.
In this case the compositions $U_{ij} \to X_i \to X$ are \'etale
(combine
Morphisms, Lemmas
\ref{morphisms-lemma-composition-etale} and
\ref{morphisms-lemma-open-immersion-etale}
and
Spaces, Lemmas
\ref{spaces-lemma-composition-representable-transformations-property} and
\ref{spaces-lemma-morphism-schemes-gives-representable-transformation-property}
).
Since $X_i \subset X$ is a subspace we see that
$U_{ij} \times_{X_i} U_{ij} = U_{ij} \times_X U_{ij}$, and hence the
condition on fibre products is preserved. And clearly
$\coprod_{i, j} U_{ij} \to X$ is surjective. Hence $X$
satisfies $(\epsilon)$.

\medskip\noindent
The case $\mathcal{P} = (\zeta)$. The condition $(\zeta)$ for $X_i$
means that $X_i$ is Zariski locally quasi-separated. It is immediately
clear that this means $X$ is Zariski locally quasi-separated.

\medskip\noindent
For $(\theta)$, see
Properties of Spaces,
Lemma \ref{spaces-properties-lemma-subscheme}.
\end{proof}

\begin{lemma}
\label{lemma-representable-properties}
Let $S$ be a scheme. Let $\mathcal{P}$ be one of the properties
$(\beta)$, $(\gamma)$, $(\delta)$, $(\epsilon)$, or
$(\theta)$ of algebraic spaces listed in
Lemma \ref{lemma-bounded-fibres}.
Let $X$, $Y$ be algebraic spaces over $S$.
Let $X \to Y$ be a representable morphism.
If $Y$ has property $\mathcal{P}$, so does $X$.
\end{lemma}

\begin{proof}
Assume $f : X \to Y$ is a representable morphism of algebraic spaces,
and assume that $Y$ has $\mathcal{P}$. Let $x \in |X|$, and set
$y = f(x) \in |Y|$.

\medskip\noindent
The case $\mathcal{P} = (\beta)$. Condition $(\beta)$ for $Y$ means
there exists a monomorphism $\Spec(k) \to Y$ representing $y$.
The fibre product $X_y = \Spec(k) \times_Y X$ is a scheme, and
$x$ corresponds to a point of $X_y$, i.e., to a monomorphism
$\Spec(k') \to X_y$. As $X_y \to X$ is a monomorphism also we see
that $x$ is represented by the monomorphism $\Spec(k') \to X_y \to X$.
In other words $(\beta)$ holds for $X$.

\medskip\noindent
The case $\mathcal{P} = (\gamma)$. Since $(\gamma) \Rightarrow (\beta)$
we have seen in the preceding paragraph that $y$ and $x$ can be represented
by monomorphisms as in the following diagram
$$
\xymatrix{
\Spec(k') \ar[r]_-x \ar[d] & X \ar[d] \\
\Spec(k) \ar[r]^-y & Y
}
$$
Also, by definition of property $(\gamma)$ via
Lemma \ref{lemma-UR-finite-above-x} (2)
there exist schemes
$V_i$ and \'etale morphisms $V_i \to Y$ such that $\coprod V_i \to Y$
is surjective and for each $i$, setting $R_i = V_i \times_Y V_i$
the fibres of both
$$
|V_i| \longrightarrow |Y|
\quad\text{and}\quad
|R_i| \longrightarrow |Y|
$$
over $y$ are finite. This means that the schemes
$(V_i)_y$ and $(R_i)_y$ are finite schemes over $y = \Spec(k)$.
As $X \to Y$ is representable, the fibre products $U_i = V_i \times_Y X$
are schemes. The morphisms $U_i \to X$ are \'etale, and
$\coprod U_i \to X$ is surjective. Finally, for each $i$ we have
$$
(U_i)_x =
(V_i \times_Y X)_x =
(V_i)_y \times_{\Spec(k)} \Spec(k')
$$
and
$$
(U_i \times_X U_i)_x =
\left((V_i \times_Y X) \times_X (V_i \times_Y X)\right)_x =
(R_i)_y \times_{\Spec(k)} \Spec(k')
$$
hence these are finite over $k'$ as base changes of the finite
schemes $(V_i)_y$ and $(R_i)_y$. This implies that $(\gamma)$ holds for $X$,
again via the second condition of
Lemma \ref{lemma-UR-finite-above-x}.

\medskip\noindent
The case $\mathcal{P} = (\delta)$. Let $V \to Y$ be an \'etale morphism with
$V$ an affine scheme. Since $Y$ has property $(\delta)$ this morphism has
universally bounded fibres. By
Lemma \ref{lemma-base-change-universally-bounded}
the base change $V \times_Y X \to X$ also has universally bounded fibres.
Hence the first part of
Lemma \ref{lemma-U-universally-bounded}
applies and we see that $Y$ also has property $(\delta)$.

\medskip\noindent
The case $\mathcal{P} = (\epsilon)$. We will repeatedly use
Spaces, Lemma
\ref{spaces-lemma-base-change-representable-transformations-property}.
Let $V_i \to Y$ be as in
Lemma \ref{lemma-characterize-very-reasonable} (2).
Set $U_i = X \times_Y V_i$. The morphisms $U_i \to X$ are \'etale,
and $\coprod U_i \to X$ is surjective. Because
$U_i \times_X U_i = X \times_Y (V_i \times_Y V_i)$ we see
that the projections $U_i \times_Y U_i \to U_i$ are
base changes of the projections $V_i \times_Y V_i \to V_i$, and so
quasi-compact as well. Hence $X$ satisfies
Lemma \ref{lemma-characterize-very-reasonable} (2).

\medskip\noindent
The case $\mathcal{P} = (\theta)$. In this case the result is
Categories, Lemma \ref{categories-lemma-representable-over-representable}.
\end{proof}











\section{Reasonable and decent algebraic spaces}
\label{section-reasonable-decent}

\noindent
In
Lemma \ref{lemma-bounded-fibres}
we have seen a number of conditions on algebraic spaces related to
the behaviour of \'etale morphisms from affine schemes into $X$
and related to the existence of special \'etale coverings of $X$ by
schemes. We tabulate the different types of conditions here:
$$
\boxed{
\begin{matrix}
(\alpha) & \text{fibres of \'etale morphisms from affines are finite} \\
(\beta) & \text{points come from monomorphisms of spectra of fields} \\
(\gamma) & \text{points come from quasi-compact monomorphisms of
spectra of fields} \\
(\delta) & \text{fibres of \'etale morphisms from affines are universally
bounded} \\
(\epsilon) & \text{cover by \'etale morphisms from schemes quasi-compact
onto their image}
\end{matrix}
}
$$

\medskip\noindent
The conditions in the following definition
are not exactly conditions on the diagonal of $X$, but they are in some
sense separation conditions on $X$.

\begin{definition}
\label{definition-very-reasonable}
Let $S$ be a scheme.
Let $X$ be an algebraic space over $S$.
\begin{enumerate}
\item We say $X$ is {\it decent} if for every point $x \in X$ the equivalent
conditions of
Lemma \ref{lemma-UR-finite-above-x}
hold, in other words property $(\gamma)$ of
Lemma \ref{lemma-bounded-fibres}
holds.
\item We say $X$ is {\it reasonable} if the equivalent conditions of
Lemma \ref{lemma-U-universally-bounded}
hold, in other words property $(\delta)$ of
Lemma \ref{lemma-bounded-fibres}
holds.
\item We say $X$ is {\it very reasonable} if the equivalent conditions of
Lemma \ref{lemma-characterize-very-reasonable}
hold, i.e., property $(\epsilon)$ of
Lemma \ref{lemma-bounded-fibres}
holds.
\end{enumerate}
\end{definition}

\noindent
We have the following implications among these conditions on algebraic spaces:
$$
\xymatrix{
\text{representable} \ar@{=>}[rd] & & & \\
 & \text{very reasonable} \ar@{=>}[r] &
\text{reasonable} \ar@{=>}[r] &
\text{decent} \\
\text{quasi-separated} \ar@{=>}[ru] & & &
}
$$
The notion of a very reasonable algebraic space is obsolete.
It was introduced because the assumption was needed to prove some results
which are now proven for the class of decent spaces.
The class of decent spaces is the largest class of spaces $X$ where one has
a good relationship between the topology of $|X|$ and
properties of $X$ itself.

\begin{example}
\label{example-not-decent}
The algebraic space $\mathbf{A}^1_{\mathbf{Q}}/\mathbf{Z}$ constructed in
Spaces, Example \ref{spaces-example-affine-line-translation}
is not decent as its ``generic point'' cannot be represented by a monomorphism
from the spectrum of a field.
\end{example}

\begin{remark}
\label{remark-reasonable}
Reasonable algebraic spaces are technically easier to work with than very
reasonable algebraic spaces. For example, if $X \to Y$ is a quasi-compact
\'etale surjective morphism of algebraic spaces and $X$ is reasonable, then
so is $Y$, see
Lemma \ref{lemma-descent-conditions}
but we don't know if this is true for the property ``very reasonable''.
Below we give another technical property enjoyed by reasonable
algebraic spaces.
\end{remark}

\begin{lemma}
\label{lemma-fun-property-reasonable}
Let $S$ be a scheme.
Let $X$ be a quasi-compact reasonable algebraic space.
Then there exists a directed system of quasi-compact and quasi-separated
algebraic spaces $X_i$ such that $X = \colim_i X_i$
(colimit in the category of sheaves). Moreover we can arrange it such that
\begin{enumerate}
\item for every quasi-compact scheme $T$ over $S$ we have
$\colim X_i(T) = X(T)$,
\item the transition morphisms $X_i \to X_{i'}$ of the system
and the coprojections $X_i \to X$ are surjective and \'etale, and
\item if $X$ is a scheme, then the algebraic spaces $X_i$ are schemes
and the transition morphisms $X_i \to X_{i'}$
and the coprojections $X_i \to X$ are local isomorphisms.
\end{enumerate}
\end{lemma}

\begin{proof}
We sketch the proof. By
Properties of Spaces, Lemma
\ref{spaces-properties-lemma-quasi-compact-affine-cover}
we have $X = U/R$ with $U$ affine.
In this case, reasonable means $U \to X$ is universally bounded.
Hence there exists an integer $N$ such that the ``fibres'' of $U \to X$
have degree at most $N$, see
Definition \ref{definition-universally-bounded}.
Denote $s, t : R \to U$ and $c : R \times_{s, U, t} R \to R$ the
groupoid structural maps.

\medskip\noindent
Claim: for every quasi-compact open $A \subset R$ there exists
an open $R' \subset R$ such that
\begin{enumerate}
\item $A \subset R'$,
\item $R'$ is quasi-compact, and
\item $(U, R', s|_{R'}, t|_{R'}, c|_{R' \times_{s, U, t} R'})$ is
a groupoid scheme.
\end{enumerate}
Note that $e : U \to R$ is open as it is a section of the \'etale morphism
$s : R \to U$, see
\'Etale Morphisms, Proposition \ref{etale-proposition-properties-sections}.
Moreover $U$ is affine hence quasi-compact. Hence we may replace $A$ by
$A \cup e(U) \subset R$, and assume that $A$ contains $e(U)$. Next, we
define inductively $A^1 = A$, and
$$
A^n = c(A^{n - 1} \times_{s, U, t} A) \subset R
$$
for $n \geq 2$. Arguing inductively, we see that $A^n$ is quasi-compact for
all $n \geq 2$, as the image of the quasi-compact fibre product
$A^{n - 1} \times_{s, U, t} A$. If $k$ is an algebraically
closed field over $S$, and we consider $k$-points then
$$
A^n(k) = \left\{(u, u') \in U(k)
:
\begin{matrix}
\text{there exist } u = u_1, u_2, \ldots, u_n \in U(k)\text{ with} \\
(u_i , u_{i + 1}) \in A \text{ for all }i = 1, \ldots, n - 1.
\end{matrix}
\right\}
$$
But as the fibres of $U(k) \to X(k)$ have size at most $N$ we see that if
$n > N$ then we get a repeat in the sequence above, and we can shorten it
proving $A^N = A^n$ for all $n \geq N$.
This implies that $R' = A^N$ gives a groupoid scheme
$(U, R', s|_{R'}, t|_{R'}, c|_{R' \times_{s, U, t} R'})$, proving the claim
above.

\medskip\noindent
Consider the map of sheaves on $(\Sch/S)_{fppf}$
$$
\colim_{R' \subset R} U/R' \longrightarrow U/R
$$
where $R' \subset R$ runs over the quasi-compact open subschemes
of $R$ which give \'etale equivalence relations as above. Each of the
quotients $U/R'$ is an algebraic space
(see Spaces, Theorem \ref{spaces-theorem-presentation}).
Since $R'$ is quasi-compact, and $U$ affine the morphism
$R' \to U \times_{\Spec(\mathbf{Z})} U$ is quasi-compact,
and hence $U/R'$ is quasi-separated. Finally, if $T$ is a quasi-compact
scheme, then
$$
\colim_{R' \subset R} U(T)/R'(T) \longrightarrow U(T)/R(T)
$$
is a bijection, since every morphism from $T$ into $R$ ends up in one
of the open subrelations $R'$ by the claim above. This clearly implies
that the colimit of the sheaves $U/R'$ is $U/R$. In other words
the algebraic space $X = U/R$ is the colimit of the quasi-separated
algebraic spaces $U/R'$.

\medskip\noindent
Properties (1) and (2) follow from the discussion above.
If $X$ is a scheme, then if we choose $U$ to be a finite
disjoint union of affine opens of $X$ we will obtain (3).
Details omitted.
\end{proof}

\begin{lemma}
\label{lemma-representable-named-properties}
Let $S$ be a scheme. Let $X$, $Y$ be algebraic spaces over $S$.
Let $X \to Y$ be a representable morphism.
If $Y$ is decent (resp.\ reasonable), then so is $X$.
\end{lemma}

\begin{proof}
Translation of Lemma \ref{lemma-representable-properties}.
\end{proof}

\begin{lemma}
\label{lemma-etale-named-properties}
Let $S$ be a scheme. Let $X \to Y$ be an \'etale morphism of
algebraic spaces over $S$. If $Y$ is decent, resp.\ reasonable,
then so is $X$.
\end{lemma}

\begin{proof}
Let $U$ be an affine scheme and $U \to X$ an \'etale morphism.
Set $R = U \times_X U$ and $R' = U \times_Y U$. Note that
$R \to R'$ is a monomorphism.

\medskip\noindent
Let $x \in |X|$. To show that $X$ is decent, we have to show that
the fibres of $|U| \to |X|$ and $|R| \to |X|$ over $x$ are finite.
But if $Y$ is decent, then the fibres of $|U| \to |Y|$ and
$|R'| \to |Y|$ are finite. Hence the result for ``decent''.

\medskip\noindent
To show that $X$ is reasonable, we have to show that the fibres of
$U \to X$ are universally bounded. However, if $Y$ is reasonable,
then the fibres of $U \to Y$ are universally bounded, which immediately
implies the same thing for the fibres of $U \to X$.
Hence the result for ``reasonable''.
\end{proof}







\section{Points and specializations}
\label{section-specializations}

\noindent
There exists an \'etale morphism of algebraic spaces $f : X \to Y$
and a nontrivial specialization between points in a fibre of
$|f| : |X| \to |Y|$, see
Examples, Lemma \ref{examples-lemma-specializations-fibre-etale}.
If the source of the morphism is a scheme we can avoid this by
imposing condition ($\alpha$) on $Y$.

\begin{lemma}
\label{lemma-no-specializations-map-to-same-point}
Let $S$ be a scheme.
Let $X$ be an algebraic space over $S$.
Let $U \to X$ be an \'etale morphism from a scheme to $X$.
Assume $u, u' \in |U|$ map to the same point $x$ of $|X|$, and
$u' \leadsto u$. If the pair $(X, x)$ satisfies the
equivalent conditions of
Lemma \ref{lemma-U-finite-above-x}
then $u = u'$.
\end{lemma}

\begin{proof}
Assume the pair $(X, x)$ satisfies the
equivalent conditions for Lemma \ref{lemma-U-finite-above-x}.
Let $U$ be a scheme, $U \to X$ \'etale, and
let $u, u' \in |U|$ map to $x$ of $|X|$, and
$u' \leadsto u$. We may and do replace $U$ by an affine
neighbourhood of $u$. Let $t, s : R = U \times_X U \to U$
be the \'etale projection maps.

\medskip\noindent
Pick a point $r \in R$ with $t(r) = u$ and $s(r) = u'$.
This is possible by
Properties of Spaces,
Lemma \ref{spaces-properties-lemma-points-presentation}.
Because generalizations lift along the \'etale morphism $t$
(Remark \ref{remark-recall}) we can find a specialization $r' \leadsto r$ with
$t(r') = u'$. Set $u'' = s(r')$. Then $u'' \leadsto u'$.
Thus we may repeat and find $r'' \leadsto r'$ with
$t(r'') = u''$. Set $u''' = s(r'')$, and so on.
Here is a picture:
$$
\xymatrix{
& r'' \ar[rd]^s \ar[ld]_t \ar@{~>}[d] & \\
u'' \ar@{~>}[d] & r' \ar[rd]^s \ar[ld]_t \ar@{~>}[d] & u''' \ar@{~>}[d] \\
u' \ar@{~>}[d] & r \ar[rd]^s \ar[ld]_t & u'' \ar@{~>}[d] \\
u & & u'
}
$$
In Remark \ref{remark-recall} we have seen that there are no specializations
among points in the fibres of the \'etale morphism $s$. Hence if
$u^{(n + 1)} = u^{(n)}$ for some $n$, then also $r^{(n)} = r^{(n - 1)}$ and
hence also (by taking $t$) $u^{(n)} = u^{(n - 1)}$. This then forces the
whole tower to collapse, in particular $u = u'$. Thus we see that if
$u \not = u'$, then all the specializations are strict and
$\{u, u', u'', \ldots\}$ is an infinite set of points in $U$ which map to the
point $x$ in $|X|$. As we chose $U$ affine this contradicts the second part of
Lemma \ref{lemma-U-finite-above-x}, as desired.
\end{proof}

\begin{lemma}
\label{lemma-no-specializations-map-to-same-point-Noetherian}
Let $S$ be a scheme. Let $X$ be an algebraic space over $S$.
Let $U \to X$ be an \'etale morphism from a scheme to $X$.
Assume $u, u' \in |U|$ map to the same point $x$ of $|X|$, and
$u' \leadsto u$. If $X$ is locally Noetherian, then $u = u'$.
\end{lemma}

\begin{proof}
The discussion in Schemes, Section \ref{schemes-section-points}
shows that $\mathcal{O}_{U, u'}$ is a localization of
the Noetherian local ring $\mathcal{O}_{U, u}$.
By Properties of Spaces, Lemma
\ref{spaces-properties-lemma-pre-dimension-local-ring}
we have $\dim(\mathcal{O}_{U, u}) = \dim(\mathcal{O}_{U, u'})$.
By dimension theory for Noetherian local rings we conclude $u = u'$.
\end{proof}

\begin{lemma}
\label{lemma-specialization}
Let $S$ be a scheme.
Let $X$ be an algebraic space over $S$.
Let $x, x' \in |X|$ and assume $x' \leadsto x$, i.e., $x$ is a
specialization of $x'$.
Assume the pair $(X, x')$ satisfies the equivalent conditions
of Lemma \ref{lemma-UR-finite-above-x}.
Then for every \'etale morphism $\varphi : U \to X$ from a scheme $U$ and any
$u \in U$ with $\varphi(u) = x$, exists a point $u'\in U$,
$u' \leadsto u$ with $\varphi(u') = x'$.
\end{lemma}

\begin{proof}
We may replace $U$ by an affine open neighbourhood of $u$.
Hence we may assume that $U$ is affine. As $x$ is in the
image of the open map $|U| \to |X|$, so is $x'$. Thus we may
replace $X$ by the Zariski open subspace corresponding to
the image of $|U| \to |X|$, see
Properties of Spaces,
Lemma \ref{spaces-properties-lemma-etale-image-open}.
In other words we may assume that
$U \to X$ is surjective and \'etale.
Let $s, t : R = U \times_X U \to U$ be the projections.
By our assumption that $(X, x')$ satisfies the equivalent conditions of
Lemma \ref{lemma-UR-finite-above-x}
we see that the fibres of $|U| \to |X|$ and $|R| \to |X|$
over $x'$ are finite. Say $\{u'_1, \ldots, u'_n\} \subset U$ and
$\{r'_1, \ldots, r'_m\} \subset R$ form the complete inverse image
of $\{x'\}$.
Consider the closed sets
$$
T = \overline{\{u'_1\}} \cup \ldots \cup \overline{\{u'_n\}} \subset |U|,
\quad
T' = \overline{\{r'_1\}} \cup \ldots \cup \overline{\{r'_m\}} \subset |R|.
$$
Trivially we have $s(T') \subset T$. Because $R$ is an equivalence
relation we also have $t(T') = s(T')$ as the set $\{r_j'\}$
is invariant under the inverse of $R$ by construction. Let $w \in T$
be any point. Then $u'_i \leadsto w$ for some $i$. Choose $r \in R$
with $s(r) = w$. Since generalizations lift along $s : R \to U$, see
Remark \ref{remark-recall}, we can find $r' \leadsto r$ with
$s(r') = u_i'$. Then $r' = r'_j$ for some $j$ and we conclude that
$w \in s(T')$. Hence $T = s(T') = t(T')$ is an $|R|$-invariant closed
set in $|U|$. This means $T$ is the inverse image of a closed (!)
subset $T'' = \varphi(T)$ of $|X|$, see
Properties of Spaces,
Lemmas \ref{spaces-properties-lemma-points-presentation} and
\ref{spaces-properties-lemma-topology-points}.
Hence $T'' = \overline{\{x'\}}$.
Thus $T$ contains some point $u_1$ mapping to $x$ as $x \in T''$.
I.e., we see that for some $i$ there exists a specialization
$u'_i \leadsto u_1$ which maps to the given specialization
$x' \leadsto x$.

\medskip\noindent
To finish the proof, choose a point $r \in R$ such that
$s(r) = u$ and $t(r) = u_1$ (using
Properties of Spaces,
Lemma \ref{spaces-properties-lemma-points-cartesian}).
As generalizations lift along $t$, and $u'_i \leadsto u_1$
we can find a specialization $r' \leadsto r$ such that $t(r') = u'_i$.
Set $u' = s(r')$. Then $u' \leadsto u$ and $\varphi(u') = x'$ as
desired.
\end{proof}

\begin{lemma}
\label{lemma-generalizations-lift-flat}
Let $S$ be a scheme. Let $f : Y \to X$ be a  flat morphism of algebraic spaces
over $S$. Let $x, x' \in |X|$ and assume $x' \leadsto x$, i.e., $x$ is a
specialization of $x'$. Assume the pair $(X, x')$ satisfies the equivalent
conditions of Lemma \ref{lemma-UR-finite-above-x} (for example if
$X$ is decent, $X$ is quasi-separated, or $X$ is representable).
Then for every $y \in |Y|$ with $f(y) = x$, there exists a point $y' \in |Y|$,
$y' \leadsto y$ with $f(y') = x'$.
\end{lemma}

\begin{proof}
(The parenthetical statement holds by the definition of decent spaces
and the implications between the different separation conditions
mentioned in Section \ref{section-reasonable-decent}.)
Choose a scheme $V$ and a surjective \'etale morphism $V \to Y$.
Choose $v \in V$ mapping to $y$. Then we see that it suffices to
prove the lemma for $V \to X$. Thus we may assume $Y$ is a scheme.
Choose a scheme $U$ and a surjective \'etale morphism $U \to X$.
Choose $u \in U$ mapping to $x$. By Lemma \ref{lemma-specialization}
we may choose $u' \leadsto u$ mapping to $x'$. By
Properties of Spaces, Lemma \ref{spaces-properties-lemma-points-cartesian}
we may choose $z \in U \times_X Y$ mapping to $y$ and $u$.
Thus we reduce to the case of the flat morphism of
schemes $U \times_X Y \to U$ which is
Morphisms, Lemma \ref{morphisms-lemma-generalizations-lift-flat}.
\end{proof}






\section{Stratifying algebraic spaces by schemes}
\label{section-stratifications}

\noindent
In this section we prove that a quasi-compact and quasi-separated
algebraic space has a finite stratification by locally closed subspaces
each of which is a scheme and such that the glueing of the parts is by
elementary distinguished squares. We first prove a slightly weaker
result for reasonable algebraic spaces.

\begin{lemma}
\label{lemma-quasi-compact-reasonable-stratification}
Let $S$ be a scheme. Let $W \to X$ be a morphism of a scheme $W$
to an algebraic space $X$ which is flat, locally of finite presentation,
separated, locally quasi-finite with universally bounded fibres. There exist
reduced closed subspaces
$$
\emptyset = Z_{-1} \subset Z_0 \subset Z_1 \subset Z_2 \subset
\ldots \subset Z_n = X
$$
such that with $X_r = Z_r \setminus Z_{r - 1}$ the stratification
$X = \coprod_{r = 0, \ldots, n} X_r$ is characterized by the following
universal property: Given $g : T \to X$ the projection
$W \times_X T \to T$ is finite locally free of degree $r$ if and only if
$g(|T|) \subset |X_r|$.
\end{lemma}

\begin{proof}
Let $n$ be an integer bounding the degrees of the fibres of $W \to X$.
Choose a scheme $U$ and a surjective \'etale morphism $U \to X$.
Apply More on Morphisms, Lemma
\ref{more-morphisms-lemma-stratify-flat-fp-lqf-universally-bounded}
to $W \times_X U \to U$. We obtain closed subsets
$$
\emptyset = Y_{-1} \subset Y_0 \subset Y_1 \subset Y_2 \subset
\ldots \subset Y_n = U
$$
characterized by the property stated in the lemma for the morphism
$W \times_X U \to U$. Clearly, the formation of these closed subsets commutes
with base change. Setting $R = U \times_X U$ with projection maps
$s, t : R \to U$ we conclude that
$$
s^{-1}(Y_r) = t^{-1}(Y_r)
$$
as closed subsets of $R$. In other words the closed subsets $Y_r \subset U$
are $R$-invariant. This means that $|Y_r|$ is the inverse image of a closed
subset $Z_r \subset |X|$. Denote $Z_r \subset X$ also the reduced induced
algebraic space structure, see
Properties of Spaces, Definition
\ref{spaces-properties-definition-reduced-induced-space}.

\medskip\noindent
Let $g : T \to X$ be a morphism of algebraic spaces. Choose a scheme $V$
and a surjective \'etale morphism $V \to T$. To prove the final
assertion of the lemma it suffices to prove the assertion for the composition
$V \to X$ (by our definition of finite locally free morphisms, see
Morphisms of Spaces, Section
\ref{spaces-morphisms-section-finite-locally-free}).
Similarly, the morphism of schemes $W \times_X V \to V$ is finite
locally free of degree $r$ if and only if the morphism of schemes
$$
W \times_X (U \times_X V)
\longrightarrow
U \times_X V
$$
is finite locally free of degree $r$ (see
Descent, Lemma \ref{descent-lemma-descending-property-finite-locally-free}).
By construction this happens if and only if $|U \times_X V| \to |U|$
maps into $|Y_r|$, which is true if and only if $|V| \to |X|$ maps
into $|Z_r|$.
\end{proof}

\begin{lemma}
\label{lemma-stratify-flat-fp-lqf}
Let $S$ be a scheme. Let $W \to X$ be a morphism of a scheme $W$ to an
algebraic space $X$ which is flat, locally of finite presentation,
separated, and locally quasi-finite. Then there
exist open subspaces
$$
X = X_0 \supset X_1 \supset X_2 \supset \ldots
$$
such that a morphism $\Spec(k) \to X$ where $k$ is a field
factors through $X_d$ if and
only if $W \times_X \Spec(k)$ has degree $\geq d$ over $k$.
\end{lemma}

\begin{proof}
Choose a scheme $U$ and a surjective \'etale morphism $U \to X$. Apply
More on Morphisms, Lemma \ref{more-morphisms-lemma-stratify-flat-fp-lqf}
to $W \times_X U \to U$. We obtain open subschemes
$$
U = U_0 \supset U_1 \supset U_2 \supset \ldots
$$
characterized by the property stated in the lemma for the morphism
$W \times_X U \to U$. Clearly, the formation of these closed subsets commutes
with base change. Setting $R = U \times_X U$ with projection maps
$s, t : R \to U$ we conclude that
$$
s^{-1}(U_d) = t^{-1}(U_d)
$$
as open subschemes of $R$. In other words the open subschemes $U_d \subset U$
are $R$-invariant. This means that $U_d$ is the inverse image of an
open subspace $X_d \subset X$
(Properties of Spaces, Lemma
\ref{spaces-properties-lemma-subspaces-presentation}).
\end{proof}

\begin{lemma}
\label{lemma-filter-quasi-compact}
Let $S$ be a scheme. Let $X$ be a quasi-compact algebraic space
over $S$. There exist open subspaces
$$
\ldots \subset U_4 \subset U_3 \subset U_2 \subset U_1 = X
$$
with the following properties:
\begin{enumerate}
\item setting $T_p = U_p \setminus U_{p + 1}$ (with reduced induced subspace
structure) there exists a separated scheme $V_p$ and a surjective \'etale
morphism $f_p : V_p \to U_p$ such that $f_p^{-1}(T_p) \to T_p$ is an
isomorphism,
\item if $x \in |X|$ can be represented by a quasi-compact morphism
$\Spec(k) \to X$ from a field, then $x \in T_p$ for some $p$.
\end{enumerate}
\end{lemma}

\begin{proof}
By Properties of Spaces, Lemma
\ref{spaces-properties-lemma-quasi-compact-affine-cover}
we can choose an affine scheme $U$ and a surjective \'etale morphism
$U \to X$. For $p \geq 0$ set
$$
W_p = U \times_X \ldots \times_X U \setminus \text{all diagonals}
$$
where the fibre product has $p$ factors. Since $U$ is separated,
the morphism $U \to X$ is separated and all fibre products
$U \times_X \ldots \times_X U$ are separated schemes. Since $U \to X$ is
separated the diagonal $U \to U \times_X U$ is a closed immersion. Since
$U \to X$ is \'etale the diagonal $U \to U \times_X U$ is an open
immersion, see Morphisms of Spaces, Lemmas
\ref{spaces-morphisms-lemma-etale-unramified} and
\ref{spaces-morphisms-lemma-diagonal-unramified-morphism}.
Similarly, all the diagonal morphisms are open and closed immersions and
$W_p$ is an open and closed subscheme of $U \times_X \ldots \times_X U$.
Moreover, the morphism
$$
U \times_X \ldots \times_X U \longrightarrow
U \times_{\Spec(\mathbf{Z})} \ldots \times_{\Spec(\mathbf{Z})} U
$$
is locally quasi-finite and separated (Morphisms of Spaces,
Lemma \ref{spaces-morphisms-lemma-fibre-product-after-map})
and its target is an affine scheme. Hence every finite set of points of
$U \times_X \ldots \times_X U$ is contained in an affine open, see
More on Morphisms, Lemma
\ref{more-morphisms-lemma-separated-locally-quasi-finite-over-affine}.
Therefore, the same is true for $W_p$.
There is a free action of the symmetric group $S_p$ on $W_p$ over $X$
(because we threw out the fix point locus from
$U \times_X \ldots \times_X U$). By the above and
Properties of Spaces, Proposition
\ref{spaces-properties-proposition-finite-flat-equivalence-global}
the quotient $V_p = W_p/S_p$ is a scheme. Since the action of
$S_p$ on $W_p$ was over $X$, there is a morphism $V_p \to X$.
Since $W_p \to X$ is \'etale and since $W_p \to V_p$ is surjective
\'etale, it follows that also $V_p \to X$ is \'etale, see
Properties of Spaces, Lemma \ref{spaces-properties-lemma-etale-local}.
Observe that $V_p$ is a separated scheme by
Properties of Spaces, Lemma
\ref{spaces-properties-lemma-quotient-separated}.

\medskip\noindent
We let $U_p \subset X$ be the open subspace which is the
image of $V_p \to X$. By construction a morphism $\Spec(k) \to X$ with
$k$ algebraically closed, factors through $U_p$ if and only if
$U \times_X \Spec(k)$ has $\geq p$ points; as usual observe that
$U \times_X \Spec(k)$ is scheme theoretically a disjoint union of
(possibly infinitely many) copies of $\Spec(k)$, see
Remark \ref{remark-recall}. It follows that
the $U_p$ give a filtration of $X$ as stated in the lemma.
Moreover, our morphism $\Spec(k) \to X$ factors through $T_p$
if and only if $U \times_X \Spec(k)$ has exactly $p$ points.
In this case we see that $V_p \times_X \Spec(k)$ has exactly one point.
Set $Z_p = f_p^{-1}(T_p) \subset V_p$. This is a closed subscheme of $V_p$.
Then $Z_p \to T_p$ is an \'etale morphism between
algebraic spaces which induces a bijection on $k$-valued
points for any algebraically closed field $k$. To be sure this
implies that $Z_p \to T_p$ is universally injective, whence an
open immersion by
Morphisms of Spaces, Lemma
\ref{spaces-morphisms-lemma-etale-universally-injective-open}
hence an isomorphism and (1) has been proved.

\medskip\noindent
Let $x : \Spec(k) \to X$ be a quasi-compact morphism where $k$ is a field.
Then the composition $\Spec(\overline{k}) \to \Spec(k) \to X$ is quasi-compact
as well (Morphisms of Spaces, Lemma
\ref{spaces-morphisms-lemma-composition-quasi-compact}).
In this case the scheme $U \times_X \Spec(\overline{k})$ is
quasi-compact. In view of the fact (seen above) that it is a disjoint union
of copies of $\Spec(\overline{k})$ we find that it has finitely many points.
If the number of points is $p$, then we see that indeed $x \in T_p$ and
the proof is finished.
\end{proof}

\begin{lemma}
\label{lemma-filter-reasonable}
Let $S$ be a scheme. Let $X$ be a quasi-compact, reasonable algebraic space
over $S$. There exist an integer $n$ and open subspaces
$$
\emptyset = U_{n + 1} \subset
U_n \subset U_{n - 1} \subset \ldots \subset U_1 = X
$$
with the following property: setting $T_p = U_p \setminus U_{p + 1}$
(with reduced induced subspace structure) there exists a separated scheme
$V_p$ and a surjective \'etale morphism $f_p : V_p \to U_p$ such that
$f_p^{-1}(T_p) \to T_p$ is an isomorphism.
\end{lemma}

\begin{proof}
The proof of this lemma is identical to the proof of
Lemma \ref{lemma-filter-quasi-compact}.
Let $n$ be an integer bounding the degrees of
the fibres of $U \to X$ which exists as $X$ is reasonable, see
Definition \ref{definition-very-reasonable}.
Then we see that $U_{n + 1} = \emptyset$ and the proof is complete.
\end{proof}

\begin{lemma}
\label{lemma-stratify-reasonable}
Let $S$ be a scheme. Let $X$ be a quasi-compact, reasonable algebraic space
over $S$. There exist an integer $n$ and open subspaces
$$
\emptyset = U_{n + 1} \subset
U_n \subset U_{n - 1} \subset \ldots \subset U_1 = X
$$
such that each $T_p = U_p \setminus U_{p + 1}$ (with reduced induced subspace
structure) is a scheme.
\end{lemma}

\begin{proof}
Immediate consequence of Lemma \ref{lemma-filter-reasonable}.
\end{proof}

\noindent
The following result is almost identical to
\cite[Proposition 5.7.8]{GruRay}.

\begin{lemma}
\label{lemma-filter-quasi-compact-quasi-separated}
\begin{reference}
This result is almost identical to \cite[Proposition 5.7.8]{GruRay}.
\end{reference}
Let $X$ be a quasi-compact and quasi-separated algebraic space over
$\Spec(\mathbf{Z})$. There exist an integer $n$ and open subspaces
$$
\emptyset = U_{n + 1} \subset
U_n \subset U_{n - 1} \subset \ldots \subset U_1 = X
$$
with the following property: setting $T_p = U_p \setminus U_{p + 1}$
(with reduced induced subspace structure) there exists a quasi-compact
separated scheme $V_p$ and a surjective \'etale morphism $f_p : V_p \to U_p$
such that $f_p^{-1}(T_p) \to T_p$ is an isomorphism.
\end{lemma}

\begin{proof}
The proof of this lemma is identical to the proof of
Lemma \ref{lemma-filter-quasi-compact}.
Observe that a quasi-separated space is reasonable, see
Lemma \ref{lemma-bounded-fibres} and
Definition \ref{definition-very-reasonable}.
Hence we find that $U_{n + 1} = \emptyset$ as in
Lemma \ref{lemma-filter-reasonable}.
At the end of the argument we add that since $X$ is quasi-separated
the schemes $U \times_X \ldots \times_X U$ are all quasi-compact.
Hence the schemes $W_p$ are quasi-compact. Hence the
quotients $V_p = W_p/S_p$ by the symmetric group $S_p$ are quasi-compact
schemes.
\end{proof}

\noindent
The following lemma probably belongs somewhere else.

\begin{lemma}
\label{lemma-locally-constructible}
Let $S$ be a scheme. Let $X$ be a quasi-separated algebraic space over $S$.
Let $E \subset |X|$ be a subset. Then $E$ is \'etale locally constructible
(Properties of Spaces, Definition
\ref{spaces-properties-definition-locally-constructible})
if and only if $E$ is a locally constructible subset of
the topological space $|X|$
(Topology, Definition \ref{topology-definition-constructible}).
\end{lemma}

\begin{proof}
Assume $E \subset |X|$ is a locally constructible subset of
the topological space $|X|$. Let $f : U \to X$ be an
\'etale morphism where $U$ is a scheme. We have to show that
$f^{-1}(E)$ is locally constructible in $U$. The question is
local on $U$ and $X$, hence we may assume that $X$ is quasi-compact,
$E \subset |X|$ is constructible, and $U$ is affine.
In this case $U \to X$ is quasi-compact, hence
$f : |U| \to |X|$ is quasi-compact. Observe that retrocompact
opens of $|X|$, resp.\ $U$ are the same thing as quasi-compact opens
of $|X|$, resp.\ $U$, see
Topology, Lemma \ref{topology-lemma-topology-quasi-separated-scheme}.
Thus $f^{-1}(E)$ is constructible by Topology, Lemma
\ref{topology-lemma-inverse-images-constructibles}.

\medskip\noindent
Conversely, assume $E$ is \'etale locally constructible.
We want to show that $E$ is locally constructible in the
topological space $|X|$.
The question is local on $X$, hence we may assume that $X$ is
quasi-compact as well as quasi-separated. We will show that
in this case $E$ is constructible in $|X|$.
Choose open subspaces
$$
\emptyset = U_{n + 1} \subset
U_n \subset U_{n - 1} \subset \ldots \subset U_1 = X
$$
and surjective \'etale morphisms $f_p : V_p \to U_p$
inducing isomorphisms $f_p^{-1}(T_p) \to T_p = U_p \setminus U_{p + 1}$
where $V_p$ is a quasi-compact separated scheme as in
Lemma \ref{lemma-filter-quasi-compact-quasi-separated}.
By definition the inverse image $E_p \subset V_p$ of $E$ is
locally constructible in $V_p$. Then $E_p$ is constructible in $V_p$
by Properties, Lemma
\ref{properties-lemma-constructible-quasi-compact-quasi-separated}.
Thus $E_p \cap |f_p^{-1}(T_p)| = E \cap |T_p|$ is constructible
in $|T_p|$ by
Topology, Lemma \ref{topology-lemma-intersect-constructible-with-closed}
(observe that $V_p \setminus f_p^{-1}(T_p)$ is quasi-compact as it is the
inverse image of the quasi-compact space $U_{p + 1}$ by the
quasi-compact morphism $f_p$).
Thus
$$
E = (|T_n| \cap E) \cup (|T_{n - 1}| \cap E) \cup \ldots \cup
(|T_1| \cap E)
$$
is constructible by
Topology, Lemma \ref{topology-lemma-collate-constructible-from-constructible}.
Here we use that $|T_p|$ is constructible in $|X|$ which is clear from
what was said above.
\end{proof}





\section{Integral cover by a scheme}
\label{section-integral-cover}

\noindent
Here we prove that given any quasi-compact and quasi-separated
algebraic space $X$, there is a scheme $Y$ and a surjective, integral
morphism $Y \to X$. After we develop some theory about limits of
algebraic spaces, we will prove that one can do this with a finite
morphism, see
Limits of Spaces, Section \ref{spaces-limits-section-finite-cover}.

\begin{lemma}
\label{lemma-extend-integral-morphism}
Let $S$ be a scheme. Let $j : V \to Y$ be a quasi-compact open immersion
of algebraic spaces over $S$. Let $\pi : Z \to V$ be an integral morphism.
Then there exists an integral morphism $\nu : Y' \to Y$ such that
$Z$ is $V$-isomorphic to the inverse image of $V$ in $Y'$.
\end{lemma}

\begin{proof}
Since both $j$ and $\pi$ are quasi-compact and separated, so is
$j \circ \pi$. Let $\nu : Y' \to Y$ be the normalization of $Y$ in $Z$, see
Morphisms of Spaces, Section
\ref{spaces-morphisms-section-normalization-X-in-Y}.
Of course $\nu$ is integral, see
Morphisms of Spaces, Lemma
\ref{spaces-morphisms-lemma-characterize-normalization}.
The final statement follows formally from
Morphisms of Spaces, Lemmas
\ref{spaces-morphisms-lemma-properties-normalization} and
\ref{spaces-morphisms-lemma-normalization-in-integral}.
\end{proof}

\begin{lemma}
\label{lemma-there-is-a-scheme-integral-over}
Let $S$ be a scheme. Let $X$ be a quasi-compact and quasi-separated
algebraic space over $S$.
\begin{enumerate}
\item There exists a surjective integral morphism $Y \to X$ where $Y$
is a scheme,
\item given a surjective \'etale morphism $U \to X$ we may choose
$Y \to X$ such that for every $y \in Y$ there is an open neighbourhood
$V \subset Y$ such that $V \to X$ factors through $U$.
\end{enumerate}
\end{lemma}

\begin{proof}
Part (1) is the special case of part (2) where $U = X$.
Choose a surjective \'etale morphism $U' \to U$
where $U'$ is a scheme. It is clear that we may replace $U$ by $U'$
and hence we may assume $U$ is a scheme. Since $X$ is quasi-compact,
there exist finitely many affine opens $U_i \subset U$ such that
$U' = \coprod U_i \to X$ is surjective.
After replacing $U$ by $U'$ again, we see that we may assume $U$ is affine.
Since $X$ is quasi-separated, hence reasonable, there exists an integer
$d$ bounding the degree of the geometric fibres of $U \to X$
(see Lemma \ref{lemma-bounded-fibres}).
We will prove the lemma by induction on $d$ for all quasi-compact
and separated schemes $U$ mapping surjective and \'etale onto $X$.
If $d = 1$, then $U = X$ and the result holds with $Y = U$.
Assume $d > 1$.

\medskip\noindent
We apply Morphisms of Spaces, Lemma
\ref{spaces-morphisms-lemma-quasi-finite-separated-quasi-affine}
and we obtain a factorization
$$
\xymatrix{
U \ar[rr]_j \ar[rd] & & Y \ar[ld]^\pi \\
& X
}
$$
with $\pi$ integral and $j$ a quasi-compact open immersion. We may and do
assume that $j(U)$ is scheme theoretically dense in $Y$. Then $U \times_X Y$
is a quasi-compact, separated scheme (being finite over $U$) and we have
$$
U \times_X Y = U \amalg W
$$
Here the first summand is the image of $U \to U \times_X Y$
(which is closed by
Morphisms of Spaces, Lemma \ref{spaces-morphisms-lemma-semi-diagonal}
and open because it is \'etale as a morphism between
algebraic spaces \'etale over $Y$) and
the second summand is the (open and closed) complement.
The image $V \subset Y$ of $W$ is an open subspace containing
$Y \setminus U$.

\medskip\noindent
The \'etale morphism $W \to Y$ has geometric fibres of cardinality $< d$.
Namely, this is clear for geometric points of $U \subset Y$ by inspection.
Since $|U| \subset |Y|$ is dense, it holds for all geometric points of $Y$
by Lemma \ref{lemma-quasi-compact-reasonable-stratification}
(the degree of the fibres of a quasi-compact \'etale morphism
does not go up under specialization). Thus we may apply the induction
hypothesis to $W \to V$ and find a surjective integral morphism
$Z \to V$ with $Z$ a scheme, which Zariski locally factors through $W$.
Choose a factorization $Z \to Z' \to Y$ with $Z' \to Y$ integral and
$Z \to Z'$ open immersion
(Lemma \ref{lemma-extend-integral-morphism}).
After replacing $Z'$ by the scheme theoretic closure of $Z$ in $Z'$
we may assume that $Z$ is scheme theoretically dense in $Z'$.
After doing this we have $Z' \times_Y V = Z$. Finally,
let $T \subset Y$ be the induced closed subspace structure on $Y \setminus V$.
Consider the morphism
$$
Z' \amalg T \longrightarrow X
$$
This is a surjective integral morphism by construction.
Since $T \subset U$ it is clear that the morphism $T \to X$
factors through $U$. On the other hand, let $z \in Z'$
be a point. If $z \not \in Z$, then $z$ maps to a point of
$Y \setminus V \subset U$ and we find a neighbourhood of $z$
on which the morphism factors through $U$.
If $z \in Z$, then we have an open neighbourhood of $z$ in $Z$
(which is also an open neighbourhood of $z$ in $Z'$)
which factors through $W \subset U \times_X Y$ and hence through $U$.
\end{proof}

\begin{lemma}
\label{lemma-there-is-a-scheme-integral-over-refined}
Let $S$ be a scheme. Let $X$ be a quasi-compact and quasi-separated
algebraic space over $S$ such that $|X|$ has finitely many irreducible
components.
\begin{enumerate}
\item There exists a surjective integral morphism $Y \to X$ where $Y$
is a scheme such that $f$ is finite \'etale over a quasi-compact
dense open $U \subset X$,
\item given a surjective \'etale morphism $V \to X$ we may choose
$Y \to X$ such that for every $y \in Y$ there is an open neighbourhood
$W \subset Y$ such that $W \to X$ factors through $V$.
\end{enumerate}
\end{lemma}

\begin{proof}
The proof is the (roughly) same as the proof of
Lemma \ref{lemma-there-is-a-scheme-integral-over}
with additional technical comments to obtain
the dense quasi-compact open $U$ (and unfortunately
changes in notation to keep track of $U$).

\medskip\noindent
Part (1) is the special case of part (2) where $V = X$.

\medskip\noindent
Proof of (2). Choose a surjective \'etale morphism $V' \to V$
where $V'$ is a scheme. It is clear that we may replace $V$ by $V'$
and hence we may assume $V$ is a scheme. Since $X$ is quasi-compact,
there exist finitely many affine opens $V_i \subset V$ such that
$V' = \coprod V_i \to X$ is surjective.
After replacing $V$ by $V'$ again, we see that we may assume $V$ is affine.
Since $X$ is quasi-separated, hence reasonable, there exists an integer
$d$ bounding the degree of the geometric fibres of $V \to X$
(see Lemma \ref{lemma-bounded-fibres}).

\medskip\noindent
By induction on $d \geq 1$ we will prove the following induction
hypothesis $(H_d)$:
\begin{itemize}
\item for any quasi-compact and quasi-separated algebraic space
$X$ with finitely many irreducible components, for any $m \geq 0$,
for any quasi-compact and separated schemes $V_j$, $j = 1, \ldots, m$,
for any \'etale morphisms $\varphi_j : V_j \to X$, $j = 1, \ldots, m$
such that $d$ bounds the degree of the geometric fibres of
$\varphi_j : V_j\to X$ and
$\varphi = \coprod \varphi_j : V = \coprod V_j \to X$
is surjective, the statement of the lemma holds for $\varphi : V \to X$.
\end{itemize}
If $d = 1$, then each $\varphi_j$ is an open immersion. Hence $X$
is a scheme and the result holds with $Y = V$.
Assume $d > 1$, assume $(H_{d - 1})$ and let
$m$, $\varphi : V_j \to X$, $j = 1, \ldots, m$ be as in $(H_d)$.

\medskip\noindent
Let $\eta_1, \ldots, \eta_n \in |X|$ be the generic points of the
irreducible components of $|X|$. By
Properties of Spaces, Proposition
\ref{spaces-properties-proposition-locally-quasi-separated-open-dense-scheme}
there is an open subscheme $U \subset X$ with $\eta_1, \ldots, \eta_n \in U$.
By shrinking $U$ we may assume $U$ affine and by
Morphisms, Lemma \ref{morphisms-lemma-generically-finite}
we may assume each $\varphi_j : V_j \to X$ is finite \'etale over $U$.
Of course, we see that $U$ is quasi-compact and dense in $X$
and that $\varphi_j^{-1}(U)$ is dense in $V_j$. In particular each $V_j$
has finitely many irreducible components.

\medskip\noindent
Fix $j \in \{1, \ldots, m\}$.
As in Morphisms of Spaces, Lemma
\ref{spaces-morphisms-lemma-quasi-finite-separated-quasi-affine}
we let $Y_j$ be the normalization of $X$ in $V_j$. We obtain a factorization
$$
\xymatrix{
V_j \ar[rr] \ar[rd]_{\varphi_j} & & Y_j \ar[ld]^{\pi_j} \\
& X
}
$$
with $\pi_j$ integral and $V_j \to Y_j$ a quasi-compact open immersion. Since
$Y_j$ is the normalization of $X$ in $V_j$, we see from
Morphisms of Spaces, Lemmas
\ref{spaces-morphisms-lemma-properties-normalization} and
\ref{spaces-morphisms-lemma-normalization-in-integral}
that $\varphi_j^{-1}(U) \to \pi_j^{-1}(U)$ is an isomorphism.
Thus $\pi_j$ is finite \'etale over $U$. Observe that $V_j$
is scheme theoretically dense in $Y_j$ because $Y_j$ is the normalization
of $X$ in $V_j$ (follows from the characterization of relative normalization
in Morphisms of Spaces, Lemma
\ref{spaces-morphisms-lemma-characterize-normalization}). Since $V_j$
is quasi-compact we see that $|V_j| \subset |Y_j|$ is dense,
see Morphisms of Spaces, Section
\ref{spaces-morphisms-section-scheme-theoretic-closure}
(and especially Morphisms of Spaces, Lemma
\ref{spaces-morphisms-lemma-quasi-compact-immersion}).
It follows that $|Y_j|$ has finitely many irreducible components.
Then $V_j \times_X Y_j$ is a quasi-compact, separated scheme
(being finite over $V_j$) and
$$
V_j \times_X Y_j = V_j \amalg W_j
$$
Here the first summand is the image of $V_j \to V_j \times_X Y_j$
(which is closed by
Morphisms of Spaces, Lemma \ref{spaces-morphisms-lemma-semi-diagonal}
and open because it is \'etale as a morphism between
algebraic spaces \'etale over $Y$) and
the second summand is the (open and closed) complement.

\medskip\noindent
The \'etale morphism $W_j \to Y_j$ has geometric fibres of cardinality $< d$.
Namely, this is clear for geometric points of $V_j \subset Y_j$ by inspection.
Since $|V_j| \subset |Y_j|$ is dense, it holds for all geometric points
of $Y_j$ by Lemma \ref{lemma-quasi-compact-reasonable-stratification}
(the degree of the fibres of a quasi-compact \'etale morphism
does not go up under specialization). By $(H_{d - 1})$ applied
to $V_j \amalg W_j \to Y_j$ we find a surjective integral morphism
$Y_j' \to Y_j$ with $Y_j'$ a scheme, which Zariski locally factors
through $V_j \amalg W_j$, and which is finite \'etale over a
quasi-compact dense open $U_j \subset Y_j$. After shrinking $U$
we may and do assume that $\pi_j^{-1}(U) \subset U_j$
(we may and do choose the same $U$ for all $j$; some details omitted).

\medskip\noindent
We claim that
$$
Y = \coprod\nolimits_{j = 1, \ldots, m} Y'_j \longrightarrow X
$$
is the solution to our problem. First, this morphism is integral
as on each summand we have the composition $Y'_j \to Y \to X$
of integral morphisms (Morphisms of Spaces, Lemma
\ref{spaces-morphisms-lemma-composition-integral}). Second, this
morphism Zariski locally factors through $V = \coprod V_j$ because
we saw above that each $Y'_j \to Y_j$ factors Zariski locally through
$V_j \amalg W_j = V_j \times_X Y_j$. Finally, since both
$Y'_j \to Y_j$ and $Y_j \to X$ are finite \'etale over
$U$, so is the composition. This finishes the proof.
\end{proof}



















\section{Schematic locus}
\label{section-schematic}

\noindent
In this section we prove that a decent algebraic space has a dense open
subspace which is a scheme. We first prove this for reasonable algebraic
spaces.


\begin{proposition}
\label{proposition-reasonable-open-dense-scheme}
Let $S$ be a scheme. Let $X$ be an algebraic space over $S$.
If $X$ is reasonable, then there exists a dense open subspace
of $X$ which is a scheme.
\end{proposition}

\begin{proof}
By Properties of Spaces,
Lemma \ref{spaces-properties-lemma-subscheme}
the question is local on $X$. Hence we may assume there exists an affine
scheme $U$ and a surjective \'etale morphism $U \to X$
(Properties of Spaces, Lemma
\ref{spaces-properties-lemma-cover-by-union-affines}).
Let $n$ be an integer bounding the degrees of the fibres of $U \to X$
which exists as $X$ is reasonable, see
Definition \ref{definition-very-reasonable}.
We will argue by induction on $n$ that whenever
\begin{enumerate}
\item $U \to X$ is a surjective \'etale morphism whose fibres have
degree $\leq n$, and
\item $U$ is isomorphic to a locally closed subscheme of an affine scheme
\end{enumerate}
then the schematic locus is dense in $X$.

\medskip\noindent
Let $X_n \subset X$ be the open subspace which is the complement of the
closed subspace $Z_{n - 1} \subset X$ constructed in
Lemma \ref{lemma-quasi-compact-reasonable-stratification}
using the morphism $U \to X$.
Let $U_n \subset U$ be the inverse image of $X_n$. Then
$U_n \to X_n$ is finite locally free of degree $n$.
Hence $X_n$ is a scheme by
Properties of Spaces, Proposition
\ref{spaces-properties-proposition-finite-flat-equivalence-global}
(and the fact that any finite set of points of $U_n$ is contained in
an affine open of $U_n$, see
Properties, Lemma \ref{properties-lemma-ample-finite-set-in-affine}).

\medskip\noindent
Let $X' \subset X$ be the open subspace such that $|X'|$ is the
interior of $|Z_{n - 1}|$ in $|X|$ (see
Topology, Definition \ref{topology-definition-nowhere-dense}).
Let $U' \subset U$ be the inverse image. Then $U' \to X'$ is surjective
\'etale and has degrees of fibres bounded by $n - 1$. By induction
we see that the schematic locus of $X'$ is an open dense $X'' \subset X'$.
By elementary topology we see that $X'' \cup X_n \subset X$ is
open and dense and we win.
\end{proof}

\begin{theorem}[David Rydh]
\label{theorem-decent-open-dense-scheme}
Let $S$ be a scheme. Let $X$ be an algebraic space over $S$.
If $X$ is decent, then there exists a dense open subspace
of $X$ which is a scheme.
\end{theorem}

\begin{proof}
Assume $X$ is a decent algebraic space for which the theorem is false. By
Properties of Spaces, Lemma \ref{spaces-properties-lemma-subscheme}
there exists a largest open subspace $X' \subset X$ which is a scheme.
Since $X'$ is not dense in $X$, there exists an open subspace
$X'' \subset X$ such that $|X''| \cap |X'| = \emptyset$. Replacing $X$
by $X''$ we get a nonempty decent algebraic space $X$ which does not
contain {\it any} open subspace which is a scheme.

\medskip\noindent
Choose a nonempty affine scheme $U$ and an \'etale morphism $U \to X$.
We may and do replace $X$ by the open subscheme corresponding to the
image of $|U| \to |X|$. Consider the sequence of open subspaces
$$
X = X_0 \supset X_1 \supset X_2 \ldots
$$
constructed in Lemma \ref{lemma-stratify-flat-fp-lqf}
for the morphism $U \to X$. Note that $X_0 = X_1$ as $U \to X$
is surjective. Let $U = U_0 = U_1 \supset U_2 \ldots$ be the induced
sequence of open subschemes of $U$.

\medskip\noindent
Choose a nonempty open affine $V_1 \subset U_1$ (for example $V_1 = U_1$).
By induction we will construct a sequence of nonempty affine opens
$V_1 \supset V_2 \supset \ldots$ with $V_n \subset U_n$. Namely, having
constructed $V_1, \ldots, V_{n - 1}$ we can always choose $V_n$ unless
$V_{n - 1} \cap U_n = \emptyset$. But if $V_{n - 1} \cap U_n = \emptyset$,
then the open subspace $X' \subset X$ with
$|X'| = \Im(|V_{n - 1}| \to |X|)$ is contained in $|X| \setminus |X_n|$.
Hence $V_{n - 1} \to X'$ is an \'etale morphism whose fibres have degree
bounded by $n - 1$. In other words, $X'$ is reasonable (by definition),
hence $X'$ contains a nonempty open subscheme by
Proposition \ref{proposition-reasonable-open-dense-scheme}.
This is a contradiction which shows that we can pick $V_n$.

\medskip\noindent
By Limits, Lemma \ref{limits-lemma-limit-nonempty}
the limit $V_\infty = \lim V_n$ is a nonempty scheme. Pick a morphism
$\Spec(k) \to V_\infty$. The composition $\Spec(k) \to V_\infty \to U \to X$
has image contained in all $X_d$ by construction. In other words, the
fibred $U \times_X \Spec(k)$ has infinite degree which contradicts
the definition of a decent space. This contradiction finishes the proof
of the theorem.
\end{proof}

\begin{lemma}
\label{lemma-when-quotient-scheme-at-point}
Let $S$ be a scheme. Let $X \to Y$ be a surjective finite locally free
morphism of algebraic spaces over $S$. For $y \in |Y|$ the following are
equivalent
\begin{enumerate}
\item $y$ is in the schematic locus of $Y$, and
\item there exists an affine open $U \subset X$
containing the preimage of $y$.
\end{enumerate}
\end{lemma}

\begin{proof}
If $y \in Y$ is in the schematic locus, then it has an affine open
neighbourhood $V \subset Y$ and the inverse image $U$ of $V$ in $X$
is an open finite over $V$, hence affine. Thus (1) implies (2).

\medskip\noindent
Conversely, assume that $U \subset X$ as in (2) is given.
Set $R = X \times_Y X$ and denote the projections $s, t : R \to X$.
Consider $Z = R \setminus s^{-1}(U) \cap t^{-1}(U)$.
This is a closed subset of $R$. The image $t(Z)$ is a closed
subset of $X$ which can loosely be described as the set of
points of $X$ which are $R$-equivalent to a point of
$X \setminus U$. Hence $U' = X \setminus t(Z)$ is an $R$-invariant,
open subspace of $X$ contained in $U$ which contains
the fibre of $X \to Y$ over $y$. Since $X \to Y$ is open
(Morphisms of Spaces, Lemma \ref{spaces-morphisms-lemma-fppf-open})
the image of $U'$ is an open subspace $V' \subset Y$.
Since $U'$ is $R$-invariant and $R = X \times_Y X$, we see that $U'$ is the
inverse image of $V'$ (use
Properties of Spaces, Lemma \ref{spaces-properties-lemma-points-cartesian}).
After replacing $Y$ by $V'$ and $X$ by $U'$ we see that we may assume
$X$ is a scheme isomorphic to an open subscheme of an affine scheme.

\medskip\noindent
Assume $X$ is a scheme isomorphic to an open subscheme of an affine scheme.
In this case the fppf quotient sheaf $X/R$ is a scheme, see
Properties of Spaces, Proposition
\ref{spaces-properties-proposition-finite-flat-equivalence-global}.
Since $Y$ is a sheaf in the fppf topology, obtain a canonical
map $X/R \to Y$ factoring $X \to Y$. Since $X \to Y$ is surjective
finite locally free, it is surjective as a map of sheaves
(Spaces, Lemma \ref{spaces-lemma-surjective-flat-locally-finite-presentation}).
We conclude that $X/R \to Y$ is surjective as a map of sheaves.
On the other hand, since $R = X \times_Y X$ as sheaves we conclude that
$X/R \to Y$ is injective as a map of sheaves. Hence $X/R \to Y$
is an isomorphism and we see that $Y$ is representable.
\end{proof}

\noindent
At this point we have several different ways for proving the following
lemma.

\begin{lemma}
\label{lemma-finite-etale-cover-dense-open-scheme}
Let $S$ be a scheme. Let $X$ be an algebraic space over $S$.
If there exists a finite, \'etale, surjective morphism
$U \to X$ where $U$ is a scheme, then there exists a dense open subspace
of $X$ which is a scheme.
\end{lemma}

\begin{proof}[First proof]
The morphism $U \to X$ is finite locally free. Hence there is a decomposition
of $X$ into open and closed subspaces $X_d \subset X$ such that
$U \times_X X_d \to X_d$ is finite locally free of degree $d$.
Thus we may assume $U \to X$ is finite locally free of degree $d$.
In this case, let $U_i \subset U$, $i \in I$ be the set of affine opens.
For each $i$ the morphism $U_i \to X$ is \'etale and has
universally bounded fibres (namely, bounded by $d$).
In other words, $X$ is reasonable and
the result follows from
Proposition \ref{proposition-reasonable-open-dense-scheme}.
\end{proof}

\begin{proof}[Second proof]
The question is local on $X$
(Properties of Spaces, Lemma \ref{spaces-properties-lemma-subscheme}),
hence may assume $X$ is quasi-compact. Then $U$ is quasi-compact.
Then there exists a dense open subscheme $W \subset U$ which is
separated (Properties, Lemma
\ref{properties-lemma-quasi-compact-dense-open-separated}).
Set $Z = U \setminus W$.
Let $R = U \times_X U$ and $s, t : R \to U$ the projections.
Then $t^{-1}(Z)$ is nowhere dense in $R$
(Topology, Lemma \ref{topology-lemma-open-inverse-image-closed-nowhere-dense})
and hence $\Delta = s(t^{-1}(Z))$ is an $R$-invariant
closed nowhere dense subset of $U$
(Morphisms, Lemma \ref{morphisms-lemma-image-nowhere-dense-finite}).
Let $u \in U \setminus \Delta$ be a generic point of an
irreducible component. Since these points are dense in $U \setminus \Delta$
and since $\Delta$ is nowhere dense, it suffices to show that the image
$x \in X$ of $u$ is in the schematic locus of $X$.
Observe that $t(s^{-1}(\{u\})) \subset W$ is a
finite set of generic points of irreducible components of $W$
(compare with
Properties of Spaces, Lemma
\ref{spaces-properties-lemma-codimension-0-points}).
By Properties, Lemma \ref{properties-lemma-maximal-points-affine}
we can find an affine open $V \subset W$ such that
$t(s^{-1}(\{u\})) \subset V$. Since $t(s^{-1}(\{u\}))$ is the fibre
of $|U| \to |X|$ over $x$, we conclude by
Lemma \ref{lemma-when-quotient-scheme-at-point}.
\end{proof}

\begin{proof}[Third proof]
(This proof is essentially the same as the second proof, but uses
fewer references.)
Assume $X$ is an algebraic space, $U$ a scheme, and $U \to X$ is a finite
\'etale surjective morphism. Write $R = U \times_X U$ and denote
$s, t : R \to U$ the projections as usual. Note that $s, t$ are surjective,
finite and \'etale. Claim: The union of the $R$-invariant affine opens of
$U$ is topologically dense in $U$.

\medskip\noindent
Proof of the claim. Let $W \subset U$ be an affine open.
Set $W' = t(s^{-1}(W)) \subset U$. Since $s^{-1}(W)$ is affine
(hence quasi-compact) we see that $W' \subset U$ is a quasi-compact open. By
Properties, Lemma \ref{properties-lemma-quasi-compact-dense-open-separated}
there exists a dense open $W'' \subset W'$ which is a separated scheme.
Set $\Delta' = W' \setminus W''$. This is a nowhere dense closed subset of
$W''$. Since $t|_{s^{-1}(W)} : s^{-1}(W) \to W'$ is open (because it is \'etale)
we see that the inverse image
$(t|_{s^{-1}(W)})^{-1}(\Delta') \subset s^{-1}(W)$
is a nowhere dense closed subset (see
Topology, Lemma \ref{topology-lemma-open-inverse-image-closed-nowhere-dense}).
Hence, by
Morphisms, Lemma \ref{morphisms-lemma-image-nowhere-dense-finite}
we see that
$$
\Delta = s\left((t|_{s^{-1}(W)})^{-1}(\Delta')\right)
$$
is a nowhere dense closed subset of $W$. Pick any point $\eta \in W$,
$\eta \not \in \Delta$ which is a generic point of an irreducible
component of $W$ (and hence of $U$). By our choices above the finite set
$t(s^{-1}(\{\eta\})) = \{\eta_1, \ldots, \eta_n\}$
is contained in the separated scheme $W''$.
Note that the fibres of $s$ is are finite discrete spaces, and that
generalizations lift along the \'etale morphism $t$, see
Morphisms, Lemmas \ref{morphisms-lemma-etale-flat}
and \ref{morphisms-lemma-generalizations-lift-flat}.
In this way we see that each $\eta_i$ is a generic point of an
irreducible component of $W''$. Thus, by
Properties, Lemma \ref{properties-lemma-maximal-points-affine}
we can find an affine open $V \subset W''$ such that
$\{\eta_1, \ldots, \eta_n\} \subset V$.
By
Groupoids, Lemma \ref{groupoids-lemma-find-invariant-affine}
this implies that $\eta$ is contained in an $R$-invariant affine
open subscheme of $U$. The claim follows as $W$ was chosen as an
arbitrary affine open of $U$ and because the set of generic points
of irreducible components of $W \setminus \Delta$ is dense in $W$.

\medskip\noindent
Using the claim we can finish the proof. Namely, if $W \subset U$ is
an $R$-invariant affine open, then the restriction $R_W$ of $R$ to $W$
equals $R_W = s^{-1}(W) = t^{-1}(W)$ (see
Groupoids, Definition \ref{groupoids-definition-invariant-open}
and discussion following it). In particular the maps $R_W \to W$ are
finite \'etale also. It follows in particular that $R_W$ is affine.
Thus we see that $W/R_W$ is a scheme, by
Groupoids, Proposition \ref{groupoids-proposition-finite-flat-equivalence}.
On the other hand, $W/R_W$ is an open subspace of $X$ by
Spaces, Lemma \ref{spaces-lemma-finding-opens}.
Hence having a dense collection of points contained in $R$-invariant
affine open of $U$ certainly implies that the schematic locus of $X$
(see Properties of Spaces, Lemma \ref{spaces-properties-lemma-subscheme})
is open dense in $X$.
\end{proof}













\section{Residue fields and henselian local rings}
\label{section-residue-fields-henselian-local-rings}

\noindent
For a decent algebraic space we can define the residue field and the
henselian local ring at a point. For example, the following lemma
tells us the residue field of a point on a decent space is defined.

\begin{lemma}
\label{lemma-decent-points-monomorphism}
Let $S$ be a scheme. Let $X$ be an algebraic space over $S$.
Consider the map
$$
\{\Spec(k) \to X \text{ monomorphism where }k\text{ is a field}\}
\longrightarrow
|X|
$$
This map is always injective. If $X$ is decent then this map
is a bijection.
\end{lemma}

\begin{proof}
We have seen in
Properties of Spaces,
Lemma \ref{spaces-properties-lemma-points-monomorphism}
that the map is an injection in general.
By Lemma \ref{lemma-bounded-fibres} it is surjective when $X$ is
decent (actually one can say this is part of the definition
of being decent).
\end{proof}

\noindent
Let $S$ be a scheme. Let $X$ be an algebraic space over $S$.
If a point $x \in |X|$ can be represented by a monomorphism
$\Spec(k) \to X$, then the field $k$ is unique up to unique
isomorphism. For a decent
algebraic space such a monomorphism exists for every point
by Lemma \ref{lemma-decent-points-monomorphism}
and hence the following definition makes sense.

\begin{definition}
\label{definition-residue-field}
Let $S$ be a scheme. Let $X$ be a decent algebraic space over $S$.
Let $x \in |X|$. The {\it residue field of $X$ at $x$}
is the unique field $\kappa(x)$ which comes equipped with a
monomorphism $\Spec(\kappa(x)) \to X$ representing $x$.
\end{definition}

\noindent
Let $S$ be a scheme. Let $f : X \to Y$ be a morphism of decent
algebraic spaces over $S$. Let $x \in |X|$ be a point.
Set $y = f(x) \in |Y|$. Then the composition $\Spec(\kappa(x)) \to Y$
is in the equivalence class defining $y$ and hence factors through
$\Spec(\kappa(y)) \to Y$. In other words we get a commutative diagram
$$
\xymatrix{
\Spec(\kappa(x)) \ar[r]_-x \ar[d] & X \ar[d]^f \\
\Spec(\kappa(y)) \ar[r]^-y & Y
}
$$
The left vertical morphism corresponds to a homomorphism
$\kappa(y) \to \kappa(x)$ of fields. We will often simply
call this the homomorphism induced by $f$.

\begin{lemma}
\label{lemma-identifies-residue-fields}
Let $S$ be a scheme. Let $f : X \to Y$ be a morphism of decent
algebraic spaces over $S$. Let $x \in |X|$ be a point
with image $y = f(x) \in |Y|$.
The following are equivalent
\begin{enumerate}
\item $f$ induces an isomorphism $\kappa(y) \to \kappa(x)$, and
\item the induced morphism $\Spec(\kappa(x)) \to Y$ is a monomorphism.
\end{enumerate}
\end{lemma}

\begin{proof}
Immediate from the discussion above.
\end{proof}

\noindent
The following lemma tells us that the henselian local ring of a point
on a decent algebraic space is defined.

\begin{lemma}
\label{lemma-decent-space-elementary-etale-neighbourhood}
Let $S$ be a scheme. Let $X$ be a decent algebraic space over $S$.
For every point $x \in |X|$ there exists an \'etale morphism
$$
(U, u) \longrightarrow (X, x)
$$
where $U$ is an affine scheme, $u$ is the only point of $U$ lying
over $x$, and the induced homomorphism $\kappa(x) \to \kappa(u)$
is an isomorphism.
\end{lemma}

\begin{proof}
We may assume that $X$ is quasi-compact by replacing $X$ with a
quasi-compact open containing $x$. Recall that $x$ can be
represented by a quasi-compact (mono)morphism
from the spectrum a field (by definition of decent spaces). Thus the
lemma follows from Lemma \ref{lemma-filter-quasi-compact}.
\end{proof}

\begin{definition}
\label{definition-elemenary-etale-neighbourhood}
Let $S$ be a scheme. Let $X$ be an algebraic space over $S$.
Let $x \in X$ be a point. An {\it elementary \'etale neighbourhood}
is an \'etale morphism $(U, u) \to (X, x)$ where $U$ is a scheme,
$u \in U$ is a point mapping to $x$, and the morphism
$u = \Spec(\kappa(u)) \to X$ is a monomorphism.
A {\it morphism of elementary \'etale neighbourhoods}
$(U, u) \to (U', u')$ is defined as a morphism $U \to U'$
over $X$ mapping $u$ to $u'$.
\end{definition}

\noindent
If $X$ is not decent then the category of elementary \'etale neighbourhoods
may be empty.

\begin{lemma}
\label{lemma-elementary-etale-neighbourhoods}
Let $S$ be a scheme. Let $X$ be a decent algebraic space over $S$.
Let $x$ be a point of $X$.
The category of elementary \'etale neighborhoods of $(X, x)$
is cofiltered (see
Categories, Definition \ref{categories-definition-codirected}).
\end{lemma}

\begin{proof}
The category is nonempty by
Lemma \ref{lemma-decent-space-elementary-etale-neighbourhood}.
Suppose that we have two elementary \'etale neighbourhoods
$(U_i, u_i) \to (X, x)$.
Then consider $U = U_1 \times_X U_2$. Since
$\Spec(\kappa(u_i)) \to X$, $i = 1, 2$ are both monomorphisms
in the class of $x$ (Lemma \ref{lemma-identifies-residue-fields})
, we see that
$$
u = \Spec(\kappa(u_1)) \times_X \Spec(\kappa(u_2))
$$
is the spectrum of a field $\kappa(u)$ such that the induced maps
$\kappa(u_i) \to \kappa(u)$ are isomorphisms. Then $u \to U$ is a point
of $U$ and we see that $(U, u) \to (X, x)$ is an elementary
\'etale neighbourhood dominating $(U_i, u_i)$.
If $a, b : (U_1, u_1) \to (U_2, u_2)$ are two morphisms between
our elementary \'etale neighbourhoods, then we consider the scheme
$$
U = U_1 \times_{(a, b), (U_2 \times_X U_2), \Delta} U_2
$$
Using Properties of Spaces, Lemma
\ref{spaces-properties-lemma-etale-permanence}
we see that $U \to X$ is \'etale. Moreover, in exactly the same manner
as before we see that $U$ has a point $u$
such that $(U, u) \to (X, x)$ is an elementary
\'etale neighbourhood. Finally, $U \to U_1$ equalizes $a$ and $b$
and the proof is finished.
\end{proof}

\begin{definition}
\label{definition-henselian-local-ring}
Let $S$ be a scheme. Let $X$ be a decent algebraic space over $S$.
Let $x \in |X|$. The {\it henselian local ring of $X$ at $x$}, is
$$
\mathcal{O}_{X, x}^h = \colim \Gamma(U, \mathcal{O}_U)
$$
where the colimit is over the elementary \'etale neighbourhoods
$(U, u) \to (X, x)$.
\end{definition}

\noindent
Here is the analogue of
Properties of Spaces, Lemma
\ref{spaces-properties-lemma-describe-etale-local-ring}.

\begin{lemma}
\label{lemma-describe-henselian-local-ring}
Let $S$ be a scheme. Let $X$ be a decent algebraic space over $S$.
Let $x \in |X|$. Let $(U, u) \to (X, x)$ be an elementary
\'etale neighbourhood. Then
$$
\mathcal{O}_{X, x}^h = \mathcal{O}_{U, u}^h
$$
In words: the henselian local ring of $X$ at $x$
is equal to the henselization $\mathcal{O}_{U, u}^h$
of the local ring $\mathcal{O}_{U, u}$ of $U$ at $u$.
\end{lemma}

\begin{proof}
Since the category of elementary \'etale neighbourhood of $(X, x)$
is cofiltered (Lemma \ref{lemma-elementary-etale-neighbourhoods})
we see that
the category of elementary \'etale neighbourhoods of $(U, u)$
is initial in
the category of elementary \'etale neighbourhood of $(X, x)$.
Then the equality follows from
More on Morphisms, Lemma \ref{more-morphisms-lemma-describe-henselization}
and
Categories, Lemma \ref{categories-lemma-cofinal}
(initial is turned into cofinal because the colimit
definining henselian local rings is over the
opposite of the category of elementary
\'etale neighbourhoods).
\end{proof}

\begin{lemma}
\label{lemma-henselian-local-ring-strict}
Let $S$ be a scheme. Let $X$ be a decent algebraic space over $S$.
Let $\overline{x}$ be a geometric point of $X$ lying over $x \in |X|$. 
The \'etale local ring $\mathcal{O}_{X, \overline{x}}$ of $X$ at $\overline{x}$
(Properties of Spaces, Definition
\ref{spaces-properties-definition-etale-local-rings})
is the strict henselization
of the henselian local ring $\mathcal{O}_{X, x}^h$ of $X$ at $x$.
\end{lemma}

\begin{proof}
Follows from Lemma \ref{lemma-describe-henselian-local-ring},
Properties of Spaces, Lemma
\ref{spaces-properties-lemma-describe-etale-local-ring}
and the fact that $(R^h)^{sh} = R^{sh}$
for a local ring $(R, \mathfrak m, \kappa)$ and a given
separable algebraic closure $\kappa^{sep}$ of $\kappa$.
This equality follows from
Algebra, Lemma \ref{algebra-lemma-uniqueness-henselian}.
\end{proof}

\begin{lemma}
\label{lemma-residue-field-henselian-local-ring}
Let $S$ be a scheme. Let $X$ be a decent algebraic space over $S$.
Let $x \in |X|$. The residue field of the
henselian local ring of $X$ at $x$
(Definition \ref{definition-henselian-local-ring})
is the residue field of $X$ at $x$
(Definition \ref{definition-residue-field}).
\end{lemma}

\begin{proof}
Choose an elementary \'etale neighbourhood $(U, u) \to (X, x)$.
Then $\kappa(u) = \kappa(x)$ and
$\mathcal{O}_{X, x}^h = \mathcal{O}_{U, u}^h$
(Lemma \ref{lemma-describe-henselian-local-ring}).
The residue field of $\mathcal{O}_{U, u}^h$
is $\kappa(u)$ by Algebra, Lemma \ref{algebra-lemma-henselization}
(the output of this lemma is the construction/definition
of the henselization of a local ring, see
Algebra, Definition \ref{algebra-definition-henselization}).
\end{proof}

\begin{remark}
\label{remark-functoriality-henselian-local-ring}
Let $S$ be a scheme. Let $f : X \to Y$ be a morphism of decent algebraic spaces
over $S$. Let $x \in |X|$ with image $y \in |Y|$. Choose an elementary
\'etale neighbourhood $(V, v) \to (Y, y)$ (possible by
Lemma \ref{lemma-decent-space-elementary-etale-neighbourhood}).
Then $V \times_Y X$ is an algebraic space \'etale over $X$
which has a unique point $x'$ mapping to $x$ in $X$ and to $v$ in $V$.
(Details omitted; use that all points can be represented by
monomorphisms from spectra of fields.)
Choose an elementary \'etale neighbourhood $(U, u) \to (V \times_Y X, x')$.
Then we obtain the following commutative diagram
$$
\xymatrix{
\Spec(\mathcal{O}_{X, \overline{x}}) \ar[r] \ar[d] &
\Spec(\mathcal{O}_{X, x}^h) \ar[r] \ar[d] &
\Spec(\mathcal{O}_{U, u}) \ar[r] \ar[d] &
U \ar[r] \ar[d] &
X \ar[d] \\
\Spec(\mathcal{O}_{Y, \overline{y}}) \ar[r] &
\Spec(\mathcal{O}_{Y, y}^h) \ar[r] &
\Spec(\mathcal{O}_{V, v}) \ar[r] &
V \ar[r] &
Y
}
$$
This comes from the identifications
$\mathcal{O}_{X, \overline{x}} = \mathcal{O}_{U, u}^{sh}$,
$\mathcal{O}_{X, x}^h = \mathcal{O}_{U, u}^h$,
$\mathcal{O}_{Y, \overline{y}} = \mathcal{O}_{V, v}^{sh}$,
$\mathcal{O}_{Y, y}^h = \mathcal{O}_{V, v}^h$
see in
Lemma \ref{lemma-describe-henselian-local-ring}
and
Properties of Spaces, Lemma
\ref{spaces-properties-lemma-describe-etale-local-ring}
and the functoriality of the (strict) henselization
discussed in Algebra, Sections \ref{algebra-section-ind-etale} and
\ref{algebra-section-henselization}.
\end{remark}









\section{Points on decent spaces}
\label{section-points}

\noindent
In this section we prove some properties of points on decent algebraic spaces.
The following lemma shows that specialization of points behaves well
on decent algebraic spaces.
Spaces, Example \ref{spaces-example-infinite-product}
shows that this is {\bf not} true in general.

\begin{lemma}
\label{lemma-decent-no-specializations-map-to-same-point}
Let $S$ be a scheme. Let $X$ be a decent algebraic space over $S$.
Let $U \to X$ be an \'etale morphism from a scheme to $X$.
If $u, u' \in |U|$ map to the same point of $|X|$, and
$u' \leadsto u$, then $u = u'$.
\end{lemma}

\begin{proof}
Combine Lemmas \ref{lemma-bounded-fibres} and
\ref{lemma-no-specializations-map-to-same-point}.
\end{proof}

\begin{lemma}
\label{lemma-decent-specialization}
Let $S$ be a scheme. Let $X$ be a decent algebraic space over $S$.
Let $x, x' \in |X|$ and assume $x' \leadsto x$, i.e., $x$ is a
specialization of $x'$. Then for every \'etale morphism
$\varphi : U \to X$ from a scheme $U$ and any $u \in U$ with
$\varphi(u) = x$, exists a point $u'\in U$, $u' \leadsto u$ with
$\varphi(u') = x'$.
\end{lemma}

\begin{proof}
Combine Lemmas \ref{lemma-bounded-fibres} and
\ref{lemma-specialization}.
\end{proof}

\begin{lemma}
\label{lemma-kolmogorov}
Let $S$ be a scheme. Let $X$ be a decent algebraic space over $S$.
Then $|X|$ is Kolmogorov (see
Topology, Definition \ref{topology-definition-generic-point}).
\end{lemma}

\begin{proof}
Let $x_1, x_2 \in |X|$ with $x_1 \leadsto x_2$ and $x_2 \leadsto x_1$.
We have to show that $x_1 = x_2$. Pick a scheme $U$ and an \'etale morphism
$U \to X$ such that $x_1, x_2$ are both in the image of $|U| \to |X|$.
By Lemma \ref{lemma-decent-specialization} we can find a specialization
$u_1 \leadsto u_2$ in $U$ mapping to $x_1 \leadsto x_2$.
By Lemma \ref{lemma-decent-specialization} we can find
$u_2' \leadsto u_1$ mapping to $x_2 \leadsto x_1$. This means that
$u_2' \leadsto u_2$ is a specialization between points of $U$ mapping to
the same point of $X$, namely $x_2$. This is not possible, unless
$u_2' = u_2$, see
Lemma \ref{lemma-decent-no-specializations-map-to-same-point}. Hence
also $u_1 = u_2$ as desired.
\end{proof}

\begin{proposition}
\label{proposition-reasonable-sober}
Let $S$ be a scheme. Let $X$ be a decent algebraic space over $S$.
Then the topological space $|X|$ is sober (see
Topology, Definition \ref{topology-definition-generic-point}).
\end{proposition}

\begin{proof}
We have seen in Lemma \ref{lemma-kolmogorov} that $|X|$ is Kolmogorov.
Hence it remains to show that every irreducible closed subset
$T \subset |X|$ has a generic point. By
Properties of Spaces,
Lemma \ref{spaces-properties-lemma-reduced-closed-subspace}
there exists a closed subspace $Z \subset X$ with $|Z| = |T|$.
By definition this means that $Z \to X$ is a representable morphism
of algebraic spaces. Hence $Z$ is a decent algebraic space
by Lemma \ref{lemma-representable-properties}. By
Theorem \ref{theorem-decent-open-dense-scheme}
we see that there exists an open dense subspace $Z' \subset Z$
which is a scheme. This means that $|Z'| \subset T$ is open dense.
Hence the topological space $|Z'|$ is irreducible, which means that
$Z'$ is an irreducible scheme. By
Schemes, Lemma \ref{schemes-lemma-scheme-sober}
we conclude that $|Z'|$ is the closure of a single point $\eta \in T$
and hence also $T = \overline{\{\eta\}}$, and we win.
\end{proof}

\noindent
For decent algebraic spaces dimension works as expected.

\begin{lemma}
\label{lemma-dimension-decent-space}
Let $S$ be a scheme. Dimension as defined in
Properties of Spaces, Section \ref{spaces-properties-section-dimension}
behaves well on decent algebraic spaces $X$ over $S$.
\begin{enumerate}
\item If $x \in |X|$, then $\dim_x(|X|) = \dim_x(X)$, and
\item $\dim(|X|) = \dim(X)$.
\end{enumerate}
\end{lemma}

\begin{proof}
Proof of (1). Choose a scheme $U$ with a point $u \in U$
and an \'etale morphism $h : U \to X$ mapping $u$ to $x$.
By definition the dimension of $X$ at $x$ is $\dim_u(|U|)$.
Thus we may pick $U$ such that $\dim_x(X) = \dim(|U|)$.
Let $d$ be an integer. If $\dim(U) \geq d$, then
there exists a sequence of nontrivial specializations
$u_d \leadsto \ldots \leadsto u_0$ in $U$. Taking the image
we find a corresponding sequence
$h(u_d) \leadsto \ldots \leadsto h(u_0)$
each of which is nontrivial by
Lemma \ref{lemma-decent-no-specializations-map-to-same-point}.
Hence we see that the image of $|U|$ in $|X|$ has dimension at least $d$.
Conversely, suppose that $x_d \leadsto \ldots \leadsto x_0$ is a
sequence of specializations in $|X|$ with $x_0$ in the image of
$|U| \to |X|$. Then we can lift this to a sequence of specializations
in $U$ by Lemma \ref{lemma-decent-specialization}.

\medskip\noindent
Part (2) is an immediate consequence of part (1),
Topology, Lemma \ref{topology-lemma-dimension-supremum-local-dimensions},
and Properties of Spaces, Section \ref{spaces-properties-section-dimension}.
\end{proof}

\begin{lemma}
\label{lemma-dimension-local-ring-quasi-finite}
Let $S$ be a scheme. Let $X \to Y$ be a locally quasi-finite morphism
of algebraic spaces over $S$. Let $x \in |X|$ with image $y \in |Y|$.
Then the dimension of the local ring of $Y$ at $y$ is $\geq$ to the
dimension of the local ring of $X$ at $x$.
\end{lemma}

\begin{proof}
The definition of the dimension of the local ring of a point on an
algebraic space is given in Properties of Spaces, Definition
\ref{spaces-properties-definition-dimension-local-ring}.
Choose an \'etale morphism $(V, v) \to (Y, y)$ where $V$ is a scheme.
Choose an \'etale morphism $U \to V \times_Y X$ and a point $u \in U$
mapping to $x \in |X|$ and $v \in V$. Then $U \to V$ is locally
quasi-finite and we have to prove that
$$
\dim(\mathcal{O}_{V, v}) \geq \dim(\mathcal{O}_{U, u})
$$
This is Algebra, Lemma \ref{algebra-lemma-dimension-inequality-quasi-finite}.
\end{proof}

\begin{lemma}
\label{lemma-dimension-quasi-finite}
Let $S$ be a scheme. Let $X \to Y$ be a locally quasi-finite morphism
of algebraic spaces over $S$. Then $\dim(X) \leq \dim(Y)$.
\end{lemma}

\begin{proof}
This follows from Lemma \ref{lemma-dimension-local-ring-quasi-finite}
and Properties of Spaces, Lemma \ref{spaces-properties-lemma-dimension}.
\end{proof}

\noindent
The following lemma is a tiny bit stronger than
Properties of Spaces,
Lemma \ref{spaces-properties-lemma-point-like-spaces}.
We will improve this lemma in Lemma \ref{lemma-when-field}.

\begin{lemma}
\label{lemma-decent-point-like-spaces}
Let $S$ be a scheme. Let $k$ be a field. Let $X$ be an algebraic space
over $S$ and assume that there exists a surjective \'etale morphism
$\Spec(k) \to X$. If $X$ is decent, then $X \cong \Spec(k')$
where $k/k'$ is a finite separable extension.
\end{lemma}

\begin{proof}
The assumption implies that $|X| = \{x\}$ is a singleton. Since
$X$ is decent we can find a quasi-compact monomorphism $\Spec(k') \to X$
whose image is $x$. Then the projection
$U = \Spec(k') \times_X \Spec(k) \to \Spec(k)$
is a monomorphism, whence $U = \Spec(k)$, see
Schemes, Lemma \ref{schemes-lemma-mono-towards-spec-field}.
Hence the projection $\Spec(k) = U \to \Spec(k')$ is \'etale and
we win.
\end{proof}






\section{Reduced singleton spaces}
\label{section-singleton}

\noindent
A {\it singleton} space is an algebraic space $X$ such that $|X|$ is
a singleton. It turns out that these can be more interesting than
just being the spectrum of a field, see
Spaces, Example \ref{spaces-example-Qbar}.
We develop a tiny bit of machinery to be able to talk about these.

\begin{lemma}
\label{lemma-flat-cover-by-field}
Let $S$ be a scheme. Let $Z$ be an algebraic space over $S$.
Let $k$ be a field and let $\Spec(k) \to Z$ be surjective and flat.
Then any morphism $\Spec(k') \to Z$ where $k'$ is a field is
surjective and flat.
\end{lemma}

\begin{proof}
Consider the fibre square
$$
\xymatrix{
T \ar[d] \ar[r] & \Spec(k) \ar[d] \\
\Spec(k') \ar[r] & Z
}
$$
Note that $T \to \Spec(k')$ is flat and surjective hence $T$
is not empty. On the other hand $T \to \Spec(k)$ is flat as
$k$ is a field. Hence $T \to Z$ is flat and surjective.
It follows from
Morphisms of Spaces, Lemma \ref{spaces-morphisms-lemma-flat-permanence}
that $\Spec(k') \to Z$ is flat. It is surjective as by assumption
$|Z|$ is a singleton.
\end{proof}

\begin{lemma}
\label{lemma-unique-point}
Let $S$ be a scheme.
Let $Z$ be an algebraic space over $S$. The following are equivalent
\begin{enumerate}
\item $Z$ is reduced and $|Z|$ is a singleton,
\item there exists a surjective flat morphism $\Spec(k) \to Z$
where $k$ is a field, and
\item there exists a locally of finite type, surjective, flat morphism
$\Spec(k) \to Z$ where $k$ is a field.
\end{enumerate}
\end{lemma}

\begin{proof}
Assume (1). Let $W$ be a scheme and
let $W \to Z$ be a surjective \'etale morphism. Then $W$ is
a reduced scheme. Let $\eta \in W$ be a generic point of an irreducible
component of $W$. Since $W$ is reduced we have
$\mathcal{O}_{W, \eta} = \kappa(\eta)$. It follows that the canonical
morphism $\eta = \Spec(\kappa(\eta)) \to W$ is flat. We see that the
composition $\eta \to Z$ is flat (see
Morphisms of Spaces, Lemma \ref{spaces-morphisms-lemma-composition-flat}).
It is also surjective as $|Z|$ is a singleton. In other words
(2) holds.

\medskip\noindent
Assume (2). Let $W$ be a scheme and
let $W \to Z$ be a surjective \'etale morphism. Choose a field
$k$ and a surjective flat morphism $\Spec(k) \to Z$.
Then $W \times_Z \Spec(k)$ is a scheme \'etale over $k$.
Hence $W \times_Z \Spec(k)$ is a disjoint union of spectra of fields
(see Remark \ref{remark-recall}), in particular reduced. Since
$W \times_Z \Spec(k) \to W$
is surjective and flat we conclude that $W$ is reduced
(Descent, Lemma \ref{descent-lemma-descend-reduced}).
In other words (1) holds.

\medskip\noindent
It is clear that (3) implies (2). Finally, assume (2). Pick a nonempty
affine scheme $W$ and an \'etale morphism $W \to Z$. Pick a closed
point $w \in W$ and set $k = \kappa(w)$. The composition
$$
\Spec(k) \xrightarrow{w} W \longrightarrow Z
$$
is locally of finite type by
Morphisms of Spaces, Lemmas
\ref{spaces-morphisms-lemma-composition-finite-type} and
\ref{spaces-morphisms-lemma-etale-locally-finite-type}.
It is also flat and surjective by
Lemma \ref{lemma-flat-cover-by-field}.
Hence (3) holds.
\end{proof}

\noindent
The following lemma singles out a slightly better class of singleton
algebraic spaces than the preceding lemma.

\begin{lemma}
\label{lemma-unique-point-better}
Let $S$ be a scheme. Let $Z$ be an algebraic space over $S$.
The following are equivalent
\begin{enumerate}
\item $Z$ is reduced, locally Noetherian, and $|Z|$
is a singleton, and
\item there exists a locally finitely presented, surjective, flat morphism
$\Spec(k) \to Z$ where $k$ is a field.
\end{enumerate}
\end{lemma}

\begin{proof}
Assume (2) holds. By
Lemma \ref{lemma-unique-point}
we see that $Z$ is reduced and $|Z|$ is a singleton.
Let $W$ be a scheme and let $W \to Z$ be a surjective \'etale
morphism. Choose a field $k$ and a locally finitely presented, surjective,
flat morphism $\Spec(k) \to Z$.
Then $W \times_Z \Spec(k)$ is a scheme
\'etale over $k$, hence a disjoint union of spectra of fields
(see Remark \ref{remark-recall}),
hence locally Noetherian. Since $W \times_Z \Spec(k) \to W$
is flat, surjective, and locally of finite presentation, we see
that $\{W \times_Z \Spec(k) \to W\}$ is an fppf covering
and we conclude that $W$ is locally Noetherian
(Descent, Lemma
\ref{descent-lemma-Noetherian-local-fppf}).
In other words (1) holds.

\medskip\noindent
Assume (1). Pick a nonempty affine scheme $W$ and an \'etale morphism
$W \to Z$. Pick a closed point $w \in W$ and set
$k = \kappa(w)$. Because $W$ is locally Noetherian the morphism
$w : \Spec(k) \to W$ is of finite presentation, see
Morphisms, Lemma \ref{morphisms-lemma-closed-immersion-finite-presentation}.
Hence the composition
$$
\Spec(k) \xrightarrow{w} W \longrightarrow Z
$$
is locally of finite presentation by
Morphisms of Spaces, Lemmas
\ref{spaces-morphisms-lemma-composition-finite-presentation} and
\ref{spaces-morphisms-lemma-etale-locally-finite-presentation}.
It is also flat and surjective by
Lemma \ref{lemma-flat-cover-by-field}.
Hence (2) holds.
\end{proof}

\begin{lemma}
\label{lemma-monomorphism-into-point}
Let $S$ be a scheme.
Let $Z' \to Z$ be a monomorphism of algebraic spaces over $S$.
Assume there exists a field $k$ and a locally finitely presented, surjective,
flat morphism $\Spec(k) \to Z$. Then either $Z'$
is empty or $Z' = Z$.
\end{lemma}

\begin{proof}
We may assume that $Z'$ is nonempty. In this case the
fibre product $T = Z' \times_Z \Spec(k)$
is nonempty, see
Properties of Spaces, Lemma \ref{spaces-properties-lemma-points-cartesian}.
Now $T$ is an algebraic space and the projection $T \to \Spec(k)$
is a monomorphism. Hence $T = \Spec(k)$, see
Morphisms of Spaces, Lemma
\ref{spaces-morphisms-lemma-monomorphism-toward-field}.
We conclude that $\Spec(k) \to Z$ factors through $Z'$.
But as $\Spec(k) \to Z$ is surjective, flat and locally of finite
presentation, we see that $\Spec(k) \to Z$ is surjective as a
map of sheaves on $(\Sch/S)_{fppf}$ (see
Spaces, Remark \ref{spaces-remark-warning})
and we conclude that $Z' = Z$.
\end{proof}

\noindent
The following lemma says that to each point of an algebraic space we
can associate a canonical reduced, locally Noetherian singleton
algebraic space.

\begin{lemma}
\label{lemma-find-singleton-from-point}
Let $S$ be a scheme. Let $X$ be an algebraic space over $S$.
Let $x \in |X|$. Then there exists a unique monomorphism
$Z \to X$ of algebraic spaces
over $S$ such that $Z$ is an algebraic space which satisfies the equivalent
conditions of
Lemma \ref{lemma-unique-point-better}
and such that the image of $|Z| \to |X|$ is $\{x\}$.
\end{lemma}

\begin{proof}
Choose a scheme $U$ and a surjective \'etale morphism $U \to X$.
Set $R = U \times_X U$ so that $X = U/R$ is a presentation (see
Spaces, Section \ref{spaces-section-presentations}).
Set
$$
U' = \coprod\nolimits_{u \in U\text{ lying over }x} \Spec(\kappa(u)).
$$
The canonical morphism $U' \to U$ is a monomorphism. Let
$$
R' = U' \times_X U' = R \times_{(U \times_S U)} (U' \times_S U').
$$
Because $U' \to U$ is a monomorphism we see that the projections
$s', t' : R' \to U'$ factor as a monomorphism followed by an
\'etale morphism. Hence, as $U'$ is a disjoint union of spectra
of fields, using
Remark \ref{remark-recall},
and using
Schemes, Lemma \ref{schemes-lemma-mono-towards-spec-field}
we conclude that $R'$ is a disjoint union of spectra of fields and
that the morphisms $s', t' : R' \to U'$ are \'etale. Hence
$Z = U'/R'$ is an algebraic space by
Spaces, Theorem \ref{spaces-theorem-presentation}.
As $R'$ is the restriction of $R$ by $U' \to U$ we see
$Z \to X$ is a monomorphism by
Groupoids, Lemma \ref{groupoids-lemma-quotient-groupoid-restrict}.
Since $Z \to X$ is a monomorphism we see that $|Z| \to |X|$ is injective, see
Morphisms of Spaces, Lemma
\ref{spaces-morphisms-lemma-monomorphism-injective-points}.
By
Properties of Spaces, Lemma \ref{spaces-properties-lemma-points-cartesian}
we see that
$$
|U'| = |Z \times_X U'| \to |Z| \times_{|X|} |U'|
$$
is surjective which implies (by our choice of $U'$) that
$|Z| \to |X|$ has image $\{x\}$. We conclude that $|Z|$ is a singleton.
Finally, by construction $U'$ is locally Noetherian and reduced, i.e.,
we see that $Z$ satisfies the equivalent conditions of
Lemma \ref{lemma-unique-point-better}.

\medskip\noindent
Let us prove uniqueness of $Z \to X$. Suppose that
$Z' \to X$ is a second such monomorphism of algebraic spaces.
Then the projections
$$
Z' \longleftarrow Z' \times_X Z \longrightarrow Z
$$
are monomorphisms. The algebraic space in the middle is nonempty by
Properties of Spaces,
Lemma \ref{spaces-properties-lemma-points-cartesian}.
Hence the two projections are isomorphisms by
Lemma \ref{lemma-monomorphism-into-point}
and we win.
\end{proof}

\noindent
We introduce the following terminology which foreshadows
the residual gerbes we will introduce later, see
Properties of Stacks, Definition
\ref{stacks-properties-definition-residual-gerbe}.

\begin{definition}
\label{definition-residual-space}
Let $S$ be a scheme.
Let $X$ be an algebraic space over $S$. Let $x \in |X|$.
The
{\it residual space of $X$ at $x$}\footnote{This is nonstandard notation.}
is the monomorphism $Z_x \to X$ constructed in
Lemma \ref{lemma-find-singleton-from-point}.
\end{definition}

\noindent
In particular we know that $Z_x$ is a
locally Noetherian, reduced, singleton algebraic space
and that there exists a field and a surjective, flat, locally
finitely presented morphism
$$
\Spec(k) \longrightarrow Z_x.
$$
The residual space is often given by a monomorphism
from the spectrum of a field.

\begin{lemma}
\label{lemma-residual-space-monomorphism}
Let $S$ be a scheme. Let $X$ be an algebraic space over $S$. Let $x \in |X|$.
The residual space $Z_x$ of $X$ at $x$ is isomorphic to the spectrum of a field
if and only if $x$ can be represented by a monomorphism $\Spec(k) \to X$
where $k$ is a field. If $X$ is decent, this holds for all $x \in |X|$.
\end{lemma}

\begin{proof}
Since $Z_x \to X$ is a monomorphism, if $Z_x = \Spec(k)$ for some field $k$,
then $x$ is represented by the monomorphism $\Spec(k) = Z_x \to X$.
Conversely, if $\Spec(k) \to X$ is a monomorphism which represents $x$,
then $Z_x \times_X \Spec(k) \to \Spec(k)$ is a monomorphism whose source
is nonempty by Properties of Spaces, Lemma
\ref{spaces-properties-lemma-points-cartesian}.
Hence $Z_x \times_X \Spec(k) = \Spec(k)$ by Morphisms of Spaces, Lemma
\ref{spaces-morphisms-lemma-monomorphism-toward-field}.
Hence we get a monomorphism $\Spec(k) \to Z_x$. This is
an isomorphism by Lemma \ref{lemma-monomorphism-into-point}.
The final statement follows from Lemma \ref{lemma-decent-points-monomorphism}.
\end{proof}

\noindent
The residual space is a regular algebraic space by the following lemma.

\begin{lemma}
\label{lemma-residual-space-regular}
A reduced, locally Noetherian singleton algebraic space $Z$ is regular.
\end{lemma}

\begin{proof}
Let $Z$ be a reduced, locally Noetherian singleton algebraic space
over a scheme $S$. Let $W \to Z$ be a surjective \'etale morphism where $W$
is a scheme. Let $k$ be a field and let $\Spec(k) \to Z$
be surjective, flat, and locally of finite presentation (see
Lemma \ref{lemma-unique-point-better}).
The scheme $T = W \times_Z \Spec(k)$ is
\'etale over $k$ in particular regular, see
Remark \ref{remark-recall}.
Since $T \to W$ is locally of finite presentation, flat, and surjective it
follows that $W$ is regular, see
Descent, Lemma \ref{descent-lemma-descend-regular}.
By definition this means that $Z$ is regular.
\end{proof}

\begin{lemma}
\label{lemma-factor-through-residual-space-Noetherian}
Let $S$ be a scheme. Let $f : Y \to X$ be a morphism of algebraic
spaces over $S$. Let $x \in |X|$ be a point. Assume
\begin{enumerate}
\item $|f|(|Y|)$ is contained in $\{x\} \subset |X|$,
\item $Y$ is reduced, and
\item $X$ is locally Noetherian.
\end{enumerate}
Then $f$ factors through the residual space $Z_x$ of $X$ at $x$.
\end{lemma}

\begin{proof}
Preliminary remark: since $Z_x \to X$ is a monomorphism, it suffices to find
a surjective \'etale morphism $Y' \to Y$ such that $Y' \to X$
factors through $Z_x$. A remark here is that $Y'$ is reduced as well.

\medskip\noindent
Let $U$ be an affine scheme and let $U \to X$ be an \'etale morphism
such that $x$ is in the image of $|U| \to |X|$. Since $X$ is locally
Noetherian, $U$ is a Noetherian affine scheme. By assumption (1)
we see that $Y' = U \times_X Y \to Y$ is surjective as well as \'etale.
Denote $E \subset |U|$ the set of points mapping to $x$.
There are no nontrivial specializations between the elements of $E$, see
Lemma \ref{lemma-no-specializations-map-to-same-point-Noetherian}.
The morphism $Y' \to U$ maps $|Y'|$ into $E$. By our construction
of $Z_x$ in the proof of Lemma \ref{lemma-find-singleton-from-point}
we know that $\coprod_{u \in E} u \to X$ factors through $Z_x$.
Hence it suffices to prove that $Y' \to U$ factors through
$\coprod_{u \in E} u \to X$. After replacing $Y'$ by an \'etale covering
by a scheme (which we are allowed by our preliminary remark), this follows
from Morphisms, Lemma \ref{morphisms-lemma-factor-through-set-of-points}.
\end{proof}

\begin{lemma}
\label{lemma-factor-through-residual-space-decent}
Let $S$ be a scheme. Let $f : Y \to X$ be a morphism of algebraic
spaces over $S$. Let $x \in |X|$ be a point. Assume
\begin{enumerate}
\item $|f|(|Y|)$ is contained in $\{x\} \subset |X|$,
\item $Y$ is reduced, and
\item $x$ can be represented by a quasi-compact monomorphism
$x : \Spec(k) \to X$ where $k$ is a field (for example if $X$ is decent).
\end{enumerate}
Then $f$ factors through
the residual space $Z_x = \Spec(k)$ of $X$ at $x$.
\end{lemma}

\begin{proof}
By Lemma \ref{lemma-residual-space-monomorphism} we have $Z_x = \Spec(k)$.

\medskip\noindent
Preliminary remark: since $\Spec(k) \to X$ is a monomorphism,
it suffices to find
a surjective \'etale morphism $Y' \to Y$ such that $Y' \to X$
factors through $Z_x$. A remark here is that $Y'$ is reduced as well.

\medskip\noindent
After replacing $X$ by a quasi-compact open neighbourhood of $x$, 
we may assume $X$ quasi-compact. By Lemma \ref{lemma-filter-quasi-compact}, $x$
is a point of $T \subset U \subset X$ where $T \to U$ (resp.\ $U \to X$)
is a closed (resp.\ open) immersion, and $T$ is a scheme.
By Properties of Spaces, Lemma
\ref{spaces-properties-lemma-factor-through-open-subspace},
$f$ factors through $U$, so we may assume $U = X$. Then $f$
factors through $T$ because $Y$ is reduced, see
Properties of Spaces, Lemma \ref{spaces-properties-lemma-map-into-reduction}.
So we may assume that $X = T$ is a scheme.
By our preliminary remark we may assume $Y$ is a scheme too.
This reduces us to Morphisms, Lemma \ref{morphisms-lemma-factor-through-point}.
\end{proof}

\begin{example}
\label{example-does-not-factor-through-residual-gerbe}
Here is a counter example to
Lemmas \ref{lemma-factor-through-residual-space-Noetherian} and
\ref{lemma-factor-through-residual-space-decent}
in case $X$ is neither locally Noetherian nor decent.
Let $k$ be a field. Let $G$ be an infinite profinite group.
Let $Y$ be $G$ viewed as a zero-dimensional affine $k$-group scheme, i.e.,
$Y = \Spec(\text{locally constant maps } G \to k)$.
Let $\Gamma$ be $G$ viewed as a discrete $k$-group scheme,
acting on $X$ by translations. Put $X = Y/\Gamma$.
This is a one-point algebraic space, with projection $q : Y \to X$.
Let $e \in G$ be the origin (any element would do), and view
it as a $k$-point of $Y$. We get a $k$-point $x :\Spec(k) \to X$
which is a monomorphism since it is a section of $X \to \Spec(k)$.
We claim that (although $Y$ is affine and reduced and $|X| = \{x\}$),
the morphism $q$ does not factor through any morphism $\Spec(K) \to X$,
where $K$ is a field. Otherwise it would factor through
$x$ by Properties of Spaces, Lemma
\ref{spaces-properties-lemma-equivalence-class-point-monomorphism}.
Now the pullback of $q$ by $x$ is $\Gamma \to \Spec(k)$, with
the projection $\Gamma \to Y$ being the orbit map $g \mapsto g \cdot e$.
The latter has no section, whence the claim.
\end{example}

\begin{lemma}
\label{lemma-residual-space-closed}
Let $S$ be a scheme. Let $X$ be an algebraic space over $S$. Let $x \in |X|$
with residual space $Z_x \subset X$. Assume $X$ is locally Noetherian.
Then $x$ is a closed point of $|X|$ if and only if
the morphism $Z_x \to X$ is a closed immersion.
\end{lemma}

\begin{proof}
If $Z_x \to X$ is a closed immersion, then $x$ is a closed point of $|X|$, see
Morphisms of Spaces, Lemma \ref{spaces-morphisms-lemma-immersion-when-closed}.
Conversely, assume $x$ is a closed point of $|X|$.
Let $Z \subset X$ be the reduced closed subspace with $|Z| = \{x\}$
(Properties of Spaces,
Lemma \ref{spaces-properties-lemma-reduced-closed-subspace}).
Then $Z$ is locally Noetherian by Morphisms of Spaces, Lemmas
\ref{spaces-morphisms-lemma-immersion-locally-finite-type} and
\ref{spaces-morphisms-lemma-locally-finite-type-locally-noetherian}.
Since also $Z$ is reduced and $|Z| = \{x\}$ it $Z = Z_x$ is the
residual space by definition.
\end{proof}











\section{Decent spaces}
\label{section-decent}

\noindent
In this section we collect some useful facts on decent spaces.

\begin{lemma}
\label{lemma-locally-Noetherian-decent-quasi-separated}
Any locally Noetherian decent algebraic space is quasi-separated.
\end{lemma}

\begin{proof}
Namely, let $X$ be an algebraic space (over some base scheme, for
example over $\mathbf{Z}$) which is decent and locally Noetherian.
Let $U \to X$ and $V \to X$ be \'etale morphisms with $U$ and $V$
affine schemes. We have to show that $W = U \times_X V$ is quasi-compact
(Properties of Spaces, Lemma
\ref{spaces-properties-lemma-characterize-quasi-separated}).
Since $X$ is locally Noetherian, the schemes $U$, $V$ are Noetherian
and $W$ is locally Noetherian. Since $X$ is decent, the fibres
of the morphism $W \to U$ are finite. Namely, we can represent
any $x \in |X|$ by a quasi-compact monomorphism $\Spec(k) \to X$.
Then $U_k$ and $V_k$ are finite disjoint unions of spectra of
finite separable extensions of $k$ (Remark \ref{remark-recall})
and we see that $W_k = U_k \times_{\Spec(k)} V_k$ is finite.
Let $n$ be the maximum degree of a fibre of $W \to U$ at a generic
point of an irreducible component of $U$. Consider the stratification
$$
U = U_0 \supset U_1 \supset U_2 \supset \ldots
$$
associated to $W \to U$ in
More on Morphisms, Lemma \ref{more-morphisms-lemma-stratify-flat-fp-lqf}.
By our choice of $n$ above we conclude that $U_{n + 1}$ is empty.
Hence we see that the fibres of $W \to U$ are universally bounded.
Then we can apply More on Morphisms, Lemma
\ref{more-morphisms-lemma-stratify-flat-fp-lqf-universally-bounded}
to find a stratification
$$
\emptyset = Z_{-1} \subset Z_0 \subset Z_1 \subset Z_2 \subset
\ldots \subset Z_n = U
$$
by closed subsets such that with $S_r = Z_r \setminus Z_{r - 1}$
the morphism $W \times_U S_r \to S_r$ is finite locally free.
Since $U$ is Noetherian, the schemes $S_r$ are Noetherian,
whence the schemes $W \times_U S_r$ are Noetherian, whence
$W = \coprod W \times_U S_r$ is quasi-compact as desired.
\end{proof}

\begin{lemma}
\label{lemma-when-field}
Let $S$ be a scheme. Let $X$ be a decent algebraic space over $S$.
\begin{enumerate}
\item If $|X|$ is a singleton then $X$ is a scheme.
\item If $|X|$ is a singleton and $X$ is reduced, then
$X \cong \Spec(k)$ for some field $k$.
\end{enumerate}
\end{lemma}

\begin{proof}
Assume $|X|$ is a singleton. It follows immediately from
Theorem \ref{theorem-decent-open-dense-scheme} that $X$ is a scheme,
but we can also argue directly as follows.
Choose an affine scheme $U$ and a surjective \'etale morphism $U \to X$.
Set $R = U \times_X U$. Then $U$ and $R$ have finitely many points by
Lemma \ref{lemma-UR-finite-above-x} (and the definition of a decent space).
All of these points are closed in $U$ and $R$ by
Lemma \ref{lemma-decent-no-specializations-map-to-same-point}.
It follows that $U$ and $R$ are affine schemes.
We may shrink $U$ to a singleton space. Then $U$ is
the spectrum of a henselian local ring, see
Algebra, Lemma \ref{algebra-lemma-local-dimension-zero-henselian}.
The projections $R \to U$ are \'etale, hence finite \'etale because
$U$ is the spectrum of a $0$-dimensional henselian local ring, see
Algebra, Lemma \ref{algebra-lemma-characterize-henselian}.
It follows that $X$ is a scheme by
Groupoids, Proposition \ref{groupoids-proposition-finite-flat-equivalence}.

\medskip\noindent
Part (2) follows from (1) and the fact that a reduced singleton
scheme is the spectrum of a field.
\end{proof}

\begin{remark}
\label{remark-one-point-decent-scheme}
We will see in
Limits of Spaces, Lemma \ref{spaces-limits-lemma-reduction-scheme}
that an algebraic space
whose reduction is a scheme is a scheme.
\end{remark}

\begin{lemma}
\label{lemma-algebraic-residue-field-extension-closed-point}
Let $S$ be a scheme. Let $X$ be a decent algebraic space over $S$.
Consider a commutative diagram
$$
\xymatrix{
\Spec(k) \ar[rr] \ar[rd] & & X \ar[ld] \\
& S
}
$$
Assume that the image point $s \in S$ of $\Spec(k) \to S$ is
a closed point and that $\kappa(s) \subset k$ is algebraic.
Then the image $x$ of $\Spec(k) \to X$ is a closed point of $|X|$.
\end{lemma}

\begin{proof}
Suppose that $x \leadsto x'$ for some $x' \in |X|$. Choose an
\'etale morphism $U \to X$ where $U$ is a scheme and a point $u' \in U'$
mapping to $x'$. Choose a specialization $u \leadsto u'$ in $U$ with $u$
mapping to $x$ in $X$, see Lemma \ref{lemma-decent-specialization}.
Then $u$ is the image of a point $w$ of the scheme
$W = \Spec(k) \times_X U$. Since the projection $W \to \Spec(k)$ is \'etale
we see that $\kappa(w) \supset k$ is finite. Hence
$\kappa(w) \supset \kappa(s)$ is algebraic. Hence $\kappa(u) \supset \kappa(s)$
is algebraic. Thus $u$ is a closed point of $U$ by
Morphisms, Lemma
\ref{morphisms-lemma-algebraic-residue-field-extension-closed-point-fibre}.
Thus $u = u'$, whence $x = x'$.
\end{proof}

\begin{lemma}
\label{lemma-finite-residue-field-extension-finite}
Let $S$ be a scheme. Let $X$ be a decent algebraic space over $S$.
Consider a commutative diagram
$$
\xymatrix{
\Spec(k) \ar[rr] \ar[rd] & & X \ar[ld] \\
& S
}
$$
Assume that the image point $s \in S$ of $\Spec(k) \to S$ is
a closed point and that the field extension $k/\kappa(s)$ is finite.
Then $\Spec(k) \to X$ is a finite morphism. If $\kappa(s) = k$
then $\Spec(k) \to X$ is a closed immersion.
\end{lemma}

\begin{proof}
By Lemma \ref{lemma-algebraic-residue-field-extension-closed-point}
the image point $x \in |X|$ is closed. Let $Z \subset X$ be the
reduced closed subspace with $|Z| = \{x\}$ (Properties of Spaces,
Lemma \ref{spaces-properties-lemma-reduced-closed-subspace}).
Note that $Z$ is a decent algebraic space by
Lemma \ref{lemma-representable-named-properties}.
By Lemma \ref{lemma-when-field} we see that $Z = \Spec(k')$
for some field $k'$. Of course $k \supset k' \supset \kappa(s)$.
Then $\Spec(k) \to Z$ is a finite morphism of schemes
and $Z \to X$ is a finite morphism as it is a closed immersion.
Hence $\Spec(k) \to X$ is finite (Morphisms of Spaces, Lemma
\ref{spaces-morphisms-lemma-composition-integral}).
If $k = \kappa(s)$, then $\Spec(k) = Z$ and $\Spec(k) \to X$
is a closed immersion.
\end{proof}

\begin{lemma}
\label{lemma-decent-space-closed-point}
Let $S$ be a scheme. Suppose $X$ is a decent algebraic space over $S$.
Let $x \in |X|$ be a closed point. Then $x$ can be represented by a
closed immersion $i : \Spec(k) \to X$ from the spectrum of a field.
\end{lemma}

\begin{proof}
We know that $x$ can be represented by a quasi-compact monomorphism
$i : \Spec(k) \to X$ where $k$ is a field
(Definition \ref{definition-very-reasonable}).
Let $U \to X$ be an \'etale morphism where $U$ is an affine scheme.
As $x$ is closed and $X$ decent, the fibre $F$ of $|U| \to |X|$ over $x$
consists of closed points
(Lemma \ref{lemma-decent-no-specializations-map-to-same-point}).
As $i$ is a monomorphism, so is $U_k = U \times_X \Spec(k) \to U$.
In particular, the map $|U_k| \to F$ is injective. Since $U_k$
is quasi-compact and \'etale over a field, we see that $U_k$ is a
finite disjoint union of spectra of fields (Remark \ref{remark-recall}).
Say $U_k = \Spec(k_1) \amalg \ldots \amalg \Spec(k_r)$.
Since $\Spec(k_i) \to U$ is a monomorphism, we see that
its image $u_i$ has residue field $\kappa(u_i) = k_i$.
Since $u_i \in F$ is a closed point we conclude the morphism
$\Spec(k_i) \to U$ is a closed immersion. As the $u_i$ are pairwise distinct,
$U_k \to U$ is a closed immersion. Hence $i$ is a closed immersion
(Morphisms of Spaces, Lemma
\ref{spaces-morphisms-lemma-closed-immersion-local}). This finishes the proof.
\end{proof}





\section{Locally separated spaces}
\label{section-locally-separated}

\noindent
It turns out that a locally separated algebraic space is decent.

\begin{lemma}
\label{lemma-infinite-number}
Let $A$ be a ring. Let $k$ be a field. Let $\mathfrak p_n$, $n \geq 1$
be a sequence of pairwise distinct primes of $A$. Moreover, for each
$n$ let $k \to \kappa(\mathfrak p_n)$ be an embedding. Then the closure
of the image of
$$
\coprod\nolimits_{n \not = m}
\Spec(\kappa(\mathfrak p_n) \otimes_k \kappa(\mathfrak p_m))
\longrightarrow
\Spec(A \otimes A)
$$
meets the diagonal.
\end{lemma}

\begin{proof}
Set $k_n = \kappa(\mathfrak p_n)$. We may assume that $A = \prod k_n$.
Denote $x_n = \Spec(k_n)$ the open and closed point corresponding to
$A \to k_n$. Then $\Spec(A) = Z \amalg \{x_n\}$ where $Z$ is a nonempty
closed subset. Namely, $Z = V(e_n; n \geq 1)$ where $e_n$
is the idempotent of $A$ corresponding to the factor $k_n$
and $Z$ is nonempty as the ideal generated by the $e_n$ is not
equal to $A$. We will show that the closure of the image
contains $\Delta(Z)$. The kernel of the map
$$
(\prod k_n) \otimes_k (\prod k_m)
\longrightarrow
\prod\nolimits_{n \not = m} k_n \otimes_k k_m
$$
is the ideal generated by $e_n \otimes e_n$, $n \geq 1$.
Hence the closure of the image of the map on spectra is 
$V(e_n \otimes e_n; n \geq 1)$ whose intersection with $\Delta(\Spec(A))$
is $\Delta(Z)$. Thus it suffices to show that
$$
\coprod\nolimits_{n \not = m} \Spec(k_n \otimes_k k_m)
\longrightarrow
\Spec(\prod\nolimits_{n \not = m} k_n \otimes_k k_m)
$$
has dense image. This follows as the family of ring maps
$\prod_{n \not = m} k_n \otimes_k k_m \to k_n \otimes_k k_m$
is jointly injective.
\end{proof}

\begin{lemma}[David Rydh]
\label{lemma-locally-separated-decent}
A locally separated algebraic space is decent.
\end{lemma}

\begin{proof}
Let $S$ be a scheme and let $X$ be a locally separated algebraic space
over $S$. We may assume $S = \Spec(\mathbf{Z})$, see
Properties of Spaces, Definition \ref{spaces-properties-definition-separated}.
Unadorned fibre products will be over $\mathbf{Z}$.
Let $x \in |X|$. Choose a scheme $U$, an \'etale
morphism $U \to X$, and a point $u \in U$ mapping to $x$ in $|X|$.
As usual we identify $u = \Spec(\kappa(u))$.
As $X$ is locally separated the morphism
$$
u \times_X u \to u \times u
$$
is an immersion (Morphisms of Spaces, Lemma
\ref{spaces-morphisms-lemma-fibre-product-after-map}).
Hence More on Groupoids, Lemma
\ref{more-groupoids-lemma-locally-closed-image-is-closed}
tells us that it is a closed immersion (use
Schemes, Lemma \ref{schemes-lemma-immersion-when-closed}).
As $u \times_X u \to u \times_X U$ is a monomorphism (base change
of $u \to U$) and as $u \times_X U \to u$ is \'etale we conclude that
$u \times_X u$ is a disjoint union of spectra of fields
(see Remark \ref{remark-recall} and
Schemes, Lemma \ref{schemes-lemma-mono-towards-spec-field}).
Since it is also closed in the affine scheme $u \times u$ we
conclude $u \times_X u$ is a finite disjoint union of spectra of fields.
Thus $x$ can be represented by a monomorphism $\Spec(k) \to X$ where $k$
is a field, see
Lemma \ref{lemma-R-finite-above-x}.

\medskip\noindent
Next, let $U = \Spec(A)$ be an affine scheme and let $U \to X$ be an
\'etale morphism. To finish the proof it suffices to show that
$F = U \times_X \Spec(k)$ is finite. Write $F = \coprod_{i \in I} \Spec(k_i)$
as the disjoint union of finite separable extensions of $k$.
We have to show that $I$ is finite.
Set $R = U \times_X U$. As $X$ is locally separated, the morphism
$j : R \to U \times U$ is an immersion. Let $U' \subset U \times U$
be an open such that $j$ factors through a closed immersion $j' : R \to U'$.
Let $e : U \to R$ be the diagonal map. Using that $e$ is a morphism between
schemes \'etale over $U$ such that $\Delta = j \circ e$ is a
closed immersion, we conclude that $R = e(U) \amalg W$ for some
open and closed subscheme $W \subset R$. Since $j'$ is a closed immersion
we conclude that $j'(W) \subset U'$ is closed and disjoint from
$j'(e(U))$. Therefore
$\overline{j(W)} \cap \Delta(U) = \emptyset$ in $U \times U$.
Note that $W$ contains $\Spec(k_i \otimes_k k_{i'})$ for all
$i \not = i'$, $i, i' \in I$. By Lemma \ref{lemma-infinite-number}
we conclude that $I$ is finite as desired.
\end{proof}








\section{Valuative criterion}
\label{section-valuative-criterion-universally-closed}

\noindent
For a quasi-compact morphism from a decent space the valuative
criterion is necessary in order for the morphism to be
universally closed.


\begin{proposition}
\label{proposition-characterize-universally-closed}
Let $S$ be a scheme. Let $f : X \to Y$ be a morphism of algebraic spaces
over $S$. Assume $f$ is quasi-compact, and $X$ is decent. Then $f$ is
universally closed if and only if the existence part of the valuative
criterion holds.
\end{proposition}

\begin{proof}
In
Morphisms of Spaces,
Lemma \ref{spaces-morphisms-lemma-quasi-compact-existence-universally-closed}
we have seen one of the implications.
To prove the other, assume that $f$ is universally closed. Let
$$
\xymatrix{
\Spec(K) \ar[r] \ar[d] & X \ar[d] \\
\Spec(A) \ar[r] & Y
}
$$
be a diagram as in
Morphisms of Spaces,
Definition \ref{spaces-morphisms-definition-valuative-criterion}.
Let $X_A = \Spec(A) \times_Y X$, so that we have
$$
\xymatrix{
\Spec(K) \ar[r] \ar[rd] & X_A \ar[d] \\
 & \Spec(A)
}
$$
By
Morphisms of Spaces,
Lemma \ref{spaces-morphisms-lemma-base-change-quasi-compact}
we see that $X_A \to \Spec(A)$ is quasi-compact. Since $X_A \to X$
is representable, we see that $X_A$ is decent also, see
Lemma \ref{lemma-representable-properties}.
Moreover, as $f$ is universally closed, we see that $X_A \to \Spec(A)$
is universally closed.
Hence we may and do replace $X$ by $X_A$ and $Y$ by $\Spec(A)$.

\medskip\noindent
Let $x' \in |X|$ be the equivalence class of
$\Spec(K) \to X$. Let $y \in |Y| = |\Spec(A)|$ be
the closed point. Set $y' = f(x')$; it is the generic point of
$\Spec(A)$. Since $f$ is universally closed we see that
$f(\overline{\{x'\}})$ contains $\overline{\{y'\}}$, and hence
contains $y$. Let $x \in \overline{\{x'\}}$ be a point such that
$f(x) = y$. Let $U$ be a scheme, and $\varphi : U \to X$
an \'etale morphism such that there exists a $u \in U$ with
$\varphi(u) = x$. By
Lemma \ref{lemma-specialization}
and our assumption that $X$ is decent
there exists a specialization $u' \leadsto u$ on $U$ with $\varphi(u') = x'$.
This means that there exists a common field extension
$K \subset K' \supset \kappa(u')$ such that
$$
\xymatrix{
\Spec(K') \ar[r] \ar[d] & U \ar[d] \\
\Spec(K) \ar[r] \ar[rd] & X \ar[d] \\
 & \Spec(A)
}
$$
is commutative. This gives the following commutative diagram of rings
$$
\xymatrix{
K' & \mathcal{O}_{U, u} \ar[l] \\
K \ar[u] & \\
 & A \ar[lu] \ar[uu]
}
$$
By
Algebra, Lemma \ref{algebra-lemma-dominate}
we can find a valuation ring $A' \subset K'$ dominating the image of
$\mathcal{O}_{U, u}$ in $K'$. Since by construction $\mathcal{O}_{U, u}$
dominates $A$ we see that $A'$ dominates $A$ also. Hence we obtain a diagram
resembling the second diagram of
Morphisms of Spaces,
Definition \ref{spaces-morphisms-definition-valuative-criterion}
and the proposition is proved.
\end{proof}







\section{Relative conditions}
\label{section-relative-conditions}

\noindent
This is a (yet another) technical section dealing with conditions on
algebraic spaces having to do with points. It is probably a good idea
to skip this section.

\begin{definition}
\label{definition-relative-conditions}
Let $S$ be a scheme. We say an algebraic space $X$ over $S$
{\it has property $(\beta)$} if $X$ has the corresponding property of
Lemma \ref{lemma-bounded-fibres}.
Let $f : X \to Y$ be a morphism of algebraic spaces over $S$.
\begin{enumerate}
\item We say $f$ {\it has property $(\beta)$} if for any scheme $T$ and
morphism $T \to Y$ the fibre product $T \times_Y X$ has property $(\beta)$.
\item We say $f$ is {\it decent} if for any scheme $T$ and
morphism $T \to Y$ the fibre product $T \times_Y X$ is a decent
algebraic space.
\item We say $f$ is {\it reasonable} if for any scheme $T$ and
morphism $T \to Y$ the fibre product $T \times_Y X$ is a reasonable
algebraic space.
\item We say $f$ is {\it very reasonable} if for any scheme $T$ and
morphism $T \to Y$ the fibre product $T \times_Y X$ is a very reasonable
algebraic space.
\end{enumerate}
\end{definition}

\noindent
We refer to Remark \ref{remark-very-reasonable} for an informal discussion.
It will turn out that the class of very reasonable morphisms is not so
useful, but that the classes of decent and reasonable morphisms are useful.

\begin{lemma}
\label{lemma-properties-trivial-implications}
Let $S$ be a scheme.
Let $f : X \to Y$ be a morphism of algebraic spaces over $S$.
We have the following implications among the conditions on $f$:
$$
\xymatrix{
\text{representable} \ar@{=>}[rd] & & & & \\
& \text{very reasonable} \ar@{=>}[r] & \text{reasonable} \ar@{=>}[r] &
\text{decent} \ar@{=>}[r] & (\beta) \\
\text{quasi-separated} \ar@{=>}[ru] & & & &
}
$$
\end{lemma}

\begin{proof}
This is clear from the definitions,
Lemma \ref{lemma-bounded-fibres}
and
Morphisms of Spaces,
Lemma \ref{spaces-morphisms-lemma-separated-local}.
\end{proof}

\noindent
Here is another sanity check.

\begin{lemma}
\label{lemma-property-for-morphism-out-of-property}
Let $S$ be a scheme. Let $f : X \to Y$ be a morphism of algebraic
spaces over $S$. If $X$ is decent (resp.\ is reasonable, resp.\ has property
$(\beta)$ of Lemma \ref{lemma-bounded-fibres}), then $f$ is
decent (resp.\ reasonable, resp.\ has property $(\beta)$).
\end{lemma}

\begin{proof}
Let $T$ be a scheme and let $T \to Y$ be a morphism. Then $T \to Y$
is representable, hence the base change $T \times_Y X \to X$ is representable.
Hence if $X$ is decent (or reasonable), then so is $T \times_Y X$, see
Lemma \ref{lemma-representable-named-properties}.
Similarly, for property $(\beta)$, see
Lemma \ref{lemma-representable-properties}.
\end{proof}

\begin{lemma}
\label{lemma-base-change-relative-conditions}
Having property $(\beta)$, being decent, or being reasonable
is preserved under arbitrary base change.
\end{lemma}

\begin{proof}
This is immediate from the definition.
\end{proof}

\begin{lemma}
\label{lemma-property-over-property}
Let $S$ be a scheme.
Let $f : X \to Y$ be a morphism of algebraic spaces over $S$.
Let $\omega \in \{\beta, decent, reasonable\}$.
Suppose that $Y$ has property $(\omega)$ and $f : X \to Y$ has $(\omega)$.
Then $X$ has $(\omega)$.
\end{lemma}

\begin{proof}
Let us prove the lemma in case $\omega = \beta$. In this case we have to show
that any $x \in |X|$ is represented by a monomorphism from the spectrum
of a field into $X$. Let $y = f(x) \in |Y|$. By assumption there exists
a field $k$ and a monomorphism $\Spec(k) \to Y$ representing $y$.
Then $x$ corresponds to a point $x'$ of $\Spec(k) \times_Y X$.
By assumption $x'$ is represented by a monomorphism
$\Spec(k') \to \Spec(k) \times_Y X$. Clearly the composition
$\Spec(k') \to X$ is a monomorphism representing $x$.

\medskip\noindent
Let us prove the lemma in case $\omega = decent$.
Let $x \in |X|$ and $y = f(x) \in |Y|$. By the result of the preceding
paragraph we can choose a diagram
$$
\xymatrix{
\Spec(k') \ar[r]_x \ar[d] & X \ar[d]^f \\
\Spec(k) \ar[r]^y & Y
}
$$
whose horizontal arrows monomorphisms. As $Y$ is decent the morphism
$y$ is quasi-compact. As $f$ is decent the algebraic space
$\Spec(k) \times_Y X$ is decent. Hence the monomorphism
$\Spec(k') \to \Spec(k) \times_Y X$ is quasi-compact.
Then the monomorphism $x : \Spec(k') \to X$ is quasi-compact
as a composition of quasi-compact morphisms (use
Morphisms of Spaces, Lemmas
\ref{spaces-morphisms-lemma-base-change-quasi-compact} and
\ref{spaces-morphisms-lemma-composition-quasi-compact}).
As the point $x$ was arbitrary this implies $X$ is decent.

\medskip\noindent
Let us prove the lemma in case $\omega = reasonable$.
Choose $V \to Y$ \'etale with $V$ an affine scheme.
Choose $U \to V \times_Y X$ \'etale with $U$ an affine scheme.
By assumption $V \to Y$ has universally bounded fibres. By
Lemma \ref{lemma-base-change-universally-bounded}
the morphism $V \times_Y X \to X$ has universally bounded fibres.
By assumption on $f$ we see that $U \to V \times_Y X$ has
universally bounded fibres. By
Lemma \ref{lemma-composition-universally-bounded}
the composition $U \to X$ has universally bounded fibres.
Hence there exists sufficiently many \'etale morphisms $U \to X$
from schemes with universally bounded fibres, and we conclude
that $X$ is reasonable.
\end{proof}

\begin{lemma}
\label{lemma-composition-relative-conditions}
Having property $(\beta)$, being decent, or being reasonable
is preserved under compositions.
\end{lemma}

\begin{proof}
Let $\omega \in \{\beta, decent, reasonable\}$.
Let $f : X \to Y$ and $g : Y \to Z$ be morphisms of algebraic spaces
over the scheme $S$. Assume $f$ and $g$ both have property
$(\omega)$. Then we have to show
that for any scheme $T$ and morphism $T \to Z$ the space $T \times_Z X$
has $(\omega)$. By
Lemma \ref{lemma-base-change-relative-conditions}
this reduces us to the following claim: Suppose that $Y$ is an algebraic
space having property $(\omega)$, and that $f : X \to Y$ is a morphism
with $(\omega)$. Then $X$ has $(\omega)$.
This is the content of Lemma \ref{lemma-property-over-property}.
\end{proof}

\begin{lemma}
\label{lemma-fibre-product-conditions}
Let $S$ be a scheme. Let $f : X \to Y$, $g : Z \to Y$ be morphisms
of algebraic spaces over $S$. If $X$ and $Z$ are decent
(resp.\ reasonable, resp.\ have property 
$(\beta)$ of Lemma \ref{lemma-bounded-fibres}), then so does $X \times_Y Z$.
\end{lemma}

\begin{proof}
Namely, by Lemma \ref{lemma-property-for-morphism-out-of-property}
the morphism $X \to Y$ has the property. Then the base change
$X \times_Y Z \to Z$ has the property by
Lemma \ref{lemma-base-change-relative-conditions}.
And finally this implies $X \times_Y Z$ has the
property by Lemma \ref{lemma-property-over-property}.
\end{proof}

\begin{lemma}
\label{lemma-descent-conditions}
Let $S$ be a scheme.
Let $f : X \to Y$ be a morphism of algebraic spaces over $S$.
Let $\mathcal{P} \in \{(\beta), decent, reasonable\}$.
Assume
\begin{enumerate}
\item $f$ is quasi-compact,
\item $f$ is \'etale,
\item $|f| : |X| \to |Y|$ is surjective, and
\item the algebraic space $X$ has property $\mathcal{P}$.
\end{enumerate}
Then $Y$ has property $\mathcal{P}$.
\end{lemma}

\begin{proof}
Let us prove this in case $\mathcal{P} = (\beta)$. Let $y \in |Y|$ be
a point. We have to show that $y$ can be represented by a monomorphism
from a field. Choose a point $x \in |X|$ with $f(x) = y$.
By assumption we may represent $x$ by a monomorphism
$\Spec(k) \to X$, with $k$ a field. By
Lemma \ref{lemma-R-finite-above-x}
it suffices to show that the projections
$\Spec(k) \times_Y \Spec(k) \to \Spec(k)$
are \'etale and quasi-compact. We can factor the first projection as
$$
\Spec(k) \times_Y \Spec(k)
\longrightarrow
\Spec(k) \times_Y X
\longrightarrow
\Spec(k)
$$
The first morphism is a monomorphism, and the second is \'etale and
quasi-compact. By
Properties of Spaces,
Lemma \ref{spaces-properties-lemma-etale-over-field-scheme}
we see that $\Spec(k) \times_Y X$ is a scheme. Hence it is a
finite disjoint union of spectra of finite separable field extensions
of $k$. By
Schemes, Lemma \ref{schemes-lemma-mono-towards-spec-field}
we see that the first arrow identifies
$\Spec(k) \times_Y \Spec(k)$ with a finite disjoint
union of spectra of finite separable field extensions of $k$.
Hence the projection morphism is \'etale and quasi-compact.

\medskip\noindent
Let us prove this in case $\mathcal{P} = decent$.
We have already seen in the first paragraph of the proof that this implies
that every $y \in |Y|$ can be represented by a monomorphism
$y : \Spec(k) \to Y$. Pick such a $y$. Pick an affine
scheme $U$ and an \'etale morphism $U \to X$ such that the image
of $|U| \to |Y|$ contains $y$. By
Lemma \ref{lemma-UR-finite-above-x}
it suffices to show that $U_y$ is a finite scheme over $k$. The fibre
product $X_y = \Spec(k) \times_Y X$ is a quasi-compact \'etale
algebraic space over $k$. Hence by
Properties of Spaces,
Lemma \ref{spaces-properties-lemma-etale-over-field-scheme}
it is a scheme. So it is a finite disjoint union of spectra of
finite separable extensions of $k$. Say $X_y = \{x_1, \ldots, x_n\}$
so $x_i$ is given by  $x_i : \Spec(k_i) \to X$ with
$[k_i : k] < \infty$. By assumption $X$ is decent, so the schemes
$U_{x_i} = \Spec(k_i) \times_X U$ are finite over $k_i$.
Finally, we note that $U_y = \coprod U_{x_i}$ as a scheme and we conclude
that $U_y$ is finite over $k$ as desired.

\medskip\noindent
Let us prove this in case $\mathcal{P} = reasonable$.
Pick an affine scheme $V$ and an \'etale morphism $V \to Y$.
We have the show the fibres of $V \to Y$ are universally bounded.
The algebraic space $V \times_Y X$ is quasi-compact.
Thus we can find an affine scheme $W$ and a surjective \'etale morphism
$W \to V \times_Y X$, see
Properties of Spaces,
Lemma \ref{spaces-properties-lemma-quasi-compact-affine-cover}.
Here is a picture (solid diagram)
$$
\xymatrix{
W \ar[r]  \ar[rd] &
V \times_Y X \ar[r] \ar[d] &
X \ar[d]_f & \Spec(k) \ar@{..>}[l]^x \ar@{..>}[ld]^y \\
 & V \ar[r] & Y
}
$$
The morphism $W \to X$ is universally bounded by our assumption that
the space $X$ is reasonable. Let $n$ be an integer bounding
the degrees of the fibres of $W \to X$. We claim that the same integer
works for bounding the fibres of $V \to Y$. Namely, suppose $y \in |Y|$
is a point. Then there exists a $x \in |X|$ with $f(x) = y$ (see above).
This means we can find a field $k$ and morphisms $x, y$ given as dotted
arrows in the diagram above. In particular we get a surjective \'etale
morphism
$$
\Spec(k) \times_{x, X} W
\to
\Spec(k) \times_{x, X} (V \times_Y X) = \Spec(k) \times_{y, Y} V
$$
which shows that the degree of $\Spec(k) \times_{y, Y} V$ over $k$
is less than or equal to the degree of $\Spec(k) \times_{x, X} W$
over $k$, i.e., $\leq n$, and we win. (This last part of the argument
is the same as the argument in the proof of
Lemma \ref{lemma-descent-universally-bounded}.
Unfortunately that lemma is not general enough because it only applies
to representable morphisms.)
\end{proof}

\begin{lemma}
\label{lemma-relative-conditions-local}
Let $S$ be a scheme.
Let $f : X \to Y$ be a morphism of algebraic spaces over $S$.
Let $\mathcal{P} \in \{(\beta), decent, reasonable, very\ reasonable\}$.
The following are equivalent
\begin{enumerate}
\item $f$ is $\mathcal{P}$,
\item for every affine scheme $Z$ and every morphism $Z \to Y$ the
base change $Z \times_Y X \to Z$ of $f$ is $\mathcal{P}$,
\item for every affine scheme $Z$ and every morphism $Z \to Y$ the
algebraic space $Z \times_Y X$ is $\mathcal{P}$, and
\item there exists a Zariski covering $Y = \bigcup Y_i$ such
that each morphism $f^{-1}(Y_i) \to Y_i$ has $\mathcal{P}$.
\end{enumerate}
If $\mathcal{P} \in \{(\beta), decent, reasonable\}$, then this is also
equivalent to
\begin{enumerate}
\item[(5)] there exists a scheme $V$ and a surjective \'etale morphism
$V \to Y$ such that the base change $V \times_Y X \to V$ has
$\mathcal{P}$.
\end{enumerate}
\end{lemma}

\begin{proof}
The implications (1) $\Rightarrow$ (2) $\Rightarrow$ (3) $\Rightarrow$ (4)
are trivial.
The implication (3) $\Rightarrow$ (1) can be seen as follows.
Let $Z \to Y$ be a morphism whose source is a scheme over $S$.
Consider the algebraic space $Z \times_Y X$. If we assume (3), then
for any affine open $W \subset Z$, the open subspace
$W \times_Y X$ of $Z \times_Y X$ has property $\mathcal{P}$. Hence by
Lemma \ref{lemma-properties-local}
the space $Z \times_Y X$ has property $\mathcal{P}$, i.e., (1) holds.
A similar argument (omitted) shows that (4) implies (1).

\medskip\noindent
The implication (1) $\Rightarrow$ (5) is trivial. Let $V \to Y$ be
an \'etale morphism from a scheme as in (5). Let $Z$ be an affine scheme,
and let $Z \to Y$ be a morphism. Consider the diagram
$$
\xymatrix{
Z \times_Y V \ar[r]_q \ar[d]_p & V \ar[d] \\
Z \ar[r] & Y
}
$$
Since $p$ is \'etale, and hence open, we can choose finitely many affine open
subschemes $W_i \subset Z \times_Y V$ such that $Z = \bigcup p(W_i)$.
Consider the commutative diagram
$$
\xymatrix{
V \times_Y X \ar[d] &
(\coprod W_i) \times_Y X \ar[l] \ar[d] \ar[r] &
Z \times_Y X \ar[d] \\
V &
\coprod W_i \ar[l] \ar[r] &
Z
}
$$
We know $V \times_Y X$ has property $\mathcal{P}$. By
Lemma \ref{lemma-representable-properties}
we see that $(\coprod W_i) \times_Y X$ has property $\mathcal{P}$.
Note that the morphism $(\coprod W_i) \times_Y X \to Z \times_Y X$
is \'etale and quasi-compact as the base change of $\coprod W_i \to Z$.
Hence by Lemma \ref{lemma-descent-conditions}
we conclude that $Z \times_Y X$ has property $\mathcal{P}$.
\end{proof}

\begin{remark}
\label{remark-very-reasonable}
An informal description of the properties $(\beta)$, decent, reasonable,
very reasonable was given in Section \ref{section-reasonable-decent}.
A morphism has one of these properties if (very) loosely speaking the
fibres of the morphism have the corresponding properties.
Being decent is useful to prove things about specializations of
points on $|X|$. Being reasonable is a bit stronger and technically
quite easy to work with.
\end{remark}

\noindent
Here is a lemma we promised earlier which uses decent morphisms.

\begin{lemma}
\label{lemma-re-characterize-universally-closed}
Let $S$ be a scheme.
Let $f : X \to Y$ be a morphism of algebraic spaces over $S$.
Assume $f$ is quasi-compact and decent.
(For example if $f$ is representable, or quasi-separated, see
Lemma \ref{lemma-properties-trivial-implications}.)
Then $f$ is universally closed if and only if the
existence part of the valuative criterion holds.
\end{lemma}

\begin{proof}
In
Morphisms of Spaces,
Lemma \ref{spaces-morphisms-lemma-quasi-compact-existence-universally-closed}
we proved that any quasi-compact morphism which satisfies the existence
part of the valuative criterion is universally closed.
To prove the other, assume that $f$ is universally closed.
In the proof of
Proposition \ref{proposition-characterize-universally-closed}
we have seen that it suffices to show, for any valuation ring $A$,
and any morphism $\Spec(A) \to Y$, that the base change
$f_A : X_A \to \Spec(A)$ satisfies the existence part of the valuative
criterion. By definition the algebraic space $X_A$ has property $(\gamma)$
and hence
Proposition \ref{proposition-characterize-universally-closed}
applies to the morphism $f_A$ and we win.
\end{proof}





\section{Points of fibres}
\label{section-points-fibres}

\noindent
Let $S$ be a scheme. Consider a cartesian diagram
\begin{equation}
\label{equation-points-fibres}
\xymatrix{
W \ar[r]_q \ar[d]_p & Z \ar[d]^g \\
X \ar[r]^f & Y
}
\end{equation}
of algebraic spaces over $S$. Let $x \in |X|$ and $z \in |Z|$
be points mapping to the same point $y \in |Y|$. We may ask:
When is the set
\begin{equation}
\label{equation-fibre}
F_{x, z} = \{ w \in |W| \text{ such that }p(w) = x\text{ and }q(w) = z\}
\end{equation}
finite?

\begin{example}
\label{example-schemes}
If $X, Y, Z$ are schemes, then the set $F_{x, z}$
is equal to the spectrum of $\kappa(x) \otimes_{\kappa(y)} \kappa(z)$
(Schemes, Lemma \ref{schemes-lemma-points-fibre-product}). Thus we
obtain a finite set if either $\kappa(y) \subset \kappa(x)$ is finite or if
$\kappa(y) \subset \kappa(z)$ is finite. In particular, this is always
the case if $g$ is quasi-finite at $z$ (Morphisms, Lemma
\ref{morphisms-lemma-residue-field-quasi-finite}).
\end{example}

\begin{example}
\label{example-not-finite}
Let $K$ be a characteristic $0$ field endowed with an automorphism
$\sigma$ of infinite order. Set $Y = \Spec(K)/\mathbf{Z}$ and
$X = \mathbf{A}^1_K/\mathbf{Z}$ where $\mathbf{Z}$ acts on $K$ via $\sigma$
and on $\mathbf{A}^1_K = \Spec(K[t])$ via $t \mapsto t + 1$.
Let $Z = \Spec(K)$. Then $W = \mathbf{A}^1_K$. Picture
$$
\xymatrix{
\mathbf{A}^1_K \ar[r]_q \ar[d]_p & \Spec(K) \ar[d]^g \\
\mathbf{A}^1_K/\mathbf{Z} \ar[r]^f & \Spec(K)/\mathbf{Z}
}
$$
Take $x$ corresponding to $t = 0$ and $z$ the unique point of $\Spec(K)$.
Then we see that $F_{x, z} = \mathbf{Z}$ as a set.
\end{example}

\begin{lemma}
\label{lemma-surjective-on-fibres}
In the situation of (\ref{equation-points-fibres}) if $Z' \to Z$
is a morphism and $z' \in |Z'|$ maps to $z$, then the induced map
$F_{x, z'} \to F_{x, z}$ is surjective.
\end{lemma}

\begin{proof}
Set $W' = X \times_Y Z' = W \times_Z Z'$. Then
$|W'| \to |W| \times_{|Z|} |Z'|$ is surjective by
Properties of Spaces, Lemma \ref{spaces-properties-lemma-points-cartesian}.
Hence the surjectivity of $F_{x, z'} \to F_{x, z}$.
\end{proof}

\begin{lemma}
\label{lemma-qf-and-qc-finite-fibre}
In diagram (\ref{equation-points-fibres}) the set (\ref{equation-fibre})
is finite if $f$ is of finite type and $f$ is quasi-finite at $x$.
\end{lemma}

\begin{proof}
The morphism $q$ is quasi-finite at every $w \in F_{x, z}$, see
Morphisms of Spaces, Lemma
\ref{spaces-morphisms-lemma-base-change-quasi-finite-locus}.
Hence the lemma follows from
Morphisms of Spaces, Lemma
\ref{spaces-morphisms-lemma-quasi-finite-at-a-finite-number-of-points}.
\end{proof}

\begin{lemma}
\label{lemma-decent-finite-fibre}
In diagram (\ref{equation-points-fibres}) the set (\ref{equation-fibre})
is finite if $y$ can be represented by a monomorphism $\Spec(k) \to Y$
where $k$ is a field and $g$ is quasi-finite at $z$.
(Special case: $Y$ is decent and $g$ is \'etale.)
\end{lemma}

\begin{proof}
By Lemma \ref{lemma-surjective-on-fibres} applied twice
we may replace $Z$ by $Z_k = \Spec(k) \times_Y Z$ and
$X$ by $X_k = \Spec(k) \times_Y X$. We may and do
replace $Y$ by $\Spec(k)$ as well. Note that $Z_k \to \Spec(k)$
is quasi-finite at $z$ by Morphisms of Spaces, Lemma
\ref{spaces-morphisms-lemma-base-change-quasi-finite-locus}.
Choose a scheme $V$, a point $v \in V$, and an \'etale morphism
$V \to Z_k$ mapping $v$ to $z$. Choose a scheme $U$, a point $u \in U$,
and an \'etale morphism $U \to X_k$ mapping $u$ to $x$.
Again by Lemma \ref{lemma-surjective-on-fibres}
it suffices to show $F_{u, v}$ is finite for the diagram
$$
\xymatrix{
U \times_{\Spec(k)} V \ar[r] \ar[d] & V \ar[d] \\
U \ar[r] & \Spec(k)
}
$$
The morphism $V \to \Spec(k)$ is quasi-finite at $v$
(follows from the general discussion in
Morphisms of Spaces, Section \ref{spaces-morphisms-section-local-source-target}
and the definition of being quasi-finite at a point).
At this point the finiteness follows from Example \ref{example-schemes}.
The parenthetical remark of the statement of the lemma follows
from the fact that on decent spaces points are represented by
monomorphisms from fields and from the fact that an \'etale
morphism of algebraic spaces is locally quasi-finite.
\end{proof}

\begin{lemma}
\label{lemma-topology-fibre}
\begin{slogan}
Fibers of field points of algebraic spaces have the
expected Zariski topologies.
\end{slogan}
Let $S$ be a scheme. Let $f : X \to Y$ be a morphism of algebraic spaces
over $S$.
Let $y \in |Y|$ and assume that $y$ is represented by a quasi-compact
monomorphism $\Spec(k) \to Y$. Then $|X_k| \to |X|$ is a
homeomorphism onto $f^{-1}(\{y\}) \subset |X|$ with induced topology.
\end{lemma}

\begin{proof}
We will use
Properties of Spaces, Lemma \ref{spaces-properties-lemma-etale-open}
and
Morphisms of Spaces, Lemma
\ref{spaces-morphisms-lemma-monomorphism-injective-points}
without further mention.
Let $V \to Y$ be an \'etale morphism with $V$ affine such that there
exists a $v \in V$ mapping to $y$. Since $\Spec(k) \to Y$ is quasi-compact
there are a finite number of points of $V$ mapping to $y$
(Lemma \ref{lemma-UR-finite-above-x}). After shrinking
$V$ we may assume $v$ is the only one. Choose a scheme $U$ and
a surjective \'etale morphism $U \to X$.
Consider the commutative diagram
$$
\xymatrix{
U \ar[d] & U_V \ar[l] \ar[d] & U_v \ar[l] \ar[d] \\
X \ar[d] & X_V \ar[l] \ar[d] & X_v \ar[l] \ar[d] \\
Y & V \ar[l] & v \ar[l]
}
$$
Since $U_v \to U_V$ identifies $U_v$ with a subset of $U_V$ with
the induced topology (Schemes, Lemma \ref{schemes-lemma-fibre-topological}),
and since $|U_V| \to |X_V|$ and $|U_v| \to |X_v|$ are surjective and open,
we see that $|X_v| \to |X_V|$ is a homeomorphism onto its image (with
induced topology).
On the other hand, the inverse image of $f^{-1}(\{y\})$
under the open map $|X_V| \to |X|$ is equal to $|X_v|$.
We conclude that $|X_v| \to f^{-1}(\{y\})$ is open.
The morphism $X_v \to X$ factors through $X_k$
and $|X_k| \to |X|$ is injective with image $f^{-1}(\{y\})$
by Properties of Spaces, Lemma
\ref{spaces-properties-lemma-points-cartesian}. Using
$|X_v| \to |X_k| \to f^{-1}(\{y\})$ the lemma follows because
$X_v \to X_k$ is surjective.
\end{proof}

\begin{lemma}
\label{lemma-conditions-on-point-on-space-over-field}
Let $X$ be an algebraic space locally of finite type over a field $k$.
Let $x \in |X|$. Consider the conditions
\begin{enumerate}
\item $\dim_x(|X|) = 0$,
\item $x$ is closed in $|X|$ and if $x' \leadsto x$  in $|X|$ then $x' = x$,
\item $x$ is an isolated point of $|X|$,
\item $\dim_x(X) = 0$,
\item $X \to \Spec(k)$ is quasi-finite at $x$.
\end{enumerate}
Then (2), (3), (4), and (5) are equivalent.
If $X$ is decent, then (1) is equivalent to the others.
\end{lemma}

\begin{proof}
Parts (4) and (5) are equivalent for example by
Morphisms of Spaces, Lemmas
\ref{spaces-morphisms-lemma-locally-finite-type-quasi-finite-part} and
\ref{spaces-morphisms-lemma-quasi-finite-at-point}.

\medskip\noindent
Let $U \to X$ be an \'etale morphism where $U$ is an affine scheme and let
$u \in U$ be a point mapping to $x$. Moreover, if $x$ is a closed
point, e.g., in case (2) or (3), then we may and do assume that $u$
is a closed point. Observe that $\dim_u(U) = \dim_x(X)$ by definition
and that this is equal to $\dim(\mathcal{O}_{U, u})$ if $u$ is a closed
point, see Algebra, Lemma
\ref{algebra-lemma-dimension-closed-point-finite-type-field}.

\medskip\noindent
If $\dim_x(X) > 0$ and $u$ is closed, by the arguments above
we can choose a nontrivial
specialization $u' \leadsto u$ in $U$. Then the transcendence degree
of $\kappa(u')$ over $k$ exceeds the transcendence degree of
$\kappa(u)$ over $k$. It follows that the images $x$ and $x'$ in $X$
are distinct, because the transcendence degree of $x/k$ and $x'/k$
are well defined, see Morphisms of Spaces, Definition
\ref{spaces-morphisms-definition-dimension-fibre}.
This applies in particular in cases (2) and (3) and we
conclude that (2) and (3) imply (4).

\medskip\noindent
Conversely, if $X \to \Spec(k)$ is locally quasi-finite at $x$, then
$U \to \Spec(k)$ is locally quasi-finite at $u$, hence $u$ is an
isolated point of $U$
(Morphisms, Lemma \ref{morphisms-lemma-quasi-finite-at-point-characterize}).
It follows that (5) implies (2) and (3) as
$|U| \to |X|$ is continuous and open.

\medskip\noindent
Assume $X$ is decent and (1) holds. Then $\dim_x(X) = \dim_x(|X|)$
by Lemma \ref{lemma-dimension-decent-space} and the proof is complete.
\end{proof}

\begin{lemma}
\label{lemma-conditions-on-space-over-field}
Let $X$ be an algebraic space locally of finite type over a field $k$.
Consider the conditions
\begin{enumerate}
\item $|X|$ is a finite set,
\item $|X|$ is a discrete space,
\item $\dim(|X|) = 0$,
\item $\dim(X) = 0$,
\item $X \to \Spec(k)$ is locally quasi-finite,
\end{enumerate}
Then (2), (3), (4), and (5) are equivalent.
If $X$ is decent, then (1) implies the others.
\end{lemma}

\begin{proof}
Parts (4) and (5) are equivalent for example by
Morphisms of Spaces, Lemma
\ref{spaces-morphisms-lemma-locally-finite-type-quasi-finite-part}.

\medskip\noindent
Let $U \to X$ be a surjective \'etale morphism where $U$ is a scheme.

\medskip\noindent
If $\dim(U) > 0$, then choose a nontrivial specialization
$u \leadsto u'$ in $U$ and the transcendence degree of $\kappa(u)$
over $k$ exceeds the transcendence degree of $\kappa(u')$ over $k$.
It follows that the images $x$ and $x'$ in $X$ are distinct, because
the transcendence degree of $x/k$ and $x'/k$ is well defined, see
Morphisms of Spaces, Definition
\ref{spaces-morphisms-definition-dimension-fibre}.
We conclude that (2) and (3) imply (4).

\medskip\noindent
Conversely, if $X \to \Spec(k)$ is locally quasi-finite, then $U$ is
locally Noetherian
(Morphisms, Lemma \ref{morphisms-lemma-finite-type-noetherian})
of dimension $0$
(Morphisms, Lemma \ref{morphisms-lemma-locally-quasi-finite-rel-dimension-0})
and hence is a disjoint union of spectra of Artinian local rings
(Properties, Lemma \ref{properties-lemma-locally-Noetherian-dimension-0}).
Hence $U$ is a discrete topological space, and since $|U| \to |X|$
is continuous and open, the same is true for $|X|$.
In other words, (4) implies (2) and (3).

\medskip\noindent
Assume $X$ is decent and (1) holds. Then we may choose $U$ above to
be affine. The fibres of $|U| \to |X|$ are finite (this is a part of the
defining property of decent spaces). Hence $U$ is a finite type scheme
over $k$ with finitely many points. Hence $U$ is quasi-finite over $k$
(Morphisms, Lemma \ref{morphisms-lemma-finite-fibre})
which by definition means that $X \to \Spec(k)$ is locally quasi-finite.
\end{proof}

\begin{lemma}
\label{lemma-conditions-on-point-in-fibre-and-qf}
Let $S$ be a scheme. Let $f : X \to Y$ be a morphism of algebraic spaces
over $S$ which is locally of finite type. Let $x \in |X|$ with image
$y \in |Y|$. Let $F = f^{-1}(\{y\})$ with induced topology from $|X|$.
Let $k$ be a field and let $\Spec(k) \to Y$ be in the
equivalence class defining $y$. Set $X_k = \Spec(k) \times_Y X$.
Let $\tilde x \in |X_k|$ map to $x \in |X|$.
Consider the following conditions
\begin{enumerate}
\item
\label{item-fibre-at-x-dim-0}
$\dim_x(F) = 0$,
\item
\label{item-isolated-in-fibre}
$x$ is isolated in $F$,
\item
\label{item-no-specializations-in-fibre}
$x$ is closed in $F$ and if $x' \leadsto x$ in $F$, then $x = x'$,
\item
\label{item-dimension-top-k-fibre}
$\dim_{\tilde x}(|X_k|) = 0$,
\item
\label{item-isolated-in-k-fibre}
$\tilde x$ is isolated in $|X_k|$,
\item
\label{item-no-specializations-in-k-fibre}
$\tilde x$ is closed in $|X_k|$ and if $\tilde x' \leadsto \tilde x$
in $|X_k|$, then $\tilde x = \tilde x'$,
\item
\label{item-k-fibre-at-x-dim-0}
$\dim_{\tilde x}(X_k) = 0$,
\item
\label{item-quasi-finite-at-x}
$f$ is quasi-finite at $x$.
\end{enumerate}
Then we have
$$
\xymatrix{
(\ref{item-dimension-top-k-fibre}) \ar@{=>}[r]_{f\text{ decent}} &
(\ref{item-isolated-in-k-fibre}) \ar@{<=>}[r] &
(\ref{item-no-specializations-in-k-fibre}) \ar@{<=>}[r] &
(\ref{item-k-fibre-at-x-dim-0}) \ar@{<=>}[r] &
(\ref{item-quasi-finite-at-x})
}
$$
If $Y$ is decent, then conditions (\ref{item-isolated-in-fibre}) and
(\ref{item-no-specializations-in-fibre}) are equivalent to each other
and to conditions
(\ref{item-isolated-in-k-fibre}),
(\ref{item-no-specializations-in-k-fibre}),
(\ref{item-k-fibre-at-x-dim-0}), and
(\ref{item-quasi-finite-at-x}).
If $Y$ and $X$ are decent, then all conditions are equivalent.
\end{lemma}

\begin{proof}
By Lemma \ref{lemma-conditions-on-point-on-space-over-field} conditions
(\ref{item-isolated-in-k-fibre}),
(\ref{item-no-specializations-in-k-fibre}), and (\ref{item-k-fibre-at-x-dim-0})
are equivalent to each other and to the condition that
$X_k \to \Spec(k)$ is quasi-finite at $\tilde x$.
Thus by Morphisms of Spaces, Lemma
\ref{spaces-morphisms-lemma-base-change-quasi-finite-locus}
they are also equivalent to (\ref{item-quasi-finite-at-x}).
If $f$ is decent, then $X_k$ is a decent algebraic space and
Lemma \ref{lemma-conditions-on-point-on-space-over-field}
shows that (\ref{item-dimension-top-k-fibre}) implies
(\ref{item-isolated-in-k-fibre}).

\medskip\noindent
If $Y$ is decent, then we can pick a quasi-compact monomorphism
$\Spec(k') \to Y$ in the equivalence class of $y$. In this case
Lemma \ref{lemma-topology-fibre}
tells us that $|X_{k'}| \to F$ is a homeomorphism.
Combined with the arguments given above this implies
the remaining statements of the lemma; details omitted.
\end{proof}

\begin{lemma}
\label{lemma-conditions-on-fibre-and-qf}
Let $S$ be a scheme. Let $f : X \to Y$ be a morphism of algebraic spaces
over $S$ which is locally of finite type. Let $y \in |Y|$. Let $k$ be a field
and let $\Spec(k) \to Y$ be in the equivalence class defining $y$.
Set $X_k = \Spec(k) \times_Y X$ and let $F = f^{-1}(\{y\})$ with the
induced topology from $|X|$. Consider the following conditions
\begin{enumerate}
\item
\label{item-fibre-finite}
$F$ is finite,
\item
\label{item-fibre-discrete}
$F$ is a discrete topological space,
\item
\label{item-fibre-no-specializations}
$\dim(F) = 0$,
\item
\label{item-k-fibre-finite}
$|X_k|$ is a finite set,
\item
\label{item-k-fibre-discrete}
$|X_k|$ is a discrete space,
\item
\label{item-k-fibre-no-specializations}
$\dim(|X_k|) = 0$,
\item
\label{item-k-fibre-dim-0}
$\dim(X_k) = 0$,
\item
\label{item-quasi-finite-at-points-fibre}
$f$ is quasi-finite at all points of $|X|$ lying over $y$.
\end{enumerate}
Then we have
$$
\xymatrix{
(\ref{item-fibre-finite}) &
(\ref{item-k-fibre-finite}) \ar@{=>}[l] \ar@{=>}[r]_{f\text{ decent}} &
(\ref{item-k-fibre-discrete}) \ar@{<=>}[r] &
(\ref{item-k-fibre-no-specializations}) \ar@{<=>}[r] &
(\ref{item-k-fibre-dim-0}) \ar@{<=>}[r] &
(\ref{item-quasi-finite-at-points-fibre})
}
$$
If $Y$ is decent, then conditions (\ref{item-fibre-discrete}) and
(\ref{item-fibre-no-specializations})
are equivalent to each other and to conditions (\ref{item-k-fibre-discrete}),
(\ref{item-k-fibre-no-specializations}), (\ref{item-k-fibre-dim-0}), and
(\ref{item-quasi-finite-at-points-fibre}).
If $Y$ and $X$ are decent, then (\ref{item-fibre-finite}) implies
all the other conditions.
\end{lemma}

\begin{proof}
By Lemma \ref{lemma-conditions-on-space-over-field}
conditions  (\ref{item-k-fibre-discrete}),
(\ref{item-k-fibre-no-specializations}), and (\ref{item-k-fibre-dim-0})
are equivalent to each other and to the condition that
$X_k \to \Spec(k)$ is locally quasi-finite.
Thus by Morphisms of Spaces, Lemma
\ref{spaces-morphisms-lemma-base-change-quasi-finite-locus}
they are also equivalent to (\ref{item-quasi-finite-at-points-fibre}).
If $f$ is decent, then $X_k$ is a decent algebraic space and
Lemma \ref{lemma-conditions-on-space-over-field}
shows that (\ref{item-k-fibre-finite}) implies (\ref{item-k-fibre-discrete}).

\medskip\noindent
The map $|X_k| \to F$ is surjective by
Properties of Spaces, Lemma \ref{spaces-properties-lemma-points-cartesian}
and we see
(\ref{item-k-fibre-finite}) $\Rightarrow$ (\ref{item-fibre-finite}).

\medskip\noindent
If $Y$ is decent, then we can pick a quasi-compact monomorphism
$\Spec(k') \to Y$ in the equivalence class of $y$. In this case
Lemma \ref{lemma-topology-fibre}
tells us that $|X_{k'}| \to F$ is a homeomorphism.
Combined with the arguments given above this implies
the remaining statements of the lemma; details omitted.
\end{proof}







\section{Monomorphisms}
\label{section-monomorphisms}

\noindent
Here is another case where monomorphisms are representable.
Please see More on Morphisms of Spaces, Section
\ref{spaces-more-morphisms-section-monomorphisms}
for more information.

\begin{lemma}
\label{lemma-monomorphism-toward-disjoint-union-dim-0-rings}
Let $S$ be a scheme. Let $Y$ be a disjoint union of spectra of
zero dimensional local rings over $S$.
Let $f : X \to Y$ be a monomorphism of algebraic spaces over $S$.
Then $f$ is representable, i.e., $X$ is a scheme.
\end{lemma}

\begin{proof}
This immediately reduces to the case $Y = \Spec(A)$ where
$A$ is a zero dimensional local ring, i.e.,
$\Spec(A) = \{\mathfrak m_A\}$
is a singleton. If $X = \emptyset$, then there is nothing to prove.
If not, choose a nonempty affine scheme $U = \Spec(B)$
and an \'etale morphism $U \to X$. As $|X|$ is a singleton (as a
subset of $|Y|$, see
Morphisms of Spaces, Lemma
\ref{spaces-morphisms-lemma-monomorphism-injective-points})
we see that $U \to X$ is surjective. Note that
$U \times_X U = U \times_Y U = \Spec(B \otimes_A B)$.
Thus we see that the ring maps $B \to B \otimes_A B$ are \'etale.
Since
$$
(B \otimes_A B)/\mathfrak m_A(B \otimes_A B)
=
(B/\mathfrak m_AB) \otimes_{A/\mathfrak m_A} (B/\mathfrak m_AB)
$$
we see that
$B/\mathfrak m_AB \to (B \otimes_A B)/\mathfrak m_A(B \otimes_A B)$
is flat and in fact free of rank equal to the dimension of
$B/\mathfrak m_AB$ as a $A/\mathfrak m_A$-vector space. Since
$B \to B \otimes_A B$ is \'etale, this can only happen if this
dimension is finite (see for example
Morphisms, Lemmas \ref{morphisms-lemma-etale-universally-bounded} and
\ref{morphisms-lemma-locally-quasi-finite-qc-source-universally-bounded}).
Every prime of $B$ lies over $\mathfrak m_A$ (the unique prime of $A$).
Hence $\Spec(B) = \Spec(B/\mathfrak m_A)$ as a topological
space, and this space is a finite discrete set as $B/\mathfrak m_A B$
is an Artinian ring, see
Algebra, Lemmas \ref{algebra-lemma-finite-dimensional-algebra} and
\ref{algebra-lemma-artinian-finite-length}.
Hence all prime ideals of $B$ are maximal and
$B = B_1 \times \ldots \times B_n$ is a product of finitely many
local rings of dimension zero, see
Algebra, Lemma \ref{algebra-lemma-product-local}.
Thus $B \to B \otimes_A B$ is finite \'etale as all the local rings
$B_i$ are henselian by
Algebra, Lemma \ref{algebra-lemma-local-dimension-zero-henselian}.
Thus $X$ is an affine scheme by
Groupoids, Proposition \ref{groupoids-proposition-finite-flat-equivalence}.
\end{proof}








\section{Generic points}
\label{section-generic-points}

\noindent
This section is a continuation of
Properties of Spaces, Section \ref{spaces-properties-section-generic-points}.

\begin{lemma}
\label{lemma-decent-generic-points}
Let $S$ be a scheme. Let $X$ be a decent algebraic space over $S$.
Let $x \in |X|$. The following are equivalent
\begin{enumerate}
\item $x$ is a generic point of an irreducible component of $|X|$,
\item for any \'etale morphism $(Y, y) \to (X, x)$ of pointed algebraic
spaces, $y$ is a generic point of an irreducible component of $|Y|$,
\item for some \'etale morphism $(Y, y) \to (X, x)$ of pointed algebraic
spaces, $y$ is a generic point of an irreducible component of $|Y|$,
\item the dimension of the local ring of $X$ at $x$ is zero, and
\item $x$ is a point of codimension $0$ on $X$
\end{enumerate}
\end{lemma}

\begin{proof}
Conditions (4) and (5) are equivalent for any algebraic space
by definition, see Properties of Spaces, Definition
\ref{spaces-properties-definition-dimension-local-ring}.
Observe that any $Y$ as in (2) and (3) is decent by
Lemma \ref{lemma-etale-named-properties}.
Thus it suffices to prove the equivalence of (1) and (4)
as then the equivalence with (2) and (3) follows since the dimension
of the local ring of $Y$ at $y$ is equal to the dimension
of the local ring of $X$ at $x$.
Let $f : U \to X$ be an \'etale morphism from an affine scheme and let
$u \in U$ be a point mapping to $x$.

\medskip\noindent
Assume (1). Let $u' \leadsto u$ be a specialization in $U$.
Then $f(u') = f(u) = x$. By
Lemma \ref{lemma-decent-no-specializations-map-to-same-point}
we see that $u' = u$. Hence $u$ is a generic point of an irreducible component
of $U$. Thus $\dim(\mathcal{O}_{U, u}) = 0$ and we see that (4) holds.

\medskip\noindent
Assume (4). The point $x$ is contained in an irreducible component
$T \subset |X|$. Since $|X|$ is sober
(Proposition \ref{proposition-reasonable-sober})
we $T$ has a generic point $x'$. Of course $x' \leadsto x$.
Then we can lift this specialization to $u' \leadsto u$ in $U$
(Lemma \ref{lemma-decent-specialization}). This contradicts the assumption
that $\dim(\mathcal{O}_{U, u}) = 0$ unless $u' = u$, i.e., $x' = x$.
\end{proof}

\begin{lemma}
\label{lemma-codimension-local-ring}
Let $S$ be a scheme. Let $X$ be a decent algebraic space over $S$.
Let $T \subset |X|$ be an irreducible closed subset. Let $\xi \in T$
be the generic point (Proposition \ref{proposition-reasonable-sober}).
Then $\text{codim}(T, |X|)$
(Topology, Definition \ref{topology-definition-codimension})
is the dimension of the local ring of $X$ at $\xi$
(Properties of Spaces, Definition
\ref{spaces-properties-definition-dimension-local-ring}).
\end{lemma}

\begin{proof}
Choose a scheme $U$, a point $u \in U$, and an \'etale morphism
$U \to X$ sending $u$ to $\xi$. Then any sequence of nontrivial
specializations $\xi_e \leadsto \ldots \leadsto \xi_0 = \xi$
can be lifted to a sequence $u_e \leadsto \ldots \leadsto u_0 = u$ in $U$
by Lemma \ref{lemma-decent-specialization}.
Conversely, any sequence of nontrivial specializations
$u_e \leadsto \ldots \leadsto u_0 = u$ in $U$
maps to a sequence of nontrivial specializations
$\xi_e \leadsto \ldots \leadsto \xi_0 = \xi$ by
Lemma \ref{lemma-decent-no-specializations-map-to-same-point}.
Because $|X|$ and $U$ are sober topological spaces
we conclude that the codimension of $T$ in $|X|$
and of $\overline{\{u\}}$ in $U$ are the same.
In this way the lemma reduces to the schemes case which
is Properties, Lemma \ref{properties-lemma-codimension-local-ring}.
\end{proof}

\begin{lemma}
\label{lemma-get-reasonable}
Let $S$ be a scheme. Let $X$ be an algebraic space over $S$. Assume
\begin{enumerate}
\item every quasi-compact scheme \'etale over $X$ has finitely many
irreducible components, and
\item every $x \in |X|$ of codimension $0$ on $X$ can be represented
by a monomorphism $\Spec(k) \to X$.
\end{enumerate}
Then $X$ is a reasonable algebraic space.
\end{lemma}

\begin{proof}
Let $U$ be an affine scheme and let $a : U \to X$ be an \'etale morphism.
We have to show that the fibres of $a$ are universally bounded. By
assumption (1) the scheme $U$ has finitely many irreducible components.
Let $u_1, \ldots, u_n \in U$ be the generic points of these irreducible
components. Let $\{x_1, \ldots, x_m\} \subset |X|$ be the image
of $\{u_1, \ldots, u_n\}$. Each $x_j$ is a point of codimension $0$.
By assumption (2) we may choose a monomorphism $\Spec(k_j) \to X$
representing $x_j$. By Properties of Spaces, Lemma
\ref{spaces-properties-lemma-codimension-0-points} we have
$$
U \times_X \Spec(k_j) = \coprod\nolimits_{a(u_i) = x_j} \Spec(\kappa(u_i))
$$
This is a scheme finite over $\Spec(k_j)$ of degree
$d_j = \sum_{a(u_i) = x_j} [\kappa(u_i) : k_j]$. Set $n = \max d_j$.

\medskip\noindent
Observe that $a$ is separated
(Properties of Spaces, Lemma \ref{spaces-properties-lemma-separated-cover}).
Consider the stratification
$$
X = X_0 \supset X_1 \supset X_2 \supset \ldots
$$
associated to $U \to X$ in Lemma \ref{lemma-stratify-flat-fp-lqf}.
By our choice of $n$ above we conclude that $X_{n + 1}$ is empty.
Namely, if not, then $a^{-1}(X_{n + 1})$ is a nonempty open
of $U$ and hence would contain one of the $x_i$. This would mean
that $X_{n + 1}$ contains $x_j = a(u_i)$ which is impossible.
Hence we see that the fibres of $U \to X$ are universally bounded
(in fact by the integer $n$).
\end{proof}

\begin{lemma}
\label{lemma-finitely-many-irreducible-components}
Let $S$ be a scheme. Let $X$ be an algebraic space over $S$.
The following are equivalent
\begin{enumerate}
\item $X$ is decent and $|X|$ has finitely many irreducible components,
\item every quasi-compact scheme \'etale over $X$ has finitely many
irreducible components, there are finitely many $x \in |X|$ of
codimension $0$ on $X$, and each of these can be represented
by a monomorphism $\Spec(k) \to X$,
\item there exists a dense open $X' \subset X$ which is
a scheme, $X'$ has finitely many irreducible components
with generic points $\{x'_1, \ldots, x'_m\}$, and
the morphism $x'_j \to X$ is quasi-compact for $j = 1, \ldots, m$.
\end{enumerate}
Moreover, if these conditions hold, then $X$ is reasonable and the
points $x'_j \in |X|$ are the generic points of the irreducible
components of $|X|$.
\end{lemma}

\begin{proof}
In the proof we use Properties of Spaces, Lemma
\ref{spaces-properties-lemma-codimension-0-points}
without further mention.
Assume (1). Then $X$ has a dense open subscheme $X'$ by
Theorem \ref{theorem-decent-open-dense-scheme}.
Since the closure of an irreducible component of $|X'|$
is an irreducible component of $|X|$, we see that $|X'|$
has finitely many irreducible components. Thus (3) holds.

\medskip\noindent
Assume $X' \subset X$ is as in (3). Let $\{x'_1, \ldots, x'_m\}$
be the generic points of the irreducible components of $X'$.
Let $a : U \to X$ be an \'etale morphism with $U$ a quasi-compact scheme.
To prove (2) it suffices to show that $U$ has
finitely many irreducible components
whose generic points lie over $\{x'_1, \ldots, x'_m\}$. It suffices
to prove this for the members of a finite affine open cover of $U$,
hence we may and do assume $U$ is affine.
Note that $U' = a^{-1}(X') \subset U$ is a dense open.
Since $U' \to X'$ is an \'etale morphism of schemes, we see
the generic points of irreducible components of $U'$ are the points
lying over $\{x'_1, \ldots, x'_m\}$. Since $x'_j \to X$ is
quasi-compact there are finitely many points of $U$ lying over $x'_j$
(Lemma \ref{lemma-UR-finite-above-x}). Hence $U'$ has finitely
many irreducible components, which implies that the closures
of these irreducible components are the irreducible components of
$U$. Thus (2) holds.

\medskip\noindent
Assume (2). This implies (1) and the final
statement by Lemma \ref{lemma-get-reasonable}.
(We also use that a reasonable algebraic space is decent, see
discussion following Definition \ref{definition-very-reasonable}.)
\end{proof}



\section{Generically finite morphisms}
\label{section-generically-finite}

\noindent
This section discusses for morphisms of algebraic spaces the material
discussed in Morphisms, Section \ref{morphisms-section-generically-finite}
and
Varieties, Section \ref{varieties-section-generically-finite}
for morphisms of schemes.

\begin{lemma}
\label{lemma-generically-finite}
Let $S$ be a scheme. Let $f : X \to Y$ be a morphism of algebraic spaces
over $S$. Assume that $f$ is quasi-separated of finite type.
Let $y \in |Y|$ be a point of codimension $0$ on $Y$.
The following are equivalent:
\begin{enumerate}
\item the space $|X_k|$ is finite where $\Spec(k) \to Y$ represents $y$,
\item $X \to Y$ is quasi-finite at all points of $|X|$ over $y$,
\item there exists an open subspace $Y' \subset Y$ with $y \in |Y'|$
such that $Y' \times_Y X \to Y'$ is finite.
\end{enumerate}
If $Y$ is decent these are also equivalent to
\begin{enumerate}
\item[(4)] the set $f^{-1}(\{y\})$ is finite.
\end{enumerate}
\end{lemma}

\begin{proof}
The equivalence of (1) and (2) follows from
Lemma \ref{lemma-conditions-on-fibre-and-qf}
(and the fact that a quasi-separated morphism is decent by
Lemma \ref{lemma-properties-trivial-implications}).

\medskip\noindent
Assume the equivalent conditions of (1) and (2). Choose an affine scheme $V$
and an \'etale morphism $V \to Y$ mapping a point $v \in V$ to $y$. Then $v$
is a generic point of an irreducible component of $V$ by
Properties of Spaces, Lemma
\ref{spaces-properties-lemma-codimension-0-points}.
Choose an affine scheme $U$
and a surjective \'etale morphism $U \to V \times_Y X$. Then $U \to V$ is of
finite type. The morphism $U \to V$ is quasi-finite at every point lying over
$v$ by (2). It follows that the fibre of $U \to V$ over $v$ is finite
(Morphisms, Lemma
\ref{morphisms-lemma-quasi-finite-at-a-finite-number-of-points}). By
Morphisms, Lemma \ref{morphisms-lemma-generically-finite}
after shrinking $V$ we may assume that $U \to V$ is finite.
Let
$$
R = U \times_{V \times_Y X} U
$$
Since $f$ is quasi-separated, we see that $V \times_Y X$ is quasi-separated
and hence $R$ is a quasi-compact scheme. Moreover the morphisms
$R \to V$ is quasi-finite as the composition of an \'etale morphism
$R \to U$ and a finite morphism $U \to V$. Hence we may apply
Morphisms, Lemma \ref{morphisms-lemma-generically-finite}
once more and after shrinking $V$ we may assume that $R \to V$ is
finite as well. This of course implies that the two projections
$R \to V$ are finite \'etale. It follows that
$V/R = V \times_Y X$ is an affine scheme, see
Groupoids, Proposition \ref{groupoids-proposition-finite-flat-equivalence}.
By Morphisms, Lemma \ref{morphisms-lemma-image-proper-is-proper}
we conclude that $V \times_Y X \to V$ is proper and by
Morphisms, Lemma \ref{morphisms-lemma-finite-proper}
we conclude that $V \times_Y X \to V$ is finite.
Finally, we let $Y' \subset Y$ be the open subspace of $Y$
corresponding to the image of $|V| \to |Y|$.
By Morphisms of Spaces, Lemma \ref{spaces-morphisms-lemma-integral-local}
we conclude that $Y' \times_Y X \to Y'$ is finite as the base
change to $V$ is finite and as $V \to Y'$ is a surjective \'etale
morphism.

\medskip\noindent
If $Y$ is decent and $f$ is quasi-separated, then we see that
$X$ is decent too; use Lemmas
\ref{lemma-properties-trivial-implications} and
\ref{lemma-property-over-property}.
Hence Lemma \ref{lemma-conditions-on-fibre-and-qf}
applies to show that (4) implies (1) and (2). On the other hand,
we see that (2) implies (4) by Morphisms of Spaces, Lemma
\ref{spaces-morphisms-lemma-quasi-finite-at-a-finite-number-of-points}.
\end{proof}

\begin{lemma}
\label{lemma-generically-finite-reprise}
Let $S$ be a scheme. Let $f : X \to Y$ be a morphism of algebraic spaces
over $S$. Assume that $f$ is quasi-separated and locally of finite type
and $Y$ quasi-separated. Let $y \in |Y|$ be a point of codimension $0$ on $Y$.
The following are equivalent:
\begin{enumerate}
\item the set $f^{-1}(\{y\})$ is finite,
\item the space $|X_k|$ is finite where $\Spec(k) \to Y$ represents $y$,
\item there exist open subspaces $X' \subset X$ and $Y' \subset Y$
with $f(X') \subset Y'$, $y \in |Y'|$, and $f^{-1}(\{y\}) \subset |X'|$
such that $f|_{X'} : X' \to Y'$ is finite.
\end{enumerate}
\end{lemma}

\begin{proof}
Since quasi-separated algebraic spaces are decent, the equivalence
of (1) and (2) follows from
Lemma \ref{lemma-conditions-on-fibre-and-qf}.
To prove that (1) and (2) imply (3)
we may and do replace $Y$ by a quasi-compact open containing $y$.
Since $f^{-1}(\{y\})$ is finite, we can find a quasi-compact
open subspace of $X' \subset X$ containing the fibre.
The restriction $f|_{X'} : X' \to Y$ is quasi-compact and quasi-separated
by Morphisms of Spaces, Lemma
\ref{spaces-morphisms-lemma-quasi-compact-quasi-separated-permanence}
(this is where we use that $Y$ is quasi-separated).
Applying Lemma \ref{lemma-generically-finite}
to $f|_{X'} : X' \to Y$ we see that (3) holds.
We omit the proof that (3) implies (2).
\end{proof}

\begin{lemma}
\label{lemma-quasi-finiteness-over-generic-point}
Let $S$ be a scheme. Let $f : X \to Y$ be a morphism of algebraic spaces
over $S$. Assume $f$ is locally of finite type.
Let $X^0 \subset |X|$, resp.\ $Y^0 \subset |Y|$ denote the set of
codimension $0$ points of $X$, resp.\ $Y$. Let $y \in Y^0$. The following are
equivalent
\begin{enumerate}
\item $f^{-1}(\{y\}) \subset X^0$,
\item $f$ is quasi-finite at all points lying over $y$,
\item $f$ is quasi-finite at all $x \in X^0$ lying over $y$.
\end{enumerate}
\end{lemma}

\begin{proof}
Let $V$ be a scheme and let $V \to Y$ be a surjective \'etale morphism.
Let $U$ be a scheme and let $U \to V \times_Y X$ be a surjective \'etale
morphism. Then $f$ is quasi-finite at the image $x$ of a point $u \in U$
if and only if $U \to V$ is quasi-finite at $u$. Moreover, $x \in X^0$
if and only if $u$ is the generic point of an irreducible component
of $U$ (Properties of Spaces, Lemma
\ref{spaces-properties-lemma-codimension-0-points}).
Thus the lemma reduces to the case of the morphism $U \to V$, i.e., to
Morphisms, Lemma \ref{morphisms-lemma-quasi-finiteness-over-generic-point}.
\end{proof}

\begin{lemma}
\label{lemma-finite-over-dense-open}
Let $S$ be a scheme. Let $f : X \to Y$ be a morphism of algebraic spaces
over $S$. Assume $f$ is locally of finite type.
Let $X^0 \subset |X|$, resp.\ $Y^0 \subset |Y|$ denote the set of
codimension $0$ points of $X$, resp.\ $Y$. Assume
\begin{enumerate}
\item $Y$ is decent,
\item $X^0$ and $Y^0$ are finite and $f^{-1}(Y^0) = X^0$,
\item either $f$ is quasi-compact or $f$ is separated.
\end{enumerate}
Then there exists a dense open $V \subset Y$
such that $f^{-1}(V) \to V$ is finite.
\end{lemma}

\begin{proof}
By Lemmas \ref{lemma-finitely-many-irreducible-components} and
\ref{lemma-decent-generic-points} we may assume $Y$ is a scheme
with finitely many irreducible components. Shrinking further we
may assume $Y$ is an irreducible affine scheme with generic point $y$.
Then the fibre of $f$ over $y$ is finite.

\medskip\noindent
Assume $f$ is quasi-compact and $Y$ affine irreducible. Then $X$ is
quasi-compact and we may choose an affine scheme $U$ and a
surjective \'etale morphism $U \to X$. Then $U \to Y$ is of finite type
and the fibre of $U \to Y$ over $y$ is the set $U^0$ of generic points of
irreducible components of $U$ (Properties of Spaces, Lemma
\ref{spaces-properties-lemma-codimension-0-points}).
Hence $U^0$ is finite
(Morphisms, Lemma
\ref{morphisms-lemma-quasi-finite-at-a-finite-number-of-points})
and after shrinking $Y$ we may assume that $U \to Y$ is finite
(Morphisms, Lemma \ref{morphisms-lemma-generically-finite}).
Next, consider $R = U \times_X U$. Since the projection
$s : R \to U$ is \'etale we see that $R^0 = s^{-1}(U^0)$
lies over $y$. Since $R \to U \times_Y U$ is a monomorphism,
we conclude that $R^0$ is finite as $U \times_Y U \to Y$ is finite.
And $R$ is separated
(Properties of Spaces, Lemma \ref{spaces-properties-lemma-separated-cover}).
Thus we may shrink $Y$ once more to reach the situation
where $R$ is finite over $Y$
(Morphisms, Lemma \ref{morphisms-lemma-finite-over-dense-open}).
In this case it follows that $X = U/R$ is finite over $Y$
by exactly the same arguments as given in the proof of
Lemma \ref{lemma-generically-finite}
(or we can simply apply that lemma because
it follows immediately that $X$ is quasi-separated as well).

\medskip\noindent
Assume $f$ is separated and $Y$ affine irreducible. Choose $V \subset Y$
and $U \subset X$ as in Lemma \ref{lemma-generically-finite-reprise}.
Since $f|_U : U \to V$ is finite, we see that $U \subset f^{-1}(V)$
is closed as well as open
(Morphisms of Spaces, Lemmas
\ref{spaces-morphisms-lemma-universally-closed-permanence} and
\ref{spaces-morphisms-lemma-finite-proper}).
Thus $f^{-1}(V) = U \amalg W$ for some
open subspace $W$ of $X$. However, since $U$ contains all the codimension
$0$ points of $X$ we conclude that $W = \emptyset$
(Properties of Spaces, Lemma
\ref{spaces-properties-lemma-codimension-0-points-dense})
as desired.
\end{proof}






\section{Birational morphisms}
\label{section-birational}

\noindent
The following definition of a birational morphism of algebraic spaces
seems to be the closest to our definition
(Morphisms, Definition \ref{morphisms-definition-birational})
of a birational morphism of schemes.

\begin{definition}
\label{definition-birational}
Let $S$ be a scheme. Let $X$ and $Y$ algebraic spaces over $S$.
Assume $X$ and $Y$ are decent and that $|X|$ and $|Y|$ have finitely many
irreducible components. We say a morphism $f : X \to Y$ is
{\it birational} if
\begin{enumerate}
\item $|f|$ induces a bijection between the set of generic points
of irreducible components of $|X|$ and the set of generic points
of the irreducible components of $|Y|$, and
\item for every generic point $x \in |X|$ of an irreducible component
the local ring map $\mathcal{O}_{Y, f(x)} \to \mathcal{O}_{X, x}$
is an isomorphism (see clarification below).
\end{enumerate}
\end{definition}

\noindent
Clarification: Since $X$ and $Y$ are decent the topological spaces
$|X|$ and $|Y|$ are sober (Proposition \ref{proposition-reasonable-sober}).
Hence condition (1) makes sense. Moreover, because we have assumed
that $|X|$ and $|Y|$ have finitely many irreducible components, we
see that the generic points $x_1, \ldots, x_n \in |X|$,
resp.\ $y_1, \ldots, y_n \in |Y|$ are contained in any dense open
of $|X|$, resp.\ $|Y|$. In particular, they are contained in
the schematic locus of $X$, resp.\ $Y$ by
Theorem \ref{theorem-decent-open-dense-scheme}.
Thus we can define $\mathcal{O}_{X, x_i}$, resp.\ $\mathcal{O}_{Y, y_i}$
to be the local ring of this scheme at $x_i$, resp.\ $y_i$.

\medskip\noindent
We conclude that if the morphism $f : X \to Y$ is birational, then
there exist dense open subspaces $X' \subset X$ and $Y' \subset Y$ such that
\begin{enumerate}
\item $f(X') \subset Y'$,
\item $X'$ and $Y'$ are representable, and
\item $f|_{X'} : X' \to Y'$ is birational in
the sense of Morphisms, Definition \ref{morphisms-definition-birational}.
\end{enumerate}
However, we do insist that $X$ and $Y$ are decent with finitely many
irreducible components. Other ways to characterize decent algebraic spaces
with finitely many irreducible components
are given in Lemma \ref{lemma-finitely-many-irreducible-components}.
In most cases birational morphisms are isomorphisms over dense opens.

\begin{lemma}
\label{lemma-birational-dominant}
Let $S$ be a scheme.
Let $f : X \to Y$ be a morphism of algebraic spaces over $S$ which
are decent and have finitely many irreducible components. If $f$ is
birational then $f$ is dominant.
\end{lemma}

\begin{proof}
Follows immediately from the definitions. See
Morphisms of Spaces, Definition \ref{spaces-morphisms-definition-dominant}.
\end{proof}

\begin{lemma}
\label{lemma-birational-generic-fibres}
Let $S$ be a scheme. Let $f : X \to Y$ be a birational morphism of
algebraic spaces over $S$ which are decent and have finitely
many irreducible components. If $y \in Y$ is the generic point of
an irreducible component, then the base change
$X \times_Y \Spec(\mathcal{O}_{Y, y}) \to \Spec(\mathcal{O}_{Y, y})$
is an isomorphism.
\end{lemma}

\begin{proof}
Let $X' \subset X$ and $Y' \subset Y$ be the maximal open subspaces
which are representable, see
Lemma \ref{lemma-finitely-many-irreducible-components}. By
Lemma \ref{lemma-quasi-finiteness-over-generic-point}
the fibre of $f$ over $y$ is consists
of points of codimension $0$ of $X$ and is therefore contained
in $X'$. Hence $X \times_Y \Spec(\mathcal{O}_{Y, y}) =
X' \times_{Y'} \Spec(\mathcal{O}_{Y', y})$ and the result follows
from Morphisms, Lemma \ref{morphisms-lemma-birational-generic-fibres}.
\end{proof}

\begin{lemma}
\label{lemma-birational-birational}
Let $S$ be a scheme. Let $f : X \to Y$ be a birational morphism of
algebraic spaces over $S$ which are decent and have finitely many
irreducible components. Assume one of the following conditions is satisfied
\begin{enumerate}
\item $f$ is locally of finite type and $Y$ reduced (i.e., integral),
\item $f$ is locally of finite presentation.
\end{enumerate}
Then there exist dense opens $U \subset X$ and $V \subset Y$
such that $f(U) \subset V$ and $f|_U : U \to V$ is an isomorphism.
\end{lemma}

\begin{proof}
By Lemma \ref{lemma-finitely-many-irreducible-components} we may assume
that $X$ and $Y$ are schemes. In this case the result is
Morphisms, Lemma \ref{morphisms-lemma-birational-birational}.
\end{proof}

\begin{lemma}
\label{lemma-birational-isomorphism-over-dense-open}
Let $S$ be a scheme. Let $f : X \to Y$ be a birational morphism of
algebraic spaces over $S$ which are decent and have finitely
many irreducible components. Assume
\begin{enumerate}
\item either $f$ is quasi-compact or $f$ is separated, and
\item either $f$ is locally of finite type and $Y$ is reduced or
$f$ is locally of finite presentation.
\end{enumerate}
Then there exists a dense open $V \subset Y$
such that $f^{-1}(V) \to V$ is an isomorphism.
\end{lemma}

\begin{proof}
By Lemma \ref{lemma-finitely-many-irreducible-components} we may assume
$Y$ is a scheme. By Lemma \ref{lemma-finite-over-dense-open} we may assume
that $f$ is finite. Then $X$ is a scheme too and the result follows from
Morphisms, Lemma \ref{morphisms-lemma-birational-isomorphism-over-dense-open}.
\end{proof}

\begin{lemma}
\label{lemma-birational-etale-localization}
Let $S$ be a scheme. Let $f : X \to Y$ be a morphism of algebraic
spaces over $S$ which are decent and have finitely many irreducible
components. If $f$ is birational and $V \to Y$ is an \'etale morphism
with $V$ affine, then $X \times_Y V$ is decent with finitely
many irreducible components and $X \times_Y V \to V$ is birational.
\end{lemma}

\begin{proof}
The algebraic space $U = X \times_Y V$ is decent
(Lemma \ref{lemma-etale-named-properties}).
The generic points of $V$ and $U$ are the elements of $|V|$ and $|U|$
which lie over generic points of $|Y|$ and $|X|$
(Lemma \ref{lemma-decent-generic-points}).
Since $Y$ is decent we conclude there are finitely many generic points
on $V$. Let $\xi \in |X|$ be a generic point of an irreducible component.
By the discussion following Definition \ref{definition-birational}
we have a cartesian square
$$
\xymatrix{
\Spec(\mathcal{O}_{X, \xi}) \ar[d] \ar[r] & X \ar[d] \\
\Spec(\mathcal{O}_{Y, f(\xi)}) \ar[r] & Y
}
$$
whose horizontal morphisms are monomorphisms identifying local rings
and where the left vertical arrow is an isomorphism. It follows that
in the diagram
$$
\xymatrix{
\Spec(\mathcal{O}_{X, \xi}) \times_X U \ar[d] \ar[r] & U \ar[d] \\
\Spec(\mathcal{O}_{Y, f(\xi)}) \times_Y V \ar[r] & V
}
$$
the vertical arrow on the left is an isomorphism. The horizonal arrows
have image contained in the schematic locus of $U$ and $V$ and
identify local rings (some details omitted). Since the image of
the horizontal arrows are the points of $|U|$, resp.\ $|V|$
lying over $\xi$, resp.\ $f(\xi)$ we conclude.
\end{proof}

\begin{lemma}
\label{lemma-birational-induced-morphism-normalizations}
Let $S$ be a scheme. Let $f : X \to Y$ be a birational morphism between
algebraic spaces over $S$ which are decent and have finitely many irreducible
components. Then the normalizations $X^\nu \to X$ and $Y^\nu \to Y$ exist
and there is a commutative diagram
$$
\xymatrix{
X^\nu \ar[r] \ar[d] &  Y^\nu \ar[d] \\
X \ar[r] & Y
}
$$
of algebraic spaces over $S$. The morphism $X^\nu \to Y^\nu$ is birational.
\end{lemma}

\begin{proof}
By Lemma \ref{lemma-finitely-many-irreducible-components} we see that
$X$ and $Y$ satisfy the equivalent conditions of
Morphisms of Spaces, Lemma \ref{spaces-morphisms-lemma-prepare-normalization}
and the normalizations are defined. By
Morphisms of Spaces, Lemma \ref{spaces-morphisms-lemma-normalization-normal}
the algebraic space $X^\nu$ is normal and maps codimension $0$ points
to codimension $0$ points. Since $f$ maps codimension $0$ points to
codimension $0$ points (this is the same as generic points on decent
spaces by Lemma \ref{lemma-decent-generic-points})
we obtain from
Morphisms of Spaces, Lemma \ref{spaces-morphisms-lemma-normalization-normal}
a factorization of the composition $X^\nu \to X \to Y$ through $Y^\nu$.

\medskip\noindent
Observe that $X^\nu$ and $Y^\nu$ are decent for example by
Lemma \ref{lemma-representable-named-properties}.
Moreover the maps $X^\nu \to X$ and $Y^\nu \to Y$
induce bijections on irreducible components (see references above)
hence $X^\nu$ and $Y^\nu$ both have a finite number of irreducible
components and the map $X^\nu \to Y^\nu$ induces a bijection
between their generic points.
To prove that $X^\nu \to Y^\nu$ is birational, it therefore
suffices to show it induces an isomorphism on local rings at
these points. To do this we may replace $X$ and $Y$ by open neighbourhoods
of their generic points, hence we may assume $X$ and $Y$ are affine
irreducible schemes with generic points $x$ and $y$. Since
$f$ is birational the map $\mathcal{O}_{X, x} \to \mathcal{O}_{Y, y}$
is an isomorphism. Let $x^\nu \in X^\nu$ and $y^\nu \in Y^\nu$ be
the points lying over $x$ and $y$.
By construction of the normalization
we see that $\mathcal{O}_{X^\nu, x^\nu} = \mathcal{O}_{X, x}/\mathfrak m_x$
and similarly on $Y$. Thus the map
$\mathcal{O}_{X^\nu, x^\nu} \to \mathcal{O}_{Y^\nu, y^\nu}$
is an isomorphism as well.
\end{proof}

\begin{lemma}
\label{lemma-finite-birational-over-normal}
Let $S$ be a scheme. Let $f : X \to Y$ be a morphism of algebraic
spaces over $S$. Assume
\begin{enumerate}
\item $X$ and $Y$ are decent and have finitely many irreducible components,
\item $f$ is integral and birational,
\item $Y$ is normal, and
\item $X$ is reduced.
\end{enumerate}
Then $f$ is an isomorphism.
\end{lemma}

\begin{proof}
Let $V \to Y$ be an \'etale morphism with $V$ affine. It suffices to show that
$U = X \times_Y V \to V$ is an isomorphism. By
Lemma \ref{lemma-birational-etale-localization} and its proof
we see that $U$ and $V$ are decent and have finitely many
irreducible components and that $U \to V$ is birational.
By Properties, Lemma
\ref{properties-lemma-normal-locally-finite-nr-irreducibles}
$V$ is a finite disjoint union of integral schemes.
Thus we may assume $V$ is integral. As $f$ is birational, we
see that $U$ is irreducible and reduced, i.e., integral
(note that $U$ is a scheme as $f$ is integral, hence representable).
Thus we may assume that $X$ and $Y$ are integral schemes
and the result follows from the case of schemes, see
Morphisms, Lemma \ref{morphisms-lemma-finite-birational-over-normal}.
\end{proof}

\begin{lemma}
\label{lemma-normalization-normal}
Let $S$ be a scheme. Let $f : X \to Y$ be an integral birational morphism of
decent algebraic spaces over $S$ which have finitely many irreducible
components. Then there exists a factorization $Y^\nu \to X \to Y$ and
$Y^\nu \to X$ is the normalization of $X$.
\end{lemma}

\begin{proof}
Consider the map $X^\nu \to Y^\nu$ of
Lemma \ref{lemma-birational-induced-morphism-normalizations}.
This map is integral by
Morphisms of Spaces, Lemma \ref{spaces-morphisms-lemma-finite-permanence}.
Hence it is an isomorphism by
Lemma \ref{lemma-finite-birational-over-normal}.
\end{proof}




\section{Jacobson spaces}
\label{section-jacobson}

\noindent
We have defined the Jacobson property for algebraic spaces in
Properties of Spaces, Remark
\ref{spaces-properties-remark-list-properties-local-etale-topology}.
For representable algebraic spaces it agrees with the property discussed in
Properties, Section \ref{properties-section-jacobson}.
The relationship between the Jacobson property and the behaviour of
the topological space $|X|$ is not evident for general algebraic spaces $|X|$.
However, a decent (for example quasi-separated or locally separated)
algebraic space $X$ is Jacobson if and only if $|X|$ is Jacobson
(see Lemma \ref{lemma-decent-Jacobson}).

\begin{lemma}
\label{lemma-Jacobson-universally-Jacobson}
Let $S$ be a scheme. Let $X$ be a Jacobson algebraic space over $S$.
Any algebraic space locally of finite type over $X$ is Jacobson.
\end{lemma}

\begin{proof}
Let $U \to X$ be a surjective \'etale morphism where $U$ is a scheme.
Then $U$ is Jacobson (by definition) and for a morphism of schemes $V \to U$
which is locally of finite type we see that $V$ is Jacobson by the
corresponding result for schemes (Morphisms, Lemma
\ref{morphisms-lemma-Jacobson-universally-Jacobson}).
Thus if $Y \to X$ is a morphism of algebraic spaces which is locally
of finite type, then setting $V = U \times_X Y$ we see that
$Y$ is Jacobson by definition.
\end{proof}

\begin{lemma}
\label{lemma-Jacobson-ft-points-lift-to-closed}
Let $S$ be a scheme. Let $X$ be a Jacobson algebraic space over $S$.
For $x \in X_{\text{ft-pts}}$ and $g : W \to X$ locally of finite type
with $W$ a scheme, if $x \in \Im(|g|)$, then there exists a closed
point of $W$ mapping to $x$.
\end{lemma}

\begin{proof}
Let $U \to X$ be an \'etale morphism with $U$ a scheme and with $u \in U$
closed mapping to $x$, see
Morphisms of Spaces, Lemma
\ref{spaces-morphisms-lemma-identify-finite-type-points}.
Observe that $W$, $W \times_X U$, and $U$ are Jacobson schemes
by Lemma \ref{lemma-Jacobson-universally-Jacobson}.
Hence finite type points on these schemes
are the same thing as closed points by
Morphisms, Lemma \ref{morphisms-lemma-jacobson-finite-type-points}.
The inverse image $T \subset W \times_X U$ of $u$ is a nonempty
(as $x$ in the image of $W \to X$) closed subset.
By Morphisms, Lemma \ref{morphisms-lemma-enough-finite-type-points}
there is a closed point $t$ of $W \times_X U$ which maps to $u$.
As $W \times_X U \to W$ is locally of finite type
the image of $t$ in $W$ is closed by
Morphisms, Lemma \ref{morphisms-lemma-jacobson-finite-type-points}.
\end{proof}

\begin{lemma}
\label{lemma-decent-Jacobson-ft-pts}
Let $S$ be a scheme. Let $X$ be a decent Jacobson algebraic space over $S$.
Then $X_{\text{ft-pts}} \subset |X|$ is the set of closed points.
\end{lemma}

\begin{proof}
If $x \in |X|$ is closed, then we can represent $x$ by a closed
immersion $\Spec(k) \to X$, see Lemma \ref{lemma-decent-space-closed-point}.
Hence $x$ is certainly a finite type point.

\medskip\noindent
Conversely, let $x \in |X|$ be a finite type point. We know that
$x$ can be represented by a quasi-compact monomorphism
$\Spec(k) \to X$ where $k$ is a field
(Definition \ref{definition-very-reasonable}). On the other hand,
by definition, there exists a morphism $\Spec(k') \to X$
which is locally of finite type and represents $x$
(Morphisms, Definition \ref{morphisms-definition-finite-type-point}).
We obtain a factorization $\Spec(k') \to \Spec(k) \to X$.
Let $U \to X$ be any \'etale morphism with $U$ affine and consider
the morphisms
$$
\Spec(k') \times_X U \to \Spec(k) \times_X U \to U
$$
The quasi-compact scheme $\Spec(k) \times_X U$ is \'etale over
$\Spec(k)$ hence is a finite disjoint union
of spectra of fields (Remark \ref{remark-recall}).
Moreover, the first morphism is surjective and locally of finite type
(Morphisms, Lemma \ref{morphisms-lemma-permanence-finite-type})
hence surjective on finite type points
(Morphisms, Lemma \ref{morphisms-lemma-finite-type-points-surjective-morphism})
and the composition (which is locally of finite type) sends
finite type points to closed points as $U$ is Jacobson
(Morphisms, Lemma \ref{morphisms-lemma-jacobson-finite-type-points}).
Thus the image of
$\Spec(k) \times_X U \to U$ is a finite set of closed points hence
closed. Since this is true for every affine $U$ and \'etale morphism
$U \to X$, we conclude that $x \in |X|$ is closed.
\end{proof}

\begin{lemma}
\label{lemma-decent-Jacobson}
Let $S$ be a scheme. Let $X$ be a decent algebraic space over $S$.
Then $X$ is Jacobson if and only if $|X|$ is Jacobson.
\end{lemma}

\begin{proof}
Assume $X$ is Jacobson and that $T \subset |X|$ is a closed subset.
By Morphisms of Spaces, Lemma
\ref{spaces-morphisms-lemma-enough-finite-type-points}
we see that $T \cap X_{\text{ft-pts}}$ is dense in $T$.
By Lemma \ref{lemma-decent-Jacobson-ft-pts} we see that
$X_{\text{ft-pts}}$ are the
closed points of $|X|$. Thus $|X|$ is indeed Jacobson.

\medskip\noindent
Assume $|X|$ is Jacobson. Let $f : U \to X$ be an \'etale
morphism with $U$ an affine scheme. We have to show that $U$
is Jacobson. If $x \in |X|$ is closed,
then the fibre $F = f^{-1}(\{x\})$ is a finite (by definition of
decent) closed (by construction of the topology on $|X|$) subset of $U$.
Since there are no specializations between points of $F$
(Lemma \ref{lemma-decent-no-specializations-map-to-same-point})
we conclude that every point of $F$ is closed in $U$.
If $U$ is not Jacobson, then there exists a non-closed point
$u \in U$ such that $\{u\}$ is locally closed (Topology, Lemma
\ref{topology-lemma-non-jacobson-Noetherian-characterize}).
We will show that $f(u) \in |X|$ is closed; by the above $u$
is closed in $U$ which is a contradiction and finishes
the proof. To prove this we may replace $U$ by an affine open
neighbourhood of $u$.
Thus we may assume that $\{u\}$ is closed in $U$.
Let $R = U \times_X U$ with projections $s, t : R \to U$.
Then $s^{-1}(\{u\}) = \{r_1, \ldots, r_m\}$ is finite (by
definition of decent spaces). After replacing $U$ by a smaller affine
open neighbourhood of $u$ we may assume that $t(r_j) = u$ for
$j = 1, \ldots, m$. It follows that $\{u\}$ is an $R$-invariant
closed subset of $U$. Hence $\{f(u)\}$ is a locally closed subset
of $X$ as it is closed in the open $|f|(|U|)$ of $|X|$. Since $|X|$
is Jacobson we conclude that $f(u)$ is closed in $|X|$ as desired.
\end{proof}

\begin{lemma}
\label{lemma-punctured-spec}
Let $S$ be a scheme. Let $X$ be a decent locally Noetherian algebraic
space over $S$. Let $x \in |X|$. Then
$$
W = \{x' \in |X| : x' \leadsto x,\ x' \not = x\}
$$
is a Noetherian, spectral, sober, Jacobson topological space.
\end{lemma}

\begin{proof}
We may replace by any open subspace containing $x$.
Thus we may assume that $X$ is quasi-compact.
Then $|X|$ is a Noetherian topological space
(Properties of Spaces, Lemma \ref{spaces-properties-lemma-Noetherian-topology}).
Thus $W$ is a Noetherian topological space
(Topology, Lemma \ref{topology-lemma-Noetherian}).

\medskip\noindent
Combining Lemma \ref{lemma-locally-Noetherian-decent-quasi-separated} with
Properties of Spaces, Lemma
\ref{spaces-properties-lemma-quasi-compact-quasi-separated-spectral}
we see that $|X|$ is a spectral toplogical space.
By Topology, Lemma \ref{topology-lemma-make-spectral-space}
we see that $W \cup \{x\}$ is a spectral topological space.
Now $W$ is a quasi-compact open of $W \cup \{x\}$ and hence $W$ is
spectral by Topology, Lemma \ref{topology-lemma-spectral-sub}.

\medskip\noindent
Let $E \subset W$ be an irreducible closed subset. Then if $Z \subset |X|$
is the closure of $E$ we see that $x \in Z$. There is a unique generic
point $\eta \in Z$ by Proposition \ref{proposition-reasonable-sober}.
Of course $\eta \in W$ and hence $\eta \in E$. We conclude that $E$
has a unique generic point, i.e., $W$ is sober.

\medskip\noindent
Let $x' \in W$ be a point such that $\{x'\}$ is locally closed in $W$.
To finish the proof we have to show that $x'$ is a closed point of $W$.
If not, then there exists a nontrivial specialization $x' \leadsto x'_1$
in $W$. Let $U$ be an affine scheme, $u \in U$ a point, and let $U \to X$
be an \'etale morphism mapping $u$ to $x$. By
Lemma \ref{lemma-decent-specialization}
we can choose specializations $u' \leadsto u'_1 \leadsto u$
mapping to $x' \leadsto x'_1 \leadsto x$.
Let $\mathfrak p' \subset \mathcal{O}_{U, u}$ be the prime ideal
corresponding to $u'$. The existence of the specializations
implies that $\dim(\mathcal{O}_{U, u}/\mathfrak p') \geq 2$.
Hence every nonempty open of $\Spec(\mathcal{O}_{U, u}/\mathfrak p')$
is infinite by Algebra, Lemma
\ref{algebra-lemma-Noetherian-local-domain-dim-2-infinite-opens}.
By Lemma \ref{lemma-decent-no-specializations-map-to-same-point}
we obtain a continuous map
$$
\Spec(\mathcal{O}_{U, u}/\mathfrak p')
\setminus \{\mathfrak m_u/\mathfrak p'\}
\longrightarrow
W
$$
Since the generic point of the LHS maps to $x'$ the image is
contained in $\overline{\{x'\}}$. We conclude the inverse image of $\{x'\}$
under the displayed arrow is nonempty open hence infinite.
However, the fibres of $U \to X$ are finite as $X$
is decent and we conclude that $\{x'\}$ is infinite.
This contradiction finishes the proof.
\end{proof}






\section{Local irreducibility}
\label{section-irreducible-local-ring}

\noindent
We have already defined the geometric number of branches of an
algebraic space at a point in Properties of Spaces, Section
\ref{spaces-properties-section-irreducible-local-ring}.
The number of branches of an algebraic space at a point can only
be defined for decent algebraic spaces.

\begin{lemma}
\label{lemma-irreducible-local-ring}
Let $S$ be a scheme. Let $X$ be a decent algebraic space over $S$.
Let $x \in |X|$ be a point. The following are equivalent
\begin{enumerate}
\item for any elementary \'etale neighbourhood $(U, u) \to (X, x)$
the local ring $\mathcal{O}_{U, u}$ has a unique minimal prime,
\item for any elementary \'etale neighbourhood $(U, u) \to (X, x)$
there is a unique irreducible component of $U$ through $u$,
\item for any elementary \'etale neighbourhood $(U, u) \to (X, x)$
the local ring $\mathcal{O}_{U, u}$ is unibranch,
\item the henselian local ring
$\mathcal{O}_{X, x}^h$ has a unique minimal prime.
\end{enumerate}
\end{lemma}

\begin{proof}
The equivalence of (1) and (2) follows from the fact that irreducible
components of $U$ passing through $u$ are in $1$-$1$ correspondence with
minimal primes of the local ring of $U$ at $u$. The ring
$\mathcal{O}_{X, x}^h$ is the henselization of $\mathcal{O}_{U, u}$, see
discussion following Definition \ref{definition-henselian-local-ring}.
In particular (3) and (4) are equivalent by
More on Algebra, Lemma \ref{more-algebra-lemma-unibranch}.
The equivalence of (2) and (3) follows from
More on Morphisms, Lemma \ref{more-morphisms-lemma-nr-branches}.
\end{proof}

\begin{definition}
\label{definition-unibranch}
Let $S$ be a scheme. Let $X$ be a decent algebraic space over $S$.
Let $x \in |X|$. We say that $X$ is {\it unibranch at $x$}
if the equivalent conditions of
Lemma \ref{lemma-irreducible-local-ring} hold.
We say that $X$ is {\it unibranch} if $X$ is
unibranch at every $x \in |X|$.
\end{definition}

\noindent
This is consistent with the definition for schemes
(Properties, Definition \ref{properties-definition-unibranch}).

\begin{lemma}
\label{lemma-nr-branches-local-ring}
Let $S$ be a scheme. Let $X$ be a decent algebraic space over $S$.
Let $x \in |X|$ be a point. Let $n \in \{1, 2, \ldots\}$ be an integer.
The following are equivalent
\begin{enumerate}
\item for any elementary \'etale neighbourhood $(U, u) \to (X, x)$
the number of minimal primes of the local ring $\mathcal{O}_{U, u}$
is $\leq n$ and for at least one choice of $(U, u)$ it is $n$,
\item for any elementary \'etale neighbourhood $(U, u) \to (X, x)$
the number irreducible components of $U$ passing through $u$ is $\leq n$
and for at least one choice of $(U, u)$ it is $n$,
\item for any elementary \'etale neighbourhood $(U, u) \to (X, x)$
the number of branches of $U$ at $u$ is $\leq n$
and for at least one choice of $(U, u)$ it is $n$,
\item the number of minimal prime ideals of
$\mathcal{O}_{X, x}^h$ is $n$.
\end{enumerate}
\end{lemma}

\begin{proof}
The equivalence of (1) and (2) follows from the fact that irreducible
components of $U$ passing through $u$ are in $1$-$1$ correspondence with
minimal primes of the local ring of $U$ at $u$.
The ring $\mathcal{O}_{X, x}$ is the henselization of $\mathcal{O}_{U, u}$, see
discussion following Definition \ref{definition-henselian-local-ring}.
In particular (3) and (4) are equivalent by
More on Algebra, Lemma \ref{more-algebra-lemma-unibranch}.
The equivalence of (2) and (3) follows from
More on Morphisms, Lemma \ref{more-morphisms-lemma-nr-branches}.
\end{proof}

\begin{definition}
\label{definition-number-of-branches}
Let $S$ be a scheme. Let $X$ be a decent algebraic space over $S$.
Let $x \in |X|$. The {\it number of branches of $X$ at $x$} is
either $n \in \mathbf{N}$ if the equivalent conditions
of Lemma \ref{lemma-nr-branches-local-ring}
hold, or else $\infty$.
\end{definition}






\section{Catenary algebraic spaces}
\label{section-catenary}

\noindent
This section extends the material in
Properties, Section \ref{properties-section-catenary}
and Morphisms, Section \ref{morphisms-section-universally-catenary}
to algebraic spaces.

\begin{definition}
\label{definition-catenary}
Let $S$ be a scheme. Let $X$ be a decent algebraic space over $S$.
We say $X$ is {\it catenary} if $|X|$ is catenary
(Topology, Definition \ref{topology-definition-catenary}).
\end{definition}

\noindent
If $X$ is representable, then this is equivalent to the corresponding notion
for the scheme representing $X$.

\begin{lemma}
\label{lemma-scheme-with-dimension-function}
Let $S$ be a locally Noetherian and universally catenary scheme.
Let $\delta : S \to \mathbf{Z}$ be a dimension function.
Let $X$ be a decent algebraic space over $S$ such that
the structure morphism $X \to S$ is locally of
finite type. Let $\delta_X : |X| \to \mathbf{Z}$ be the map
sending $x$ to $\delta(f(x))$ plus the transcendence degree
of $x/f(x)$. Then $\delta_X$ is a dimension function on $|X|$.
\end{lemma}

\begin{proof}
Let $\varphi : U \to X$ be a surjective \'etale morphism where $U$ is a scheme.
Then the similarly defined function $\delta_U$ is a
dimension function on $U$ by
Morphisms, Lemma \ref{morphisms-lemma-dimension-function-propagates}.
On the other hand, by the definition of relative transcendence degree in
(Morphisms of Spaces, Definition
\ref{spaces-morphisms-definition-dimension-fibre}) we see
that $\delta_U(u) = \delta_X(\varphi(u))$.

\medskip\noindent
Let $x \leadsto x'$ be a specialization of points in $|X|$.
by Lemma \ref{lemma-decent-specialization} we can find
a specialization $u \leadsto u'$ of points of $U$ with
$\varphi(u) = x$ and $\varphi(u') = x'$. Moreover, we see
that $x = x'$ if and only if $u = u'$, see
Lemma \ref{lemma-decent-no-specializations-map-to-same-point}.
Thus the fact that $\delta_U$ is a dimension function implies that
$\delta_X$ is a dimension function, see
Topology, Definition \ref{topology-definition-dimension-function}.
\end{proof}

\begin{lemma}
\label{lemma-universally-catenary-scheme}
Let $S$ be a locally Noetherian and universally catenary scheme.
Let $X$ be an algebraic space over $S$ such that $X$ is decent
and such that the structure morphism $X \to S$ is locally of
finite type. Then $X$ is catenary.
\end{lemma}

\begin{proof}
The question is local on $S$ (use
Topology, Lemma \ref{topology-lemma-catenary}).
Thus we may assume that $S$ has a
dimension function, see Topology, Lemma
\ref{topology-lemma-locally-dimension-function}.
Then we conclude that $|X|$ has a dimension function by
Lemma \ref{lemma-scheme-with-dimension-function}.
Since $|X|$ is sober (Proposition \ref{proposition-reasonable-sober})
we conclude that $|X|$ is catenary by
Topology, Lemma \ref{topology-lemma-dimension-function-catenary}.
\end{proof}

\noindent
By Lemma \ref{lemma-universally-catenary-scheme}
the following definition is compatible with the
already existing notion for representable algebraic spaces.

\begin{definition}
\label{definition-universally-catenary}
Let $S$ be a scheme. Let $X$ be a decent and locally Noetherian
algebraic space over $S$. We say $X$ is {\it universally catenary}
if for every morphism $Y \to X$ of algebraic spaces which is 
locally of finite type and with $Y$ decent, the algebraic space
$Y$ is catenary.
\end{definition}

\noindent
If $X$ is an algebraic space, then the condition
``$X$ is decent and locally Noetherian'' is equivalent to
``$X$ is quasi-separated and locally Noetherian''. This is
Lemma \ref{lemma-locally-Noetherian-decent-quasi-separated}.
Thus another way to understand the definition above is that $X$
is universally catenary if and only if $Y$ is catenary for
all morphisms $Y \to X$ which are quasi-separated and locally of finite type.

\begin{lemma}
\label{lemma-universally-catenary}
Let $S$ be a scheme. Let $X$ be a decent, locally Noetherian, and
universally catenary algebraic space over $S$. Then any decent algebraic
space locally of finite type over $X$ is universally catenary.
\end{lemma}

\begin{proof}
This is formal from the definitions and the fact that
compositions of morphisms locally of finite type are
locally of finite type (Morphisms of Spaces, Lemma
\ref{spaces-morphisms-lemma-composition-finite-type}).
\end{proof}

\begin{lemma}
\label{lemma-check-dimension-function-finite-cover}
Let $S$ be a scheme. Let $f : Y \to X$ be a surjective finite morphism of
decent and locally Noetherian algebraic spaces. Let
$\delta : |X| \to \mathbf{Z}$ be a function. If $\delta \circ |f|$ is a
dimension function, then $\delta$ is a dimension function.
\end{lemma}

\begin{proof}
Let $x \mapsto x'$, $x \not = x'$ be a specialization in $|X|$.
Choose $y \in |Y|$ with $|f|(y) = x$. Since $|f|$ is closed
(Morphisms of Spaces, Lemma \ref{spaces-morphisms-lemma-finite-proper})
we find a specialization $y \leadsto y'$ with $|f|(y') = x'$.
Thus we conclude that
$\delta(x) = \delta(|f|(y)) > \delta(|f|(y')) = \delta(x')$
(see Topology, Definition \ref{topology-definition-dimension-function}).
If $x \leadsto x'$ is an immediate specialization, then
$y \leadsto y'$ is an immediate specialization too:
namely if $y \leadsto y'' \leadsto y'$, then $|f|(y'')$
must be either $x$ or $x'$ and there are no nontrivial
specializations between points of fibres of $|f|$ by
Lemma \ref{lemma-conditions-on-fibre-and-qf}.
\end{proof}

\noindent
The discussion will be continued in
More on Morphisms of Spaces, Section
\ref{spaces-more-morphisms-section-catenary}.










\begin{multicols}{2}[\section{Other chapters}]
\noindent
Preliminaries
\begin{enumerate}
\item \hyperref[introduction-section-phantom]{Introduction}
\item \hyperref[conventions-section-phantom]{Conventions}
\item \hyperref[sets-section-phantom]{Set Theory}
\item \hyperref[categories-section-phantom]{Categories}
\item \hyperref[topology-section-phantom]{Topology}
\item \hyperref[sheaves-section-phantom]{Sheaves on Spaces}
\item \hyperref[sites-section-phantom]{Sites and Sheaves}
\item \hyperref[stacks-section-phantom]{Stacks}
\item \hyperref[fields-section-phantom]{Fields}
\item \hyperref[algebra-section-phantom]{Commutative Algebra}
\item \hyperref[brauer-section-phantom]{Brauer Groups}
\item \hyperref[homology-section-phantom]{Homological Algebra}
\item \hyperref[derived-section-phantom]{Derived Categories}
\item \hyperref[simplicial-section-phantom]{Simplicial Methods}
\item \hyperref[more-algebra-section-phantom]{More on Algebra}
\item \hyperref[smoothing-section-phantom]{Smoothing Ring Maps}
\item \hyperref[modules-section-phantom]{Sheaves of Modules}
\item \hyperref[sites-modules-section-phantom]{Modules on Sites}
\item \hyperref[injectives-section-phantom]{Injectives}
\item \hyperref[cohomology-section-phantom]{Cohomology of Sheaves}
\item \hyperref[sites-cohomology-section-phantom]{Cohomology on Sites}
\item \hyperref[dga-section-phantom]{Differential Graded Algebra}
\item \hyperref[dpa-section-phantom]{Divided Power Algebra}
\item \hyperref[sdga-section-phantom]{Differential Graded Sheaves}
\item \hyperref[hypercovering-section-phantom]{Hypercoverings}
\end{enumerate}
Schemes
\begin{enumerate}
\setcounter{enumi}{25}
\item \hyperref[schemes-section-phantom]{Schemes}
\item \hyperref[constructions-section-phantom]{Constructions of Schemes}
\item \hyperref[properties-section-phantom]{Properties of Schemes}
\item \hyperref[morphisms-section-phantom]{Morphisms of Schemes}
\item \hyperref[coherent-section-phantom]{Cohomology of Schemes}
\item \hyperref[divisors-section-phantom]{Divisors}
\item \hyperref[limits-section-phantom]{Limits of Schemes}
\item \hyperref[varieties-section-phantom]{Varieties}
\item \hyperref[topologies-section-phantom]{Topologies on Schemes}
\item \hyperref[descent-section-phantom]{Descent}
\item \hyperref[perfect-section-phantom]{Derived Categories of Schemes}
\item \hyperref[more-morphisms-section-phantom]{More on Morphisms}
\item \hyperref[flat-section-phantom]{More on Flatness}
\item \hyperref[groupoids-section-phantom]{Groupoid Schemes}
\item \hyperref[more-groupoids-section-phantom]{More on Groupoid Schemes}
\item \hyperref[etale-section-phantom]{\'Etale Morphisms of Schemes}
\end{enumerate}
Topics in Scheme Theory
\begin{enumerate}
\setcounter{enumi}{41}
\item \hyperref[chow-section-phantom]{Chow Homology}
\item \hyperref[intersection-section-phantom]{Intersection Theory}
\item \hyperref[pic-section-phantom]{Picard Schemes of Curves}
\item \hyperref[weil-section-phantom]{Weil Cohomology Theories}
\item \hyperref[adequate-section-phantom]{Adequate Modules}
\item \hyperref[dualizing-section-phantom]{Dualizing Complexes}
\item \hyperref[duality-section-phantom]{Duality for Schemes}
\item \hyperref[discriminant-section-phantom]{Discriminants and Differents}
\item \hyperref[derham-section-phantom]{de Rham Cohomology}
\item \hyperref[local-cohomology-section-phantom]{Local Cohomology}
\item \hyperref[algebraization-section-phantom]{Algebraic and Formal Geometry}
\item \hyperref[curves-section-phantom]{Algebraic Curves}
\item \hyperref[resolve-section-phantom]{Resolution of Surfaces}
\item \hyperref[models-section-phantom]{Semistable Reduction}
\item \hyperref[functors-section-phantom]{Functors and Morphisms}
\item \hyperref[equiv-section-phantom]{Derived Categories of Varieties}
\item \hyperref[pione-section-phantom]{Fundamental Groups of Schemes}
\item \hyperref[etale-cohomology-section-phantom]{\'Etale Cohomology}
\item \hyperref[crystalline-section-phantom]{Crystalline Cohomology}
\item \hyperref[proetale-section-phantom]{Pro-\'etale Cohomology}
\item \hyperref[relative-cycles-section-phantom]{Relative Cycles}
\item \hyperref[more-etale-section-phantom]{More \'Etale Cohomology}
\item \hyperref[trace-section-phantom]{The Trace Formula}
\end{enumerate}
Algebraic Spaces
\begin{enumerate}
\setcounter{enumi}{64}
\item \hyperref[spaces-section-phantom]{Algebraic Spaces}
\item \hyperref[spaces-properties-section-phantom]{Properties of Algebraic Spaces}
\item \hyperref[spaces-morphisms-section-phantom]{Morphisms of Algebraic Spaces}
\item \hyperref[decent-spaces-section-phantom]{Decent Algebraic Spaces}
\item \hyperref[spaces-cohomology-section-phantom]{Cohomology of Algebraic Spaces}
\item \hyperref[spaces-limits-section-phantom]{Limits of Algebraic Spaces}
\item \hyperref[spaces-divisors-section-phantom]{Divisors on Algebraic Spaces}
\item \hyperref[spaces-over-fields-section-phantom]{Algebraic Spaces over Fields}
\item \hyperref[spaces-topologies-section-phantom]{Topologies on Algebraic Spaces}
\item \hyperref[spaces-descent-section-phantom]{Descent and Algebraic Spaces}
\item \hyperref[spaces-perfect-section-phantom]{Derived Categories of Spaces}
\item \hyperref[spaces-more-morphisms-section-phantom]{More on Morphisms of Spaces}
\item \hyperref[spaces-flat-section-phantom]{Flatness on Algebraic Spaces}
\item \hyperref[spaces-groupoids-section-phantom]{Groupoids in Algebraic Spaces}
\item \hyperref[spaces-more-groupoids-section-phantom]{More on Groupoids in Spaces}
\item \hyperref[bootstrap-section-phantom]{Bootstrap}
\item \hyperref[spaces-pushouts-section-phantom]{Pushouts of Algebraic Spaces}
\end{enumerate}
Topics in Geometry
\begin{enumerate}
\setcounter{enumi}{81}
\item \hyperref[spaces-chow-section-phantom]{Chow Groups of Spaces}
\item \hyperref[groupoids-quotients-section-phantom]{Quotients of Groupoids}
\item \hyperref[spaces-more-cohomology-section-phantom]{More on Cohomology of Spaces}
\item \hyperref[spaces-simplicial-section-phantom]{Simplicial Spaces}
\item \hyperref[spaces-duality-section-phantom]{Duality for Spaces}
\item \hyperref[formal-spaces-section-phantom]{Formal Algebraic Spaces}
\item \hyperref[restricted-section-phantom]{Algebraization of Formal Spaces}
\item \hyperref[spaces-resolve-section-phantom]{Resolution of Surfaces Revisited}
\end{enumerate}
Deformation Theory
\begin{enumerate}
\setcounter{enumi}{89}
\item \hyperref[formal-defos-section-phantom]{Formal Deformation Theory}
\item \hyperref[defos-section-phantom]{Deformation Theory}
\item \hyperref[cotangent-section-phantom]{The Cotangent Complex}
\item \hyperref[examples-defos-section-phantom]{Deformation Problems}
\end{enumerate}
Algebraic Stacks
\begin{enumerate}
\setcounter{enumi}{93}
\item \hyperref[algebraic-section-phantom]{Algebraic Stacks}
\item \hyperref[examples-stacks-section-phantom]{Examples of Stacks}
\item \hyperref[stacks-sheaves-section-phantom]{Sheaves on Algebraic Stacks}
\item \hyperref[criteria-section-phantom]{Criteria for Representability}
\item \hyperref[artin-section-phantom]{Artin's Axioms}
\item \hyperref[quot-section-phantom]{Quot and Hilbert Spaces}
\item \hyperref[stacks-properties-section-phantom]{Properties of Algebraic Stacks}
\item \hyperref[stacks-morphisms-section-phantom]{Morphisms of Algebraic Stacks}
\item \hyperref[stacks-limits-section-phantom]{Limits of Algebraic Stacks}
\item \hyperref[stacks-cohomology-section-phantom]{Cohomology of Algebraic Stacks}
\item \hyperref[stacks-perfect-section-phantom]{Derived Categories of Stacks}
\item \hyperref[stacks-introduction-section-phantom]{Introducing Algebraic Stacks}
\item \hyperref[stacks-more-morphisms-section-phantom]{More on Morphisms of Stacks}
\item \hyperref[stacks-geometry-section-phantom]{The Geometry of Stacks}
\end{enumerate}
Topics in Moduli Theory
\begin{enumerate}
\setcounter{enumi}{107}
\item \hyperref[moduli-section-phantom]{Moduli Stacks}
\item \hyperref[moduli-curves-section-phantom]{Moduli of Curves}
\end{enumerate}
Miscellany
\begin{enumerate}
\setcounter{enumi}{109}
\item \hyperref[examples-section-phantom]{Examples}
\item \hyperref[exercises-section-phantom]{Exercises}
\item \hyperref[guide-section-phantom]{Guide to Literature}
\item \hyperref[desirables-section-phantom]{Desirables}
\item \hyperref[coding-section-phantom]{Coding Style}
\item \hyperref[obsolete-section-phantom]{Obsolete}
\item \hyperref[fdl-section-phantom]{GNU Free Documentation License}
\item \hyperref[index-section-phantom]{Auto Generated Index}
\end{enumerate}
\end{multicols}


\bibliography{my}
\bibliographystyle{amsalpha}

\end{document}
