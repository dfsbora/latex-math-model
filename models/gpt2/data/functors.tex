\IfFileExists{stacks-project.cls}{%
\documentclass{stacks-project}
}{%
\documentclass{amsart}
}

% For dealing with references we use the comment environment
\usepackage{verbatim}
\newenvironment{reference}{\comment}{\endcomment}
%\newenvironment{reference}{}{}
\newenvironment{slogan}{\comment}{\endcomment}
\newenvironment{history}{\comment}{\endcomment}

% For commutative diagrams we use Xy-pic
\usepackage[all]{xy}

% We use 2cell for 2-commutative diagrams.
\xyoption{2cell}
\UseAllTwocells

% We use multicol for the list of chapters between chapters
\usepackage{multicol}

% This is generall recommended for better output
\usepackage{lmodern}
\usepackage[T1]{fontenc}

% For cross-file-references
\usepackage{xr-hyper}

% Package for hypertext links:
\usepackage{hyperref}

% For any local file, say "hello.tex" you want to link to please
% use \externaldocument[hello-]{hello}
\externaldocument[introduction-]{introduction}
\externaldocument[conventions-]{conventions}
\externaldocument[sets-]{sets}
\externaldocument[categories-]{categories}
\externaldocument[topology-]{topology}
\externaldocument[sheaves-]{sheaves}
\externaldocument[sites-]{sites}
\externaldocument[stacks-]{stacks}
\externaldocument[fields-]{fields}
\externaldocument[algebra-]{algebra}
\externaldocument[brauer-]{brauer}
\externaldocument[homology-]{homology}
\externaldocument[derived-]{derived}
\externaldocument[simplicial-]{simplicial}
\externaldocument[more-algebra-]{more-algebra}
\externaldocument[smoothing-]{smoothing}
\externaldocument[modules-]{modules}
\externaldocument[sites-modules-]{sites-modules}
\externaldocument[injectives-]{injectives}
\externaldocument[cohomology-]{cohomology}
\externaldocument[sites-cohomology-]{sites-cohomology}
\externaldocument[dga-]{dga}
\externaldocument[dpa-]{dpa}
\externaldocument[sdga-]{sdga}
\externaldocument[hypercovering-]{hypercovering}
\externaldocument[schemes-]{schemes}
\externaldocument[constructions-]{constructions}
\externaldocument[properties-]{properties}
\externaldocument[morphisms-]{morphisms}
\externaldocument[coherent-]{coherent}
\externaldocument[divisors-]{divisors}
\externaldocument[limits-]{limits}
\externaldocument[varieties-]{varieties}
\externaldocument[topologies-]{topologies}
\externaldocument[descent-]{descent}
\externaldocument[perfect-]{perfect}
\externaldocument[more-morphisms-]{more-morphisms}
\externaldocument[flat-]{flat}
\externaldocument[groupoids-]{groupoids}
\externaldocument[more-groupoids-]{more-groupoids}
\externaldocument[etale-]{etale}
\externaldocument[chow-]{chow}
\externaldocument[intersection-]{intersection}
\externaldocument[pic-]{pic}
\externaldocument[weil-]{weil}
\externaldocument[adequate-]{adequate}
\externaldocument[dualizing-]{dualizing}
\externaldocument[duality-]{duality}
\externaldocument[discriminant-]{discriminant}
\externaldocument[derham-]{derham}
\externaldocument[local-cohomology-]{local-cohomology}
\externaldocument[algebraization-]{algebraization}
\externaldocument[curves-]{curves}
\externaldocument[resolve-]{resolve}
\externaldocument[models-]{models}
\externaldocument[functors-]{functors}
\externaldocument[equiv-]{equiv}
\externaldocument[pione-]{pione}
\externaldocument[etale-cohomology-]{etale-cohomology}
\externaldocument[proetale-]{proetale}
\externaldocument[relative-cycles-]{relative-cycles}
\externaldocument[more-etale-]{more-etale}
\externaldocument[trace-]{trace}
\externaldocument[crystalline-]{crystalline}
\externaldocument[spaces-]{spaces}
\externaldocument[spaces-properties-]{spaces-properties}
\externaldocument[spaces-morphisms-]{spaces-morphisms}
\externaldocument[decent-spaces-]{decent-spaces}
\externaldocument[spaces-cohomology-]{spaces-cohomology}
\externaldocument[spaces-limits-]{spaces-limits}
\externaldocument[spaces-divisors-]{spaces-divisors}
\externaldocument[spaces-over-fields-]{spaces-over-fields}
\externaldocument[spaces-topologies-]{spaces-topologies}
\externaldocument[spaces-descent-]{spaces-descent}
\externaldocument[spaces-perfect-]{spaces-perfect}
\externaldocument[spaces-more-morphisms-]{spaces-more-morphisms}
\externaldocument[spaces-flat-]{spaces-flat}
\externaldocument[spaces-groupoids-]{spaces-groupoids}
\externaldocument[spaces-more-groupoids-]{spaces-more-groupoids}
\externaldocument[bootstrap-]{bootstrap}
\externaldocument[spaces-pushouts-]{spaces-pushouts}
\externaldocument[spaces-chow-]{spaces-chow}
\externaldocument[groupoids-quotients-]{groupoids-quotients}
\externaldocument[spaces-more-cohomology-]{spaces-more-cohomology}
\externaldocument[spaces-simplicial-]{spaces-simplicial}
\externaldocument[spaces-duality-]{spaces-duality}
\externaldocument[formal-spaces-]{formal-spaces}
\externaldocument[restricted-]{restricted}
\externaldocument[spaces-resolve-]{spaces-resolve}
\externaldocument[formal-defos-]{formal-defos}
\externaldocument[defos-]{defos}
\externaldocument[cotangent-]{cotangent}
\externaldocument[examples-defos-]{examples-defos}
\externaldocument[algebraic-]{algebraic}
\externaldocument[examples-stacks-]{examples-stacks}
\externaldocument[stacks-sheaves-]{stacks-sheaves}
\externaldocument[criteria-]{criteria}
\externaldocument[artin-]{artin}
\externaldocument[quot-]{quot}
\externaldocument[stacks-properties-]{stacks-properties}
\externaldocument[stacks-morphisms-]{stacks-morphisms}
\externaldocument[stacks-limits-]{stacks-limits}
\externaldocument[stacks-cohomology-]{stacks-cohomology}
\externaldocument[stacks-perfect-]{stacks-perfect}
\externaldocument[stacks-introduction-]{stacks-introduction}
\externaldocument[stacks-more-morphisms-]{stacks-more-morphisms}
\externaldocument[stacks-geometry-]{stacks-geometry}
\externaldocument[moduli-]{moduli}
\externaldocument[moduli-curves-]{moduli-curves}
\externaldocument[examples-]{examples}
\externaldocument[exercises-]{exercises}
\externaldocument[guide-]{guide}
\externaldocument[desirables-]{desirables}
\externaldocument[coding-]{coding}
\externaldocument[obsolete-]{obsolete}
\externaldocument[fdl-]{fdl}
\externaldocument[index-]{index}

% Theorem environments.
%
\theoremstyle{plain}
\newtheorem{theorem}[subsection]{Theorem}
\newtheorem{proposition}[subsection]{Proposition}
\newtheorem{lemma}[subsection]{Lemma}

\theoremstyle{definition}
\newtheorem{definition}[subsection]{Definition}
\newtheorem{example}[subsection]{Example}
\newtheorem{exercise}[subsection]{Exercise}
\newtheorem{situation}[subsection]{Situation}

\theoremstyle{remark}
\newtheorem{remark}[subsection]{Remark}
\newtheorem{remarks}[subsection]{Remarks}

\numberwithin{equation}{subsection}

% Macros
%
\def\lim{\mathop{\mathrm{lim}}\nolimits}
\def\colim{\mathop{\mathrm{colim}}\nolimits}
\def\Spec{\mathop{\mathrm{Spec}}}
\def\Hom{\mathop{\mathrm{Hom}}\nolimits}
\def\Ext{\mathop{\mathrm{Ext}}\nolimits}
\def\SheafHom{\mathop{\mathcal{H}\!\mathit{om}}\nolimits}
\def\SheafExt{\mathop{\mathcal{E}\!\mathit{xt}}\nolimits}
\def\Sch{\mathit{Sch}}
\def\Mor{\mathop{\mathrm{Mor}}\nolimits}
\def\Ob{\mathop{\mathrm{Ob}}\nolimits}
\def\Sh{\mathop{\mathit{Sh}}\nolimits}
\def\NL{\mathop{N\!L}\nolimits}
\def\CH{\mathop{\mathrm{CH}}\nolimits}
\def\proetale{{pro\text{-}\acute{e}tale}}
\def\etale{{\acute{e}tale}}
\def\QCoh{\mathit{QCoh}}
\def\Ker{\mathop{\mathrm{Ker}}}
\def\Im{\mathop{\mathrm{Im}}}
\def\Coker{\mathop{\mathrm{Coker}}}
\def\Coim{\mathop{\mathrm{Coim}}}

% Boxtimes
%
\DeclareMathSymbol{\boxtimes}{\mathbin}{AMSa}{"02}

%
% Macros for moduli stacks/spaces
%
\def\QCohstack{\mathcal{QC}\!\mathit{oh}}
\def\Cohstack{\mathcal{C}\!\mathit{oh}}
\def\Spacesstack{\mathcal{S}\!\mathit{paces}}
\def\Quotfunctor{\mathrm{Quot}}
\def\Hilbfunctor{\mathrm{Hilb}}
\def\Curvesstack{\mathcal{C}\!\mathit{urves}}
\def\Polarizedstack{\mathcal{P}\!\mathit{olarized}}
\def\Complexesstack{\mathcal{C}\!\mathit{omplexes}}
% \Pic is the operator that assigns to X its picard group, usage \Pic(X)
% \Picardstack_{X/B} denotes the Picard stack of X over B
% \Picardfunctor_{X/B} denotes the Picard functor of X over B
\def\Pic{\mathop{\mathrm{Pic}}\nolimits}
\def\Picardstack{\mathcal{P}\!\mathit{ic}}
\def\Picardfunctor{\mathrm{Pic}}
\def\Deformationcategory{\mathcal{D}\!\mathit{ef}}


% OK, start here.
%
\begin{document}

\title{Functors and Morphisms}


\maketitle

\phantomsection
\label{section-phantom}

\tableofcontents

\section{Introduction}
\label{section-introduction}

\noindent
Let $X$ and $Y$ be schemes. This chapter circles around the relationship
between functors $\QCoh(\mathcal{O}_Y) \to \QCoh(\mathcal{O}_X)$ and
morphisms of schemes $X \to Y$. More broadly speaking we study the
relationship between $\QCoh(\mathcal{O}_X)$ and $X$ or, if $X$ is Noetherian,
the relationship between $\textit{Coh}(\mathcal{O}_X)$ and $X$.
This relationship was studied in \cite{Gabriel}.






\section{Functors on module categories}
\label{section-preliminary}

\noindent
For a ring $A$ let us denote $\text{Mod}^{fp}_A$ the category of
finitely presented $A$-modules.

\begin{lemma}
\label{lemma-functor-on-fp-modules}
Let $A$ be a ring. Let $\mathcal{B}$ be a category having filtered
colimits. Let $F : \text{Mod}^{fp}_A \to \mathcal{B}$ be a functor. Then $F$
extends uniquely to a functor $F' : \text{Mod}_A \to \mathcal{B}$
which commutes with filtered colimits.
\end{lemma}

\begin{proof}
This follows from
Categories, Lemma \ref{categories-lemma-extend-functor-by-colim}.
To see that the lemma applies observe that
finitely presented $A$-modules are
categorically compact objects of $\text{Mod}_A$ by
Algebra, Lemma \ref{algebra-lemma-characterize-finitely-presented-module-hom}.
Also, every $A$-module is a filtered colimit
of finitely presented $A$-modules by
Algebra, Lemma \ref{algebra-lemma-module-colimit-fp}.
\end{proof}

\noindent
If a category $\mathcal{B}$ is additive and has filtered colimits,
then $\mathcal{B}$ has arbitrary direct sums: any direct sum can be written
as a filtered colimit of finite direct sums.

\begin{lemma}
\label{lemma-functor-on-fp-modules-additive}
Let $A$, $\mathcal{B}$, $F$ be as in Lemma \ref{lemma-functor-on-fp-modules}.
Assume $\mathcal{B}$ is additive and $F$ is additive. Then
$F'$ is additive and commutes with arbitrary direct sums.
\end{lemma}

\begin{proof}
To show that $F'$ is additive it suffices to show
that $F'(M) \oplus F'(M') \to F'(M \oplus M')$ is an isomorphism for
any $A$-modules $M$, $M'$, see
Homology, Lemma \ref{homology-lemma-additive-functor}.
Write $M = \colim_i M_i$ and $M' = \colim_j M'_j$ as filtered colimits
of finitely presented $A$-modules $M_i$. Then
$F'(M) = \colim_i F(M_i)$, $F'(M') = \colim_j F(M'_j)$, and
\begin{align*}
F'(M \oplus M')
& =
F'(\colim_{i, j} M_i \oplus M'_j) \\
& =
\colim_{i, j} F(M_i \oplus M'_j) \\
& =
\colim_{i, j} F(M_i) \oplus F(M'_j) \\
& =
F'(M) \oplus F'(M')
\end{align*}
as desired. To show that $F'$ commutes with direct sums, assume
we have $M = \bigoplus_{i \in I} M_i$. Then
$M = \colim_{I' \subset I\text{ finite}} \bigoplus_{i \in I'} M_i$
is a filtered colimit. We obtain
\begin{align*}
F'(M)
& =
\colim_{I' \subset I\text{ finite}}
F'(\bigoplus\nolimits_{i \in I'} M_i) \\
& =
\colim_{I' \subset I\text{ finite}}
\bigoplus\nolimits_{i \in I'} F'(M_i) \\
& =
\bigoplus\nolimits_{i \in I} F'(M_i)
\end{align*}
The second equality holds by the additivity of $F'$ already shown.
\end{proof}

\noindent
If a category $\mathcal{B}$ is additive, has filtered colimits, and
has cokernels, then $\mathcal{B}$ has arbitrary colimits, see
discussion above and Categories, Lemma
\ref{categories-lemma-colimits-coproducts-coequalizers}.

\begin{lemma}
\label{lemma-functor-on-fp-modules-right-exact}
Let $A$, $\mathcal{B}$, $F$ be as in Lemma \ref{lemma-functor-on-fp-modules}.
Assume $\mathcal{B}$ is additive, has cokernels, and $F$ is right exact. Then
$F'$ is additive, right exact, and commutes with arbitrary direct sums.
\end{lemma}

\begin{proof}
Since $F$ is right exact, $F$ commutes with coproducts of pairs, which are
represented by direct sums. Hence $F$ is additive by
Homology, Lemma \ref{homology-lemma-additive-functor}.
Hence $F'$ is additive and commutes with direct sums by
Lemma \ref{lemma-functor-on-fp-modules-additive}.
We urge the reader to prove that $F'$ is right exact themselves
instead of reading the proof below.

\medskip\noindent
To show that $F'$ is right exact, it suffices to show that $F'$ commutes
with coequalizers, see
Categories, Lemma \ref{categories-lemma-characterize-right-exact}.
Now, if $a, b : K \to L$ are maps of $A$-modules, then the
coequalizer of $a$ and $b$ is the cokernel of $a - b : K \to L$.
Thus let $K \to L \to M \to 0$ be an exact sequence
of $A$-modules. We have to show that in
$$
F'(K) \to F'(L) \to F'(M) \to 0
$$
the second arrow is a cokernel for the first arrow in $\mathcal{B}$
(if $\mathcal{B}$ were abelian we would say that the displayed sequence
is exact).
Write $M = \colim_{i \in I} M_i$ as a filtered colimit of finitely
presented $A$-modules, see
Algebra, Lemma \ref{algebra-lemma-module-colimit-fp}.
Let $L_i = L \times_M M_i$. 
We obtain a system of exact sequences $K \to L_i \to M_i \to 0$ over $I$.
Since colimits commute with colimits by
Categories, Lemma \ref{categories-lemma-colimits-commute}
and since cokernels are a type of coequalizer,
it suffices to show that $F'(L_i) \to F(M_i)$ is a cokernel of
$F'(K) \to F'(L_i)$ in $\mathcal{B}$ for all $i \in I$. In other words, we may
assume $M$ is finitely presented. Write $L = \colim_{i \in I} L_i$
as a filtered colimit of finitely presented $A$-modules
with the property that each $L_i$ surjects onto $M$.
Let $K_i = K \times_L L_i$. We obtain a system of short exact sequences
$K_i \to L_i \to M \to 0$ over $I$.
Repeating the argument already given, we reduce to showing
$F(L_i) \to F(M_i)$ is a cokernel of
$F'(K) \to F(L_i)$ in $\mathcal{B}$ for all $i \in I$.
In other words, we may assume both
$L$ and $M$ are finitely presented $A$-modules.
In this case the module $\Ker(L \to M)$ is finite
(Algebra, Lemma \ref{algebra-lemma-extension}).
Thus we can write $K = \colim_{i \in I} K_i$ as a filtered colimit
of finitely presented $A$-modules each surjecting onto $\Ker(L \to M)$.
We obtain a system of short exact sequences
$K_i \to L \to M \to 0$ over $I$.
Repeating the argument already given, we reduce to showing
$F(L) \to F(M)$ is a cokernel of
$F(K_i) \to F(L)$ in $\mathcal{B}$ for all $i \in I$.
In other words, we may assume $K$, $L$, and $M$
are finitely presented $A$-modules. This final case follows
from the assumption that $F$ is right exact.
\end{proof}

\noindent
If a category $\mathcal{B}$ is additive and has kernels,
then $\mathcal{B}$ has finite limits. Namely, finite products
are direct sums which exist and the equalizer of $a, b : L \to M$
is the kernel of $a - b : K \to L$ which exists. Thus all finite
limits exist by Categories, Lemma \ref{categories-lemma-finite-limits-exist}.

\begin{lemma}
\label{lemma-functor-on-fp-modules-left-exact}
Let $A$, $\mathcal{B}$, $F$ be as in Lemma \ref{lemma-functor-on-fp-modules}.
Assume $A$ is a coherent ring (Algebra, Definition
\ref{algebra-definition-coherent}), $\mathcal{B}$ is additive, has kernels,
filtered colimits commute with taking kernels, and $F$ is left exact. Then
$F'$ is additive, left exact, and commutes with arbitrary direct sums.
\end{lemma}

\begin{proof}
Since $A$ is coherent, the category $\text{Mod}^{fp}_A$ is abelian
with same kernels and cokernels as in $\text{Mod}_A$, see
Algebra, Lemmas \ref{algebra-lemma-coherent-ring} and
\ref{algebra-lemma-coherent}. Hence all finite limits exist in
$\text{Mod}^{fp}_A$ and
Categories, Definition \ref{categories-definition-exact} applies.
Since $F$ is left exact, $F$ commutes with products of pairs, which are
represented by direct sums. Hence $F$ is additive by
Homology, Lemma \ref{homology-lemma-additive-functor}.
Hence $F'$ is additive and commutes with direct sums by
Lemma \ref{lemma-functor-on-fp-modules-additive}.
We urge the reader to prove that $F'$ is left exact themselves
instead of reading the proof below.

\medskip\noindent
To show that $F'$ is left exact, it suffices to show that $F'$ commutes
with equalizers, see
Categories, Lemma \ref{categories-lemma-characterize-left-exact}.
Now, if $a, b : L \to M$ are maps of $A$-modules, then the
equalizer of $a$ and $b$ is the kernel of $a - b : L \to M$.
Thus let $0 \to K \to L \to M$ be an exact sequence
of $A$-modules. We have to show that in
$$
0 \to F'(K) \to F'(L) \to F'(M)
$$
the arrow $F'(K) \to F'(L)$ is a kernel for $F'(L) \to F'(M)$ in $\mathcal{B}$
(if $\mathcal{B}$ were abelian we would say that the displayed sequence
is exact).
Write $M = \colim_{i \in I} M_i$ as a filtered colimit of finitely
presented $A$-modules, see
Algebra, Lemma \ref{algebra-lemma-module-colimit-fp}.
Let $L_i = L \times_M M_i$. 
We obtain a system of exact sequences $0 \to K \to L_i \to M_i$
over $I$. Since filtered colimits commute with taking kernels
in $\mathcal{B}$ by assumption, 
it suffices to show that $F'(K) \to F'(L_i)$ is a kernel of
$F'(L_i) \to F(M_i)$ in $\mathcal{B}$ for all $i \in I$. In other words, we may
assume $M$ is finitely presented. Write $L = \colim_{i \in I} L_i$
as a filtered colimit of finitely presented $A$-modules.
Let $K_i = K \times_L L_i$. We obtain a system of short exact sequences
$0 \to K_i \to L_i \to M$ over $I$.
Repeating the argument already given, we reduce to showing
$F'(K_i) \to F(L_i)$ is a kernel of
$F(L_i) \to F(M)$ in $\mathcal{B}$ for all $i \in I$.
In other words, we may assume both
$L$ and $M$ are finitely presented $A$-modules.
Since $A$ is coherent, the $A$-module $K = \Ker(L \to M)$ is
of finite presentation as the category of finitely presented
$A$-modules is abelian (see references given above).
In other words, all three modules $K$, $L$, and $M$
are finitely presented $A$-modules. This final case follows
from the assumption that $F$ is left exact.
\end{proof}

\noindent
If a category $\mathcal{B}$ is additive and has cokernels,
then $\mathcal{B}$ has finite colimits. Namely, finite coproducts
are direct sums which exist and the coequalizer of $a, b : K \to L$
is the cokernel of $a - b : K \to L$ which exists. Thus all finite
colimits exist by Categories, Lemma \ref{categories-lemma-colimits-exist}.

\begin{lemma}
\label{lemma-functor-on-modules-fp}
Let $A$ be a ring. Let $\mathcal{B}$ be an additive category
with cokernels. There is an equivalence of categories between
\begin{enumerate}
\item the category of functors $F : \text{Mod}^{fp}_A \to \mathcal{B}$
which are right exact, and
\item the category of pairs $(K, \kappa)$ where $K \in \Ob(\mathcal{B})$
and $\kappa : A \to \text{End}_\mathcal{B}(K)$ is a ring homomorphism
\end{enumerate}
given by the rule sending $F$ to $F(A)$ with its natural $A$-action.
\end{lemma}

\begin{proof}
Let $(K, \kappa)$ be as in (2). We will construct a functor
$F : \text{Mod}^{fp}_A \to \mathcal{B}$ such that $F(A) = K$
endowed with the given $A$-action $\kappa$. Namely, given an
integer $n \geq 0$ let us set
$$
F(A^{\oplus n}) = K^{\oplus n}
$$
Given an $A$-linear map $\varphi : A^{\oplus m} \to A^{\oplus n}$
with matrix $(a_{ij}) \in \text{Mat}(n \times m, A)$ we define
$$
F(\varphi) :
F(A^{\oplus m}) = K^{\oplus m}
\longrightarrow
K^{\oplus n} = F(A^{\oplus n})
$$
to be the map with matrix $(\kappa(a_{ij}))$. This defines an additive
functor $F$ from the full subcategory of
$\text{Mod}^{fp}_A$ with objects $0$, $A$, $A^{\oplus 2}$, $\ldots$
to $\mathcal{B}$; we omit the verification.

\medskip\noindent
For each object $M$ of $\text{Mod}^{fp}_A$ choose a presentation
$$
A^{\oplus m_M} \xrightarrow{\varphi_M} A^{\oplus n_M} \to M \to 0
$$
of $M$ as an $A$-module. Let us use the trivial presentation
$0 \to A^{\oplus n} \xrightarrow{1} A^{\oplus n} \to 0$ if $M = A^{\oplus n}$
(this isn't necessary but simplifies the exposition).
For each morphism $f : M \to N$ of
$\text{Mod}^{fp}_A$ we can choose a commutative diagram
\begin{equation}
\label{equation-map}
\vcenter{
\xymatrix{
A^{\oplus m_M} \ar[r]_{\varphi_M} \ar[d]_{\psi_f} &
A^{\oplus n_M} \ar[r] \ar[d]_{\chi_f} &
M \ar[r] \ar[d]_f & 0 \\
A^{\oplus m_N} \ar[r]^{\varphi_N} &
A^{\oplus n_N} \ar[r] &
N \ar[r] & 0
}
}
\end{equation}
Having made these choices we can define: for an object
$M$ of $\text{Mod}^{fp}_A$ we set
$$
F(M) = \Coker(F(\varphi_M) : F(A^{\oplus m_M}) \to F(A^{\oplus n_M}))
$$
and for a morphism $f : M \to N$ of $\text{Mod}^{fp}_A$ we set
$$
F(f) = \text{the map }F(M) \to F(N)\text{ induced by }
F(\psi_f)\text{ and }F(\chi_f)\text{ on cokernels}
$$
Note that this rule extends the given functor $F$ on
the full subcategory consisting of the free modules $A^{\oplus n}$.
We still have to show that $F$ is a functor, that $F$ is additive,
and that $F$ is right exact.

\medskip\noindent
Let $f : M \to N$ be a morphism $\text{Mod}^{fp}_A$. We claim that the map
$F(f)$ defined above is independent of the choices of $\psi_f$ and $\chi_f$
in (\ref{equation-map}). Namely, say
$$
\xymatrix{
A^{\oplus m_M} \ar[r]_{\varphi_M} \ar[d]_\psi &
A^{\oplus n_M} \ar[r] \ar[d]_\chi &
M \ar[r] \ar[d]_f & 0 \\
A^{\oplus m_N} \ar[r]^{\varphi_N} &
A^{\oplus n_N} \ar[r] &
N \ar[r] & 0
}
$$
is also commutative. Denote $F(f)' : F(M) \to F(N)$ the map
induced by $F(\psi)$ and $F(\chi)$. Looking at the commutative
diagrams, by elementary commutative algebra there exists a map
$\omega : A^{\oplus n_M} \to A^{\oplus m_N}$ such that
$\chi = \chi_f + \varphi_N \circ \omega$. Applying $F$ we
find that $F(\chi) = F(\chi_f) + F(\varphi_N) \circ F(\omega)$.
As $F(N)$ is the cokernel of $F(\varphi_N)$ we find
that the map $F(A^{\oplus n_M}) \to F(M)$ equalizes $F(f)$ and $F(f)'$.
Since a cokernel is an epimorphism, we conclude that $F(f) = F(f)'$.

\medskip\noindent
Let us prove $F$ is a functor. First, observe that
$F(\text{id}_M) = \text{id}_{F(M)}$ because we may pick
the identities for $\psi_f$ and $\chi_f$ in the diagram above
in case $f = \text{id}_M$. Second, suppose we have
$f : M \to N$ and $g : L \to M$. Then we see that
$\psi = \psi_f \circ \psi_g$ and $\chi = \chi_f \circ \chi_g$
fit into (\ref{equation-map}) for $f \circ g$.
Hence these induce the correct map which exactly
says that $F(f) \circ F(g) = F(f \circ g)$.

\medskip\noindent
Let us prove that $F$ is additive. Namely, suppose we have
$f, g : M \to N$. Then we see that $\psi = \psi_f + \psi_g$ and
$\chi = \chi_f + \chi_g$ fit into (\ref{equation-map}) for $f + g$.
Hence these induce the correct map which exactly
says that $F(f) + F(g) = F(f + g)$.

\medskip\noindent
Finally, let us prove that $F$ is right exact. It suffices to show that $F$
commutes with coequalizers, see
Categories, Lemma \ref{categories-lemma-characterize-right-exact}.
For this, it suffices to prove that $F$ commutes with cokernels.
Let $K \to L \to M \to 0$ be an exact sequence of $A$-modules
with $K$, $L$, $M$ finitely presented. Since $F$ is an additive
functor, this certainly gives a complex
$$
F(K) \to F(L) \to F(M) \to 0
$$
and we have to show that the second arrow is the cokernel of the first
in $\mathcal{B}$. In any case, we obtain a map
$\Coker(F(K) \to F(L)) \to F(M)$.
By elementary commutative algebra there exists a commutative diagram
$$
\xymatrix{
A^{\oplus m_M} \ar[r]_{\varphi_M} \ar[d]_\psi &
A^{\oplus n_M} \ar[r] \ar[d]_\chi &
M \ar[r] \ar[d]_1 & 0 \\
K \ar[r] &
L \ar[r] &
M \ar[r] & 0
}
$$
Applying $F$ to this diagram and using the construction of $F(M)$ as
the cokernel of $F(\varphi_M)$ we find there exists a map
$F(M) \to \Coker(F(K) \to F(L))$ which is a right inverse
to the map $\Coker(F(K) \to F(L)) \to F(M)$. This first implies
that $F(L) \to F(M)$ is an epimorphism always. Next, the above shows
we have
$$
\Coker(F(K) \to F(L)) = F(M) \oplus E
$$
where the direct sum decomposition is compatible with both
$F(M) \to \Coker(F(K) \to F(L))$ and $\Coker(F(K) \to F(L)) \to F(M)$.
However, then the epimorphism $p : F(L) \to E$ becomes zero both
after composition with $F(K) \to F(L)$ and after composition
with $F(A^{n_M}) \to F(L)$. However, since $K \oplus A^{n_M} \to L$
is surjective (algebra argument omitted), we conclude that
$F(K \oplus A^{n_M}) \to F(L)$ is an epimorphism (by the above)
whence $E = 0$. This finishes the proof.
\end{proof}

\begin{lemma}
\label{lemma-functor-on-modules}
Let $A$ be a ring. Let $\mathcal{B}$ be an additive category
with arbitrary direct sums and cokernels. There is an equivalence
of categories between
\begin{enumerate}
\item the category of functors $F : \text{Mod}_A \to \mathcal{B}$
which are right exact and commute with arbitrary direct sums, and
\item the category of pairs $(K, \kappa)$ where $K \in \Ob(\mathcal{B})$
and $\kappa : A \to \text{End}_\mathcal{B}(K)$ is a ring homomorphism
\end{enumerate}
given by the rule sending $F$ to $F(A)$ with its natural $A$-action.
\end{lemma}

\begin{proof}
Combine Lemmas \ref{lemma-functor-on-modules-fp} and
\ref{lemma-functor-on-fp-modules-right-exact}.
\end{proof}






\section{Functors between categories of modules}
\label{section-functors}

\noindent
The following lemma is archetypical of the results in this chapter.

\begin{lemma}
\label{lemma-functor}
Let $A$ and $B$ be rings. Let $F : \text{Mod}_A \to \text{Mod}_B$
be a functor. The following are equivalent
\begin{enumerate}
\item $F$ is isomorphic to the functor $M \mapsto M \otimes_A K$
for some $A \otimes_\mathbf{Z} B$-module $K$,
\item $F$ is right exact and commutes with all direct sums,
\item $F$ commutes with all colimits,
\item $F$ has a right adjoint $G$.
\end{enumerate}
\end{lemma}

\begin{proof}
If (1), then (4) as a right adjoint for $M \mapsto M \otimes_A K$
is $N \mapsto \Hom_B(K, N)$, see
Differential Graded Algebra, Lemma \ref{dga-lemma-tensor-hom-adjunction}.
If (4), then (3) by Categories, Lemma \ref{categories-lemma-adjoint-exact}.
The implication (3) $\Rightarrow$ (2) is immediate from the definitions.

\medskip\noindent
Assume (2). We will prove (1). By the discussion in
Homology, Section \ref{homology-section-functors}
the functor $F$ is additive. Hence $F$ induces
a ring map $A \to \text{End}_B(F(M))$, $a \mapsto F(a \cdot \text{id}_M)$
for every $A$-module $M$. We conclude that $F(M)$ is an
$A \otimes_\mathbf{Z} B$-module functorially in $M$.
Set $K = F(A)$. Define
$$
M \otimes_A K = M \otimes_A F(A) \longrightarrow F(M),
\quad m \otimes k \longmapsto F(\varphi_m)(k)
$$
Here $\varphi_m : A \to M$ sends $a \to am$. The rule
$(m, k) \mapsto F(\varphi_m)(k)$ is $A$-bilinear (and $B$-linear
on the right) as required to obtain the displayed
$A \otimes_\mathbf{Z} B$-linear map.
This construction is functorial in $M$, hence defines a transformation
of functors $- \otimes_A K \to F(-)$ which is an isomorphism when
evaluated on $A$. For every $A$-module $M$ we can choose an exact sequence
$$
\bigoplus\nolimits_{j \in J} A \to
\bigoplus\nolimits_{i \in I} A \to
M \to 0
$$
Using the maps constructed above we find a commutative diagram
$$
\xymatrix{
(\bigoplus\nolimits_{j \in J} A) \otimes_A K \ar[r] \ar[d] &
(\bigoplus\nolimits_{i \in I} A) \otimes_A K \ar[r] \ar[d] &
M \otimes_A K \ar[r] \ar[d] &
0 \\
F(\bigoplus\nolimits_{j \in J} A) \ar[r] &
F(\bigoplus\nolimits_{i \in I} A) \ar[r] &
F(M) \ar[r] &  0
}
$$
The lower row is exact as $F$ is right exact.
The upper row is exact as tensor product with $K$ is right exact.
Since $F$ commutes with direct sums the left two vertical arrows
are bijections. Hence we conclude.
\end{proof}

\begin{example}
\label{example-functor-modules}
Let $R$ be a ring. Let $A$ and $B$ be $R$-algebras. Let $K$ be a
$A \otimes_R B$-module. Then we can consider the functor
\begin{equation}
\label{equation-FM-modules}
F : \text{Mod}_A \longrightarrow \text{Mod}_B,\quad
M \longmapsto M \otimes_A K
\end{equation}
This functor is $R$-linear, right exact,
commutes with arbitrary direct sums, commutes
with all colimits, has a right adjoint (Lemma \ref{lemma-functor}).
\end{example}

\begin{lemma}
\label{lemma-functor-modules}
Let $R$ be a ring. Let $A$ and $B$ be $R$-algebras. There is an
equivalence of categories between
\begin{enumerate}
\item the category of $R$-linear functors
$F : \text{Mod}_A \to \text{Mod}_B$ which
are right exact and commute with arbitrary direct sums, and
\item the category $\text{Mod}_{A \otimes_R B}$.
\end{enumerate}
given by sending $K$ to the functor $F$ in (\ref{equation-FM-modules}).
\end{lemma}

\begin{proof}
Let $F$ be an object of the first category. By
Lemma \ref{lemma-functor} we may assume $F(M) = M \otimes_A K$
functorially in $M$ for some $A \otimes_\mathbf{Z} B$-module $K$.
The $R$-linearity of $F$ immediately implies that the
$A \otimes_\mathbf{Z} B$-module structure on $K$ comes
from a (unique) $A \otimes_R B$-module structure on $K$.
Thus we see that sending $K$ to $F$ as in (\ref{equation-FM-modules})
is essentially surjective.

\medskip\noindent
To prove that our functor is fully faithful, we have to show that
given $A \otimes_R B$-modules $K$ and $K'$ any transformation
$t : F \to F'$ between the corresponding functors, comes from
a unique $\varphi : K \to K'$. Since $K = F(A)$ and $K' = F'(A)$
we can take $\varphi$ to be the value $t_A : F(A) \to F'(A)$
of $t$ at $A$. This maps is $A \otimes_R B$-linear by the
definition of the $A \otimes B$-module structure on $F(A)$
and $F'(A)$ given in the proof of Lemma \ref{lemma-functor}.
\end{proof}

\begin{remark}
\label{remark-composition}
Let $R$ be a ring. Let $A$, $B$, $C$ be $R$-algebras.
Let $F : \text{Mod}_A \to \text{Mod}_B$ and
$F' : \text{Mod}_B \to \text{Mod}_C$ be
$R$-linear, right exact functors which commute with arbitrary direct sums.
If by the equivalence of Lemma \ref{lemma-functor-modules} the object
$K$ in $\text{Mod}_{A \otimes_R B}$ corresponds to $F$ and the object
$K'$ in $\text{Mod}_{B \otimes_R C}$ corresponds to $F'$, then
$K \otimes_B K'$ viewed as an object of
$\text{Mod}_{A \otimes_R C}$ corresponds to $F' \circ F$.
\end{remark}

\begin{remark}
\label{remark-exact-flat}
In the situation of Lemma \ref{lemma-functor-modules}
suppose that $F$ corresponds to $K$. Then
$F$ is exact $\Leftrightarrow$ $K$ is flat over $A$.
\end{remark}

\begin{remark}
\label{remark-finite}
In the situation of Lemma \ref{lemma-functor-modules}
suppose that $F$ corresponds to $K$. Then
$F$ sends finite $A$-modules to finite $B$-modules
$\Leftrightarrow$ $K$ is finite as a $B$-module.
\end{remark}

\begin{remark}
\label{remark-finite-presentation}
In the situation of Lemma \ref{lemma-functor-modules}
suppose that $F$ corresponds to $K$. Then
$F$ sends finitely presented $A$-modules to finitely presented $B$-modules
$\Leftrightarrow$ $K$ is finitely presented as a $B$-module.
\end{remark}

\begin{lemma}
\label{lemma-functor-equivalence}
Let $A$ and $B$ be rings. If
$$
F : \text{Mod}_A \longrightarrow \text{Mod}_B
$$
is an equivalence of categories, then there exists an isomorphism
$A \to B$ of rings and an invertible $B$-module $L$ such that
$F$ is isomorphic to the functor $M \mapsto (M \otimes_A B) \otimes_B L$.
\end{lemma}

\begin{proof}
Since an equivalence commutes with all colimits, we see that
Lemmas \ref{lemma-functor} applies. Let $K$ be the
$A \otimes_\mathbf{Z} B$-module such that $F$ is
isomorphic to the functor $M \mapsto M \otimes_A K$.
Let $K'$ be the $B \otimes_\mathbf{Z} A$-module such that
a quasi-inverse of $F$ is
isomorphic to the functor $N \mapsto N \otimes_B K'$.
By Remark \ref{remark-composition} and
Lemma \ref{lemma-functor-modules} we have an isomorphism
$$
\psi : K \otimes_B K' \longrightarrow A
$$
of $A \otimes_\mathbf{Z} A$-modules.
Similarly, we have an isomorphism
$$
\psi' : K' \otimes_A K \longrightarrow B
$$
of $B \otimes_\mathbf{Z} B$-modules. Choose an element
$\xi = \sum_{i = 1, \ldots, n} x_i \otimes y_i \in K \otimes_B K'$
such that $\psi(\xi) = 1$. Consider the isomorphisms
$$
K \xrightarrow{\psi^{-1} \otimes \text{id}_K}
K \otimes_B K' \otimes_A K \xrightarrow{\text{id}_K \otimes \psi'} K
$$
The composition is an isomorphism and given by
$$
k \longmapsto \sum x_i \psi'(y_i \otimes k)
$$
We conclude this automorphism factors as
$$
K \to B^{\oplus n} \to K
$$
as a map of $B$-modules. It follows that $K$ is finite
projective as a $B$-module.

\medskip\noindent
We claim that $K$ is invertible as a $B$-module. This is equivalent
to asking the rank of $K$ as a $B$-module to have the constant value $1$,
see More on Algebra, Lemma \ref{more-algebra-lemma-invertible} and
Algebra, Lemma \ref{algebra-lemma-finite-projective}.
If not, then there exists a maximal ideal $\mathfrak m \subset B$
such that either (a) $K \otimes_B B/\mathfrak m = 0$ or
(b) there is a surjection $K \to (B/\mathfrak m)^{\oplus 2}$ of
$B$-modules. Case (a) is absurd as $K' \otimes_A K \otimes_B N = N$
for all $B$-modules $N$. Case (b) would imply we get a surjection
$$
A = K \otimes_B K' \longrightarrow (B/\mathfrak m \otimes_B K')^{\oplus 2}
$$
of (right) $A$-modules. This is impossible as the target is an $A$-module
which needs at least two generators: $B/\mathfrak m \otimes_B K'$
is nonzero as the image of the nonzero module $B/\mathfrak m$ under
the quasi-inverse of $F$.

\medskip\noindent
Since $K$ is invertible as a $B$-module we see that $\Hom_B(K, K) = B$.
Since $K = F(A)$ the action of $A$ on $K$ defines a ring isomorphism
$A \to B$. The lemma follows.
\end{proof}

\begin{lemma}
\label{lemma-functor-equivalence-linear}
Let $R$ be a ring. Let $A$ and $B$ be $R$-algebras. If
$$
F : \text{Mod}_A \longrightarrow \text{Mod}_B
$$
is an $R$-linear equivalence of categories, then there exists an isomorphism
$A \to B$ of $R$-algebras and an invertible $B$-module $L$ such that
$F$ is isomorphic to the functor $M \mapsto (M \otimes_A B) \otimes_B L$.
\end{lemma}

\begin{proof}
We get $A \to B$ and $L$ from Lemma \ref{lemma-functor-equivalence}.
To finish the proof, we need to show that the $R$-linearity
of $F$ forces $A \to B$ to be an $R$-algebra map. We omit the details.
\end{proof}

\begin{remark}
\label{remark-monoidal}
Let $A$ and $B$ be rings. Let us endow $\text{Mod}_A$ and $\text{Mod}_B$
with the usual monoidal structure given by tensor products of modules.
Let $F : \text{Mod}_A \to \text{Mod}_B$ be a functor of
monoidal categories, see
Categories, Definition \ref{categories-definition-functor-monoidal-categories}.
Here are some comments:
\begin{enumerate}
\item Since $F(A)$ is a unit (by our definitions) we have $F(A) = B$.
\item We obtain a multiplicative map $\varphi : A \to B$
by sending $a \in A$ to its action on $F(A) = B$.
\item Take $A = B$ and $F(M) = M \otimes_A M$. In this case $\varphi(a) = a^2$.
\item If $F$ is additive, then $\varphi$ is a ring map.
\item Take $A = B = \mathbf{Z}$ and $F(M) = M/\text{torsion}$. Then
$\varphi = \text{id}_\mathbf{Z}$ but $F$ is not the identity functor.
\item If $F$ is right exact and commutes with direct sums,
then $F(M) = M \otimes_{A, \varphi} B$ by Lemma \ref{lemma-functor}.
\end{enumerate}
In other words, ring maps $A \to B$ are in bijection with isomorphism classes
of functors of monoidal categories $\text{Mod}_A \to \text{Mod}_B$
which commute with all colimits.
\end{remark}




\section{Extending functors on categories of modules}
\label{section-functors-extend}

\noindent
For a ring $A$ let us denote $\text{Mod}^{fp}_A$ the category of
finitely presented $A$-modules.

\begin{lemma}
\label{lemma-functor-fp-modules}
Let $A$ and $B$ be rings. Let
$F : \text{Mod}^{fp}_A \to \text{Mod}^{fp}_B$ be a functor.
Then $F$ extends uniquely to a functor
$F' : \text{Mod}_A \to \text{Mod}_B$
which commutes with filtered colimits.
\end{lemma}

\begin{proof}
Special case of Lemma \ref{lemma-functor-on-fp-modules}.
\end{proof}

\begin{remark}
\label{remark-monoidal-extension}
With $A$, $B$, $F$, and $F'$ as in Lemma \ref{lemma-functor-fp-modules}.
Observe that the tensor product of two finitely presented modules is
finitely presented, see Algebra, Lemma \ref{algebra-lemma-tensor-finiteness}.
Thus we may endow $\text{Mod}^{fp}_A$, $\text{Mod}^{fp}_B$,
$\text{Mod}_A$, and $\text{Mod}_B$ with the usual monoidal structure
given by tensor products of modules. In this case, if $F$ is
a functor of monoidal categories, so is $F'$. This follows immediately
from the fact that tensor products of modules commutes with filtered
colimits.
\end{remark}

\begin{lemma}
\label{lemma-functor-fp-modules-exact}
With $A$, $B$, $F$, and $F'$ as in Lemma \ref{lemma-functor-fp-modules}.
\begin{enumerate}
\item If $F$ is additive, then $F'$ is additive and
commutes with arbitrary direct sums, and
\item if $F$ is right exact, then $F'$ is right exact.
\end{enumerate}
\end{lemma}

\begin{proof}
Follows from Lemmas \ref{lemma-functor-on-fp-modules-additive} and
\ref{lemma-functor-on-fp-modules-right-exact}.
\end{proof}

\begin{remark}
\label{remark-monoidal-extension-exact}
Combining Remarks \ref{remark-monoidal} and \ref{remark-monoidal-extension}
and Lemma \ref{lemma-functor-fp-modules-exact}
we find the following. Given rings $A$ and $B$ the set of ring maps $A \to B$
is in bijection with the set of isomorphism classes
of functors of monoidal categories $\text{Mod}^{fp}_A \to \text{Mod}^{fp}_B$
which are right exact.
\end{remark}

\begin{lemma}
\label{lemma-functor-fp-modules-left-exact}
With $A$, $B$, $F$, and $F'$ as in Lemma \ref{lemma-functor-fp-modules}.
Assume $A$ is a coherent ring
(Algebra, Definition \ref{algebra-definition-coherent}).
If $F$ is left exact, then $F'$ is left exact.
\end{lemma}

\begin{proof}
Special case of Lemma \ref{lemma-functor-on-fp-modules-left-exact}.
\end{proof}

\noindent
For a ring $A$ let us denote $\text{Mod}^{fg}_A$ the category of
finitely generated $A$-modules (AKA finite $A$-modules).

\begin{lemma}
\label{lemma-functor-finite-modules}
Let $A$ and $B$ be Noetherian rings. Let
$F : \text{Mod}^{fg}_A \to \text{Mod}^{fg}_B$ be a functor.
Then $F$ extends uniquely to a functor $F' : \text{Mod}_A \to \text{Mod}_B$
which commutes with filtered colimits. If $F$ is additive, then
$F'$ is additive and commutes with arbitrary direct sums.
If $F$ is exact, left exact, or right exact, so is $F'$.
\end{lemma}

\begin{proof}
See Lemmas \ref{lemma-functor-fp-modules-exact} and
\ref{lemma-functor-fp-modules-left-exact}.
Also, use the finite $A$-modules are finitely presented $A$-modules,
see Algebra, Lemma
\ref{algebra-lemma-Noetherian-finite-type-is-finite-presentation},
and use that Noetherian rings are coherent, see
Algebra, Lemma \ref{algebra-lemma-Noetherian-coherent}.
\end{proof}









\section{Functors between categories of quasi-coherent modules}
\label{section-functor-quasi-coherent}

\noindent
In this section we briefly study functors between categories of
quasi-coherent modules.

\begin{example}
\label{example-functor-quasi-coherent}
Let $R$ be a ring. Let $X$ and $Y$ be
schemes over $R$ with $X$ quasi-compact and quasi-separated.
Let $\mathcal{K}$ be a quasi-coherent $\mathcal{O}_{X \times_R Y}$-module.
Then we can consider the functor
\begin{equation}
\label{equation-FM-QCoh}
F : \QCoh(\mathcal{O}_X) \longrightarrow \QCoh(\mathcal{O}_Y),\quad
\mathcal{F} \longmapsto
\text{pr}_{2, *}(\text{pr}_1^*\mathcal{F}
\otimes_{\mathcal{O}_{X \times_R Y}} \mathcal{K})
\end{equation}
The morphism $\text{pr}_2$ is quasi-compact and quasi-separated
(Schemes, Lemmas \ref{schemes-lemma-quasi-compact-preserved-base-change}
and \ref{schemes-lemma-separated-permanence}). Hence pushforward along
this morphism preserves quasi-coherent modules, see
Schemes, Lemma \ref{schemes-lemma-push-forward-quasi-coherent}.
Moreover, our functor is $R$-linear and commutes with arbitrary direct sums,
see Cohomology of Schemes, Lemma \ref{coherent-lemma-colimit-cohomology}.
\end{example}

\noindent
The following lemma is a natural generalization of
Lemma \ref{lemma-functor-modules}.

\begin{lemma}
\label{lemma-functor-quasi-coherent-from-affine}
Let $R$ be a ring. Let $X$ and $Y$ be schemes over $R$ with $X$ affine.
There is an equivalence of categories between
\begin{enumerate}
\item the category of $R$-linear functors
$F : \QCoh(\mathcal{O}_X) \to \QCoh(\mathcal{O}_Y)$
which are right exact and commute with arbitrary direct sums, and
\item the category $\QCoh(\mathcal{O}_{X \times_R Y})$
\end{enumerate}
given by sending $\mathcal{K}$ to the functor $F$ in (\ref{equation-FM-QCoh}).
\end{lemma}

\begin{proof}
Let $\mathcal{K}$ be an object of $\QCoh(\mathcal{O}_{X \times_R Y})$
and $F_\mathcal{K}$ the functor (\ref{equation-FM-QCoh}). By the discussion in
Example \ref{example-functor-quasi-coherent} we already know that
$F$ is $R$-linear and commutes with arbitrary direct sums.
Since $\text{pr}_2 : X \times_R Y \to Y$ is affine
(Morphisms, Lemma \ref{morphisms-lemma-base-change-affine}) the functor
$\text{pr}_{2, *}$ is exact, see Cohomology of Schemes, Lemma
\ref{coherent-lemma-relative-affine-vanishing}.
Hence $F$ is right exact as well, in other words $F$ is as in (1).

\medskip\noindent
Let $F$ be as in (1). Say $X = \Spec(A)$. Consider the quasi-coherent
$\mathcal{O}_Y$-module $\mathcal{G} = F(\mathcal{O}_X)$.
The functor $F$ induces an $R$-linear map
$A \to \text{End}_{\mathcal{O}_Y}(\mathcal{G})$,
$a \mapsto F(a \cdot \text{id})$. Thus $\mathcal{G}$ is a sheaf of modules over
$$
A \otimes_R \mathcal{O}_Y = \text{pr}_{2, *}\mathcal{O}_{X \times_R Y}
$$
By Morphisms, Lemma \ref{morphisms-lemma-affine-equivalence-modules}
we find that there is a unique quasi-coherent module $\mathcal{K}$
on $X \times_R Y$ such that $F(\mathcal{O}_X) = \mathcal{G} =
\text{pr}_{2, *}\mathcal{K}$ compatible with action
of $A$ and $\mathcal{O}_Y$. Denote $F_\mathcal{K}$ the functor
given by (\ref{equation-FM-QCoh}). There is an equivalence
$\text{Mod}_A \to \QCoh(\mathcal{O}_X)$ sending $A$ to $\mathcal{O}_X$, see
Schemes, Lemma \ref{schemes-lemma-equivalence-quasi-coherent}.
Hence we find an isomorphism $F \cong F_\mathcal{K}$ by
Lemma \ref{lemma-functor-on-modules} because we have an isomorphism
$F(\mathcal{O}_X) \cong F_\mathcal{K}(\mathcal{O}_X)$ compatible with
$A$-action by construction.

\medskip\noindent
This shows that the functor sending $\mathcal{K}$ to $F_\mathcal{K}$
is essentially surjective. We omit the verification of fully faithfulness.
\end{proof}

\begin{remark}
\label{remark-affine-morphism}
Below we will use that for an affine morphism
$h : T \to S$ we have $h_*\mathcal{G} \otimes_{\mathcal{O}_S} \mathcal{H} =
h_*(\mathcal{G} \otimes_{\mathcal{O}_T} h^*\mathcal{H})$ for
$\mathcal{G} \in \QCoh(\mathcal{O}_T)$ and
$\mathcal{H} \in \QCoh(\mathcal{O}_S)$. This follows
immediately on translating into algebra.
\end{remark}

\begin{lemma}
\label{lemma-functor-quasi-coherent-from-affine-compose}
In Lemma \ref{lemma-functor-quasi-coherent-from-affine} let $F$
correspond to $\mathcal{K}$ in $\QCoh(\mathcal{O}_{X \times_R Y})$.
We have
\begin{enumerate}
\item If $f : X' \to X$ is an affine morphism, then $F \circ f_*$
corresponds to $(f \times \text{id}_Y)^*\mathcal{K}$.
\item If $g : Y' \to Y$ is a flat morphism, then $g^* \circ F$ corresponds to
$(\text{id}_X \times g)^*\mathcal{K}$.
\item If $j : V \to Y$ is an open immersion, then $j^* \circ F$
corresponds to $\mathcal{K}|_{X \times_R V}$.
\end{enumerate}
\end{lemma}

\begin{proof}
Proof of (1). Consider the commutative diagram
$$
\xymatrix{
X' \times_R Y \ar[rrd]^{\text{pr}'_2} \ar[rd]_{f \times \text{id}_Y}
\ar[dd]_{\text{pr}'_1} \\
& X \times_R Y \ar[r]_{\text{pr}_2} \ar[d]_{\text{pr}_1} & Y \\
X' \ar[r]^f & X
}
$$
Let $\mathcal{F}'$ be a quasi-coherent module on $X'$. We have
\begin{align*}
\text{pr}_{2, *}(\text{pr}_1^*f_*\mathcal{F}'
\otimes_{\mathcal{O}_{X \times_R Y}} \mathcal{K})
& =
\text{pr}_{2, *}((f \times \text{id}_Y)_*
(\text{pr}'_1)^*\mathcal{F}'
\otimes_{\mathcal{O}_{X \times_R Y}} \mathcal{K}) \\
& =
\text{pr}_{2, *}(f \times \text{id}_Y)_*
\left((\text{pr}'_1)^*\mathcal{F}'
\otimes_{\mathcal{O}_{X' \times_R Y}}
(f \times \text{id}_Y)^*\mathcal{K})\right)  \\
& =
\text{pr}'_{2, *}((\text{pr}'_1)^*\mathcal{F}'
\otimes_{\mathcal{O}_{X' \times_R Y}} (f \times \text{id}_Y)^*\mathcal{K})
\end{align*}
Here the first equality is affine base change for the left hand
square in the diagram, see
Cohomology of Schemes, Lemma \ref{coherent-lemma-affine-base-change}.
The second equality hold by Remark \ref{remark-affine-morphism}.
The third equality is functoriality of pushforwards for modules.
This proves (1).

\medskip\noindent
Proof of (2). Consider the commutative diagram
$$
\xymatrix{
X \times_R Y' \ar[rr]_-{\text{pr}'_2} \ar[rd]^{\text{id}_X \times g}
\ar[rdd]_{\text{pr}'_1} & & Y' \ar[d]^g \\
& X \times_R Y \ar[r]_-{\text{pr}_2} \ar[d]^{\text{pr}_1} & Y \\
& X
}
$$
We have
\begin{align*}
g^*\text{pr}_{2, *}(\text{pr}_1^*\mathcal{F}
\otimes_{\mathcal{O}_{X \times_R Y}} \mathcal{K})
& =
\text{pr}'_{2, *}(
(\text{id}_X \times g)^*(
\text{pr}_1^*\mathcal{F} \otimes_{\mathcal{O}_{X \times_R Y}} \mathcal{K})) \\
& =
\text{pr}'_{2, *}((\text{pr}'_1)^*\mathcal{F}
\otimes_{\mathcal{O}_{X \times_R Y'}}
(\text{id}_X \times g)^*\mathcal{K})
\end{align*}
The first equality by flat base change for the square in the diagram, see
Cohomology of Schemes, Lemma \ref{coherent-lemma-flat-base-change-cohomology}.
The second equality by functoriality of pullback and the fact that
a pullback of tensor products it the tensor product of the pullbacks.

\medskip\noindent
Part (3) is a special case of (2).
\end{proof}

\begin{lemma}
\label{lemma-functor-quasi-coherent-from-affine-diagonal-pre}
Let $R$ be a ring. Let $X$ and $Y$ be schemes over $R$. Assume $X$
is quasi-compact with affine diagonal. Let
$F : \QCoh(\mathcal{O}_X) \to \QCoh(\mathcal{O}_Y)$
be an $R$-linear, right exact functor which commutes
with arbitrary direct sums. Then we can construct
\begin{enumerate}
\item a quasi-coherent module $\mathcal{K}$ on $X \times_R Y$, and
\item a natural transformation $t : F \to F_\mathcal{K}$
where $F_\mathcal{K}$ denotes the functor (\ref{equation-FM-QCoh})
\end{enumerate}
such that $t : F \circ f_* \to F_\mathcal{K} \circ f_*$ is an isomorphism
for every morphism $f : X' \to X$ whose source is an affine scheme.
\end{lemma}

\begin{proof}
Consider a morphism $f' : X' \to X$ with $X'$ affine. Since the
diagonal of $X$ is affine, we see that $f'$ is an affine morphism
(Morphisms, Lemma \ref{morphisms-lemma-affine-permanence}).
Thus $f'_* : \QCoh(\mathcal{O}_{X'}) \to \QCoh(\mathcal{O}_X)$
is an $R$-linear exact functor
(Cohomology of Schemes, Lemma \ref{coherent-lemma-relative-affine-vanishing})
which commutes with direct sums
(Cohomology of Schemes, Lemma \ref{coherent-lemma-colimit-cohomology}).
Thus $F \circ f'_*$ is an $R$-linear, right exact functor which commutes
with arbitrary direct sums. Whence
$F \circ f'_* = F_{\mathcal{K}'}$ for some $\mathcal{K}'$
on $X' \times_R Y$ by Lemma \ref{lemma-functor-quasi-coherent-from-affine}.
Moreover, given a morphism $f'' : X'' \to X'$ with $X''$ affine
we obtain a canonical identification
$(f'' \times \text{id}_Y)^*\mathcal{K}' = \mathcal{K}''$
by the references already given combined with
Lemma \ref{lemma-functor-quasi-coherent-from-affine-compose}.
These identifications satisfy a cocycle condition given
another morphism $f''' : X''' \to X''$ which we leave it to
the reader to spell out.

\medskip\noindent
Choose an affine open covering $X = \bigcup_{i = 1, \ldots, n} U_i$.
Since the diagonal of $X$ is affine, we see that the intersections
$U_{i_0 \ldots i_p} = U_{i_0} \cap \ldots \cap U_{i_p}$ are affine.
As above the inclusion morphisms
$j_{i_0 \ldots i_p} : U_{i_0 \ldots i_p} \to X$ are affine.
Denote $\mathcal{K}_{i_0 \ldots i_p}$ the quasi-coherent module
on $U_{i_0 \ldots i_p} \times_R Y$ corresponding to
$F \circ j_{i_0 \ldots i_p *}$ as above.
By the above we obtain identifications
$$
\mathcal{K}_{i_0 \ldots i_p} =
\mathcal{K}_{i_0 \ldots \hat i_j \ldots i_p}|_{U_{i_0 \ldots i_p} \times_R Y}
$$
which satisfy the usual compatibilites for glueing. In other words, we obtain
a unique quasi-coherent module $\mathcal{K}$ on $X \times_R Y$
whose restriction to $U_{i_0 \ldots i_p} \times_R Y$ is
$\mathcal{K}_{i_0 \ldots i_p}$ compatible with the displayed identifications.

\medskip\noindent
Next, we construct the transformation $t$. Given a quasi-coherent
$\mathcal{O}_X$-module $\mathcal{F}$ denote $\mathcal{F}_{i_0 \ldots i_p}$
the restriction of $\mathcal{F}$ to $U_{i_0 \ldots i_p}$ and denote
$(\text{pr}_1^*\mathcal{F} \otimes \mathcal{K})_{i_0 \ldots i_p}$
the restriction of $\text{pr}_1^*\mathcal{F} \otimes \mathcal{K}$ to
$U_{i_0 \ldots i_p} \times_R Y$.
Observe that
\begin{align*}
F(j_{i_0 \ldots i_p *}\mathcal{F}_{i_0 \ldots i_p})
& =
\text{pr}_{i_0 \ldots i_p, 2, *}(
\text{pr}_{i_0 \ldots i_p, 1}^*\mathcal{F}_{i_0 \ldots i_p}
\otimes \mathcal{K}_{i_0 \ldots i_p}) \\
& =
\text{pr}_{i_0 \ldots i_p, 2, *}
(\text{pr}_1^*\mathcal{F} \otimes \mathcal{K})_{i_0 \ldots i_p}
\end{align*}
where $\text{pr}_{i_0 \ldots i_p, 2} : U_{i_0 \ldots i_p} \times_R Y \to Y$
is the projection and similarly for the other projection. Moreover, these
identifications are compatible with the displayed identifications
in the previous paragraph. Recall, from Cohomology of Schemes, Lemma
\ref{coherent-lemma-separated-case-relative-cech}
that the relative {\v C}ech complex
$$
\bigoplus 
\text{pr}_{i_0, 2, *}
(\text{pr}_1^*\mathcal{F} \otimes \mathcal{K})_{i_0}
\to
\bigoplus 
\text{pr}_{i_0i_1, 2, *}
(\text{pr}_1^*\mathcal{F} \otimes \mathcal{K})_{i_0i_1}
\to
\bigoplus 
\text{pr}_{i_0i_1i_2, 2, *}
(\text{pr}_1^*\mathcal{F} \otimes \mathcal{K})_{i_0i_1i_2}
\to \ldots
$$
computes $R\text{pr}_{2, *}(\text{pr}_1^*\mathcal{F} \otimes \mathcal{K})$.
Hence the cohomology sheaf in degree $0$ is $F_\mathcal{K}(\mathcal{F})$.
Thus we obtain the desired map
$t : F(\mathcal{F}) \to F_\mathcal{K}(\mathcal{F})$
by contemplating the following commutative diagram
$$
\xymatrix{
&
F(\mathcal{F}) \ar[r] \ar@{..>}[d] &
\bigoplus F(j_{i_0*}\mathcal{F}_{i_0}) \ar[r] \ar[d] &
\bigoplus F(j_{i_0i_1*}\mathcal{F}_{i_0i_1}) \ar[d] \\
0 \ar[r] &
F_\mathcal{K}(\mathcal{F}) \ar[r] &
\bigoplus 
\text{pr}_{i_0, 2, *}
(\text{pr}_1^*\mathcal{F} \otimes \mathcal{K})_{i_0}
\ar[r] &
\bigoplus 
\text{pr}_{i_0i_1, 2, *}
(\text{pr}_1^*\mathcal{F} \otimes \mathcal{K})_{i_0i_1}
}
$$
We obtain the top row by applying $F$ to the (exact) complex
$0 \to \mathcal{F} \to \bigoplus j_{i_0*}\mathcal{F}_{i_0} \to
\bigoplus j_{i_0i_1*}\mathcal{F}_{i_0i_1}$ (but since
$F$ is not exact, the top row is just a complex and not
necessarily exact).
The solid vertical arrows are the identifications above.
This does indeed define the dotted arrow as desired.
The arrow is functorial in $\mathcal{F}$; we omit the details.

\medskip\noindent
We still have to prove the final assertion. Let $f : X' \to X$
be as in the statement of the lemma and let $\mathcal{K}'$
be the quasi-coherent module on $X' \times_R Y$ constructed
in the first paragraph of the proof. If the morphism
$f : X' \to X$ maps into one of the opens $U_i$, then the
result follows from
Lemma \ref{lemma-functor-quasi-coherent-from-affine-compose}
because in this case we know
that $\mathcal{K}_i = \mathcal{K}|_{U_i \times_R Y}$
pulls back to $\mathcal{K}$. In general, we obtain an
affine open covering $X' = \bigcup U'_i$ with $U'_i = f^{-1}(U_i)$
and we obtain isomorphisms
$\mathcal{K}'|_{U'_i} = f_i^*\mathcal{K}_i$ where
$f_i : U'_i \to U_i$ is the induced morphism.
These morphisms satisfy the compatibility conditions needed
to glue to an isomorphism $\mathcal{K}' = f^*\mathcal{K}$
and we conclude. Some details omitted.
\end{proof}

\begin{lemma}
\label{lemma-coh-noetherian-from-affine-flat}
In Lemma \ref{lemma-functor-quasi-coherent-from-affine}
or in Lemma \ref{lemma-functor-quasi-coherent-from-affine-diagonal-pre}
if $F$ is an exact functor, then the corresponding object
$\mathcal{K}$ of $\QCoh(\mathcal{O}_{X \times_R Y})$ is flat over $X$.
\end{lemma}

\begin{proof}
We may assume $X$ is affine, so we are in the case of
Lemma \ref{lemma-functor-quasi-coherent-from-affine}.
By Lemma \ref{lemma-functor-quasi-coherent-from-affine-compose}
we may assume $Y$ is affine. In the affine case the statement
translates into Remark \ref{remark-exact-flat}.
\end{proof}

\begin{lemma}
\label{lemma-functor-quasi-coherent-from-affine-diagonal}
Let $R$ be a ring. Let $X$ and $Y$ be schemes over $R$. Assume $X$ is
quasi-compact with affine diagonal.
There is an equivalence of categories between
\begin{enumerate}
\item the category of $R$-linear exact functors
$F : \QCoh(\mathcal{O}_X) \to \QCoh(\mathcal{O}_Y)$
which commute with arbitrary direct sums, and
\item the full subcategory of $\QCoh(\mathcal{O}_{X \times_R Y})$ consisting
of $\mathcal{K}$ such that
\begin{enumerate}
\item $\mathcal{K}$ is flat over $X$,
\item for $\mathcal{F} \in \QCoh(\mathcal{O}_X)$ we have
$R^q\text{pr}_{2, *}(\text{pr}_1^*\mathcal{F}
\otimes_{\mathcal{O}_{X \times_R Y}} \mathcal{K}) = 0$ for $q > 0$.
\end{enumerate}
\end{enumerate}
given by sending $\mathcal{K}$ to the functor $F$ in (\ref{equation-FM-QCoh}).
\end{lemma}

\begin{proof}
Let $\mathcal{K}$ be as in (2). The functor $F$ in
(\ref{equation-FM-QCoh}) commutes with direct sums.
Since by (1) (a) the modules $\mathcal{K}$ is $X$-flat,
we see that given a short exact
sequence $0 \to \mathcal{F}_1 \to \mathcal{F}_2 \to \mathcal{F}_3 \to 0$
we obtain a short exact sequence
$$
0 \to
\text{pr}_1^*\mathcal{F}_1 \otimes_{\mathcal{O}_{X \times_R Y}} \mathcal{K} \to
\text{pr}_1^*\mathcal{F}_2 \otimes_{\mathcal{O}_{X \times_R Y}} \mathcal{K} \to
\text{pr}_1^*\mathcal{F}_3 \otimes_{\mathcal{O}_{X \times_R Y}} \mathcal{K} \to
0
$$
Since by (2)(b) the higher direct image $R^1\text{pr}_{2, *}$
on the first term is zero, we conclude that
$0 \to F(\mathcal{F}_1) \to F(\mathcal{F}_2) \to F(\mathcal{F}_3) \to 0$
is exact and we see that $F$ is as in (1).

\medskip\noindent
Let $F$ be as in (1). Let $\mathcal{K}$ and $t : F \to F_\mathcal{K}$ be as
in Lemma \ref{lemma-functor-quasi-coherent-from-affine-diagonal-pre}.
By Lemma \ref{lemma-coh-noetherian-from-affine-flat} we see
that $\mathcal{K}$ is flat over $X$. To finish the proof we have
to show that $t$ is an isomorphism and the statement on higher
direct images. Both of these follow from the fact that the
relative {\v C}ech complex
$$
\bigoplus 
\text{pr}_{i_0, 2, *}
(\text{pr}_1^*\mathcal{F} \otimes \mathcal{K})_{i_0}
\to
\bigoplus 
\text{pr}_{i_0i_1, 2, *}
(\text{pr}_1^*\mathcal{F} \otimes \mathcal{K})_{i_0i_1}
\to
\bigoplus 
\text{pr}_{i_0i_1i_2, 2, *}
(\text{pr}_1^*\mathcal{F} \otimes \mathcal{K})_{i_0i_1i_2}
\to \ldots
$$
computes $R\text{pr}_{2, *}(\text{pr}_1^*\mathcal{F} \otimes \mathcal{K})$.
Please see proof of
Lemma \ref{lemma-functor-quasi-coherent-from-affine-diagonal-pre}
for notation and for the reason why this is so. In the proof of
Lemma \ref{lemma-functor-quasi-coherent-from-affine-diagonal-pre}
we also found that this complex is equal to $F$ applied to the complex
$$
\bigoplus j_{i_0*}\mathcal{F}_{i_0} \to
\bigoplus j_{i_0i_1*}\mathcal{F}_{i_0i_1} \to
\bigoplus j_{i_0i_1i_2*}\mathcal{F}_{i_0i_1i_2} \to \ldots
$$
This complex is exact except in degree zero with cohomology
sheaf equal to $\mathcal{F}$. Hence since $F$ is an exact functor
we conclude $F = F_\mathcal{K}$ and that (2)(b) holds.

\medskip\noindent
We omit the proof that the construction that sends $F$ to
$\mathcal{K}$ is functorial and a quasi-inverse to the
functor sending $\mathcal{K}$ to the functor $F_\mathcal{K}$
determined by (\ref{equation-FM-QCoh}).
\end{proof}

\begin{remark}
\label{remark-characterize-FM-QCoh}
Let $R$ be a ring. Let $X$ and $Y$ be schemes over $R$. Assume $X$
is quasi-compact with affine diagonal.
Lemma \ref{lemma-functor-quasi-coherent-from-affine-diagonal} may
be generalized as follows: the functors
(\ref{equation-FM-QCoh}) associated to quasi-coherent modules on
$X \times_R Y$ are exactly those
$F : \QCoh(\mathcal{O}_X) \to \QCoh(\mathcal{O}_Y)$
which have the following properties
\begin{enumerate}
\item $F$ is $R$-linear and commutes with arbitrary direct sums,
\item $F \circ j_*$ is right exact when $j : U \to X$ is the
inclusion of an affine open, and
\item $0 \to F(\mathcal{F}) \to F(\mathcal{G}) \to F(\mathcal{H})$
is exact whenever $0 \to \mathcal{F} \to \mathcal{G} \to \mathcal{H} \to 0$
is an exact sequence such that for all $x \in X$ the sequence on stalks
$0 \to \mathcal{F}_x \to \mathcal{G}_x \to \mathcal{H}_x \to 0$
is a split short exact sequence.
\end{enumerate}
Namely, these assumptions are enough to get construct a transformation
$t : F \to F_\mathcal{K}$ as in
Lemma \ref{lemma-functor-quasi-coherent-from-affine-diagonal-pre}
and to show that it is an isomorphism. Moreover, properties (1), (2), and (3)
do hold for functors (\ref{equation-FM-QCoh}).
If we ever need this we will carefully state and prove this here.
\end{remark}

\begin{lemma}
\label{lemma-compose-FM-QCoh}
Let $R$ be a ring. Let $X$, $Y$, $Z$ be schemes over $R$. Assume
$X$ and $Y$ are quasi-compact and have affine diagonal. Let
$$
F : \QCoh(\mathcal{O}_X) \to \QCoh(\mathcal{O}_Y)
\quad\text{and}\quad
G : \QCoh(\mathcal{O}_Y) \to \QCoh(\mathcal{O}_Z)
$$
be $R$-linear exact functors which commute with arbitrary direct sums.
Let $\mathcal{K}$ in $\QCoh(\mathcal{O}_{X \times_R Y})$
and $\mathcal{L}$ in $\QCoh(\mathcal{O}_{Y \times_R Z})$
be the corresponding ``kernels'', see
Lemma \ref{lemma-functor-quasi-coherent-from-affine-diagonal}.
Then $G \circ F$ corresponds to
$\text{pr}_{13, *}(\text{pr}_{12}^*\mathcal{K}
\otimes_{\mathcal{O}_{X \times_R Y \times_R Z}}
\text{pr}_{23}^*\mathcal{L})$ in $\QCoh(\mathcal{O}_{X \times_R Z})$.
\end{lemma}

\begin{proof}
Since $G \circ F : \QCoh(\mathcal{O}_X) \to \QCoh(\mathcal{O}_Z)$
is $R$-linear, exact, and commutes with arbitrary direct sums,
we find by Lemma \ref{lemma-functor-quasi-coherent-from-affine-diagonal}
that there exists an $\mathcal{M}$ in
$\QCoh(\mathcal{O}_{X \times_R Z})$ corresponding to $G \circ F$.
On the other hand, denote
$\mathcal{E} = \text{pr}_{13, *}(\text{pr}_{12}^*\mathcal{K}
\otimes \text{pr}_{23}^*\mathcal{L})$. Here and in the rest of
the proof we omit the subscript from the tensor products.
Let $U \subset X$ and $W \subset Z$ be affine open subschemes.
To prove the lemma, we will construct an isomorphism
$$
\Gamma(U \times_R W, \mathcal{E})
\cong
\Gamma(U \times_R W, \mathcal{M})
$$
compatible with restriction mappings for varying $U$ and $W$.

\medskip\noindent
First, we observe that
$$
\Gamma(U \times_R W, \mathcal{E}) =
\Gamma(U \times_R Y \times_R W,
\text{pr}_{12}^*\mathcal{K} \otimes \text{pr}_{23}^*\mathcal{L})
$$
by construction. Thus we have to show that the same thing is true
for $\mathcal{M}$.

\medskip\noindent
Write $U = \Spec(A)$ and denote $j : U \to X$ the inclusion morphism.
Recall from the construction of $\mathcal{M}$ in the proof of
Lemma \ref{lemma-functor-quasi-coherent-from-affine} that
$$
\Gamma(U \times_R W, \mathcal{M}) =
\Gamma(W, G(F(j_*\mathcal{O}_U)))
$$
where the $A$-module action on the right hand side is given by
the action of $A$ on $\mathcal{O}_U$. The correspondence between
$F$ and $\mathcal{K}$ tells us that
$F(j_*\mathcal{O}_U) = b_*(a^*j_*\mathcal{O}_U \otimes \mathcal{K})$
where $a : X \times_R Y \to X$ and $b : X \times_R Y \to Y$ are
the projection morphisms. Since $j$ is an affine morphism, we have
$a^*j_*\mathcal{O}_U = (j \times \text{id}_Y)_*\mathcal{O}_{U \times_R Y}$
by Cohomology of Schemes, Lemma
\ref{coherent-lemma-affine-base-change}.
Next, we have
$(j \times \text{id}_Y)_*\mathcal{O}_{U \times_R Y} \otimes \mathcal{K} =
(j \times \text{id}_Y)_*\mathcal{K}|_{U \times_R Y}$
by Remark \ref{remark-affine-morphism} for example.
Putting what we have found together we find
$$
F(j_*\mathcal{O}_U) =
(U \times_R Y \to Y)_*\mathcal{K}|_{U \times_R Y}
$$
with obvious $A$-action. (This formula is implicit in the proof
of Lemma \ref{lemma-functor-quasi-coherent-from-affine}.)
Applying the functor $G$ we obtain
$$
G(F(j_*\mathcal{O}_U)) =
t_*(s^*((U \times_R Y \to Y)_*\mathcal{K}|_{U \times_R Y})
\otimes \mathcal{L})
$$
where $s : Y \times_R Z \to Y$ and $t : Y \times_R Z \to Z$ are
the projection morphisms. Again using affine base change
(Cohomology of Schemes, Lemma \ref{coherent-lemma-affine-base-change})
but this time for the square
$$
\xymatrix{
U \times_R Y \times_R Z \ar[r] \ar[d] & U \times_R Y \ar[d] \\
Y \times_R Z \ar[r] & Y
}
$$
we obtain
$$
s^*((U \times_R Y \to Y)_*\mathcal{K}|_{U \times_R Y}) =
(U \times_R Y \times_R Z \to Y \times_R Z)_*
\text{pr}_{12}^*\mathcal{K}|_{U \times_R Y \times_R Z}
$$
Using Remark \ref{remark-affine-morphism} again we find
\begin{align*}
(U \times_R Y \times_R Z \to Y \times_R Z)_*
\text{pr}_{12}^*\mathcal{K}|_{U \times_R Y \times_R Z}
\otimes \mathcal{L} \\
=
(U \times_R Y \times_R Z \to Y \times_R Z)_*
\left(\text{pr}_{12}^*\mathcal{K} \otimes
\text{pr}_{23}^*\mathcal{L}\right)|_{U \times_R Y \times_R Z}
\end{align*}
Applying the functor $\Gamma(W, t_*(-)) = \Gamma(Y \times_R W, -)$
to this we obtain
\begin{align*}
\Gamma(U \times_R W, \mathcal{M})
& =
\Gamma(W, G(F(j_*\mathcal{O}_U))) \\
& =
\Gamma(Y \times_R W, (U \times_R Y \times_R Z \to Y \times_R Z)_*
(\text{pr}_{12}^*\mathcal{K} \otimes
\text{pr}_{23}^*\mathcal{L})|_{U \times_R Y \times_R Z}) \\
& =
\Gamma(U \times_R Y \times_R W,
\text{pr}_{12}^*\mathcal{K} \otimes \text{pr}_{23}^*\mathcal{L})
\end{align*}
as desired. We omit the verication that these isomorphisms are
compatible with restriction mappings.
\end{proof}

\begin{lemma}
\label{lemma-persistence-exactness}
Let $R$, $X$, $Y$, and $\mathcal{K}$ be as in
Lemma \ref{lemma-functor-quasi-coherent-from-affine-diagonal} part (2).
Then for any scheme $T$ over $R$ we have
$$
R^q\text{pr}_{13, *}(\text{pr}_{12}^*\mathcal{F}
\otimes_{\mathcal{O}_{T \times_R X \times_R Y}}
\text{pr}_{23}^*\mathcal{K}) = 0
$$
for $\mathcal{F}$ quasi-coherent on $T \times_R X$ and $q > 0$.
\end{lemma}

\begin{proof}
The question is local on $T$ hence we may assume $T$ is affine.
In this case we can consider the diagram
$$
\xymatrix{
T \times_R X \ar[d] &
T \times_R X \times_R Y \ar[d] \ar[l] \ar[r] &
T \times_R Y \ar[d] \\
X &
X \times_R Y \ar[l] \ar[r] &
Y
}
$$
whose vertical arrows are affine. In particular the pushforward along
$T \times_R Y \to Y$ is faithful and exact (Cohomology of Schemes,
Lemma \ref{coherent-lemma-relative-affine-vanishing} and
Morphisms, Lemma \ref{morphisms-lemma-affine-equivalence-modules}).
Chasing around in the diagram using that higher direct images along
affine morphisms vanish (see reference above) we see that
it suffices to prove
$$
R^q\text{pr}_{2, *}(
\text{pr}_{23, *}(\text{pr}_{12}^*\mathcal{F}
\otimes_{\mathcal{O}_{T \times_R X \times_R Y}}
\text{pr}_{23}^*\mathcal{K})) =
R^q\text{pr}_{2, *}(
\text{pr}_{23, *}(\text{pr}_{12}^*\mathcal{F})
\otimes_{\mathcal{O}_{X \times_R Y}}
\mathcal{K}))
$$
is zero which is true by assumption on $\mathcal{K}$.
The equality holds by Remark \ref{remark-affine-morphism}.
\end{proof}

\begin{lemma}
\label{lemma-functor-quasi-coherent-from-separated}
In Lemma \ref{lemma-functor-quasi-coherent-from-affine-diagonal}
let $F$ and $\mathcal{K}$ correspond. If $X$ is separated and
flat over $R$, then there is a surjection
$\mathcal{O}_X \boxtimes F(\mathcal{O}_X) \to \mathcal{K}$.
\end{lemma}

\begin{proof}
Let $\Delta : X \to X \times_R X$ be the diagonal morphism and
set $\mathcal{O}_\Delta = \Delta_*\mathcal{O}_X$.
Since $\Delta$ is a closed immersion have a short exact sequence
$$
0 \to \mathcal{I} \to 
\mathcal{O}_{X \times_R X} \to \mathcal{O}_\Delta \to 0
$$
Since $\mathcal{K}$ is flat over $X$, the pullback
$\text{pr}_{23}^*\mathcal{K}$ to $X \times_R X \times_R Y$
is flat over $X \times_R X$. We obtain a short exact sequence
$$
0 \to 
\text{pr}_{12}^*\mathcal{I}
\otimes
\text{pr}_{23}^*\mathcal{K} \to
\text{pr}_{23}^*\mathcal{K} \to
\text{pr}_{12}^*\mathcal{O}_\Delta
\otimes
\text{pr}_{23}^*\mathcal{K} \to 0
$$
on $X \times_R X \times_R Y$, see
Modules, Lemma \ref{modules-lemma-pullback-tensor-flat-module}.
Thus, by Lemma \ref{lemma-persistence-exactness}
we obtain a surjection
$$
\text{pr}_{13, *}(\text{pr}_{23}^*\mathcal{K})
\to
\text{pr}_{13, *}(
\text{pr}_{12}^*\mathcal{O}_\Delta
\otimes
\text{pr}_{23}^*\mathcal{K})
$$
By flat base change
(Cohomology of Schemes, Lemma \ref{coherent-lemma-flat-base-change-cohomology})
the source of this arrow is equal to $\text{pr}_2^*\text{pr}_{2, *}\mathcal{K}
= \mathcal{O}_X \boxtimes F(\mathcal{O}_X)$. On the other hand the target is
equal to
$$
\text{pr}_{13, *}(
\text{pr}_{12}^*\mathcal{O}_\Delta
\otimes
\text{pr}_{23}^*\mathcal{K}) =
\text{pr}_{13, *} (\Delta \times \text{id}_Y)_* \mathcal{K} =
\mathcal{K}
$$
which finishes the proof. The first equality holds for example by
Cohomology, Lemma \ref{cohomology-lemma-projection-formula-closed-immersion}
and the fact that $\text{pr}_{12}^*\mathcal{O}_\Delta =
(\Delta \times \text{id}_Y)_*\mathcal{O}_{X \times_R Y}$.
\end{proof}










\section{Gabriel-Rosenberg reconstruction}
\label{section-gabriel}

\noindent
The title of this section refers to results like
Proposition \ref{proposition-gabriel-rosenberg}.
Besides Gabriel's original paper \cite{Gabriel}, please consult
\cite{Brandenburg} which has a proof of the result for quasi-separated schemes
and discusses the literature. In this section we will only prove
Gabriel-Rosenberg reconstruction for quasi-compact and quasi-separated schemes.

\begin{lemma}
\label{lemma-categorically-compact-QCoh}
Let $X$ be a quasi-compact and quasi-separated scheme.
Let $\mathcal{F}$ be a quasi-coherent $\mathcal{O}_X$-module.
Then $\mathcal{F}$ is a categorically compact object of
$\QCoh(\mathcal{O}_X)$ if and only if $\mathcal{F}$ is of
finite presentation.
\end{lemma}

\begin{proof}
See Categories, Definition \ref{categories-definition-compact-object}
for our notion of categorically compact objects in a category.
If $\mathcal{F}$ is of finite presentation then it is categorically compact by
Modules, Lemma \ref{modules-lemma-finite-presentation-quasi-compact-colimit}.
Conversely, any quasi-coherent module $\mathcal{F}$ can be written
as a filtered colimit $\mathcal{F} = \colim \mathcal{F}_i$ of
finitely presented (hence quasi-coherent)
$\mathcal{O}_X$-modules, see Properties, Lemma
\ref{properties-lemma-directed-colimit-finite-presentation}.
If $\mathcal{F}$ is categorically compact, then we find
some $i$ and a morphism $\mathcal{F} \to \mathcal{F}_i$
which is a right inverse to the given map $\mathcal{F}_i \to \mathcal{F}$.
We conclude that $\mathcal{F}$ is a direct summand of a
finitely presented module, and hence finitely presented itself.
\end{proof}

\begin{lemma}
\label{lemma-supported-on-support-pre}
Let $X$ be an affine scheme. Let $\mathcal{F}$ be a finitely presented
$\mathcal{O}_X$-module. Let $\mathcal{E}$ be a nonzero quasi-coherent
$\mathcal{O}_X$-module. If
$\text{Supp}(\mathcal{E}) \subset \text{Supp}(\mathcal{F})$,
then there exists a nonzero map $\mathcal{F} \to \mathcal{E}$.
\end{lemma}

\begin{proof}
Let us translate the statement into algebra. Let $A$ be a ring. Let $M$ be a
finitely presented $A$-module. Let $N$ be a nonzero $A$-module. Assume
$\text{Supp}(N) \subset \text{Supp}(M)$. To show: $\Hom_A(M, N)$ is nonzero.
We may assume $N = A/I$ is cyclic (replace $N$ by any nonzero
cyclic submodule). Choose a presentation
$$
A^{\oplus m} \xrightarrow{T} A^{\oplus n} \to M \to 0
$$
Recall that $\text{Supp}(M)$ is cut out by $\text{Fit}_0(M)$ which
is the ideal generated by the $n \times n$ minors of the matrix $T$.
See More on Algebra, Lemma \ref{more-algebra-lemma-fitting-ideal-basics}.
The assumption $\text{Supp}(N) \subset \text{Supp}(M)$ now means that
the elements of $\text{Fit}_0(M)$ are nilpotent in $A/I$.
Consider the exact sequence
$$
0 \to \Hom_A(M, A/I) \to (A/I)^{\oplus n} \xrightarrow{T^t} (A/I)^{\oplus m}
$$
We have to show that $T^t$ cannot be injective; we urge the reader to
find their own proof of this using the nilpotency of elements of
$\text{Fit}_0(M)$ in $A/I$. Here is our proof.
Since $\text{Fit}_0(M)$ is finitely generated, the nilpotency means that
the annihilator $J \subset A/I$ of $\text{Fit}_0(M)$ in $A/I$ is nonzero.
To show the non-injectivity of $T^t$ we may localize at a prime.
Choosing a suitable prime we may assume $A$ is local and $J$ is still
nonzero. Then $T^t$ has a nonzero kernel by
More on Algebra, Lemma \ref{more-algebra-lemma-exact-length-1}.
\end{proof}

\begin{lemma}
\label{lemma-supported-on-support}
Let $X$ be a quasi-compact and quasi-separated scheme.
Let $\mathcal{F}$ be a finitely presented $\mathcal{O}_X$-module.
The following two subcategories of $\QCoh(\mathcal{O}_X)$ are equal
\begin{enumerate}
\item the full subcategory $\mathcal{A} \subset \QCoh(\mathcal{O}_X)$
whose objects are the quasi-coherent modules
whose support is (set theoretically) contained in $\text{Supp}(\mathcal{F})$,
\item the smallest Serre subcategory $\mathcal{B} \subset \QCoh(\mathcal{O}_X)$
containing
$\mathcal{F}$ closed under extensions and arbitrary direct sums.
\end{enumerate}
\end{lemma}

\begin{proof}
Observe that the statement makes sense as finitely presented
$\mathcal{O}_X$-modules are quasi-coherent.
Since $\mathcal{A}$ is a Serre subcategory closed under extensions and
direct sums and since $\mathcal{F}$ is an object of $\mathcal{A}$
we see that $\mathcal{B} \subset \mathcal{A}$. Thus it remains to
show that $\mathcal{A}$ is contained in $\mathcal{B}$.

\medskip\noindent
Let $\mathcal{E}$ be an object of $\mathcal{A}$. There exists
a maximal submodule $\mathcal{E}' \subset \mathcal{E}$
which is in $\mathcal{B}$. Namely, suppose
$\mathcal{E}_i \subset \mathcal{E}$, $i \in I$ is the set of
subobjects which are objects of $\mathcal{B}$. Then
$\bigoplus \mathcal{E}_i$ is in $\mathcal{B}$ and so is
$$
\mathcal{E}' = \Im(\bigoplus \mathcal{E}_i \longrightarrow \mathcal{E})
$$
This is clearly the maximal submodule we were looking for.

\medskip\noindent
Now suppose that we have a nonzero map
$\mathcal{G} \to \mathcal{E}/\mathcal{E}'$
with $\mathcal{G}$ in $\mathcal{B}$. Then
$\mathcal{G}' = \mathcal{E} \times_{\mathcal{E}/\mathcal{E}'} \mathcal{G}$
is in $\mathcal{B}$ as an extension of $\mathcal{E}'$ and $\mathcal{G}$.
Then the image $\mathcal{G}' \to \mathcal{E}$ would be strictly
bigger than $\mathcal{E}'$, contradicting the maximality of $\mathcal{E}'$.
Thus it suffices to show the claim in the following paragraph.

\medskip\noindent
Let $\mathcal{E}$ be an nonzero object of $\mathcal{A}$. We claim that there
is a nonzero map $\mathcal{G} \to \mathcal{E}$ with $\mathcal{G}$ in
$\mathcal{B}$. We will prove this by induction on the minimal
number $n$ of affine opens $U_i$ of $X$ such that
$\text{Supp}(\mathcal{E}) \subset U_1 \cup \ldots \cup U_n$.
Set $U = U_n$ and denote $j : U \to X$ the inclusion morphism.
Denote $\mathcal{E}' = \Im(\mathcal{E} \to j_*\mathcal{E}|_U)$.
Then the kernel $\mathcal{E}''$ of the surjection
$\mathcal{E} \to \mathcal{E}'$ has support contained in
$U_1 \cup \ldots \cup U_{n - 1}$. Thus if $\mathcal{E}''$ is nonzero,
then we win. In other words, we may assume that
$\mathcal{E} \subset j_*\mathcal{E}|_U$.
In particular, we see that $\mathcal{E}|_U$ is nonzero.
By Lemma \ref{lemma-supported-on-support-pre}
there exists a nonzero map $\mathcal{F}|_U \to \mathcal{E}|_U$.
This corresponds to a map
$$
\varphi : \mathcal{F} \longrightarrow j_*(\mathcal{E}|_U)
$$
whose restriction to $U$ is nonzero.
Setting $\mathcal{G} = \varphi^{-1}(\mathcal{E})$ we conclude.
\end{proof}

\begin{lemma}
\label{lemma-quotient-supported-on-closed}
Let $X$ be a quasi-compact and quasi-separated scheme.
Let $Z \subset X$ be a closed subset such that $U = X \setminus Z$
is quasi-compact. Let $\mathcal{A} \subset \QCoh(\mathcal{O}_X)$
be the full subcategory whose objects are the quasi-coherent modules
supported on $Z$. Then the restriction functor
$\QCoh(\mathcal{O}_X) \to \QCoh(\mathcal{O}_U)$ induces
an equivalence $\QCoh(\mathcal{O}_X)/\mathcal{A} \cong \QCoh(\mathcal{O}_U)$.
\end{lemma}

\begin{proof}
By the universal property of the quotient construction
(Homology, Lemma \ref{homology-lemma-serre-subcategory-is-kernel})
we certainly obtain an induced functor
$\QCoh(\mathcal{O}_X)/\mathcal{A} \cong \QCoh(\mathcal{O}_U)$.
Denote $j : U \to X$ the inclusion morphism. Since $j$ is quasi-compact
and quasi-separated we obtain a functor
$j_* : \QCoh(\mathcal{O}_U) \to \QCoh(\mathcal{O}_X)$.
The reader shows that this defines a quasi-inverse; details omitted.
\end{proof}

\begin{lemma}
\label{lemma-characterize-affine}
Let $X$ be a quasi-compact and quasi-separated scheme.
If $\QCoh(\mathcal{O}_X)$ is equivalent to the category
of modules over a ring, then $X$ is affine.
\end{lemma}

\begin{proof}
Say $F : \text{Mod}_R \to \QCoh(\mathcal{O}_X)$ is an equivalence.
Then $\mathcal{F} = F(R)$ has the following properties:
\begin{enumerate}
\item it is a finitely presented $\mathcal{O}_X$-module
(Lemma \ref{lemma-categorically-compact-QCoh}),
\item $\Hom_X(\mathcal{F}, -)$ is exact,
\item $\Hom_X(\mathcal{F}, \mathcal{F})$ is a commutative ring,
\item every object of $\QCoh(\mathcal{O}_X)$ is a quotient of
a direct sum of copies of $\mathcal{F}$.
\end{enumerate}
Let $x \in X$ be a closed point. Consider the surjection
$$
\mathcal{O}_X \to i_*\kappa(x)
$$
where the target is the pushforward of $\kappa(x)$ by the
inclusion morphism $i : x \to X$. We have
$$
\Hom_X(\mathcal{F}, i_*\kappa(x)) =
\Hom_{\mathcal{O}_{X, x}}(\mathcal{F}_x, \kappa(x))
$$
This first by (4) implies that $\mathcal{F}_x$ is nonzero.
From (2) we deduce that every map $\mathcal{F}_x \to \kappa(x)$
lifts to a map $\mathcal{F}_x \to \mathcal{O}_{X, x}$ (as it
even lifts to a global map $\mathcal{F} \to \mathcal{O}_X$).
Since $\mathcal{F}_x$ is a finite $\mathcal{O}_{X, x}$-module,
this implies that $\mathcal{F}_x$ is a (nonzero)
finite free $\mathcal{O}_{X, x}$-module.
Then since $\mathcal{F}$ is of finite presentation, this implies
that $\mathcal{F}$ is finite free of positive rank
in an open neighbourhood of $x$
(Modules, Lemma \ref{modules-lemma-finite-presentation-stalk-free}).
Since every closed subset of $X$ contains a closed point 
(Topology, Lemma \ref{topology-lemma-quasi-compact-closed-point})
this implies that $\mathcal{F}$ is finite locally free of positive rank.
Similarly, the map
$$
\Hom_X(\mathcal{F}, \mathcal{F}) \to
\Hom_X(\mathcal{F}, i_*i^*\mathcal{F}) =
\Hom_{\kappa(x)}(\mathcal{F}_x/\mathfrak m_x \mathcal{F}_x,
\mathcal{F}_x/\mathfrak m_x \mathcal{F}_x)
$$
is surjective. By property (3) we conclude that the rank $\mathcal{F}_x$
must be $1$. Hence $\mathcal{F}$ is an invertible $\mathcal{O}_X$-module.
But then we conclude that the functor
$$
\mathcal{H}
\longmapsto
\Gamma(X, \mathcal{H}) = \Hom_X(\mathcal{O}_X, \mathcal{H}) =
\Hom_X(\mathcal{F}, \mathcal{H} \otimes_{\mathcal{O}_X} \mathcal{F})
$$
on $\QCoh(\mathcal{O}_X)$ is exact too. This implies that the first
$\Ext$ group
$$
\Ext^1_{\QCoh(\mathcal{O}_X)}(\mathcal{O}_X, \mathcal{H}) = 0
$$
computed in the abelian category $\QCoh(\mathcal{O}_X)$ vanishes for all
$\mathcal{H}$ in $\QCoh(\mathcal{O}_X)$. However, since
$\QCoh(\mathcal{O}_X) \subset \textit{Mod}(\mathcal{O}_X)$
is closed under extensions
(Schemes, Section \ref{schemes-section-quasi-coherent})
we see that $\Ext^1$ between quasi-coherent modules
computed in $\QCoh(\mathcal{O}_X)$ is the same as computed
in $\textit{Mod}(\mathcal{O}_X)$. Hence we conclude that
$$
H^1(X, \mathcal{H}) =
\Ext^1_{\textit{Mod}(\mathcal{O}_X)}(\mathcal{O}_X, \mathcal{H}) = 0
$$
for all $\mathcal{H}$ in $\QCoh(\mathcal{O}_X)$.
This implies that $X$ is affine for example by
Cohomology of Schemes, Lemma
\ref{coherent-lemma-quasi-compact-h1-zero-covering}.
\end{proof}

\begin{proposition}
\label{proposition-gabriel-rosenberg}
\begin{reference}
Special case of \cite[Theorem 1.2]{Brandenburg}
\end{reference}
Let $X$ and $Y$ be quasi-compact and quasi-separated schemes.
If $F : \QCoh(\mathcal{O}_X) \to \QCoh(\mathcal{O}_Y)$
is an equivalence, then there exists an isomorphism
$f : Y \to X$ of schemes and an invertible $\mathcal{O}_Y$-module
$\mathcal{L}$ such that $F(\mathcal{F}) = f^*\mathcal{F} \otimes \mathcal{L}$.
\end{proposition}

\begin{proof}
Of course $F$ is additive, exact, commutes with all limits,
commutes with all colimits, commutes with direct sums, etc.
Let $U \subset X$ be an affine open subscheme.
Let $\mathcal{I} \subset \mathcal{O}_X$ be a finite type
quasi-coherent sheaf of ideals such that $Z = V(\mathcal{I})$
is the complement of $U$ in $X$, see
Properties, Lemma \ref{properties-lemma-quasi-coherent-finite-type-ideals}.
Then $\mathcal{O}_X/\mathcal{I}$ is a finitely presented
$\mathcal{O}_X$-module. Hence $\mathcal{G} = F(\mathcal{O}_X/\mathcal{I})$
is a finitely presented $\mathcal{O}_Y$-module by
Lemma \ref{lemma-categorically-compact-QCoh}.
Denote $T \subset Y$ the support of $\mathcal{G}$ and set
$V = Y \setminus T$. Since $\mathcal{G}$ is of finite presentation,
the scheme $V$ is a quasi-compact open of $Y$.
By Lemma \ref{lemma-supported-on-support} we see that $F$ induces an
equivalence between
\begin{enumerate}
\item the full subcategory of $\QCoh(\mathcal{O}_X)$ consisting
of modules supported on $Z$, and
\item the full subcategory of $\QCoh(\mathcal{O}_Y)$ consisting
of modules supported on $T$.
\end{enumerate}
By Lemma \ref{lemma-quotient-supported-on-closed} we obtain a commutative
diagram
$$
\xymatrix{
\QCoh(\mathcal{O}_X) \ar[r]_F \ar[d] &
\QCoh(\mathcal{O}_Y) \ar[d] \\
\QCoh(\mathcal{O}_U) \ar[r]^{F_U} &
\QCoh(\mathcal{O}_V)
}
$$
where the vertical arrows are the restruction functors and the
horizontal arrows are equivalences. By Lemma \ref{lemma-characterize-affine}
we conclude that $V$ is affine. For the affine case we have
Lemma \ref{lemma-functor-equivalence}. Thus we find that there
is an isomorphism $f_U : V \to U$ and an invertible
$\mathcal{O}_V$-module $\mathcal{L}_U$ such that
$F_U$ is the functor
$\mathcal{F} \mapsto f_U^*\mathcal{F} \otimes \mathcal{L}_U$.

\medskip\noindent
The proof can be finished by noticing that the diagrams above
satisfy an obvious compatibility with regards to inclusions
of affine open subschemes of $X$. Thus the morphisms $f_U$ and the
invertible modules $\mathcal{L}_U$ glue. We omit the details.
\end{proof}










\section{Functors between categories of coherent modules}
\label{section-functor-coherent}

\noindent
The following lemma guarantees that we can use the material on
functors between categories of quasi-coherent modules when
we are given a functor between categories of coherent modules.

\begin{lemma}
\label{lemma-functor-coherent}
Let $X$ and $Y$ be Noetherian schemes. Let
$F : \textit{Coh}(\mathcal{O}_X) \to \textit{Coh}(\mathcal{O}_Y)$
be a functor. Then $F$ extends uniquely to a functor
$\QCoh(\mathcal{O}_X) \to \QCoh(\mathcal{O}_Y)$
which commutes with filtered colimits.
If $F$ is additive, then its extension commutes with arbitrary direct sums.
If $F$ is exact, left exact, or right exact, so is its extension.
\end{lemma}

\begin{proof}
The existence and uniqueness of the extension is a general fact, see
Categories, Lemma \ref{categories-lemma-extend-functor-by-colim}.
To see that the lemma applies observe that coherent modules
are of finite presentation
(Modules, Lemma \ref{modules-lemma-coherent-finite-presentation}) and hence
categorically compact objects of $\textit{Mod}(\mathcal{O}_X)$ by
Modules, Lemma \ref{modules-lemma-finite-presentation-quasi-compact-colimit}.
Finally, every quasi-coherent module is a filtered colimit
of coherent ones for example by
Properties, Lemma \ref{properties-lemma-quasi-coherent-colimit-finite-type}.

\medskip\noindent
Assume $F$ is additive. If $\mathcal{F} = \bigoplus_{j \in J} \mathcal{H}_j$
with $\mathcal{H}_j$ quasi-coherent, then
$\mathcal{F} = \colim_{J' \subset J\text{ finite}}
\bigoplus_{j \in J'} \mathcal{H}_j$.
Denoting the extension of $F$ also by $F$ we obtain
\begin{align*}
F(\mathcal{F})
& =
\colim_{J' \subset J\text{ finite}}
F(\bigoplus\nolimits_{j \in J'} \mathcal{H}_j) \\
& =
\colim_{J' \subset J\text{ finite}}
\bigoplus\nolimits_{j \in J'} F(\mathcal{H}_j) \\
& =
\bigoplus\nolimits_{j \in J} F(\mathcal{H}_j)
\end{align*}
Thus $F$ commutes with arbitrary direct sums.

\medskip\noindent
Suppose $0 \to \mathcal{F} \to \mathcal{F}' \to \mathcal{F}'' \to 0$
is a short exact sequence of quasi-coherent $\mathcal{O}_X$-modules.
Then we write $\mathcal{F}' = \bigcup \mathcal{F}'_i$ as the
union of its coherent submodules, see
Properties, Lemma \ref{properties-lemma-quasi-coherent-colimit-finite-type}.
Denote $\mathcal{F}''_i \subset \mathcal{F}''$ the image of $\mathcal{F}'_i$
and denote $\mathcal{F}_i = \mathcal{F} \cap \mathcal{F}'_i =
\Ker(\mathcal{F}'_i \to \mathcal{F}''_i)$. Then it is clear that
$\mathcal{F} = \bigcup \mathcal{F}_i$ and
$\mathcal{F}'' = \bigcup \mathcal{F}''_i$
and that we have short exact sequences
$$
0 \to \mathcal{F}_i \to \mathcal{F}_i' \to \mathcal{F}_i'' \to 0
$$
Since the extension commutes with filtered colimits we have
$F(\mathcal{F}) = \colim_{i \in I} F(\mathcal{F}_i)$,
$F(\mathcal{F}') = \colim_{i \in I} F(\mathcal{F}'_i)$, and
$F(\mathcal{F}'') = \colim_{i \in I} F(\mathcal{F}''_i)$.
Since filtered colimits are exact
(Modules, Lemma \ref{modules-lemma-limits-colimits}) we
conclude that exactness properties of $F$ are inherited by
its extension.
\end{proof}

\begin{lemma}
\label{lemma-equivalence-coherent}
Let $X$ and $Y$ be Noetherian schemes. Let
$F : \textit{Coh}(\mathcal{O}_X) \to \textit{Coh}(\mathcal{O}_Y)$
be an equivalence of categories. Then there is an isomorphism $f : Y \to X$
and an invertible $\mathcal{O}_Y$-module $\mathcal{L}$
such that $F(\mathcal{F}) = f^*\mathcal{F} \otimes \mathcal{L}$.
\end{lemma}

\begin{proof}
By Lemma \ref{lemma-functor-coherent} we obtain a unique functor
$F' : \QCoh(\mathcal{O}_X) \to \QCoh(\mathcal{O}_Y)$ extending $F$.
The same is true for the quasi-inverse of $F$ and by the uniqueness
we conclude that $F'$ is an equivalence. By
Proposition \ref{proposition-gabriel-rosenberg}
we find an isomorphism $f : Y \to X$
and an invertible $\mathcal{O}_Y$-module $\mathcal{L}$
such that $F'(\mathcal{F}) = f^*\mathcal{F} \otimes \mathcal{L}$.
Then $f$ and $\mathcal{L}$ work for $F$ as well.
\end{proof}

\begin{remark}
\label{remark-equivalence-coherent-linear}
In Lemma \ref{lemma-equivalence-coherent} if $X$ and $Y$ are defined
over a common base ring $R$ and $F$ is $R$-linear, then the isomorphism
$f$ will be a morphism of schemes over $R$.
\end{remark}

\begin{lemma}
\label{lemma-characterize-finite}
Let $f : V \to X$ be a quasi-finite separated morphism of Noetherian
schemes. If there exists a coherent $\mathcal{O}_V$-module $\mathcal{K}$
whose support is $V$ such that $f_*\mathcal{K}$ is coherent and
$R^qf_*\mathcal{K} = 0$, then $f$ is finite.
\end{lemma}

\begin{proof}
By Zariski's main theorem we can find an open immersion
$j : V \to Y$ over $X$ with $\pi : Y \to X$ finite, see
More on Morphisms, Lemma
\ref{more-morphisms-lemma-quasi-finite-separated-pass-through-finite}.
Since $\pi$ is affine the functor $\pi_*$ is exact and faithful
on the category of coherent $\mathcal{O}_X$-modules.
Hence we see that $j_*\mathcal{K}$ is coherent and
that $R^qj_*\mathcal{K}$ is zero for $q > 0$.
In other words, we reduce to the case discussed in the next paragraph.

\medskip\noindent
Assume $f$ is an open immersion. We may replace $X$ by the
scheme theoretic closure of $V$. Assume $X \setminus V$ is nonempty
to get a contradiction. Choose a generic point $\xi \in X \setminus V$
of an irreducible component of $X \setminus V$. Looking at the situation
after base change by $\Spec(\mathcal{O}_{X, \xi}) \to X$ using flat base
change and using
Local Cohomology, Lemma
\ref{local-cohomology-lemma-finiteness-pushforwards-and-H1-local}
we reduce to the algebra problem discussed in the next paragraph.

\medskip\noindent
Let $(A, \mathfrak m)$ be a Noetherian local ring. Let $M$ be a finite
$A$-module whose support is $\Spec(A)$. Then $H^i_\mathfrak m(M) \not = 0$
for some $i$. This is true by
Dualizing Complexes, Lemma \ref{dualizing-lemma-depth}
and the fact that $M$ is not zero hence has finite depth.
\end{proof}

\noindent
The next lemma can be generalized to the case where $k$ is a
Noetherian ring and $X$ flat over $k$ (all other assumptions
stay the same).

\begin{lemma}
\label{lemma-functor-coherent-over-field}
Let $k$ be a field. Let $X$, $Y$ be finite type schemes over $k$ with
$X$ separated. There is an equivalence of categories between
\begin{enumerate}
\item the category of $k$-linear exact functors
$F : \textit{Coh}(\mathcal{O}_X) \to \textit{Coh}(\mathcal{O}_Y)$, and
\item the category of coherent $\mathcal{O}_{X \times Y}$-modules
$\mathcal{K}$ which are flat over $X$ and have support finite over $Y$
\end{enumerate}
given by sending $\mathcal{K}$ to the restriction of the functor
(\ref{equation-FM-QCoh}) to $\textit{Coh}(\mathcal{O}_X)$.
\end{lemma}

\begin{proof}
Let $\mathcal{K}$ be as in (2). By
Lemma \ref{lemma-functor-quasi-coherent-from-affine-diagonal}
the functor $F$ given by (\ref{equation-FM-QCoh}) is exact and $k$-linear.
Moreover, $F$ sends $\textit{Coh}(\mathcal{O}_X)$ into
$\textit{Coh}(\mathcal{O}_Y)$ for example by
Cohomology of Schemes, Lemma
\ref{coherent-lemma-support-proper-over-base-pushforward}.

\medskip\noindent
Let us construct the quasi-inverse to the construction. Let $F$ be
as in (1). By Lemma \ref{lemma-functor-coherent} we can extend $F$
to a $k$-linear exact functor on the
categories of quasi-coherent modules which commutes with arbitrary direct sums.
By Lemma \ref{lemma-functor-quasi-coherent-from-affine-diagonal}
the extension corresponds to a unique quasi-coherent module
$\mathcal{K}$, flat over $X$, such that
$R^q\text{pr}_{2, *}(\text{pr}_1^*\mathcal{F}
\otimes_{\mathcal{O}_{X \times Y}} \mathcal{K}) = 0$ for $q > 0$
for all quasi-coherent $\mathcal{O}_X$-modules $\mathcal{F}$.
Since $F(\mathcal{O}_X)$ is a coherent $\mathcal{O}_Y$-module, we
conclude from Lemma \ref{lemma-functor-quasi-coherent-from-separated}
that $\mathcal{K}$ is coherent.

\medskip\noindent
For a closed point $x \in X$ denote $\mathcal{O}_x$ the skyscraper sheaf
at $x$ with value the residue field of $x$. We have
$$
F(\mathcal{O}_x) =
\text{pr}_{2, *}(\text{pr}_1^*\mathcal{O}_x \otimes \mathcal{K}) =
(x \times Y \to Y)_*(\mathcal{K}|_{x \times Y})
$$
Since $x \times Y \to Y$ is finite, we see that the pushforward along
this morphism is faithful. Hence if $y \in Y$ is in the image of the
support of $\mathcal{K}|_{x \times Y}$, then $y$ is in the support of
$F(\mathcal{O}_x)$.

\medskip\noindent
Let $Z \subset X \times Y$ be the scheme theoretic support $Z$ of
$\mathcal{K}$, see
Morphisms, Definition \ref{morphisms-definition-scheme-theoretic-support}.
We first prove that $Z \to Y$ is quasi-finite, by proving that its fibres
over closed points are finite. Namely, if the fibre of $Z \to Y$ over a
closed point $y \in Y$ has dimension $> 0$, then we can find infinitely
many pairwise distinct closed points $x_1, x_2, \ldots$ in the image of
$Z_y \to X$. Since we have a surjection
$\mathcal{O}_X \to \bigoplus_{i = 1, \ldots, n} \mathcal{O}_{x_i}$
we obtain a surjection
$$
F(\mathcal{O}_X) \to \bigoplus\nolimits_{i = 1, \ldots, n} F(\mathcal{O}_{x_i})
$$
By what we said above, the point $y$ is in the support of each
of the coherent modules $F(\mathcal{O}_{x_i})$. Since $F(\mathcal{O}_X)$
is a coherent module, this will lead to a contradiction because
the stalk of $F(\mathcal{O}_X)$ at $y$ will be generated by $< n$ elements
if $n$ is large enough. Hence $Z \to Y$ is quasi-finite.
Since $\text{pr}_{2, *}\mathcal{K}$ is coherent and
$R^q\text{pr}_{2, *}\mathcal{K} = 0$ for $q > 0$ we conclude
that $Z \to Y$ is finite by Lemma \ref{lemma-characterize-finite}.
\end{proof}

\begin{lemma}
\label{lemma-pushforward-invertible-pre}
Let $f : X \to Y$ be a finite type separated morphism of schemes. Let
$\mathcal{F}$ be a finite type quasi-coherent module on $X$
with support finite over $Y$
and with $\mathcal{L} = f_*\mathcal{F}$ an invertible $\mathcal{O}_X$-module.
Then there exists a section $s : Y \to X$ such that
$\mathcal{F} \cong s_*\mathcal{L}$.
\end{lemma}

\begin{proof}
Looking affine locally this translates into the following algebra problem.
Let $A \to B$ be a ring map and let $N$ be a $B$-module which is
invertible as an $A$-module. Then the annihilator $J$ of $N$ in $B$
has the property that $A \to B/J$ is an isomorphism. We omit the details.
\end{proof}

\begin{lemma}
\label{lemma-pushforward-invertible}
Let $f : X \to Y$ be a finite type separated morphism of schemes with a section
$s : Y \to X$. Let $\mathcal{F}$ be a finite type quasi-coherent module
on $X$, set theoretically supported on $s(Y)$ with
$\mathcal{L} = f_*\mathcal{F}$
an invertible $\mathcal{O}_X$-module. If $Y$ is reduced, then
$\mathcal{F} \cong s_*\mathcal{L}$.
\end{lemma}

\begin{proof}
By Lemma \ref{lemma-pushforward-invertible-pre}
there exists a section $s' : Y  \to X$ such that
$\mathcal{F} = s'_*\mathcal{L}$. Since $s'(Y)$ and $s(Y)$
have the same underlying closed subset
and since both are reduced closed subschemes of $X$, they have to be equal.
Hence $s = s'$ and the lemma holds.
\end{proof}

\begin{lemma}
\label{lemma-equivalence-coherent-over-field}
\begin{reference}
Weak version of the result in \cite{Gabriel}
stating that the category of quasi-coherent modules
determines the isomorphism class of a scheme.
\end{reference}
Let $k$ be a field. Let $X$, $Y$ be finite type schemes over $k$ with
$X$ separated and $Y$ reduced. If there is a $k$-linear equivalence
$F : \textit{Coh}(\mathcal{O}_X) \to \textit{Coh}(\mathcal{O}_Y)$
of categories, then there is an isomorphism $f : Y \to X$
over $k$ and an invertible $\mathcal{O}_Y$-module $\mathcal{L}$
such that $F(\mathcal{F}) = f^*\mathcal{F} \otimes \mathcal{L}$.
\end{lemma}

\begin{proof}[Proof using Gabriel-Rosenberg reconstruction]
This lemma is a weak form of the results discussed in
Lemma \ref{lemma-equivalence-coherent} and
Remark \ref{remark-equivalence-coherent-linear}.
\end{proof}

\begin{proof}[Proof not relying on Gabriel-Rosenberg reconstruction]
By Lemma \ref{lemma-functor-coherent-over-field} we obtain a coherent
$\mathcal{O}_{X \times Y}$-module $\mathcal{K}$ which is flat
over $X$ with support finite over $Y$ such that $F$ is given by
the restriction of the functor
(\ref{equation-FM-QCoh}) to $\textit{Coh}(\mathcal{O}_X)$.
If we can show that $F(\mathcal{O}_X)$ is an invertible $\mathcal{O}_Y$-module,
then by Lemma \ref{lemma-pushforward-invertible-pre}
we see that $\mathcal{K} = s_*\mathcal{L}$
for some section $s : Y \to X \times Y$ of $\text{pr}_2$ and some
invertible $\mathcal{O}_Y$-module $\mathcal{L}$.
This will show that $F$ has the form indicated with
$f = \text{pr}_1 \circ s$. Some details omitted.

\medskip\noindent
It remains to show that $F(\mathcal{O}_X)$ is invertible. We only
sketch the proof and we omit some of the details.
For a closed point $x \in X$ we denote
$\mathcal{O}_x$ in $\textit{Coh}(\mathcal{O}_X)$
the skyscraper sheaf at $x$ with value $\kappa(x)$.
First we observe that the only simple objects
of the category $\textit{Coh}(\mathcal{O}_X)$
are these skyscraper sheaves $\mathcal{O}_x$.
The same is true for $Y$. Hence for every closed point $y \in Y$
there exists a closed point $x \in X$ such that
$\mathcal{O}_y \cong F(\mathcal{O}_x)$. Moreover, looking at endomorphisms
we find that $\kappa(x) \cong \kappa(y)$ as finite extensions of $k$.
Then
$$
\Hom_Y(F(\mathcal{O}_X), \mathcal{O}_y) \cong
\Hom_Y(F(\mathcal{O}_X), F(\mathcal{O}_x)) \cong
\Hom_X(\mathcal{O}_X, \mathcal{O}_x) \cong \kappa(x) \cong \kappa(y)
$$
This implies that the stalk of the coherent $\mathcal{O}_Y$-module
$F(\mathcal{O}_X)$ at $y \in Y$ can be generated by $1$ generator
(and no less) for each closed point $y \in Y$. It follows immediately
that $F(\mathcal{O}_X)$ is locally generated by $1$ element (and no less)
and since $Y$ is reduced this indeed tells us it is an invertible module.
\end{proof}









\begin{multicols}{2}[\section{Other chapters}]
\noindent
Preliminaries
\begin{enumerate}
\item \hyperref[introduction-section-phantom]{Introduction}
\item \hyperref[conventions-section-phantom]{Conventions}
\item \hyperref[sets-section-phantom]{Set Theory}
\item \hyperref[categories-section-phantom]{Categories}
\item \hyperref[topology-section-phantom]{Topology}
\item \hyperref[sheaves-section-phantom]{Sheaves on Spaces}
\item \hyperref[sites-section-phantom]{Sites and Sheaves}
\item \hyperref[stacks-section-phantom]{Stacks}
\item \hyperref[fields-section-phantom]{Fields}
\item \hyperref[algebra-section-phantom]{Commutative Algebra}
\item \hyperref[brauer-section-phantom]{Brauer Groups}
\item \hyperref[homology-section-phantom]{Homological Algebra}
\item \hyperref[derived-section-phantom]{Derived Categories}
\item \hyperref[simplicial-section-phantom]{Simplicial Methods}
\item \hyperref[more-algebra-section-phantom]{More on Algebra}
\item \hyperref[smoothing-section-phantom]{Smoothing Ring Maps}
\item \hyperref[modules-section-phantom]{Sheaves of Modules}
\item \hyperref[sites-modules-section-phantom]{Modules on Sites}
\item \hyperref[injectives-section-phantom]{Injectives}
\item \hyperref[cohomology-section-phantom]{Cohomology of Sheaves}
\item \hyperref[sites-cohomology-section-phantom]{Cohomology on Sites}
\item \hyperref[dga-section-phantom]{Differential Graded Algebra}
\item \hyperref[dpa-section-phantom]{Divided Power Algebra}
\item \hyperref[sdga-section-phantom]{Differential Graded Sheaves}
\item \hyperref[hypercovering-section-phantom]{Hypercoverings}
\end{enumerate}
Schemes
\begin{enumerate}
\setcounter{enumi}{25}
\item \hyperref[schemes-section-phantom]{Schemes}
\item \hyperref[constructions-section-phantom]{Constructions of Schemes}
\item \hyperref[properties-section-phantom]{Properties of Schemes}
\item \hyperref[morphisms-section-phantom]{Morphisms of Schemes}
\item \hyperref[coherent-section-phantom]{Cohomology of Schemes}
\item \hyperref[divisors-section-phantom]{Divisors}
\item \hyperref[limits-section-phantom]{Limits of Schemes}
\item \hyperref[varieties-section-phantom]{Varieties}
\item \hyperref[topologies-section-phantom]{Topologies on Schemes}
\item \hyperref[descent-section-phantom]{Descent}
\item \hyperref[perfect-section-phantom]{Derived Categories of Schemes}
\item \hyperref[more-morphisms-section-phantom]{More on Morphisms}
\item \hyperref[flat-section-phantom]{More on Flatness}
\item \hyperref[groupoids-section-phantom]{Groupoid Schemes}
\item \hyperref[more-groupoids-section-phantom]{More on Groupoid Schemes}
\item \hyperref[etale-section-phantom]{\'Etale Morphisms of Schemes}
\end{enumerate}
Topics in Scheme Theory
\begin{enumerate}
\setcounter{enumi}{41}
\item \hyperref[chow-section-phantom]{Chow Homology}
\item \hyperref[intersection-section-phantom]{Intersection Theory}
\item \hyperref[pic-section-phantom]{Picard Schemes of Curves}
\item \hyperref[weil-section-phantom]{Weil Cohomology Theories}
\item \hyperref[adequate-section-phantom]{Adequate Modules}
\item \hyperref[dualizing-section-phantom]{Dualizing Complexes}
\item \hyperref[duality-section-phantom]{Duality for Schemes}
\item \hyperref[discriminant-section-phantom]{Discriminants and Differents}
\item \hyperref[derham-section-phantom]{de Rham Cohomology}
\item \hyperref[local-cohomology-section-phantom]{Local Cohomology}
\item \hyperref[algebraization-section-phantom]{Algebraic and Formal Geometry}
\item \hyperref[curves-section-phantom]{Algebraic Curves}
\item \hyperref[resolve-section-phantom]{Resolution of Surfaces}
\item \hyperref[models-section-phantom]{Semistable Reduction}
\item \hyperref[functors-section-phantom]{Functors and Morphisms}
\item \hyperref[equiv-section-phantom]{Derived Categories of Varieties}
\item \hyperref[pione-section-phantom]{Fundamental Groups of Schemes}
\item \hyperref[etale-cohomology-section-phantom]{\'Etale Cohomology}
\item \hyperref[crystalline-section-phantom]{Crystalline Cohomology}
\item \hyperref[proetale-section-phantom]{Pro-\'etale Cohomology}
\item \hyperref[relative-cycles-section-phantom]{Relative Cycles}
\item \hyperref[more-etale-section-phantom]{More \'Etale Cohomology}
\item \hyperref[trace-section-phantom]{The Trace Formula}
\end{enumerate}
Algebraic Spaces
\begin{enumerate}
\setcounter{enumi}{64}
\item \hyperref[spaces-section-phantom]{Algebraic Spaces}
\item \hyperref[spaces-properties-section-phantom]{Properties of Algebraic Spaces}
\item \hyperref[spaces-morphisms-section-phantom]{Morphisms of Algebraic Spaces}
\item \hyperref[decent-spaces-section-phantom]{Decent Algebraic Spaces}
\item \hyperref[spaces-cohomology-section-phantom]{Cohomology of Algebraic Spaces}
\item \hyperref[spaces-limits-section-phantom]{Limits of Algebraic Spaces}
\item \hyperref[spaces-divisors-section-phantom]{Divisors on Algebraic Spaces}
\item \hyperref[spaces-over-fields-section-phantom]{Algebraic Spaces over Fields}
\item \hyperref[spaces-topologies-section-phantom]{Topologies on Algebraic Spaces}
\item \hyperref[spaces-descent-section-phantom]{Descent and Algebraic Spaces}
\item \hyperref[spaces-perfect-section-phantom]{Derived Categories of Spaces}
\item \hyperref[spaces-more-morphisms-section-phantom]{More on Morphisms of Spaces}
\item \hyperref[spaces-flat-section-phantom]{Flatness on Algebraic Spaces}
\item \hyperref[spaces-groupoids-section-phantom]{Groupoids in Algebraic Spaces}
\item \hyperref[spaces-more-groupoids-section-phantom]{More on Groupoids in Spaces}
\item \hyperref[bootstrap-section-phantom]{Bootstrap}
\item \hyperref[spaces-pushouts-section-phantom]{Pushouts of Algebraic Spaces}
\end{enumerate}
Topics in Geometry
\begin{enumerate}
\setcounter{enumi}{81}
\item \hyperref[spaces-chow-section-phantom]{Chow Groups of Spaces}
\item \hyperref[groupoids-quotients-section-phantom]{Quotients of Groupoids}
\item \hyperref[spaces-more-cohomology-section-phantom]{More on Cohomology of Spaces}
\item \hyperref[spaces-simplicial-section-phantom]{Simplicial Spaces}
\item \hyperref[spaces-duality-section-phantom]{Duality for Spaces}
\item \hyperref[formal-spaces-section-phantom]{Formal Algebraic Spaces}
\item \hyperref[restricted-section-phantom]{Algebraization of Formal Spaces}
\item \hyperref[spaces-resolve-section-phantom]{Resolution of Surfaces Revisited}
\end{enumerate}
Deformation Theory
\begin{enumerate}
\setcounter{enumi}{89}
\item \hyperref[formal-defos-section-phantom]{Formal Deformation Theory}
\item \hyperref[defos-section-phantom]{Deformation Theory}
\item \hyperref[cotangent-section-phantom]{The Cotangent Complex}
\item \hyperref[examples-defos-section-phantom]{Deformation Problems}
\end{enumerate}
Algebraic Stacks
\begin{enumerate}
\setcounter{enumi}{93}
\item \hyperref[algebraic-section-phantom]{Algebraic Stacks}
\item \hyperref[examples-stacks-section-phantom]{Examples of Stacks}
\item \hyperref[stacks-sheaves-section-phantom]{Sheaves on Algebraic Stacks}
\item \hyperref[criteria-section-phantom]{Criteria for Representability}
\item \hyperref[artin-section-phantom]{Artin's Axioms}
\item \hyperref[quot-section-phantom]{Quot and Hilbert Spaces}
\item \hyperref[stacks-properties-section-phantom]{Properties of Algebraic Stacks}
\item \hyperref[stacks-morphisms-section-phantom]{Morphisms of Algebraic Stacks}
\item \hyperref[stacks-limits-section-phantom]{Limits of Algebraic Stacks}
\item \hyperref[stacks-cohomology-section-phantom]{Cohomology of Algebraic Stacks}
\item \hyperref[stacks-perfect-section-phantom]{Derived Categories of Stacks}
\item \hyperref[stacks-introduction-section-phantom]{Introducing Algebraic Stacks}
\item \hyperref[stacks-more-morphisms-section-phantom]{More on Morphisms of Stacks}
\item \hyperref[stacks-geometry-section-phantom]{The Geometry of Stacks}
\end{enumerate}
Topics in Moduli Theory
\begin{enumerate}
\setcounter{enumi}{107}
\item \hyperref[moduli-section-phantom]{Moduli Stacks}
\item \hyperref[moduli-curves-section-phantom]{Moduli of Curves}
\end{enumerate}
Miscellany
\begin{enumerate}
\setcounter{enumi}{109}
\item \hyperref[examples-section-phantom]{Examples}
\item \hyperref[exercises-section-phantom]{Exercises}
\item \hyperref[guide-section-phantom]{Guide to Literature}
\item \hyperref[desirables-section-phantom]{Desirables}
\item \hyperref[coding-section-phantom]{Coding Style}
\item \hyperref[obsolete-section-phantom]{Obsolete}
\item \hyperref[fdl-section-phantom]{GNU Free Documentation License}
\item \hyperref[index-section-phantom]{Auto Generated Index}
\end{enumerate}
\end{multicols}


\bibliography{my}
\bibliographystyle{amsalpha}

\end{document}

