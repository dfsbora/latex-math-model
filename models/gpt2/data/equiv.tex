\IfFileExists{stacks-project.cls}{%
\documentclass{stacks-project}
}{%
\documentclass{amsart}
}

% For dealing with references we use the comment environment
\usepackage{verbatim}
\newenvironment{reference}{\comment}{\endcomment}
%\newenvironment{reference}{}{}
\newenvironment{slogan}{\comment}{\endcomment}
\newenvironment{history}{\comment}{\endcomment}

% For commutative diagrams we use Xy-pic
\usepackage[all]{xy}

% We use 2cell for 2-commutative diagrams.
\xyoption{2cell}
\UseAllTwocells

% We use multicol for the list of chapters between chapters
\usepackage{multicol}

% This is generall recommended for better output
\usepackage{lmodern}
\usepackage[T1]{fontenc}

% For cross-file-references
\usepackage{xr-hyper}

% Package for hypertext links:
\usepackage{hyperref}

% For any local file, say "hello.tex" you want to link to please
% use \externaldocument[hello-]{hello}
\externaldocument[introduction-]{introduction}
\externaldocument[conventions-]{conventions}
\externaldocument[sets-]{sets}
\externaldocument[categories-]{categories}
\externaldocument[topology-]{topology}
\externaldocument[sheaves-]{sheaves}
\externaldocument[sites-]{sites}
\externaldocument[stacks-]{stacks}
\externaldocument[fields-]{fields}
\externaldocument[algebra-]{algebra}
\externaldocument[brauer-]{brauer}
\externaldocument[homology-]{homology}
\externaldocument[derived-]{derived}
\externaldocument[simplicial-]{simplicial}
\externaldocument[more-algebra-]{more-algebra}
\externaldocument[smoothing-]{smoothing}
\externaldocument[modules-]{modules}
\externaldocument[sites-modules-]{sites-modules}
\externaldocument[injectives-]{injectives}
\externaldocument[cohomology-]{cohomology}
\externaldocument[sites-cohomology-]{sites-cohomology}
\externaldocument[dga-]{dga}
\externaldocument[dpa-]{dpa}
\externaldocument[sdga-]{sdga}
\externaldocument[hypercovering-]{hypercovering}
\externaldocument[schemes-]{schemes}
\externaldocument[constructions-]{constructions}
\externaldocument[properties-]{properties}
\externaldocument[morphisms-]{morphisms}
\externaldocument[coherent-]{coherent}
\externaldocument[divisors-]{divisors}
\externaldocument[limits-]{limits}
\externaldocument[varieties-]{varieties}
\externaldocument[topologies-]{topologies}
\externaldocument[descent-]{descent}
\externaldocument[perfect-]{perfect}
\externaldocument[more-morphisms-]{more-morphisms}
\externaldocument[flat-]{flat}
\externaldocument[groupoids-]{groupoids}
\externaldocument[more-groupoids-]{more-groupoids}
\externaldocument[etale-]{etale}
\externaldocument[chow-]{chow}
\externaldocument[intersection-]{intersection}
\externaldocument[pic-]{pic}
\externaldocument[weil-]{weil}
\externaldocument[adequate-]{adequate}
\externaldocument[dualizing-]{dualizing}
\externaldocument[duality-]{duality}
\externaldocument[discriminant-]{discriminant}
\externaldocument[derham-]{derham}
\externaldocument[local-cohomology-]{local-cohomology}
\externaldocument[algebraization-]{algebraization}
\externaldocument[curves-]{curves}
\externaldocument[resolve-]{resolve}
\externaldocument[models-]{models}
\externaldocument[functors-]{functors}
\externaldocument[equiv-]{equiv}
\externaldocument[pione-]{pione}
\externaldocument[etale-cohomology-]{etale-cohomology}
\externaldocument[proetale-]{proetale}
\externaldocument[relative-cycles-]{relative-cycles}
\externaldocument[more-etale-]{more-etale}
\externaldocument[trace-]{trace}
\externaldocument[crystalline-]{crystalline}
\externaldocument[spaces-]{spaces}
\externaldocument[spaces-properties-]{spaces-properties}
\externaldocument[spaces-morphisms-]{spaces-morphisms}
\externaldocument[decent-spaces-]{decent-spaces}
\externaldocument[spaces-cohomology-]{spaces-cohomology}
\externaldocument[spaces-limits-]{spaces-limits}
\externaldocument[spaces-divisors-]{spaces-divisors}
\externaldocument[spaces-over-fields-]{spaces-over-fields}
\externaldocument[spaces-topologies-]{spaces-topologies}
\externaldocument[spaces-descent-]{spaces-descent}
\externaldocument[spaces-perfect-]{spaces-perfect}
\externaldocument[spaces-more-morphisms-]{spaces-more-morphisms}
\externaldocument[spaces-flat-]{spaces-flat}
\externaldocument[spaces-groupoids-]{spaces-groupoids}
\externaldocument[spaces-more-groupoids-]{spaces-more-groupoids}
\externaldocument[bootstrap-]{bootstrap}
\externaldocument[spaces-pushouts-]{spaces-pushouts}
\externaldocument[spaces-chow-]{spaces-chow}
\externaldocument[groupoids-quotients-]{groupoids-quotients}
\externaldocument[spaces-more-cohomology-]{spaces-more-cohomology}
\externaldocument[spaces-simplicial-]{spaces-simplicial}
\externaldocument[spaces-duality-]{spaces-duality}
\externaldocument[formal-spaces-]{formal-spaces}
\externaldocument[restricted-]{restricted}
\externaldocument[spaces-resolve-]{spaces-resolve}
\externaldocument[formal-defos-]{formal-defos}
\externaldocument[defos-]{defos}
\externaldocument[cotangent-]{cotangent}
\externaldocument[examples-defos-]{examples-defos}
\externaldocument[algebraic-]{algebraic}
\externaldocument[examples-stacks-]{examples-stacks}
\externaldocument[stacks-sheaves-]{stacks-sheaves}
\externaldocument[criteria-]{criteria}
\externaldocument[artin-]{artin}
\externaldocument[quot-]{quot}
\externaldocument[stacks-properties-]{stacks-properties}
\externaldocument[stacks-morphisms-]{stacks-morphisms}
\externaldocument[stacks-limits-]{stacks-limits}
\externaldocument[stacks-cohomology-]{stacks-cohomology}
\externaldocument[stacks-perfect-]{stacks-perfect}
\externaldocument[stacks-introduction-]{stacks-introduction}
\externaldocument[stacks-more-morphisms-]{stacks-more-morphisms}
\externaldocument[stacks-geometry-]{stacks-geometry}
\externaldocument[moduli-]{moduli}
\externaldocument[moduli-curves-]{moduli-curves}
\externaldocument[examples-]{examples}
\externaldocument[exercises-]{exercises}
\externaldocument[guide-]{guide}
\externaldocument[desirables-]{desirables}
\externaldocument[coding-]{coding}
\externaldocument[obsolete-]{obsolete}
\externaldocument[fdl-]{fdl}
\externaldocument[index-]{index}

% Theorem environments.
%
\theoremstyle{plain}
\newtheorem{theorem}[subsection]{Theorem}
\newtheorem{proposition}[subsection]{Proposition}
\newtheorem{lemma}[subsection]{Lemma}

\theoremstyle{definition}
\newtheorem{definition}[subsection]{Definition}
\newtheorem{example}[subsection]{Example}
\newtheorem{exercise}[subsection]{Exercise}
\newtheorem{situation}[subsection]{Situation}

\theoremstyle{remark}
\newtheorem{remark}[subsection]{Remark}
\newtheorem{remarks}[subsection]{Remarks}

\numberwithin{equation}{subsection}

% Macros
%
\def\lim{\mathop{\mathrm{lim}}\nolimits}
\def\colim{\mathop{\mathrm{colim}}\nolimits}
\def\Spec{\mathop{\mathrm{Spec}}}
\def\Hom{\mathop{\mathrm{Hom}}\nolimits}
\def\Ext{\mathop{\mathrm{Ext}}\nolimits}
\def\SheafHom{\mathop{\mathcal{H}\!\mathit{om}}\nolimits}
\def\SheafExt{\mathop{\mathcal{E}\!\mathit{xt}}\nolimits}
\def\Sch{\mathit{Sch}}
\def\Mor{\mathop{\mathrm{Mor}}\nolimits}
\def\Ob{\mathop{\mathrm{Ob}}\nolimits}
\def\Sh{\mathop{\mathit{Sh}}\nolimits}
\def\NL{\mathop{N\!L}\nolimits}
\def\CH{\mathop{\mathrm{CH}}\nolimits}
\def\proetale{{pro\text{-}\acute{e}tale}}
\def\etale{{\acute{e}tale}}
\def\QCoh{\mathit{QCoh}}
\def\Ker{\mathop{\mathrm{Ker}}}
\def\Im{\mathop{\mathrm{Im}}}
\def\Coker{\mathop{\mathrm{Coker}}}
\def\Coim{\mathop{\mathrm{Coim}}}

% Boxtimes
%
\DeclareMathSymbol{\boxtimes}{\mathbin}{AMSa}{"02}

%
% Macros for moduli stacks/spaces
%
\def\QCohstack{\mathcal{QC}\!\mathit{oh}}
\def\Cohstack{\mathcal{C}\!\mathit{oh}}
\def\Spacesstack{\mathcal{S}\!\mathit{paces}}
\def\Quotfunctor{\mathrm{Quot}}
\def\Hilbfunctor{\mathrm{Hilb}}
\def\Curvesstack{\mathcal{C}\!\mathit{urves}}
\def\Polarizedstack{\mathcal{P}\!\mathit{olarized}}
\def\Complexesstack{\mathcal{C}\!\mathit{omplexes}}
% \Pic is the operator that assigns to X its picard group, usage \Pic(X)
% \Picardstack_{X/B} denotes the Picard stack of X over B
% \Picardfunctor_{X/B} denotes the Picard functor of X over B
\def\Pic{\mathop{\mathrm{Pic}}\nolimits}
\def\Picardstack{\mathcal{P}\!\mathit{ic}}
\def\Picardfunctor{\mathrm{Pic}}
\def\Deformationcategory{\mathcal{D}\!\mathit{ef}}


% OK, start here.
%
\begin{document}

\title{Derived Categories of Varieties}


\maketitle

\phantomsection
\label{section-phantom}

\tableofcontents

\section{Introduction}
\label{section-introduction}

\noindent
In this chapter we continue the discussion started in
Derived Categories of Schemes, Section \ref{perfect-section-introduction}.
We will discuss Fourier-Mukai transforms,
first studied by Mukai in \cite{Mukai}.
We will prove Orlov's theorem on derived equivalences (\cite{Orlov-K3}).
We also discuss the countability of derived equivalence
classes proved by Anel and To\"en in \cite{AT}.

\medskip\noindent
A good introduction to this material is the book
\cite{Huybrechts} by Daniel Huybrechts. Some other
papers which helped popularize this topic are
\begin{enumerate}
\item the paper by Bondal and Kapranov, see \cite{Bondal-Kapranov}
\item the paper by Bondal and Orlov, see \cite{Bondal-Orlov}
\item the paper by Bondal and Van den Bergh, see \cite{BvdB}
\item the papers by Beilinson, see
\cite{Beilinson} and \cite{Beilinson-derived}
\item the paper by Orlov, see \cite{Orlov-AV}
\item the paper by Orlov, see \cite{Orlov-motives}
\item the paper by Rouquier, see \cite{Rouquier-dimensions}
\item there are many more we could mention here.
\end{enumerate}




\section{Conventions and notation}
\label{section-conventions}

\noindent
Let $k$ be a field. A $k$-linear triangulated category $\mathcal{T}$
is a triangulated category (Derived Categories, Section
\ref{derived-section-triangulated-definitions})
which is endowed with a $k$-linear structure
(Differential Graded Algebra, Section \ref{dga-section-linear})
such that the translation functors $[n] : \mathcal{T} \to \mathcal{T}$
are $k$-linear for all $n \in \mathbf{Z}$.

\medskip\noindent
Let $k$ be a field. We denote $\text{Vect}_k$ the category of
$k$-vector spaces. For a $k$-vector space $V$ we denote
$V^\vee$ the $k$-linear dual of $V$, i.e., $V^\vee = \Hom_k(V, k)$.

\medskip\noindent
Let $X$ be a scheme. We denote $D_{perf}(\mathcal{O}_X)$ the full
subcategory of $D(\mathcal{O}_X)$ consisting of perfect complexes
(Cohomology, Section \ref{cohomology-section-perfect}).
If $X$ is Noetherian then
$D_{perf}(\mathcal{O}_X) \subset D^b_{\textit{Coh}}(\mathcal{O}_X)$, see
Derived Categories of Schemes, Lemma \ref{perfect-lemma-perfect-on-noetherian}.
If $X$ is Noetherian and regular, then
$D_{perf}(\mathcal{O}_X) = D^b_{\textit{Coh}}(\mathcal{O}_X)$, see
Derived Categories of Schemes, Lemma \ref{perfect-lemma-perfect-on-regular}.

\medskip\noindent
Let $k$ be a field. Let $X$ and $Y$ be schemes over $k$. In this
situation we will write $X \times Y$ instead of $X \times_{\Spec(k)} Y$.


\medskip\noindent
Let $S$ be a scheme. Let $X$, $Y$ be schemes over $S$.
Let $\mathcal{F}$ be a $\mathcal{O}_X$-module and let
$\mathcal{G}$ be a $\mathcal{O}_Y$-module. We set
$$
\mathcal{F} \boxtimes \mathcal{G} =
\text{pr}_1^*\mathcal{F} \otimes_{\mathcal{O}_{X \times_S Y}}
\text{pr}_2^*\mathcal{G}
$$
as $\mathcal{O}_{X \times_S Y}$-modules.
If $K \in D(\mathcal{O}_X)$ and $M \in D(\mathcal{O}_Y)$ then we set
$$
K \boxtimes M =
L\text{pr}_1^*K \otimes_{\mathcal{O}_{X \times_S Y}}^\mathbf{L} L\text{pr}_2^*M
$$
as an object of $D(\mathcal{O}_{X \times_S Y})$.
Thus our notation is potentially ambiguous, but context should make it clear
which of the two is meant.





\section{Serre functors}
\label{section-Serre-functors}

\noindent
The material in this section is taken from \cite{Bondal-Kapranov}.

\begin{lemma}
\label{lemma-Serre-functor-exists}
Let $k$ be a field. Let $\mathcal{T}$ be a $k$-linear
triangulated category such that $\dim_k \Hom_\mathcal{T}(X, Y) < \infty$
for all $X, Y \in \Ob(\mathcal{T})$. The following are equivalent
\begin{enumerate}
\item there exists a $k$-linear equivalence
$S : \mathcal{T} \to \mathcal{T}$ and $k$-linear isomorphisms
$c_{X, Y} : \Hom_\mathcal{T}(X, Y) \to \Hom_\mathcal{T}(Y, S(X))^\vee$
functorial in $X, Y \in \Ob(\mathcal{T})$,
\item for every $X \in \Ob(\mathcal{T})$
the functor $Y \mapsto \Hom_\mathcal{T}(X, Y)^\vee$
is representable and the functor $Y \mapsto \Hom_\mathcal{T}(Y, X)^\vee$
is corepresentable.
\end{enumerate}
\end{lemma}

\begin{proof}
Condition (1) implies (2) since given $(S, c)$ and $X \in \Ob(\mathcal{T})$
the object $S(X)$ represents the functor
$Y \mapsto \Hom_\mathcal{T}(X, Y)^\vee$ and the object $S^{-1}(X)$ corepresents
the functor $Y \mapsto \Hom_\mathcal{T}(Y, X)^\vee$.

\medskip\noindent
Assume (2). We will repeatedly use the Yoneda lemma, see
Categories, Lemma \ref{categories-lemma-yoneda}.
For every $X$ denote $S(X)$ the object representing the
functor $Y \mapsto \Hom_\mathcal{T}(X, Y)^\vee$. Given
$\varphi : X \to X'$, we obtain a unique arrow $S(\varphi) : S(X) \to S(X')$
determined by the corresponding transformation of functors
$\Hom_\mathcal{T}(X, -)^\vee \to \Hom_\mathcal{T}(X', -)^\vee$.
Thus $S$ is a functor and we obtain the isomorphisms $c_{X, Y}$
by construction. It remains to show that $S$ is an equivalence.
For every $X$ denote $S'(X)$ the object corepresenting the
functor $Y \mapsto \Hom_\mathcal{T}(Y, X)^\vee$. Arguing as
above we find that $S'$ is a functor. We claim that $S'$
is quasi-inverse to $S$. To see this observe that
$$
\Hom_\mathcal{T}(X, Y) = \Hom_\mathcal{T}(Y, S(X))^\vee =
\Hom_\mathcal{T}(S'(S(X)), Y)
$$
bifunctorially, i.e., we find $S' \circ S \cong \text{id}_\mathcal{T}$.
Similarly, we have
$$
\Hom_\mathcal{T}(Y, X) = \Hom_\mathcal{T}(S'(X), Y)^\vee =
\Hom_\mathcal{T}(Y, S(S'(X)))
$$
and we find $S \circ S' \cong \text{id}_\mathcal{T}$.
\end{proof}

\begin{definition}
\label{definition-Serre-functor}
Let $k$ be a field. Let $\mathcal{T}$ be a $k$-linear
triangulated category such that $\dim_k \Hom_\mathcal{T}(X, Y) < \infty$
for all $X, Y \in \Ob(\mathcal{T})$. We say {\it a Serre functor
exists} if the equivalent conditions of Lemma \ref{lemma-Serre-functor-exists}
are satisfied. In this case a {\it Serre functor} is a $k$-linear equivalence
$S : \mathcal{T} \to \mathcal{T}$ endowed with $k$-linear isomorphisms
$c_{X, Y} : \Hom_\mathcal{T}(X, Y) \to \Hom_\mathcal{T}(Y, S(X))^\vee$
functorial in $X, Y \in \Ob(\mathcal{T})$.
\end{definition}

\begin{lemma}
\label{lemma-Serre-functor}
In the situation of Definition \ref{definition-Serre-functor}.
If a Serre functor exists, then it is unique up to unique isomorphism and
it is an exact functor of triangulated categories.
\end{lemma}

\begin{proof}
Given a Serre functor $S$ the object $S(X)$ represents
the functor $Y \mapsto \Hom_\mathcal{T}(X, Y)^\vee$.
Thus the object $S(X)$ together with the functorial identification
$\Hom_\mathcal{T}(X, Y)^\vee = \Hom_\mathcal{T}(Y, S(X))$
is determined up to unique isomorphism by the Yoneda lemma
(Categories, Lemma \ref{categories-lemma-yoneda}).
Moreover, for $\varphi : X \to X'$, the arrow $S(\varphi) : S(X) \to S(X')$
is uniquely determined by the corresponding transformation of functors
$\Hom_\mathcal{T}(X, -)^\vee \to \Hom_\mathcal{T}(X', -)^\vee$.

\medskip\noindent
For objects $X, Y$ of $\mathcal{T}$ we have
\begin{align*}
\Hom(Y, S(X)[1])^\vee
& =
\Hom(Y[-1], S(X))^\vee \\
& =
\Hom(X, Y[-1]) \\
& =
\Hom(X[1], Y) \\
& =
\Hom(Y, S(X[1]))^\vee
\end{align*}
By the Yoneda lemma we conclude that there is a unique isomorphism
$S(X[1]) \to S(X)[1]$ inducing the isomorphism from top left to bottom right.
Since each of the isomorphisms above is functorial in both $X$ and $Y$
we find that this defines an isomorphism of functors
$S \circ [1] \to [1] \circ S$.

\medskip\noindent
Let $(A, B, C, f, g, h)$ be a distinguished triangle in $\mathcal{T}$.
We have to show that the triangle $(S(A), S(B), S(C), S(f), S(g), S(h))$
is distinguished. Here we use the canonical isomorphism $S(A[1]) \to S(A)[1]$
constructed above to identify the target $S(A[1])$ of $S(h)$ with $S(A)[1]$.
We first observe that for any $X$ in $\mathcal{T}$
the triangle $(S(A), S(B), S(C), S(f), S(g), S(h))$ induces
a long exact sequence
$$
\ldots \to
\Hom(X, S(A)) \to
\Hom(X, S(B)) \to
\Hom(X, S(C)) \to
\Hom(X, S(A)[1]) \to \ldots
$$
of finite dimensional $k$-vector spaces. Namely, this sequence is
$k$-linear dual of the sequence
$$
\ldots \leftarrow
\Hom(A, X) \leftarrow
\Hom(B, X) \leftarrow
\Hom(C, X) \leftarrow
\Hom(A[1], X) \leftarrow
\ldots
$$
which is exact by Derived Categories, Lemma
\ref{derived-lemma-representable-homological}.
Next, we choose a distinguished triangle $(S(A), E, S(C), i, p, S(h))$
which is possible by axioms TR1 and TR2. We want to construct the dotted
arrow making following diagram commute
$$
\xymatrix{
S(C)[-1] \ar[r]_-{S(h[-1])} &
S(A) \ar[r]_{S(f)} &
S(B) \ar[r]_{S(g)} &
S(C) \ar[r]_{S(h)} &
S(A)[1] \\
S(C)[-1] \ar[r]^-{S(h[-1])} \ar@{=}[u] &
S(A) \ar[r]^i \ar@{=}[u] &
E \ar[r]^p \ar@{..>}[u]^\varphi &
S(C) \ar[r]^{S(h)} \ar@{=}[u] &
S(A)[1] \ar@{=}[u]
}
$$
Namely, if we have $\varphi$, then we claim for any $X$ the resulting
map $\Hom(X, E) \to \Hom(X, S(B))$ will be an isomorphism of $k$-vector
spaces. Namely, we will obtain a commutative diagram
$$
\xymatrix{
\Hom(X, S(C)[-1]) \ar[r] &
\Hom(X, S(A)) \ar[r] &
\Hom(X, S(B)) \ar[r] &
\Hom(X, S(C)) \ar[r] &
\Hom(X, S(A)[1]) \\
\Hom(X, S(C)[-1]) \ar[r] \ar@{=}[u] &
\Hom(X, S(A)) \ar[r] \ar@{=}[u] &
\Hom(X, E) \ar[r] \ar[u]^\varphi &
\Hom(X, S(C)) \ar[r] \ar@{=}[u] &
\Hom(X, S(A)[1]) \ar@{=}[u]
}
$$
with exact rows (see above) and we can apply the 5 lemma
(Homology, Lemma \ref{homology-lemma-five-lemma}) to see
that the middle arrow is an isomorphism. By the Yoneda lemma
we conclude that $\varphi$ is an isomorphism.
To find $\varphi$ consider the following diagram
$$
\xymatrix{
\Hom(E, S(C)) \ar[r] &
\Hom(S(A), S(C)) \\
\Hom(E, S(B)) \ar[u] \ar[r] &
\Hom(S(A), S(B)) \ar[u]
}
$$
The elements $p$ and $S(f)$ in positions $(0, 1)$ and
$(1, 0)$ define a cohomology class $\xi$ in the total complex
of this double complex. The existence of $\varphi$ is
equivalent to whether $\xi$ is zero. If we take $k$-linear duals
of this and we use the defining property of $S$ we obtain
$$
\xymatrix{
\Hom(C, E) \ar[d] &
\Hom(C, S(A)) \ar[l] \ar[d] \\
\Hom(B, E) &
\Hom(B, S(A)) \ar[l]
}
$$
Since both $A \to B \to C$ and $S(A) \to E \to S(C)$ are distinguished
triangles, we know by TR3 that given elements $\alpha \in \Hom(C, E)$
and $\beta \in \Hom(B, S(A))$ mapping to the same element in
$\Hom(B, E)$, there exists an element in $\Hom(C, S(A))$ mapping
to both $\alpha$ and $\beta$. In other words, the cohomology of
the total complex associated to this double complex is zero in degree
$1$, i.e., the degree corresponding to $\Hom(C, E) \oplus \Hom(B, S(A))$.
Taking duals the same must be true for the previous one which concludes
the proof.
\end{proof}






\section{Examples of Serre functors}
\label{section-examples-Serre-functors}

\noindent
The lemma below is the standard example.

\begin{lemma}
\label{lemma-Serre-functor-Gorenstein-proper}
Let $k$ be a field. Let $X$ be a proper scheme over $k$ which is Gorenstein.
Consider the complex $\omega_X^\bullet$ of
Duality for Schemes, Lemmas \ref{duality-lemma-duality-proper-over-field}.
Then the functor
$$
S : D_{perf}(\mathcal{O}_X) \longrightarrow D_{perf}(\mathcal{O}_X),\quad
K \longmapsto S(K) = \omega_X^\bullet \otimes_{\mathcal{O}_X}^\mathbf{L} K
$$
is a Serre functor.
\end{lemma}

\begin{proof}
The statement make sense because $\dim \Hom_X(K, L) < \infty$
for $K, L \in D_{perf}(\mathcal{O}_X)$ by
Derived Categories of Schemes, Lemma \ref{perfect-lemma-ext-finite}.
Since $X$ is Gorenstein the dualizing complex $\omega_X^\bullet$
is an invertible object of $D(\mathcal{O}_X)$, see
Duality for Schemes, Lemma \ref{duality-lemma-gorenstein}.
In particular, locally on $X$ the complex $\omega_X^\bullet$
has one nonzero cohomology sheaf which is an invertible module, see
Cohomology, Lemma \ref{cohomology-lemma-invertible-derived}.
Thus $S(K)$ lies in $D_{perf}(\mathcal{O}_X)$.
On the other hand, the invertibility of $\omega_X^\bullet$
clearly implies that $S$ is a self-equivalence of $D_{perf}(\mathcal{O}_X)$.
Finally, we have to find an isomorphism
$$
c_{K, L} : \Hom_X(K, L) \longrightarrow
\Hom_X(L, \omega_X^\bullet \otimes_{\mathcal{O}_X}^\mathbf{L} K)^\vee
$$
bifunctorially in $K, L$. To do this we use the canonical isomorphisms
$$
\Hom_X(K, L) = H^0(X, L \otimes_{\mathcal{O}_X}^\mathbf{L} K^\vee)
$$
and
$$
\Hom_X(L, \omega_X^\bullet \otimes_{\mathcal{O}_X}^\mathbf{L} K) =
H^0(X, 
\omega_X^\bullet \otimes_{\mathcal{O}_X}^\mathbf{L} K
\otimes_{\mathcal{O}_X}^\mathbf{L} L^\vee)
$$
given in Cohomology, Lemma \ref{cohomology-lemma-dual-perfect-complex}.
Since $(L \otimes_{\mathcal{O}_X}^\mathbf{L} K^\vee)^\vee =
(K^\vee)^\vee \otimes_{\mathcal{O}_X}^\mathbf{L} L^\vee$
and since there is a canonical isomorphism $K \to (K^\vee)^\vee$
we find these $k$-vector spaces are canonically dual by
Duality for Schemes, Lemma
\ref{duality-lemma-duality-proper-over-field-perfect}.
This produces the isomorphisms $c_{K, L}$.
We omit the proof that these isomorphisms are functorial.
\end{proof}





\section{Characterizing coherent modules}
\label{section-coherent}

\noindent
This section is in some sense a continuation of the discussion
in Derived Categories of Schemes, Section \ref{perfect-section-pseudo-coherent}
and More on Morphisms, Section
\ref{more-morphisms-section-characterize-pseudo-coherent}.

\medskip\noindent
Before we can state the result we need some notation.
Let $k$ be a field. Let $n \geq 0$ be an integer.
Let $S = k[X_0, \ldots, X_n]$. For an integer $e$ denote
$S_e \subset S$ the homogeneous polynomials of degree $e$.
Consider the (noncommutative) $k$-algebra
$$
R =
\left(
\begin{matrix}
S_0 & S_1 & S_2 & \ldots & \ldots \\
0 & S_0 & S_1 & \ldots & \ldots\\
0 & 0 & S_0 & \ldots & \ldots \\
\ldots & \ldots & \ldots & \ldots & \ldots \\
0 & \ldots & \ldots & \ldots & S_0
\end{matrix}
\right)
$$
(with $n + 1$ rows and columns) with obvious multiplication and addition.

\begin{lemma}
\label{lemma-perfect-for-R}
With $k$, $n$, and $R$ as above, for an object $K$ of $D(R)$
the following are equivalent
\begin{enumerate}
\item $\sum_{i \in \mathbf{Z}} \dim_k H^i(K) < \infty$, and
\item $K$ is a compact object.
\end{enumerate}
\end{lemma}

\begin{proof}
If $K$ is a compact object, then $K$ can be represented by a complex
$M^\bullet$ which is finite projective as a graded $R$-module, see
Differential Graded Algebra, Lemma \ref{dga-lemma-compact}.
Since $\dim_k R < \infty$ we conclude $\sum \dim_k M^i < \infty$
and a fortiori $\sum \dim_k H^i(M^\bullet) < \infty$.
(One can also easily deduce this implication from the easier
Differential Graded Algebra, Proposition \ref{dga-proposition-compact}.)

\medskip\noindent
Assume $K$ satisfies (1). Consider the distinguished triangle
of trunctions $\tau_{\leq m}K \to K \to \tau_{\geq m + 1}K$, see
Derived Categories, Remark
\ref{derived-remark-truncation-distinguished-triangle}.
It is clear that both $\tau_{\leq m}K$ and $\tau_{\geq m + 1} K$
satisfy (1). If we can show both are compact, then so is $K$, see
Derived Categories, Lemma \ref{derived-lemma-compact-objects-subcategory}.
Hence, arguing on the number of nonzero cohomology modules of $K$
we may assume $H^i(K)$ is nonzero only for one $i$.
Shifting, we may assume $K$ is given by the complex
consisting of a single finite dimensional $R$-module $M$ sitting
in degree $0$.

\medskip\noindent
Since $\dim_k(M) < \infty$ we see that $M$ is Artinian as an $R$-module.
Thus it suffices to show that every simple $R$-module represents a
compact object of $D(R)$. Observe that
$$
I =
\left(
\begin{matrix}
0 & S_1 & S_2 & \ldots & \ldots \\
0 & 0 & S_1 & \ldots & \ldots\\
0 & 0 & 0 & \ldots & \ldots \\
\ldots & \ldots & \ldots & \ldots & \ldots \\
0 & \ldots & \ldots & \ldots & 0
\end{matrix}
\right)
$$
is a nilpotent two sided ideal of $R$ and that $R/I$
is a commutative $k$-algebra isomorphic to a product of $n + 1$ copies of
$k$ (placed along the diagonal in the matrix, i.e., $R/I$ can be lifted
to a $k$-subalgebra of $R$). It follows that $R$ has exactly $n + 1$
isomorphism classes of simple modules $M_0, \ldots, M_n$ (sitting along
the diagonal). Consider the right $R$-module $P_i$ of row vectors
$$
P_i =
\left(
\begin{matrix}
0 &
\ldots &
0 &
S_0 &
\ldots &
S_{i - 1} &
S_i
\end{matrix}
\right)
$$
with obvious multiplication $P_i \times R \to P_i$. Then we see that
$R \cong P_0 \oplus \ldots \oplus P_n$ as a right $R$-module. Since clearly
$R$ is a compact object of $D(R)$, we conclude each $P_i$ is a compact
object of $D(R)$. (We of course also conclude each $P_i$ is projective
as an $R$-module, but this isn't what we have to show in this proof.)
Clearly, $P_0 = M_0$ is the first of our simple $R$-modules.
For $P_1$ we have a short exact sequence
$$
0 \to P_0^{\oplus n + 1} \to P_1 \to M_1 \to 0
$$
which proves that $M_1$ fits into a distinguished triangle whose
other members are compact objects and hence $M_1$ is a compact
object of $D(R)$. More generally, there exists a short exact sequence
$$
0 \to C_i \to P_i \to M_i \to 0
$$
where $C_i$ is a finite dimensional $R$-module whose simple constituents
are isomorphic to $M_j$ for $j < i$. By induction, we first conclude that
$C_i$ determines a compact object of $D(R)$ whereupon we conclude that $M_i$
does too as desired.
\end{proof}

\begin{lemma}
\label{lemma-coherent-on-projective-space}
Let $k$ be a field. Let $n \geq 0$. Let
$K \in D_\QCoh(\mathcal{O}_{\mathbf{P}^n_k})$.
The following are equivalent
\begin{enumerate}
\item $K$ is in $D^b_{\textit{Coh}}(\mathcal{O}_{\mathbf{P}^n_k})$,
\item $\sum_{i \in \mathbf{Z}}
\dim_k H^i(\mathbf{P}^n_k, E \otimes^\mathbf{L} K) < \infty$
for each perfect object $E$ of
$D(\mathcal{O}_{\mathbf{P}^n_k})$,
\item $\sum_{i \in \mathbf{Z}}
\dim_k \Ext^i_{\mathbf{P}^n_k}(E, K) < \infty$
for each perfect object $E$ of $D(\mathcal{O}_{\mathbf{P}^n_k})$,
\item $\sum_{i \in \mathbf{Z}} \dim_k H^i(\mathbf{P}^n_k,
K \otimes^\mathbf{L} \mathcal{O}_{\mathbf{P}^n_k}(d)) < \infty$
for $d = 0, 1, \ldots, n$.
\end{enumerate}
\end{lemma}

\begin{proof}
Parts (2) and (3) are equivalent by
Cohomology, Lemma \ref{cohomology-lemma-dual-perfect-complex}.
If (1) is true, then for $E$ perfect the derived tensor product
$E \otimes^\mathbf{L} K$ is in
$D^b_{\textit{Coh}}(\mathcal{O}_{\mathbf{P}^n_k})$
and we see that (2) holds by 
Derived Categories of Schemes, Lemma \ref{perfect-lemma-direct-image-coherent}.
It is clear that (2) implies (4) as $\mathcal{O}_{\mathbf{P}^n_k}(d)$
can be viewed
as a perfect object of the derived category of $\mathbf{P}^n_k$.
Thus it suffices to prove that (4) implies (1).

\medskip\noindent
Assume (4). Let $R$ be as in Lemma \ref{lemma-perfect-for-R}.
Let $P = \bigoplus_{d = 0, \ldots, n} \mathcal{O}_{\mathbf{P}^n_k}(-d)$.
Recall that $R = \text{End}_{\mathbf{P}^n_k}(P)$ whereas all other
self-Exts of $P$ are zero and that $P$ determines an equivalence
$- \otimes^\mathbf{L} P : D(R) \to D_\QCoh(\mathcal{O}_{\mathbf{P}^n_k})$
by Derived Categories of Schemes, Lemma \ref{perfect-lemma-Pn-module-category}.
Say $K$ corresponds to $L$ in $D(R)$. Then
\begin{align*}
H^i(L)
& =
\Ext^i_{D(R)}(R, L) \\
& =
\Ext^i_{\mathbf{P}^n_k}(P, K) \\
& =
H^i(\mathbf{P}^n_k, K \otimes P^\vee) \\
& =
\bigoplus\nolimits_{d = 0, \ldots, n}
H^i(\mathbf{P}^n_k, K \otimes \mathcal{O}(d))
\end{align*}
by Differential Graded Algebra, Lemma
\ref{dga-lemma-upgrade-tensor-with-complex-derived}
(and the fact that $- \otimes^\mathbf{L} P$ is an equivalence)
and Cohomology, Lemma \ref{cohomology-lemma-dual-perfect-complex}.
Thus our assumption (4) implies that $L$ satisfies condition (2) of
Lemma \ref{lemma-perfect-for-R} and hence is a compact object of $D(R)$.
Therefore $K$ is a compact object of
$D_\QCoh(\mathcal{O}_{\mathbf{P}^n_k})$.
Thus $K$ is perfect by
Derived Categories of Schemes, Proposition
\ref{perfect-proposition-compact-is-perfect}.
Since $D_{perf}(\mathcal{O}_{\mathbf{P}^n_k}) =
D^b_{\textit{Coh}}(\mathcal{O}_{\mathbf{P}^n_k})$
by
Derived Categories of Schemes, Lemma \ref{perfect-lemma-perfect-on-regular}
we conclude (1) holds.
\end{proof}

\begin{lemma}
\label{lemma-finiteness}
Let $X$ be a scheme proper over a field $k$. Let
$K \in D^b_{\textit{Coh}}(\mathcal{O}_X)$ and let $E$ in $D(\mathcal{O}_X)$
be perfect. Then
$\sum_{i \in \mathbf{Z}} \dim_k \Ext^i_X(E, K) < \infty$.
\end{lemma}

\begin{proof}
This follows for example by combining
Derived Categories of Schemes, Lemmas \ref{perfect-lemma-ext-finite} and
\ref{perfect-lemma-ext-from-perfect-into-bounded-QCoh}.
Alternative proof: combine
Derived Categories of Schemes, Lemmas
\ref{perfect-lemma-perfect-on-noetherian} and
\ref{perfect-lemma-direct-image-coherent}.
\end{proof}

\begin{lemma}
\label{lemma-characterize-dbcoh-projective}
\begin{reference}
In the projective case this is \cite[Lemma 7.46]{Rouquier-dimensions}
and implicit in \cite[Theorem A.1]{BvdB}
\end{reference}
Let $X$ be a proper scheme over a field $k$. Let
$K \in \Ob(D_\QCoh(\mathcal{O}_X))$. The following are equivalent
\begin{enumerate}
\item $K \in D^b_{\textit{Coh}}(\mathcal{O}_X)$, and
\item $\sum_{i \in \mathbf{Z}} \dim_k \Ext^i_X(E, K) < \infty$
for all perfect $E$ in $D(\mathcal{O}_X)$.
\end{enumerate}
\end{lemma}

\begin{proof}
The implication (1) $\Rightarrow$ (2) follows from
Lemma \ref{lemma-finiteness}.
The implication (2) $\Rightarrow$ (1) follows from
More on Morphisms, Lemma
\ref{more-morphisms-lemma-characterize-relatively-perfect}
(see Derived Categories of Schemes, Example
\ref{perfect-example-relatively-perfect-field}
for the meaning of a relatively perfect object over a field);
the easier proof in the projective case is in the next paragraph.

\medskip\noindent
Assume (2) and $X$ projective over $k$.
Choose a closed immersion $i : X \to \mathbf{P}^n_k$. It suffices to show
that $Ri_*K$ is in $D^b_{\textit{Coh}}(\mathbf{P}^n_k)$ since a quasi-coherent
module $\mathcal{F}$ on $X$ is coherent, resp.\ zero if and only if
$i_*\mathcal{F}$ is coherent, resp.\ zero. For a perfect object $E$
of $D(\mathcal{O}_{\mathbf{P}^n_k})$, $Li^*E$ is a perfect object of
$D(\mathcal{O}_X)$ and
$$
\Ext^q_{\mathbf{P}^n_k}(E, Ri_*K) = \Ext^q_X(Li^*E, K)
$$
Hence by our assumption we see that
$\sum_{q \in \mathbf{Z}} \dim_k \Ext^q_{\mathbf{P}^n_k}(E, Ri_*K) < \infty$.
We conclude by Lemma \ref{lemma-coherent-on-projective-space}.
\end{proof}





\section{A representability theorem}
\label{section-bondal-van-den-bergh}

\noindent
The material in this section is taken from \cite{BvdB}.

\medskip\noindent
Let $\mathcal{T}$ be a $k$-linear triangulated category.
In this section we consider $k$-linear cohomological functors
$H$ from $\mathcal{T}$ to the category of $k$-vector spaces.
This will mean $H$ is a functor
$$
H : \mathcal{T}^{opp} \longrightarrow \text{Vect}_k
$$
which is $k$-linear such that for any distinguished triangle
$X \to Y \to Z$ in $\mathcal{T}$ the sequence $H(Z) \to H(Y) \to H(X)$
is an exact sequence of $k$-vector spaces. See
Derived Categories, Definition \ref{derived-definition-homological}
and Differential Graded Algebra, Section \ref{dga-section-linear}.

\begin{lemma}
\label{lemma-maps-from-compact-filtered}
Let $\mathcal{D}$ be a triangulated category. Let
$\mathcal{D}' \subset \mathcal{D}$ be a full triangulated subcategory. Let
$X \in \Ob(\mathcal{D})$. The category of arrows $E \to X$ with
$E \in \Ob(\mathcal{D}')$ is filtered.
\end{lemma}

\begin{proof}
We check the conditions of
Categories, Definition \ref{categories-definition-directed}.
The category is nonempty because it contains $0 \to X$.
If $E_i \to X$, $i = 1, 2$ are objects, then $E_1 \oplus E_2 \to X$
is an object and there are morphisms $(E_i \to X) \to (E_1 \oplus E_2 \to X)$.
Finally, suppose that $a, b : (E \to X) \to (E' \to X)$ are morphisms.
Choose a distinguished triangle $E \xrightarrow{a - b} E' \to E''$
in $\mathcal{D}'$. By Axiom TR3 we obtain a morphism of triangles
$$
\xymatrix{
E \ar[r]_{a - b} \ar[d] &
E' \ar[d] \ar[r] & E'' \ar[d] \\
0 \ar[r] &
X \ar[r] &
X
}
$$
and we find that the resulting arrow $(E' \to X) \to (E'' \to X)$
equalizes $a$ and $b$.
\end{proof}

\begin{lemma}
\label{lemma-van-den-bergh}
\begin{reference}
\cite[Lemma 2.14]{CKN}
\end{reference}
Let $k$ be a field. Let $\mathcal{D}$ be a $k$-linear triangulated category
which has direct sums and is compactly generated.
Denote $\mathcal{D}_c$ the full
subcategory of compact objects. Let $H : \mathcal{D}_c^{opp} \to \text{Vect}_k$
be a $k$-linear cohomological functor such that
$\dim_k H(X) < \infty$ for all $X \in \Ob(\mathcal{D}_c)$.
Then $H$ is isomorphic to the functor $X \mapsto \Hom(X, Y)$
for some $Y \in \Ob(\mathcal{D})$.
\end{lemma}

\begin{proof}
We will use Derived Categories, Lemma
\ref{derived-lemma-compact-objects-subcategory} without further mention.
Denote $G : \mathcal{D}_c \to \text{Vect}_k$ the $k$-linear homological
functor which sends $X$ to $H(X)^\vee$. For any object $Y$ of $\mathcal{D}$
we set
$$
G'(Y) = \colim_{X \to Y, X \in \Ob(\mathcal{D}_c)} G(X)
$$
The colimit is filtered by Lemma \ref{lemma-maps-from-compact-filtered}.
We claim that $G'$ is a $k$-linear homological functor,
the restriction of $G'$ to $\mathcal{D}_c$ is $G$, and $G'$
sends direct sums to direct sums.

\medskip\noindent
Namely, suppose that $Y_1 \to Y_2 \to Y_3$ is a distinguished triangle.
Let $\xi \in G'(Y_2)$ map to zero in $G'(Y_3)$. Since the colimit is
filtered $\xi$ is represented by some $X \to Y_2$ with
$X \in \Ob(\mathcal{D}_c)$ and $g \in G(X)$.
The fact that $\xi$ maps to zero in $G'(Y_3)$ means the composition
$X \to Y_2 \to Y_3$ factors as $X \to X' \to Y_3$ with $X' \in \mathcal{D}_c$
and $g$ mapping to zero in $G(X')$. Choose a distinguished
triangle $X'' \to X \to X'$. Then $X'' \in \Ob(\mathcal{D}_c)$.
Since $G$ is homological we find that $g$ is the image of some
$g'' \in G'(X'')$. By Axiom TR3 the maps $X \to Y_2$ and $X' \to Y_3$ fit into
a morphism of distinguished triangles
$(X'' \to X \to X') \to (Y_1 \to Y_2 \to Y_3)$
and we find that indeed $\xi$ is the image of the
element of $G'(Y_1)$ represented by $X'' \to Y_1$ and $g'' \in G(X'')$.

\medskip\noindent
If $Y \in \Ob(\mathcal{D}_c)$, then $\text{id} : Y \to Y$ is the final
object in the category of arrows $X \to Y$ with $X \in \Ob(\mathcal{D}_c)$.
Hence we see that $G'(Y) = G(Y)$ in this case and the
statement on restriction holds. Let $Y = \bigoplus_{i \in I} Y_i$
be a direct sum. Let $a : X \to Y$ with $X \in \Ob(\mathcal{D}_c)$
and $g \in G(X)$ represent an element $\xi$ of $G'(Y)$.
The morphism $a : X \to Y$ can be uniquely written as a sum of morphisms
$a_i : X \to Y_i$ almost all zero as $X$ is a compact object of $\mathcal{D}$.
Let $I' = \{i \in I \mid a_i \not = 0\}$. Then we can factor
$a$ as the composition
$$
X \xrightarrow{(1, \ldots, 1)}
\bigoplus\nolimits_{i \in I'} X
\xrightarrow{\bigoplus_{i \in I'} a_i}
\bigoplus\nolimits_{i \in I} Y_i = Y
$$
We conclude that $\xi = \sum_{i \in I'} \xi_i$
is the sum of the images of the elements
$\xi_i \in G'(Y_i)$ corresponding to $a_i : X \to Y_i$
and $g \in G(X)$. Hence $\bigoplus G'(Y_i) \to G'(Y)$
is surjective. We omit the (trivial) verification that it is injective.

\medskip\noindent
It follows that the functor $Y \mapsto G'(Y)^\vee$ is cohomological
and sends direct sums to direct products. Hence by Brown representability,
see Derived Categories, Proposition \ref{derived-proposition-brown}
we conclude that there exists a $Y \in \Ob(\mathcal{D})$
and an isomorphism
$G'(Z)^\vee = \Hom(Z, Y)$ functorially in $Z$.
For $X \in \Ob(\mathcal{D}_c)$ we have
$G'(X)^\vee = G(X)^\vee = (H(X)^\vee)^\vee = H(X)$
because $\dim_k H(X) < \infty$ and the proof is complete.
\end{proof}

\begin{theorem}
\label{theorem-bondal-van-den-bergh}
\begin{reference}
In the projective case this is \cite[Theorem A.1]{BvdB}
\end{reference}
Let $X$ be a proper scheme over a field $k$.
Let $F : D_{perf}(\mathcal{O}_X)^{opp} \to \text{Vect}_k$
be a $k$-linear cohomological functor such that
$$
\sum\nolimits_{n \in \mathbf{Z}} \dim_k F(E[n]) < \infty
$$
for all $E \in D_{perf}(\mathcal{O}_X)$. Then $F$ is isomorphic to a functor
of the form $E \mapsto \Hom_X(E, K)$ for some
$K \in D^b_{\textit{Coh}}(\mathcal{O}_X)$.
\end{theorem}

\begin{proof}
The derived category $D_\QCoh(\mathcal{O}_X)$ has direct sums,
is compactly generated, and $D_{perf}(\mathcal{O}_X)$ is the full subcategory
of compact objects, see
Derived Categories of Schemes, Lemma
\ref{perfect-lemma-quasi-coherence-direct-sums},
Theorem \ref{perfect-theorem-bondal-van-den-Bergh}, and
Proposition \ref{perfect-proposition-compact-is-perfect}.
By Lemma \ref{lemma-van-den-bergh} we may assume
$F(E) = \Hom_X(E, K)$ for some $K \in \Ob(D_\QCoh(\mathcal{O}_X))$.
Then it follows that $K$ is in $D^b_{\textit{Coh}}(\mathcal{O}_X)$
by Lemma \ref{lemma-characterize-dbcoh-projective}.
\end{proof}

\begin{lemma}
\label{lemma-homological-representable}
Let $X$ be a proper scheme over a field $k$ which is regular.
Let $G : D_{perf}(\mathcal{O}_X) \to \text{Vect}_k$
be a $k$-linear homological functor such that
$$
\sum\nolimits_{n \in \mathbf{Z}} \dim_k G(E[n]) < \infty
$$
for all $E \in D_{perf}(\mathcal{O}_X)$. Then $G$ is isomorphic to a functor
of the form $E \mapsto \Hom_X(K, E)$ for some $K \in D_{perf}(\mathcal{O}_X)$.
\end{lemma}

\begin{proof}
Consider the contravariant functor $E \mapsto E^\vee$
on $D_{perf}(\mathcal{O}_X)$, see
Cohomology, Lemma \ref{cohomology-lemma-dual-perfect-complex}.
This functor is an exact anti-self-equivalence of $D_{perf}(\mathcal{O}_X)$.
Hence we may apply Theorem \ref{theorem-bondal-van-den-bergh}
to the functor $F(E) = G(E^\vee)$ to find
$K \in D_{perf}(\mathcal{O}_X)$ such that $G(E^\vee) = \Hom_X(E, K)$.
It follows that $G(E) = \Hom_X(E^\vee, K) = \Hom_X(K^\vee, E)$
and we conclude that taking $K^\vee$ works.
\end{proof}






\section{Existence of adjoints}
\label{section-adjoints}

\noindent
As a consequence of the results in the paper of Bondal and van den Bergh
we get the following automatic existence of adjoints.

\begin{lemma}
\label{lemma-always-right-adjoints}
Let $k$ be a field. Let $X$ and $Y$ be proper schemes over $k$.
If $X$ is regular, then any $k$-linear exact functor
$F : D_{perf}(\mathcal{O}_X) \to D_{perf}(\mathcal{O}_Y)$
has an exact right adjoint and an exact left adjoint.
\end{lemma}

\begin{proof}
If an adjoint exists it is an exact functor by the very general
Derived Categories, Lemma \ref{derived-lemma-adjoint-is-exact}.

\medskip\noindent
Let us prove the existence of a right adjoint.
To see existence, it suffices to show that for
$M \in D_{perf}(\mathcal{O}_Y)$ the contravariant functor
$K \mapsto \Hom_Y(F(K), M)$ is representable.
This functor is contravariant, $k$-linear, and cohomological.
Hence by Theorem \ref{theorem-bondal-van-den-bergh}
it suffices to show that
$$
\sum\nolimits_{i \in \mathbf{Z}} \dim_k \Ext^i_Y(F(K), M) < \infty
$$
This follows from Lemma \ref{lemma-finiteness}.

\medskip\noindent
For the existence of the left adjoint we argue in the same
manner using Lemma \ref{lemma-homological-representable}
in stead of Theorem \ref{theorem-bondal-van-den-bergh}.
\end{proof}






\section{Fourier-Mukai functors}
\label{section-fourier-mukai}

\noindent
These functors were first introduced in \cite{Mukai}.

\begin{definition}
\label{definition-fourier-mukai-functor}
Let $S$ be a scheme. Let $X$ and $Y$ be schemes over $S$.
Let $K \in D(\mathcal{O}_{X \times_S Y})$. The exact functor
$$
\Phi_K : D(\mathcal{O}_X) \longrightarrow D(\mathcal{O}_Y),\quad
M \longmapsto R\text{pr}_{2, *}(
L\text{pr}_1^*M \otimes_{\mathcal{O}_{X \times_S Y}}^\mathbf{L} K)
$$
of triangulated categories is called a {\it Fourier-Mukai functor}
and $K$ is called a {\it Fourier-Mukai kernel} for this functor.
Moreover,
\begin{enumerate}
\item if $\Phi_K$ sends $D_\QCoh(\mathcal{O}_X)$ into $D_\QCoh(\mathcal{O}_Y)$
then the resulting exact functor
$\Phi_K : D_\QCoh(\mathcal{O}_X) \to D_\QCoh(\mathcal{O}_Y)$
is called a Fourier-Mukai functor,
\item if $\Phi_K$ sends $D_{perf}(\mathcal{O}_X)$ into
$D_{perf}(\mathcal{O}_Y)$ then the resulting exact functor
$\Phi_K : D_{perf}(\mathcal{O}_X) \to D_{perf}(\mathcal{O}_Y)$
is called a Fourier-Mukai functor, and
\item if $X$ and $Y$ are Noetherian and $\Phi_K$ sends
$D^b_{\textit{Coh}}(\mathcal{O}_X)$ into $D^b_{\textit{Coh}}(\mathcal{O}_Y)$
then the resulting exact functor
$\Phi_K : D^b_{\textit{Coh}}(\mathcal{O}_X) \to
D^b_{\textit{Coh}}(\mathcal{O}_Y)$
is called a Fourier-Mukai functor.
Similarly for $D_{\textit{Coh}}$, $D^+_{\textit{Coh}}$, $D^-_{\textit{Coh}}$.
\end{enumerate}
\end{definition}

\begin{lemma}
\label{lemma-fourier-Mukai-QCoh}
Let $S$ be a scheme. Let $X$ and $Y$ be schemes over $S$.
Let $K \in D(\mathcal{O}_{X \times_S Y})$.
The corresponding Fourier-Mukai functor $\Phi_K$ sends
$D_\QCoh(\mathcal{O}_X)$ into $D_\QCoh(\mathcal{O}_Y)$
if $K$ is in $D_\QCoh(\mathcal{O}_{X \times_S Y})$ and $X \to S$ is
quasi-compact and quasi-separated.
\end{lemma}

\begin{proof}
This follows from the fact that derived pullback preserves
$D_\QCoh$
(Derived Categories of Schemes, Lemma
\ref{perfect-lemma-quasi-coherence-pullback}),
derived tensor products preserve $D_\QCoh$
(Derived Categories of Schemes, Lemma
\ref{perfect-lemma-quasi-coherence-tensor-product}),
the projection $\text{pr}_2 : X \times_S Y \to Y$ is
quasi-compact and quasi-separated
(Schemes, Lemmas
\ref{schemes-lemma-quasi-compact-preserved-base-change} and
\ref{schemes-lemma-separated-permanence}), and
total direct image along a quasi-separated and quasi-compact
morphism preserves $D_\QCoh$
(Derived Categories of Schemes, Lemma
\ref{perfect-lemma-quasi-coherence-direct-image}).
\end{proof}

\begin{lemma}
\label{lemma-compose-fourier-mukai}
Let $S$ be a scheme. Let $X, Y, Z$ be schemes over $S$. Assume
$X \to S$, $Y \to S$, and $Z \to S$ are quasi-compact and quasi-separated.
Let $K \in D_\QCoh(\mathcal{O}_{X \times_S Y})$.
Let $K' \in D_\QCoh(\mathcal{O}_{Y \times_S Z})$.
Consider the Fourier-Mukai functors
$\Phi_K : D_\QCoh(\mathcal{O}_X) \to D_\QCoh(\mathcal{O}_Y)$
and $\Phi_{K'} : D_\QCoh(\mathcal{O}_Y) \to D_\QCoh(\mathcal{O}_Z)$.
If $X$ and $Z$ are tor independent over $S$ and $Y \to S$ is flat,
then
$$
\Phi_{K'} \circ \Phi_K = \Phi_{K''} :
D_\QCoh(\mathcal{O}_X)
\longrightarrow
D_\QCoh(\mathcal{O}_Z)
$$
where
$$
K'' = R\text{pr}_{13, *}(
L\text{pr}_{12}^*K
\otimes_{\mathcal{O}_{X \times_S Y \times_S Z}}^\mathbf{L}
L\text{pr}_{23}^*K')
$$
in $D_\QCoh(\mathcal{O}_{X \times_S Z})$.
\end{lemma}

\begin{proof}
The statement makes sense by Lemma \ref{lemma-fourier-Mukai-QCoh}.
We are going to use
Derived Categories of Schemes, Lemmas
\ref{perfect-lemma-quasi-coherence-pullback},
\ref{perfect-lemma-quasi-coherence-tensor-product}, and
\ref{perfect-lemma-quasi-coherence-direct-image}
and Schemes, Lemmas
\ref{schemes-lemma-quasi-compact-preserved-base-change} and
\ref{schemes-lemma-separated-permanence}
without further mention.
By Derived Categories of Schemes, Lemma
\ref{perfect-lemma-flat-base-change-tor-independent}
we see that $X \times_S Y$ and $Y \times_S Z$ are tor independent
over $Y$. This means that we have base change for the cartesian diagram
$$
\xymatrix{
X \times_S Y \times_S Z \ar[d] \ar[r] &
Y \times_S Z \ar[d]^{p^{YZ}_Y} \\
X \times_S Y \ar[r]^{p^{XY}_Y} & Y
}
$$
for complexes with quasi-coherent cohomology sheaves, see
Derived Categories of Schemes, Lemma \ref{perfect-lemma-compare-base-change}.
Abbreviating $p^* = Lp^*$, $p_* = Rp_*$ and $\otimes = \otimes^\mathbf{L}$
we have for $M \in D_\QCoh(\mathcal{O}_X)$ the sequence of equalities
\begin{align*}
\Phi_{K'}(\Phi_K(M))
& =
p^{YZ}_{Z, *}(p^{YZ, *}_Y p^{XY}_{Y, *}(p^{XY, *}_X M \otimes K) \otimes K') \\
& =
p^{YZ}_{Z, *}(\text{pr}_{23, *} \text{pr}_{12}^*(p^{XY, *}_X M \otimes K)
\otimes K') \\
& =
p^{YZ}_{Z, *}(\text{pr}_{23, *}(\text{pr}_1^*M \otimes \text{pr}_{12}^*K)
\otimes K') \\
& =
p^{YZ}_{Z, *}(\text{pr}_{23, *}(\text{pr}_1^*M \otimes \text{pr}_{12}^*K
\otimes \text{pr}_{23}^*K')) \\
& =
\text{pr}_{3, *}(\text{pr}_1^*M \otimes \text{pr}_{12}^*K
\otimes \text{pr}_{23}^*K') \\
& =
p^{XZ}_{Z, *}\text{pr}_{13, *}(\text{pr}_1^*M \otimes \text{pr}_{12}^*K
\otimes \text{pr}_{23}^*K') \\
& =
p^{XZ}_{Z, *} (p^{XZ, *}_X M \otimes \text{pr}_{13, *}(\text{pr}_{12}^*K
\otimes \text{pr}_{23}^*K'))
\end{align*}
as desired. Here we have used the remark on base change in the
second equality and we have use Derived Categories of Schemes, Lemma
\ref{perfect-lemma-cohomology-base-change} in the $4$th and
last equality.
\end{proof}

\begin{lemma}
\label{lemma-fourier-mukai}
Let $S$ be a scheme. Let $X$ and $Y$ be schemes over $S$.
Let $K \in D(\mathcal{O}_{X \times_S Y})$.
The corresponding Fourier-Mukai functor $\Phi_K$ sends
$D_{perf}(\mathcal{O}_X)$ into $D_{perf}(\mathcal{O}_Y)$ if at least
one of the following conditions is satisfied:
\begin{enumerate}
\item $S$ is Noetherian, $X \to S$ and $Y \to S$ are of finite type,
$K \in D^b_{\textit{Coh}}(\mathcal{O}_{X \times_S Y})$, the support of $H^i(K)$
is proper over $Y$ for all $i$, and $K$ has finite tor dimension
as an object of $D(\text{pr}_2^{-1}\mathcal{O}_Y)$,
\item $X \to S$ is of finite presentation and $K$ can be represented
by a bounded complex $\mathcal{K}^\bullet$ of finitely presented
$\mathcal{O}_{X \times_S Y}$-modules, flat over $Y$, with support
proper over $Y$,
\item $X \to S$ is a proper flat morphism of finite presentation
and $K$ is perfect,
\item $S$ is Noetherian, $X \to S$ is flat and proper, and $K$ is perfect
\item $X \to S$ is a proper flat morphism of finite presentation
and $K$ is $Y$-perfect,
\item $S$ is Noetherian, $X \to S$ is flat and proper, and $K$ is
$Y$-perfect.
\end{enumerate}
\end{lemma}

\begin{proof}
If $M$ is perfect on $X$, then $L\text{pr}_1^*M$
is perfect on $X \times_S Y$, see
Cohomology, Lemma \ref{cohomology-lemma-perfect-pullback}.
We will use this without further mention below.
We will also use that if $X \to S$ is of finite type, or proper, or
flat, or of finite presentation, then the same thing is true for
the base change $\text{pr}_2 : X \times_S Y \to Y$, see
Morphisms, Lemmas
\ref{morphisms-lemma-base-change-finite-type},
\ref{morphisms-lemma-base-change-proper},
\ref{morphisms-lemma-base-change-flat}, and
\ref{morphisms-lemma-base-change-finite-presentation}.

\medskip\noindent
Part (1) follows from
Derived Categories of Schemes, Lemma \ref{perfect-lemma-perfect-direct-image}
combined with
Derived Categories of Schemes, Lemma \ref{perfect-lemma-perfect-on-noetherian}.

\medskip\noindent
Part (2) follows from
Derived Categories of Schemes, Lemma
\ref{perfect-lemma-base-change-tensor-perfect}.

\medskip\noindent
Part (3) follows from
Derived Categories of Schemes, Lemma
\ref{perfect-lemma-flat-proper-perfect-direct-image-general}.

\medskip\noindent
Part (4) follows from part (3) and the fact that a finite type
morphism of Noetherian schemes is of finite presentation by Morphisms, Lemma
\ref{morphisms-lemma-noetherian-finite-type-finite-presentation}.

\medskip\noindent
Part (5) follows from
Derived Categories of Schemes, Lemma
\ref{perfect-lemma-derived-pushforward-rel-perfect} combined with
Derived Categories of Schemes, Lemma
\ref{perfect-lemma-perfect-relatively-perfect}.

\medskip\noindent
Part (6) follows from part (5) in the same way that part (4) follows from
part (3).
\end{proof}

\begin{lemma}
\label{lemma-fourier-mukai-Coh}
Let $S$ be a Noetherian scheme. Let $X$ and $Y$ be schemes of finite type
over $S$. Let $K \in D^b_{\textit{Coh}}(\mathcal{O}_{X \times_S Y})$.
The corresponding Fourier-Mukai functor $\Phi_K$ sends
$D^b_{\textit{Coh}}(\mathcal{O}_X)$ into $D^b_{\textit{Coh}}(\mathcal{O}_Y)$
if at least one of the following conditions is satisfied:
\begin{enumerate}
\item the support of $H^i(K)$ is proper over $Y$ for all $i$, and $K$
has finite tor dimension as an object of $D(\text{pr}_1^{-1}\mathcal{O}_X)$,
\item $K$ can be represented by a bounded complex $\mathcal{K}^\bullet$
of coherent $\mathcal{O}_{X \times_S Y}$-modules, flat over $X$, with support
proper over $Y$,
\item the support of $H^i(K)$ is proper over $Y$ for all $i$
and $X$ is a regular scheme,
\item $K$ is perfect, the support of $H^i(K)$ is proper over $Y$ for all $i$,
and $Y \to S$ is flat.
\end{enumerate}
Furthermore in each case the support condition is automatic
if $X \to S$ is proper.
\end{lemma}

\begin{proof}
Let $M$ be an object of $D^b_{\textit{Coh}}(\mathcal{O}_X)$.
In each case we will use Derived Categories of Schemes, Lemma
\ref{perfect-lemma-direct-image-coherent} to show that
$$
\Phi_K(M) = R\text{pr}_{2, *}(
L\text{pr}_1^*M
\otimes_{\mathcal{O}_{X \times_S Y}}^\mathbf{L}
K)
$$
is in $D^b_{\textit{Coh}}(\mathcal{O}_Y)$. The derived tensor product
$L\text{pr}_1^*M \otimes_{\mathcal{O}_{X \times_S Y}}^\mathbf{L} K$
is a pseudo-coherent object of $D(\mathcal{O}_{X \times_S Y})$
(by
Cohomology, Lemma \ref{cohomology-lemma-pseudo-coherent-pullback},
Derived Categories of Schemes, Lemma
\ref{perfect-lemma-identify-pseudo-coherent-noetherian}, and
Cohomology, Lemma \ref{cohomology-lemma-tensor-pseudo-coherent})
whence has coherent cohomology sheaves (by
Derived Categories of Schemes, Lemma
\ref{perfect-lemma-identify-pseudo-coherent-noetherian} again).
In each case the supports of the cohomology sheaves
$H^i(L\text{pr}_1^*M \otimes_{\mathcal{O}_{X \times_S Y}}^\mathbf{L} K)$
is proper over $Y$ as these supports are contained in the
union of the supports of the $H^i(K)$. Hence in each case
it suffices to prove that this tensor product is bounded below.

\medskip\noindent
Case (1). By Cohomology, Lemma \ref{cohomology-lemma-variant-derived-pullback}
we have
$$
L\text{pr}_1^*M
\otimes_{\mathcal{O}_{X \times_S Y}}^\mathbf{L}
K
\cong
\text{pr}_1^{-1}M
\otimes_{\text{pr}_1^{-1}\mathcal{O}_X}^\mathbf{L}
K
$$
with obvious notation. Hence the assumption on tor dimension
and the fact that $M$ has only a finite number of nonzero
cohomology sheaves, implies the bound we want.

\medskip\noindent
Case (2) follows because here the assumption implies that $K$ has
finite tor dimension as an object of $D(\text{pr}_1^{-1}\mathcal{O}_X)$
hence the argument in the previous paragraph applies.

\medskip\noindent
In Case (3) it is also the case that $K$ has finite tor dimension
as an object of $D(\text{pr}_1^{-1}\mathcal{O}_X)$. Namely, choose
affine opens $U = \Spec(A)$ and $V = \Spec(B)$ of $X$ and $Y$ mapping into the
affine open $W = \Spec(R)$ of $S$. Then
$K|_{U \times V}$ is given by a bounded complex of finite
$A \otimes_R B$-modules $M^\bullet$. Since $A$ is a regular ring
of finite dimension we see that each $M^i$ has finite projective dimension
as an $A$-module (Algebra, Lemma
\ref{algebra-lemma-finite-gl-dim-finite-dim-regular})
and hence finite tor dimension as an $A$-module.
Thus $M^\bullet$ has finite tor dimension as a complex of $A$-modules
(More on Algebra, Lemma
\ref{more-algebra-lemma-complex-finite-tor-dimension-modules}).
Since $X \times Y$ is quasi-compact we conclude there exist $[a, b]$
such that for every point $z \in X \times Y$ the stalk $K_z$
has tor amplitude in $[a, b]$ over $\mathcal{O}_{X, \text{pr}_1(z)}$.
This implies $K$ has bounded tor dimension as an object of
$D(\text{pr}_1^{-1}\mathcal{O}_X)$, see
Cohomology, Lemma \ref{cohomology-lemma-tor-amplitude-stalk}.
We conclude as in the previous to paragraphs.

\medskip\noindent
Case (4). With notation as above, the ring map $R \to B$ is flat.
Hence the ring map $A \to A \otimes_R B$ is flat. Hence any projective
$A \otimes_R B$-module is $A$-flat. Thus any perfect complex of
$A \otimes_R B$-modules has finite tor dimension as a complex
of $A$-modules and we conclude as before.
\end{proof}

\begin{example}
\label{example-diagonal-fourier-mukai}
Let $X \to S$ be a separated morphism of schemes. Then the diagonal
$\Delta : X \to X \times_S X$ is a closed immersion and hence
$\mathcal{O}_\Delta = \Delta_*\mathcal{O}_X = R\Delta_*\mathcal{O}_X$
is a quasi-coherent $\mathcal{O}_{X \times_S X}$-module of finite type
which is flat over $X$ (under either projection). The Fourier-Mukai functor
$\Phi_{\mathcal{O}_\Delta}$ is equal to the identity in this case.
Namely, for any $M \in D(\mathcal{O}_X)$ we have
\begin{align*}
L\text{pr}_1^*M \otimes_{\mathcal{O}_{X \times_S X}}^\mathbf{L}
\mathcal{O}_\Delta
& =
L\text{pr}_1^*M \otimes_{\mathcal{O}_{X \times_S X}}^\mathbf{L}
R\Delta_*\mathcal{O}_X \\
& =
R\Delta_*(
L\Delta^*L\text{pr}_1^*M \otimes_{\mathcal{O}_X}^\mathbf{L} \mathcal{O}_X) \\
& =
R\Delta_*(M)
\end{align*}
The first equality we discussed above.
The second equality is Cohomology, Lemma
\ref{cohomology-lemma-projection-formula-closed-immersion}.
The third because $\text{pr}_1 \circ \Delta = \text{id}_X$ and we have
Cohomology, Lemma \ref{cohomology-lemma-derived-pullback-composition}.
If we push this to $X$ using $R\text{pr}_{2, *}$
we obtain $M$ by
Cohomology, Lemma \ref{cohomology-lemma-derived-pushforward-composition}
and the fact that $\text{pr}_2 \circ \Delta = \text{id}_X$.
\end{example}

\begin{lemma}
\label{lemma-fourier-mukai-right-adjoint}
\begin{reference}
Compare with discussion in \cite{Rizzardo}.
\end{reference}
Let $X \to S$ and $Y \to S$ be morphisms of quasi-compact and quasi-separated
schemes. Let $\Phi : D_\QCoh(\mathcal{O}_X) \to D_\QCoh(\mathcal{O}_Y)$
be a Fourier-Mukai functor with pseudo-coherent kernel
$K \in D_\QCoh(\mathcal{O}_{X \times_S Y})$.
Let $a : D_\QCoh(\mathcal{O}_Y) \to  D_\QCoh(\mathcal{O}_{X \times_S Y})$
be the right adjoint to $R\text{pr}_{2, *}$, see
Duality for Schemes, Lemma \ref{duality-lemma-twisted-inverse-image}.
Denote
$$
K' = (Y \times_S X \to X \times_S Y)^*
R\SheafHom_{\mathcal{O}_{X \times_S Y}}(K, a(\mathcal{O}_Y)) \in
D_\QCoh(\mathcal{O}_{Y \times_S X})
$$
and denote $\Phi' : D_\QCoh(\mathcal{O}_Y) \to D_\QCoh(\mathcal{O}_X)$
the corresponding Fourier-Mukai transform. There is a canonical map
$$
\Hom_X(M, \Phi'(N)) \longrightarrow \Hom_Y(\Phi(M), N)
$$
functorial in $M$ in $D_\QCoh(\mathcal{O}_X)$ and $N$ in
$D_\QCoh(\mathcal{O}_Y)$ which is an isomorphism if
\begin{enumerate}
\item $N$ is perfect, or
\item $K$ is perfect and $X \to S$ is proper flat and of finite presentation.
\end{enumerate}
\end{lemma}

\begin{proof}
By Lemma \ref{lemma-fourier-Mukai-QCoh} we obtain a functor $\Phi$
as in the statement. Observe that $a(\mathcal{O}_Y)$ is in
$D^+_\QCoh(\mathcal{O}_{X \times_S Y})$ by Duality for Schemes,
Lemma \ref{duality-lemma-twisted-inverse-image-bounded-below}.
Hence for $K$ pseudo-coherent we have
$K' \in D_\QCoh(\mathcal{O}_{Y \times_S X})$
by Derived Categories of Schemes, Lemma
\ref{perfect-lemma-quasi-coherence-internal-hom}
we we obtain $\Phi'$ as indicated.

\medskip\noindent
We abbreviate
$\otimes^\mathbf{L} = \otimes_{\mathcal{O}_{X \times_S Y}}^\mathbf{L}$
and
$\SheafHom = R\SheafHom_{\mathcal{O}_{X \times_S Y}}$.
Let $M$ be in $D_\QCoh(\mathcal{O}_X)$ and let
$N$ be in $D_\QCoh(\mathcal{O}_Y)$. We have
\begin{align*}
\Hom_Y(\Phi(M), N)
& =
\Hom_Y(R\text{pr}_{2, *}(L\text{pr}_1^*M \otimes^\mathbf{L} K), N) \\
& =
\Hom_{X \times_S Y}(L\text{pr}_1^*M \otimes^\mathbf{L} K, a(N)) \\
& =
\Hom_{X \times_S Y}(L\text{pr}_1^*M,
R\SheafHom(K, a(N))) \\
& =
\Hom_X(M, R\text{pr}_{1, *}R\SheafHom(K, a(N)))
\end{align*}
where we have used Cohomology, Lemmas \ref{cohomology-lemma-internal-hom}
and \ref{cohomology-lemma-adjoint}. There are canonical maps
$$
L\text{pr}_2^*N \otimes^\mathbf{L} R\SheafHom(K, a(\mathcal{O}_Y))
\xrightarrow{\alpha}
R\SheafHom(K, L\text{pr}_2^*N \otimes^\mathbf{L} a(\mathcal{O}_Y))
\xrightarrow{\beta}
R\SheafHom(K, a(N))
$$
Here $\alpha$ is
Cohomology, Lemma \ref{cohomology-lemma-internal-hom-diagonal-better}
and $\beta$ is Duality for Schemes, Equation
(\ref{duality-equation-compare-with-pullback}).
Combining all of these arrows we obtain the functorial displayed
arrow in the statement of the lemma.

\medskip\noindent
The arrow $\alpha$ is an isomorphism by
Derived Categories of Schemes, Lemma
\ref{perfect-lemma-internal-hom-evaluate-tensor-isomorphism}
as soon as either $K$ or $N$ is perfect.
The arrow $\beta$ is an isomorphism if $N$ is perfect by
Duality for Schemes, Lemma \ref{duality-lemma-compare-with-pullback-perfect}
or in general if $X \to S$ is
flat proper of finite presentation by
Duality for Schemes, Lemma
\ref{duality-lemma-compare-with-pullback-flat-proper}.
\end{proof}

\begin{lemma}
\label{lemma-fourier-mukai-left-adjoint}
\begin{reference}
Compare with discussion in \cite{Rizzardo}.
\end{reference}
Let $S$ be a Noetherian scheme. Let $Y \to S$ be a flat proper
Gorenstein morphism and let $X \to S$ be a finite type morphism.
Denote $\omega^\bullet_{Y/S}$ the relative dualizing complex of
$Y$ over $S$. Let $\Phi : D_\QCoh(\mathcal{O}_X) \to D_\QCoh(\mathcal{O}_Y)$
be a Fourier-Mukai functor with perfect kernel
$K \in D_\QCoh(\mathcal{O}_{X \times_S Y})$. Denote
$$
K' = (Y \times_S X \to X \times_S Y)^*(K^\vee
\otimes_{\mathcal{O}_{X \times_S Y}}^\mathbf{L}
L\text{pr}_2^*\omega^\bullet_{Y/S})
\in
D_\QCoh(\mathcal{O}_{Y \times_S X})
$$
and denote $\Phi' : D_\QCoh(\mathcal{O}_Y) \to D_\QCoh(\mathcal{O}_X)$
the corresponding Fourier-Mukai transform. There is a canonical
isomorphism
$$
\Hom_Y(N, \Phi(M)) \longrightarrow \Hom_X(\Phi'(N), M)
$$
functorial in $M$ in $D_\QCoh(\mathcal{O}_X)$ and $N$ in
$D_\QCoh(\mathcal{O}_Y)$.
\end{lemma}

\begin{proof}
By Lemma \ref{lemma-fourier-Mukai-QCoh} we obtain a functor $\Phi$
as in the statement.

\medskip\noindent
Observe that formation of the relative dualizing complex commutes
with base change in our setting, see Duality for Schemes,
Remark \ref{duality-remark-relative-dualizing-complex}.
Thus $L\text{pr}_2^*\omega^\bullet_{Y/S} = \omega^\bullet_{X \times_S Y/X}$.
Moreover, we observe that $\omega^\bullet_{Y/S}$ is an
invertible object of the derived category, see Duality for Schemes, Lemma
\ref{duality-lemma-affine-flat-Noetherian-gorenstein}, and a fortiori
perfect.

\medskip\noindent
To actually prove the lemma we're going to cheat. Namely, we will
show that if we replace the roles of $X$ and $Y$ and $K$ and $K'$
then these are as in Lemma \ref{lemma-fourier-mukai-right-adjoint}
and we get the result. It is clear that $K'$ is perfect as a
tensor product of perfect objects so that the discussion in
Lemma \ref{lemma-fourier-mukai-right-adjoint} applies to it.
To show that the procedure of
Lemma \ref{lemma-fourier-mukai-right-adjoint} applied to $K'$
on $Y \times_S X$ produces a complex isomorphic to $K$ it suffices
(details omitted) to show that
$$
R\SheafHom(R\SheafHom(K, \omega^\bullet_{X \times_S Y/X}),
\omega^\bullet_{X \times_S Y/X}) = K
$$
This is clear because $K$ is perfect and $\omega^\bullet_{X \times_S Y/X}$
is invertible; details omitted. Thus
Lemma \ref{lemma-fourier-mukai-right-adjoint} produces a map
$$
\Hom_Y(N, \Phi(M)) \longrightarrow \Hom_X(\Phi'(N), M)
$$
functorial in $M$ in $D_\QCoh(\mathcal{O}_X)$ and $N$ in
$D_\QCoh(\mathcal{O}_Y)$ which is an isomorphism because
$K'$ is perfect. This finishes the proof.
\end{proof}

\begin{lemma}
\label{lemma-fourier-mukai-flat-proper-over-noetherian}
Let $S$ be a Noetherian scheme.
\begin{enumerate}
\item For $X$, $Y$ proper and flat over $S$ and $K$ in
$D_{perf}(\mathcal{O}_{X \times_S Y})$ we obtain a Fourier-Mukai functor
$\Phi_K : D_{perf}(\mathcal{O}_X) \to D_{perf}(\mathcal{O}_Y)$.
\item For $X$, $Y$, $Z$ proper and flat over $S$, $K \in
D_{perf}(\mathcal{O}_{X \times_S Y})$, $K' \in
D_{perf}(\mathcal{O}_{Y \times_S Z})$ the composition
$\Phi_{K'} \circ \Phi_K : D_{perf}(\mathcal{O}_X) \to D_{perf}(\mathcal{O}_Z)$
is equal to $\Phi_{K''}$ with $K'' \in D_{perf}(\mathcal{O}_{X \times_S Z})$
computed as in Lemma \ref{lemma-compose-fourier-mukai},
\item For $X$, $Y$, $K$, $\Phi_K$ as in (1) if $X \to S$ is Gorenstein, then
$\Phi_{K'} : D_{perf}(\mathcal{O}_Y) \to D_{perf}(\mathcal{O}_X)$ is a right
adjoint to $\Phi_K$ where $K' \in D_{perf}(\mathcal{O}_{Y \times_S X})$
is the pullback of $L\text{pr}_1^*\omega_{X/S}^\bullet
\otimes_{\mathcal{O}_{X \times_S Y}}^\mathbf{L} K^\vee$ by
$Y \times_S X \to X \times_S Y$.
\item For $X$, $Y$, $K$, $\Phi_K$ as in (1) if $Y \to S$ is Gorenstein, then
$\Phi_{K''} : D_{perf}(\mathcal{O}_Y) \to D_{perf}(\mathcal{O}_X)$ is a left
adjoint to $\Phi_K$ where $K'' \in D_{perf}(\mathcal{O}_{Y \times_S X})$
is the pullback of $L\text{pr}_2^*\omega_{Y/S}^\bullet
\otimes_{\mathcal{O}_{X \times_S Y}}^\mathbf{L} K^\vee$ by
$Y \times_S X \to X \times_S Y$.
\end{enumerate}
\end{lemma}

\begin{proof}
Part (1) is immediate from Lemma \ref{lemma-fourier-mukai} part (4).

\medskip\noindent
Part (2) follows from Lemma \ref{lemma-compose-fourier-mukai} and the
fact that
$K'' = R\text{pr}_{13, *}(
L\text{pr}_{12}^*K
\otimes_{\mathcal{O}_{X \times_S Y \times_S Z}}^\mathbf{L}
L\text{pr}_{23}^*K')$ is perfect for example by
Derived Categories of Schemes, Lemma
\ref{perfect-lemma-flat-proper-perfect-direct-image}.

\medskip\noindent
The adjointness in part (3) on all complexes with quasi-coherent cohomology
sheaves follows from Lemma \ref{lemma-fourier-mukai-right-adjoint} with
$K'$ equal to the pullback of
$R\SheafHom_{\mathcal{O}_{X \times_S Y}}(K, a(\mathcal{O}_Y))$
by $Y \times_S X \to X \times_S Y$ where $a$ is the right adjoint
to $R\text{pr}_{2, *} : D_\QCoh(\mathcal{O}_{X \times_S Y}) \to
D_\QCoh(\mathcal{O}_Y)$. Denote $f : X \to S$ the structure morphism of $X$.
Since $f$ is proper the functor
$f^! : D_\QCoh^+(\mathcal{O}_S) \to D_\QCoh^+(\mathcal{O}_X)$
is the restriction to $D_\QCoh^+(\mathcal{O}_S)$
of the right adjoint to
$Rf_* : D_\QCoh(\mathcal{O}_X) \to D_\QCoh(\mathcal{O}_S)$, see
Duality for Schemes, Section \ref{duality-section-upper-shriek}.
Hence the relative dualizing complex $\omega_{X/S}^\bullet$ as defined in
Duality for Schemes, Remark
\ref{duality-remark-relative-dualizing-complex}
is equal to $\omega_{X/S}^\bullet = f^!\mathcal{O}_S$.
Since formation of the relative dualizing complex
commutes with base change (see Duality for Schemes, Remark
\ref{duality-remark-relative-dualizing-complex}) we see that
$a(\mathcal{O}_Y) = L\text{pr}_1^*\omega_{X/S}^\bullet$.
Thus
$$
R\SheafHom_{\mathcal{O}_{X \times_S Y}}(K, a(\mathcal{O}_Y))
\cong
L\text{pr}_1^*\omega_{X/S}^\bullet
\otimes_{\mathcal{O}_{X \times_S Y}}^\mathbf{L} K^\vee
$$
by Cohomology, Lemma \ref{cohomology-lemma-dual-perfect-complex}.
Finally, since $X \to S$ is assumed Gorenstein the relative dualizing complex
is invertible: this follows from Duality for Schemes, Lemma
\ref{duality-lemma-affine-flat-Noetherian-gorenstein}.
We conclude that $\omega_{X/S}^\bullet$ is perfect
(Cohomology, Lemma \ref{cohomology-lemma-invertible-derived})
and hence $K'$ is perfect.
Therefore $\Phi_{K'}$ does indeed map $D_{perf}(\mathcal{O}_Y)$
into $D_{perf}(\mathcal{O}_X)$ which finishes the proof of (3).

\medskip\noindent
The proof of (4) is the same as the proof of (3) except one uses
Lemma \ref{lemma-fourier-mukai-left-adjoint} instead of
Lemma \ref{lemma-fourier-mukai-right-adjoint}.
\end{proof}














\section{Resolutions and bounds}
\label{section-tricks-smooth}

\noindent
The diagonal of a smooth proper scheme has a nice resolution.

\begin{lemma}
\label{lemma-on-product}
Let $R$ be a Noetherian ring. Let $X$, $Y$ be finite type schemes over $R$
having the resolution property. For any coherent
$\mathcal{O}_{X \times_R Y}$-module $\mathcal{F}$ there exist
a surjection $\mathcal{E} \boxtimes \mathcal{G} \to \mathcal{F}$
where $\mathcal{E}$ is a finite locally free $\mathcal{O}_X$-module
and $\mathcal{G}$ is a finite locally free $\mathcal{O}_Y$-module.
\end{lemma}

\begin{proof}
Let $U \subset X$ and $V \subset Y$ be affine open subschemes. Let
$\mathcal{I} \subset \mathcal{O}_X$ be the ideal sheaf of the
reduced induced closed subscheme structure on $X \setminus U$.
Similarly, let $\mathcal{I}' \subset \mathcal{O}_Y$ be the ideal sheaf of the
reduced induced closed subscheme structure on $Y \setminus V$.
Then the ideal sheaf
$$
\mathcal{J} = \Im(\text{pr}_1^*\mathcal{I} \otimes_{\mathcal{O}_{X \times_R Y}}
\text{pr}_2^*\mathcal{I}' \to \mathcal{O}_{X \times_R Y})
$$
satisfies $V(\mathcal{J}) = X \times_R Y \setminus U \times_R V$.
For any section $s \in \mathcal{F}(U \times_R V)$ we can find an integer
$n > 0$ and a map $\mathcal{J}^n \to \mathcal{F}$ whose restriction to
$U \times_R V$ gives $s$, see
Cohomology of Schemes, Lemma \ref{coherent-lemma-homs-over-open}.
By assumption we can choose surjections
$\mathcal{E} \to \mathcal{I}$ and $\mathcal{G} \to \mathcal{I}'$.
These produce corresponding surjections
$$
\mathcal{E} \boxtimes \mathcal{G} \to \mathcal{J}
\quad\text{and}\quad
\mathcal{E}^{\otimes n} \boxtimes \mathcal{G}^{\otimes n} \to \mathcal{J}^n
$$
and hence a map
$\mathcal{E}^{\otimes n} \boxtimes \mathcal{G}^{\otimes n} \to \mathcal{F}$
whose image contains the section $s$ over $U \times_R V$.
Since we can cover $X \times_R Y$ by a finite number of affine opens
of the form $U \times_R V$ and since $\mathcal{F}|_{U \times_R V}$
is generated by finitely many sections (Properties, Lemma
\ref{properties-lemma-finite-type-module})
we conclude that there exists a surjection
$$
\bigoplus\nolimits_{j = 1, \ldots, N}
\mathcal{E}_j^{\otimes n_j} \boxtimes \mathcal{G}_j^{\otimes n_j}
\to \mathcal{F}
$$
where $\mathcal{E}_j$ is finite locally free on $X$ and
$\mathcal{G}_j$ is finite locally free on $Y$.
Setting $\mathcal{E} = \bigoplus \mathcal{E}_j^{\otimes n_j}$
and $\mathcal{G} = \bigoplus \mathcal{G}_j^{\otimes n_j}$
we conclude that the lemma is true.
\end{proof}

\begin{lemma}
\label{lemma-on-product-general}
Let $R$ be a ring. Let $X$, $Y$ be quasi-compact and quasi-separated
schemes over $R$ having the resolution property. For any finite
type quasi-coherent $\mathcal{O}_{X \times_R Y}$-module $\mathcal{F}$
there exist a surjection $\mathcal{E} \boxtimes \mathcal{G} \to \mathcal{F}$
where $\mathcal{E}$ is a finite locally free $\mathcal{O}_X$-module
and $\mathcal{G}$ is a finite locally free $\mathcal{O}_Y$-module.
\end{lemma}

\begin{proof}
Follows from Lemma \ref{lemma-on-product} by a limit argument.
We urge the reader to skip the proof.
Since $X \times_R Y$ is a closed subscheme of $X \times_\mathbf{Z} Y$
it is harmless if we replace $R$ by $\mathbf{Z}$.
We can write $\mathcal{F}$ as the quotient of
a finitely presented $\mathcal{O}_{X \times_R Y}$-module by
Properties, Lemma
\ref{properties-lemma-finite-directed-colimit-surjective-maps}.
Hence we may assume $\mathcal{F}$ is of
finite presentation. Next we can write $X = \lim X_i$
with $X_i$ of finite presentation over $\mathbf{Z}$ and similarly
$Y = \lim Y_j$, see Limits, Proposition \ref{limits-proposition-approximate}.
Then $\mathcal{F}$ will descend to $\mathcal{F}_{ij}$ on some $X_i \times_R Y_j$
(Limits, Lemma \ref{limits-lemma-descend-modules-finite-presentation}) and
so does the property of having the resolution property
(Derived Categories of Schemes, Lemma
\ref{perfect-lemma-resolution-property-descends}).
Then we apply Lemma \ref{lemma-on-product}
to $\mathcal{F}_{ij}$ and we pullback.
\end{proof}

\begin{lemma}
\label{lemma-diagonal-resolution}
Let $R$ be a Noetherian ring. Let $X$ be a separated finite type scheme
over $R$ which has the resolution property. Set
$\mathcal{O}_\Delta = \Delta_*(\mathcal{O}_X)$ where
$\Delta : X \to X \times_R X$ is the diagonal of $X/k$.
There exists a resolution
$$
\ldots \to
\mathcal{E}_2 \boxtimes \mathcal{G}_2 \to
\mathcal{E}_1 \boxtimes \mathcal{G}_1 \to
\mathcal{E}_0 \boxtimes \mathcal{G}_0 \to
\mathcal{O}_\Delta \to 0
$$
where each $\mathcal{E}_i$ and $\mathcal{G}_i$ is a finite locally
free $\mathcal{O}_X$-module.
\end{lemma}

\begin{proof}
Since $X$ is separated, the diagonal morphism $\Delta$ is a closed
immersion and hence $\mathcal{O}_\Delta$ is a coherent
$\mathcal{O}_{X \times_R X}$-module (Cohomology of Schemes, Lemma
\ref{coherent-lemma-i-star-equivalence}).
Thus the lemma follows immediately from Lemma \ref{lemma-on-product}.
\end{proof}

\begin{lemma}
\label{lemma-Ext-0-regular}
Let $X$ be a regular Noetherian scheme of dimension $d < \infty$. Then
\begin{enumerate}
\item for $\mathcal{F}$, $\mathcal{G}$ coherent $\mathcal{O}_X$-modules
we have $\Ext^n_X(\mathcal{F}, \mathcal{G}) = 0$ for $n > d$, and
\item for $K, L \in D^b_{\textit{Coh}}(\mathcal{O}_X)$ and $a \in \mathbf{Z}$
if $H^i(K) = 0$ for $i < a + d$ and $H^i(L) = 0$ for $i \geq a$ then
$\Hom_X(K, L) = 0$.
\end{enumerate}
\end{lemma}

\begin{proof}
To prove (1) we use the spectral sequence
$$
H^p(X, \SheafExt^q(\mathcal{F}, \mathcal{G})) \Rightarrow
\Ext^{p + q}_X(\mathcal{F}, \mathcal{G})
$$
of Cohomology, Section \ref{cohomology-section-ext}. Let $x \in X$.
We have
$$
\SheafExt^q(\mathcal{F}, \mathcal{G})_x =
\SheafExt^q_{\mathcal{O}_{X, x}}(\mathcal{F}_x, \mathcal{G}_x)
$$
see Cohomology, Lemma \ref{cohomology-lemma-stalk-internal-hom}
(this also uses that $\mathcal{F}$ is pseudo-coherent by
Derived Categories of Schemes, Lemma
\ref{perfect-lemma-identify-pseudo-coherent-noetherian}).
Set $d_x = \dim(\mathcal{O}_{X, x})$.
Since $\mathcal{O}_{X, x}$ is regular the ring
$\mathcal{O}_{X, x}$ has global dimension $d_x$, see
Algebra, Proposition \ref{algebra-proposition-regular-finite-gl-dim}.
Thus $\SheafExt^q_{\mathcal{O}_{X, x}}(\mathcal{F}_x, \mathcal{G}_x)$
is zero for $q > d_x$. It follows that the modules
$\SheafExt^q(\mathcal{F}, \mathcal{G})$ have support
of dimension at most $d - q$. Hence we have
$H^p(X, \SheafExt^q(\mathcal{F}, \mathcal{G})) = 0$ for $p > d - q$
by Cohomology, Proposition \ref{cohomology-proposition-vanishing-Noetherian}.
This proves (1).

\medskip\noindent
Proof of (2).
We may use induction on the number of nonzero cohomology sheaves
of $K$ and $L$. The case where these numbers are $0, 1$ follows
from (1). If the number of nonzero cohomology sheaves of $K$
is $> 1$, then we let $i \in \mathbf{Z}$ be minimal such that
$H^i(K)$ is nonzero. We obtain a distinguished triangle
$$
H^i(K)[-i] \to K \to \tau_{\geq i + 1}K
$$
(Derived Categories, Remark
\ref{derived-remark-truncation-distinguished-triangle})
and we get the vanishing of $\Hom(K, L)$ from the vanishing
of $\Hom(H^i(K)[-i], L)$ and $\Hom(\tau_{\geq i + 1}K, L)$
by Derived Categories, Lemma \ref{derived-lemma-representable-homological}.
Simlarly if $L$ has more than one nonzero cohomology sheaf.
\end{proof}

\begin{lemma}
\label{lemma-split-complex-regular}
Let $X$ be a regular Noetherian scheme of dimension $d < \infty$.
Let $K \in D^b_{\textit{Coh}}(\mathcal{O}_X)$ and $a \in \mathbf{Z}$.
If $H^i(K) = 0$ for $a < i < a + d$, then
$K = \tau_{\leq a}K \oplus \tau_{\geq a + d}K$.
\end{lemma}

\begin{proof}
We have $\tau_{\leq a}K = \tau_{\leq a + d - 1}K$ by the assumed
vanishing of cohomology sheaves. By Derived Categories, Remark
\ref{derived-remark-truncation-distinguished-triangle}
we have a distinguished triangle
$$
\tau_{\leq a}K \to K \to \tau_{\geq a + d}K \xrightarrow{\delta}
(\tau_{\leq a}K)[1]
$$
By Derived Categories, Lemma \ref{derived-lemma-split} it
suffices to show that the morphism $\delta$ is zero.
This follows from Lemma \ref{lemma-Ext-0-regular}.
\end{proof}

\begin{lemma}
\label{lemma-diagonal-trick}
Let $k$ be a field. Let $X$ be a quasi-compact separated smooth scheme over $k$.
There exist finite locally free $\mathcal{O}_X$-modules
$\mathcal{E}$ and $\mathcal{G}$ such that
$$
\mathcal{O}_\Delta \in \langle \mathcal{E} \boxtimes \mathcal{G} \rangle
$$
in $D(\mathcal{O}_{X \times X})$ where the notation is as in
Derived Categories, Section \ref{derived-section-generators}.
\end{lemma}

\begin{proof}
Recall that $X$ is regular by
Varieties, Lemma \ref{varieties-lemma-smooth-regular}.
Hence $X$ has the resolution property by
Derived Categories of Schemes, Lemma
\ref{perfect-lemma-regular-resolution-property}.
Hence we may choose a resolution as in Lemma \ref{lemma-diagonal-resolution}.
Say $\dim(X) = d$. Since $X \times X$ is smooth over $k$ it is regular.
Hence $X \times X$ is a regular Noetherian scheme with
$\dim(X \times X) = 2d$. The object
$$
K = (\mathcal{E}_{2d} \boxtimes \mathcal{G}_{2d} \to
\ldots \to
\mathcal{E}_0 \boxtimes \mathcal{G}_0)
$$
of $D_{perf}(\mathcal{O}_{X \times X})$
has cohomology sheaves $\mathcal{O}_\Delta$
in degree $0$ and $\Ker(\mathcal{E}_{2d} \boxtimes \mathcal{G}_{2d} \to
\mathcal{E}_{2d-1} \boxtimes \mathcal{G}_{2d-1})$ in degree $-2d$ and zero
in all other degrees.
Hence by Lemma \ref{lemma-split-complex-regular} we see that
$\mathcal{O}_\Delta$ is a summand of $K$ in
$D_{perf}(\mathcal{O}_{X \times X})$.
Clearly, the object $K$ is in
$$
\left\langle
\bigoplus\nolimits_{i = 0, \ldots, 2d} \mathcal{E}_i \boxtimes \mathcal{G}_i
\right\rangle
\subset
\left\langle
\left(\bigoplus\nolimits_{i = 0, \ldots, 2d} \mathcal{E}_i\right)
\boxtimes
\left(\bigoplus\nolimits_{i = 0, \ldots, 2d} \mathcal{G}_i\right)
\right\rangle
$$
which finishes the proof. (The reader may consult
Derived Categories, Lemmas \ref{derived-lemma-generated-by-E-explicit} and
\ref{derived-lemma-in-cone-n} to see that our object is contained in this
category.)
\end{proof}

\begin{lemma}
\label{lemma-smooth-proper-strong-generator}
Let $k$ be a field. Let $X$ be a scheme proper and smooth over $k$.
Then $D_{perf}(\mathcal{O}_X)$
has a strong generator.
\end{lemma}

\begin{proof}
Using Lemma \ref{lemma-diagonal-trick} choose finite locally free
$\mathcal{O}_X$-modules $\mathcal{E}$ and $\mathcal{G}$ such that
$\mathcal{O}_\Delta \in \langle \mathcal{E} \boxtimes \mathcal{G} \rangle$
in $D(\mathcal{O}_{X \times X})$. We claim that $\mathcal{G}$
is a strong generator for $D_{perf}(\mathcal{O}_X)$. With notation as in
Derived Categories, Section \ref{derived-section-operate-on-full}
choose $m, n \geq 1$ such that
$$
\mathcal{O}_\Delta \in
smd(add(\mathcal{E} \boxtimes \mathcal{G}[-m, m])^{\star n})
$$
This is possible by Derived Categories, Lemma
\ref{derived-lemma-find-smallest-containing-E}.
Let $K$ be an object of $D_{perf}(\mathcal{O}_X)$. Since
$L\text{pr}_1^*K \otimes_{\mathcal{O}_{X \times X}}^\mathbf{L} -$
is an exact functor and since
$$
L\text{pr}_1^*K \otimes_{\mathcal{O}_{X \times X}}^\mathbf{L}
(\mathcal{E} \boxtimes \mathcal{G}) =
(K \otimes_{\mathcal{O}_X}^\mathbf{L} \mathcal{E}) \boxtimes \mathcal{G}
$$
we conclude from
Derived Categories, Remark \ref{derived-remark-operations-functor} that
$$
L\text{pr}_1^*K
\otimes_{\mathcal{O}_{X \times X}}^\mathbf{L}
\mathcal{O}_\Delta
\in
smd(add(
(K \otimes_{\mathcal{O}_X}^\mathbf{L} \mathcal{E})
\boxtimes \mathcal{G}[-m, m])^{\star n})
$$
Applying the exact functor $R\text{pr}_{2, *}$ and observing that
$$
R\text{pr}_{2, *}
\left((K \otimes_{\mathcal{O}_X}^\mathbf{L} \mathcal{E}) \boxtimes
\mathcal{G}\right) =
R\Gamma(X, K \otimes_{\mathcal{O}_X}^\mathbf{L} \mathcal{E})
\otimes_k \mathcal{G}
$$
by Derived Categories of Schemes, Lemma
\ref{perfect-lemma-cohomology-base-change} we conclude that
$$
K = R\text{pr}_{2, *}(L\text{pr}_1^*K
\otimes_{\mathcal{O}_{X \times X}}^\mathbf{L} \mathcal{O}_\Delta)
\in
smd(add(R\Gamma(X, K \otimes_{\mathcal{O}_X}^\mathbf{L} \mathcal{E})
\otimes_k \mathcal{G}[-m, m])^{\star n})
$$
The equality follows from the discussion in
Example \ref{example-diagonal-fourier-mukai}.
Since $K$ is perfect, there exist $a \leq b$ such that
$H^i(X, K)$ is nonzero only for $i \in [a, b]$. Since $X$ is proper,
each $H^i(X, K)$ is finite dimensional. We conclude that
the right hand side is contained in
$smd(add(\mathcal{G}[-m + a, m + b])^{\star n})$ which is
itself contained in $\langle \mathcal{G} \rangle_n$ by one of the
references given above. This finishes the proof.
\end{proof}

\begin{lemma}
\label{lemma-diagonal-trick-proper}
Let $k$ be a field. Let $X$ be a proper smooth scheme over $k$.
There exists integers $m, n \geq 1$ and a finite locally free
$\mathcal{O}_X$-module $\mathcal{G}$ such that every coherent
$\mathcal{O}_X$-module is contained in $smd(add(\mathcal{G}[-m, m])^{\star n})$
with notation as in Derived Categories, Section
\ref{derived-section-operate-on-full}.
\end{lemma}

\begin{proof}
In the proof of Lemma \ref{lemma-smooth-proper-strong-generator}
we have shown that there exist $m', n \geq 1$ such that for any
coherent $\mathcal{O}_X$-module $\mathcal{F}$,
$$
\mathcal{F} \in smd(add(\mathcal{G}[-m' + a, m' + b])^{\star n})
$$
for any $a \leq b$ such that $H^i(X, \mathcal{F})$ is nonzero only
for $i \in [a, b]$. Thus we can take $a = 0$ and $b = \dim(X)$.
Taking $m = \max(m', m' + b)$ finishes the proof.
\end{proof}

\noindent
The following lemma is the boundedness result referred to
in the title of this section.

\begin{lemma}
\label{lemma-boundedness}
Let $k$ be a field. Let $X$ be a smooth proper scheme over $k$.
Let $\mathcal{A}$ be an abelian category. Let
$H : D_{perf}(\mathcal{O}_X) \to \mathcal{A}$ be a homological
functor (Derived Categories, Definition \ref{derived-definition-homological})
such that for all $K$ in $D_{perf}(\mathcal{O}_X)$ the object
$H^i(K)$ is nonzero for only a finite number of $i \in \mathbf{Z}$.
Then there exists an integer $m \geq 1$ such that
$H^i(\mathcal{F}) = 0$ for any coherent $\mathcal{O}_X$-module
$\mathcal{F}$ and $i \not \in [-m, m]$.
Similarly for cohomological functors.
\end{lemma}

\begin{proof}
Combine Lemma \ref{lemma-diagonal-trick-proper} with
Derived Categories, Lemma \ref{derived-lemma-forward-cone-n}.
\end{proof}

\begin{lemma}
\label{lemma-bounded-fibres}
Let $k$ be a field. Let $X$, $Y$ be finite type schemes over $k$.
Let $K_0 \to K_1 \to K_2 \to \ldots$ be a system of objects
of $D_{perf}(\mathcal{O}_{X \times Y})$ and $m \geq 0$ an integer such that
\begin{enumerate}
\item $H^q(K_i)$ is nonzero only for $q \leq m$,
\item for every coherent $\mathcal{O}_X$-module $\mathcal{F}$ with
$\dim(\text{Supp}(\mathcal{F})) = 0$ the object
$$
R\text{pr}_{2, *}(
\text{pr}_1^*\mathcal{F} \otimes_{\mathcal{O}_{X \times Y}}^\mathbf{L}
K_n)
$$
has vanishing cohomology sheaves in degrees outside
$[-m, m] \cup [-m - n, m - n]$ and for $n > 2m$ the transition maps
induce isomorphisms on cohomology sheaves in degrees in $[-m, m]$.
\end{enumerate}
Then $K_n$ has vanishing cohomology sheaves in degrees outside
$[-m, m] \cup [-m - n, m - n]$ and for $n > 2m$ the
transition maps induce isomorphisms on cohomology sheaves in degrees in
$[-m, m]$. Moreover, if $X$ and $Y$ are smooth over $k$, then for $n$
large enough we find $K_n = K \oplus C_n$ in
$D_{perf}(\mathcal{O}_{X \times Y})$
where $K$ has cohomology only indegrees $[-m, m]$ and $C_n$ only in
degrees $[-m - n, m - n]$ and the transition maps
define isomorphisms between various copies of $K$.
\end{lemma}

\begin{proof}
Let $Z$ be the scheme theoretic support of an $\mathcal{F}$ as in (2).
Then $Z \to \Spec(k)$ is finite, hence $Z \times Y \to Y$ is finite.
It follows that for an object $M$ of $D_\QCoh(\mathcal{O}_{X \times Y})$
with cohomology sheaves supported on $Z \times Y$ we have
$H^i(R\text{pr}_{2, *}(M)) = \text{pr}_{2, *}H^i(M)$ and the functor
$\text{pr}_{2, *}$ is faithful on quasi-coherent modules supported
on $Z \times Y$; details omitted. Hence we see that the objects
$$
\text{pr}_1^*\mathcal{F} \otimes_{\mathcal{O}_{X \times Y}}^\mathbf{L} K_n
$$
in $D_{perf}(\mathcal{O}_{X \times Y})$ have vanishing cohomology sheaves
outside $[-m, m] \cup [-m - n, m - n]$ and for $n > 2m$ the transition maps
induce isomorphisms on cohomology sheaves in $[-m, m]$.
Let $z \in X \times Y$ be a closed point mapping to the closed point
$x \in X$. Then we know that
$$
K_{n, z} \otimes_{\mathcal{O}_{X \times Y, z}}^\mathbf{L}
\mathcal{O}_{X \times Y, z}/\mathfrak m_x^t\mathcal{O}_{X \times Y, z}
$$
has nonzero cohomology only in the intervals
$[-m, m] \cup [-m - n, m - n]$.
We conclude by More on Algebra, Lemma
\ref{more-algebra-lemma-kollar-kovacs-pseudo-coherent}
that $K_{n, z}$ only has nonzero cohomology
in degrees $[-m, m] \cup [-m - n, m - n]$. Since this holds for all
closed points of $X \times Y$, we conclude $K_n$ only has nonzero
cohomology sheaves in degrees $[-m, m] \cup [-m - n, m - n]$.
In exactly the same way we see that the maps $K_n \to K_{n + 1}$
are isomorphisms on cohomology sheaves in degrees $[-m, m]$
for $n > 2m$.

\medskip\noindent
If $X$ and $Y$ are smooth over $k$, then $X \times Y$ is smooth
over $k$ and hence regular by
Varieties, Lemma \ref{varieties-lemma-smooth-regular}.
Thus we will obtain the direct sum decomposition of $K_n$
as soon as $n > 2m + \dim(X \times Y)$ from
Lemma \ref{lemma-split-complex-regular}. The final statement
is clear from this.
\end{proof}





\section{Sibling functors}
\label{section-sibling}

\noindent
In this section we prove some categorical result on the following notion.

\begin{definition}
\label{definition-siblings}
Let $\mathcal{A}$ be an abelian category. Let $\mathcal{D}$ be a
triangulated category. We say two exact functors of triangulated categories
$$
F, F' : D^b(\mathcal{A}) \longrightarrow \mathcal{D}
$$
are {\it siblings}, or we say $F'$ is a {\it sibling} of $F$,
if the following two conditions are satisfied
\begin{enumerate}
\item the functors $F \circ i$ and $F' \circ i$ are isomorphic
where $i : \mathcal{A} \to D^b(\mathcal{A})$ is the inclusion functor, and
\item $F(K) \cong F'(K)$ for any $K$ in $D^b(\mathcal{A})$.
\end{enumerate}
\end{definition}

\noindent
Sometimes the second condition is a consequence of the first.

\begin{lemma}
\label{lemma-sibling-fully-faithful}
Let $\mathcal{A}$ be an abelian category. Let $\mathcal{D}$ be a
triangulated category. Let
$F, F' : D^b(\mathcal{A}) \longrightarrow \mathcal{D}$
be exact functors of triangulated categories. Assume
\begin{enumerate}
\item the functors $F \circ i$ and $F' \circ i$ are isomorphic
where $i : \mathcal{A} \to D^b(\mathcal{A})$ is the inclusion functor, and
\item for all $X, Y \in \Ob(\mathcal{A})$ we have
$\Ext^q_\mathcal{D}(F(X), F(Y)) = 0$ for $q < 0$ (for example
if $F$ is fully faithful).
\end{enumerate}
Then $F$ and $F'$ are siblings.
\end{lemma}

\begin{proof}
Let $K \in D^b(\mathcal{A})$. We will show $F(K)$ is isomorphic to $F'(K)$.
We can represent $K$ by a bounded complex $A^\bullet$ of objects of
$\mathcal{A}$. After replacing $K$ by a translation we may
assume $A^i = 0$ for $i > 0$. Choose $n \geq 0$ such that $A^{-i} = 0$
for $i > n$. The objects
$$
M_i = (A^{-i} \to \ldots \to A^0)[-i],\quad i = 0, \ldots, n
$$
form a Postnikov system in $D^b(\mathcal{A})$ for the complex
$A^\bullet = A^{-n} \to \ldots \to A^0$ in $D^b(\mathcal{A})$.
See Derived Categories, Example \ref{derived-example-key-postnikov}.
Since both $F$ and $F'$ are exact functors of triangulated categories both
$$
F(M_i)
\quad\text{and}\quad
F'(M_i)
$$
form a Postnikov system in $\mathcal{D}$ for the complex
$$
F(A^{-n}) \to \ldots \to F(A^0) =
F'(A^{-n}) \to \ldots \to F'(A^0)
$$
Since all negative $\Ext$s between these objects vanish by assumption
we conclude by uniqueness of Postnikov systems
(Derived Categories, Lemma \ref{derived-lemma-existence-postnikov-system})
that $F(K) = F(M_n[n]) \cong F'(M_n[n]) = F'(K)$.
\end{proof}

\begin{lemma}
\label{lemma-sibling-faithful}
Let $F$ and $F'$ be siblings as in Definition \ref{definition-siblings}.
Then
\begin{enumerate}
\item if $F$ is essentially surjective, then $F'$ is essentially
surjective,
\item if $F$ is fully faithful, then $F'$ is fully faithful.
\end{enumerate}
\end{lemma}

\begin{proof}
Part (1) is immediate from property (2) for siblings.

\medskip\noindent
Assume $F$ is fully faithful. Denote $\mathcal{D}' \subset \mathcal{D}$
the essential image of $F$ so that $F : D^b(\mathcal{A}) \to \mathcal{D}'$
is an equivalence. Since the functor $F'$ factors through $\mathcal{D}'$
by property (2) for siblings, we can consider the functor
$H = F^{-1} \circ F' : D^b(\mathcal{A}) \to D^b(\mathcal{A})$.
Observe that $H$ is a sibling of the identity functor.
Since it suffices to prove that $H$ is fully faithful,
we reduce to the problem discussed in the next paragraph.

\medskip\noindent
Set $\mathcal{D} = D^b(\mathcal{A})$. We have to show a sibling
$F : \mathcal{D} \to \mathcal{D}$ of the identity functor is fully faithful.
Denote $a_X : X \to F(X)$ the functorial isomorphism for
$X \in \Ob(\mathcal{A})$ given to us by Definition \ref{definition-siblings}.
For any $K$ in $\mathcal{D}$ and distinguished triangle
$K_1 \to K_2 \to K_3$ of $\mathcal{D}$
if the maps
$$
F : \Hom(K, K_i[n]) \to \Hom(F(K), F(K_i[n]))
$$
are isomorphisms for all $n \in \mathbf{Z}$ and $i = 1, 3$, then the
same is true for $i = 2$ and all $n \in \mathbf{Z}$. This uses the
$5$-lemma Homology, Lemma \ref{homology-lemma-five-lemma} and
Derived Categories, Lemma \ref{derived-lemma-representable-homological};
details omitted. Similarly, if the maps
$$
F : \Hom(K_i[n], K) \to \Hom(F(K_i[n]), F(K))
$$
are isomorphisms for all $n \in \mathbf{Z}$ and $i = 1, 3$, then the
same is true for $i = 2$ and all $n \in \mathbf{Z}$. Using the canonical
truncations and induction on the number of nonzero cohomology objects,
we see that it is enough to show
$$
F : \Ext^q(X, Y) \to \Ext^q(F(X), F(Y))
$$
is bijective for all $X, Y \in \Ob(\mathcal{A})$ and all $q \in \mathbf{Z}$.
Since $F$ is a sibling of $\text{id}$ we have $F(X) \cong X$ and
$F(Y) \cong Y$ hence the right hand side is zero for $q < 0$.
The case $q = 0$ is OK by our assumption that $F$ is a sibling of
the identity functor. It remains to prove the cases $q > 0$.

\medskip\noindent
The case $q = 1$: Injectivity. An element $\xi$ of $\Ext^1(X, Y)$
gives rise to a distinguished triangle
$$
Y \to E \to X \xrightarrow{\xi} Y[1]
$$
Observe that $E \in \Ob(\mathcal{A})$. Since $F$ is a sibling of the
identity functor we obtain a commutative diagram
$$
\xymatrix{
E \ar[d] \ar[r] & X \ar[d] \\
F(E) \ar[r] & F(X)
}
$$
whose vertical arrows are the isomorphisms $a_E$ and $a_X$.
By TR3 the distinguished triangle associated to $\xi$ we started
with is isomorphic to the distinguished triangle
$$
F(Y) \to F(E) \to F(X) \xrightarrow{F(\xi)} F(Y[1]) = F(Y)[1]
$$
Thus $\xi = 0$ if and only if $F(\xi)$ is zero, i.e., we see that
$F : \Ext^1(X, Y) \to \Ext^1(F(X), F(Y))$ is injective.

\medskip\noindent
The case $q = 1$: Surjectivity. Let $\theta$ be an element of
$\Ext^1(F(X), F(Y))$. This defines an extension of $F(X)$ by $F(Y)$
in $\mathcal{A}$ which we may write as $F(E)$
as $F$ is a sibling of the identity functor. We thus get a distinguished
triangle
$$
F(Y) \xrightarrow{F(\alpha)} F(E)
\xrightarrow{F(\beta)} F(X)
\xrightarrow{\theta} F(Y[1]) = F(Y)[1]
$$
for some morphisms $\alpha : Y \to E$ and $\beta : E \to X$.
Since $F$ is a sibling of the identity functor, the sequence
$0 \to Y \to E \to X \to 0$
is a short exact sequence in $\mathcal{A}$! Hence we obtain a
distinguished triangle
$$
Y \xrightarrow{\alpha} E \xrightarrow{\beta} X \xrightarrow{\delta} Y[1]
$$
for some morphism $\delta : X \to Y[1]$. Applying the exact functor
$F$ we obtain the distinguished triangle
$$
F(Y) \xrightarrow{F(\alpha)} F(E) \xrightarrow{F(\beta)} F(X)
\xrightarrow{F(\delta)} F(Y)[1]
$$
Arguing as above, we see that these triangles are isomorphic.
Hence there exists a commutative diagram
$$
\xymatrix{
F(X) \ar[d]^\gamma \ar[r]_{F(\delta)} & F(Y[1]) \ar[d]_\epsilon \\
F(X) \ar[r]^\theta & F(Y[1])
}
$$
for some isomorphisms $\gamma$, $\epsilon$ (we can say more but we won't
need more information). We may write $\gamma = F(\gamma')$ and
$\epsilon = F(\epsilon')$. Then we have
$\theta = F(\epsilon' \circ \delta \circ (\gamma')^{-1})$
and we see the surjectivity holds.

\medskip\noindent
The case $q > 1$: surjectivity. Using Yoneda extensions, see
Derived Categories, Section \ref{derived-section-ext}, we find that for any
element $\xi$ in $\Ext^q(F(X), F(Y))$ we can find
$F(X) = B_0, B_1, \ldots, B_{q - 1}, B_q = F(Y) \in \Ob(\mathcal{A})$ and
elements
$$
\xi_i \in \Ext^1(B_{i - 1}, B_i)
$$
such that $\xi$ is the composition $\xi_q \circ \ldots \circ \xi_1$.
Write $B_i = F(A_i)$ (of course we have $A_i = B_i$ but we don't
need to use this) so that
$$
\xi_i = F(\eta_i) \in \Ext^1(F(A_{i - 1}), F(A_i))
\quad\text{with}\quad
\eta_i \in \Ext^1(A_{i - 1}, A_i)
$$
by surjectivity for $q = 1$. Then $\eta = \eta_q \circ \ldots \circ \eta_1$
is an element of $\Ext^q(X, Y)$ with $F(\eta) = \xi$.

\medskip\noindent
The case $q > 1$: injectivity. An element $\xi$ of $\Ext^q(X, Y)$
gives rise to a distinguished triangle
$$
Y[q - 1] \to E \to X \xrightarrow{\xi} Y[q]
$$
Applying $F$ we obtain a distinguished triangle
$$
F(Y)[q - 1] \to F(E) \to F(X) \xrightarrow{F(\xi)} F(Y)[q]
$$
If $F(\xi) = 0$, then $F(E) \cong F(Y)[q - 1] \oplus F(X)$
in $\mathcal{D}$, see
Derived Categories, Lemma \ref{derived-lemma-split}.
Since $F$ is a sibling of the identity functor we have
$E \cong F(E)$ and hence
$$
E \cong F(E) \cong F(Y)[q - 1] \oplus F(X) \cong Y[q - 1] \oplus X
$$
In other words, $E$ is isomorphic to the
direct sum of its cohomology objects. This implies that the
initial distinguished triangle is split, i.e., $\xi = 0$.
\end{proof}

\noindent
Let us make a nonstandard definition. Let $\mathcal{A}$ be an abelian
category. Let us say $\mathcal{A}$ {\it has enough negative objects}
if given any $X \in \Ob(\mathcal{A})$ there exists an object $N$ such that
\begin{enumerate}
\item there is a surjection $N \to X$ and
\item $\Hom(X, N) = 0$.
\end{enumerate}
Let us prove a couple of lemmas about this notion in order to
help with the proof of Proposition \ref{proposition-siblings-isomorphic}.

\begin{lemma}
\label{lemma-good-map}
Let $\mathcal{A}$ be an abelian category with enough negative objects.
Let $X \in D^b(\mathcal{A})$. Let $b \in \mathbf{Z}$ with
$H^i(X) = 0$ for $i > b$. Then
there exists a map $N[-b] \to X$ such that the induced map
$N \to H^b(X)$ is surjective and $\Hom(H^b(X), N) = 0$.
\end{lemma}

\begin{proof}
Using the truncation functors we can represent $X$ by a complex
$A^a \to A^{a + 1} \to \ldots \to A^b$ of objects of $\mathcal{A}$.
Choose $N$ in $\mathcal{A}$ such that there exists a surjection
$t : N \to A^b$ and such that $\Hom(A^b, N) = 0$. Then the surjection $t$
defines a map $N[-b] \to X$ as desired.
\end{proof}

\begin{lemma}
\label{lemma-good-map-zero}
Let $\mathcal{A}$ be an abelian category with enough negative objects.
Let $f : X \to X'$ be a morphism of $D^b(\mathcal{A})$. Let $b \in \mathbf{Z}$
such that $H^i(X) = 0$ for $i > b$ and $H^i(X') = 0$ for $i \geq b$.
Then there exists a map $N[-b] \to X$ such that the induced map
$N \to H^b(X)$ is surjective, such that $\Hom(H^b(X), N) = 0$, and
such that the composition $N[-b] \to X \to X'$ is zero.
\end{lemma}

\begin{proof}
We can represent $f$ by a map $f^\bullet : A^\bullet \to B^\bullet$
of bounded complexes of objects of $\mathcal{A}$, see for example
Derived Categories, Lemma \ref{derived-lemma-bounded-derived}.
Consider the object
$$
C = \Ker(A^b \to A^{b + 1}) \times_{\Ker(B^b \to B^{b + 1})} B^{b - 1}
$$
of $\mathcal{A}$. Since $H^b(B^\bullet) = 0$ we see that
$C \to H^b(A^\bullet)$ is surjective. On the other hand, the map
$C \to A^b \to B^b$ is the same as the map $C \to B^{b - 1} \to B^b$
and hence the composition $C[-b] \to X \to X'$ is zero.
Since $\mathcal{A}$ has enough negative objects, we can find an object $N$
which has a surjection $N \to C \oplus H^b(X)$ such that
$\Hom(C \oplus H^b(X), N) = 0$. Then $N$ together with the map
$N[-b] \to X$ is a solution to the problem posed by the lemma.
\end{proof}

\noindent
We encourage the reader to read the original
\cite[Proposition 2.16]{Orlov-K3} for the marvellous ideas
that go into the proof of the following proposition.

\begin{proposition}
\label{proposition-siblings-isomorphic}
\begin{reference}
\cite[Proposition 2.16]{Orlov-K3}; the fact that we do not need
to assume vanishing of $\Ext^q(N, X)$ for $q > 0$ in the definition
of negative objects above is due to \cite{Canonaco-Stellari}.
\end{reference}
Let $F$ and $F'$ be siblings as in Definition \ref{definition-siblings}.
Assume that $F$ is fully faithful and that $\mathcal{A}$ has enough
negative objects (see above). Then $F$ and $F'$ are isomorphic functors.
\end{proposition}

\begin{proof}
By part (2) of Definition \ref{definition-siblings} the image of the functor
$F'$ is contained in the essential image of the functor $F$. Hence
the functor $H = F^{-1} \circ F'$ is a sibling of the identity functor.
This reduces us to the case described in the next paragraph.

\medskip\noindent
Let $\mathcal{D} = D^b(\mathcal{A})$. We have to show a sibling
$F : \mathcal{D} \to \mathcal{D}$ of the identity functor is
isomorphic to the identity functor. Given an object $X$ of $\mathcal{D}$
let us say $X$ has {\it width} $w = w(X)$ if $w \geq 0$ is minimal
such that there exists an integer $a \in \mathbf{Z}$ with $H^i(X) = 0$
for $i \not \in [a, a + w - 1]$. Since $F$ is a sibling of the identity
and since $F \circ [n] = [n] \circ F$ we are aready given isomorphisms
$$
c_X : X \to F(X)
$$
for $w(X) \leq 1$ compatible with shifts. Moreover, if $X = A[-a]$ and
$X' = A'[-a]$ for some $A, A' \in \Ob(\mathcal{A})$ then for any morphism
$f : X \to X'$ the diagram
\begin{equation}
\label{equation-to-show}
\vcenter{
\xymatrix{
X \ar[d]_{c_X} \ar[r]_f &
X' \ar[d]^{c_{X'}} \\
F(X) \ar[r]^{F(f)} &
F(X')
}
}
\end{equation}
is commutative.

\medskip\noindent
Next, let us show that for any morphism $f : X \to X'$  with
$w(X), w(X') \leq 1$ the diagram (\ref{equation-to-show}) commutes.
If $X$ or $X'$ is zero, this is clear. If not then we can write
$X = A[-a]$ and $X' = A'[-a']$ for unique $A, A'$ in $\mathcal{A}$
and $a, a' \in \mathbf{Z}$. The case $a = a'$ was discussed above.
If $a' > a$, then $f = 0$ (Derived Categories, Lemma
\ref{derived-lemma-negative-exts}) and the result is clear.
If $a' < a$ then $f$ corresponds to an element
$\xi \in \Ext^q(A, A')$ with $q = a - a'$. Using Yoneda extensions, see
Derived Categories, Section \ref{derived-section-ext}, we can find
$A = A_0, A_1, \ldots, A_{q - 1}, A_q = A' \in \Ob(\mathcal{A})$ and
elements
$$
\xi_i \in \Ext^1(A_{i - 1}, A_i)
$$
such that $\xi$ is the composition $\xi_q \circ \ldots \circ \xi_1$.
In other words, setting $X_i = A_i[-a + i]$
we obtain morphisms
$$
X = X_0 \xrightarrow{f_1} X_1 \to \ldots \to X_{q - 1}
\xrightarrow{f_q} X_q = X'
$$
whose compostion is $f$. Since the commutativity of  (\ref{equation-to-show})
for $f_1, \ldots, f_q$ implies it for $f$, this reduces us to the case $q = 1$.
In this case after shifting we may assume we have a distinguished triangle
$$
A' \to E \to A \xrightarrow{f} A'[1]
$$
Observe that $E$ is an object of $\mathcal{A}$. Consider the following
diagram
$$
\xymatrix{
E \ar[d]_{c_E} \ar[r] &
A \ar[d]_{c_A} \ar[r]_f &
A'[1] \ar[d]^{c_{A'}[1]}
\ar@{..>}@<-1ex>[d]_\gamma \ar@{..>}[ld]^\epsilon \ar[r] &
E[1] \ar[d]^{c_E[1]} \\
F(E) \ar[r] &
F(A) \ar[r]^{F(f)} &
F(A')[1] \ar[r] &
F(E)[1]
}
$$
whose rows are distinguished triangles.
The square on the right commutes already but we don't yet know that
the middle square does. By the axioms of a triangulated category
we can find a morphism $\gamma$ which does make the diagram commute.
Then $\gamma - c_{A'}[1]$ composed with
$F(A')[1] \to F(E)[1]$ is zero hence we
can find $\epsilon : A'[1] \to F(A)$ such that
$\gamma - c_{A'}[1] = F(f) \circ \epsilon$. However, any arrow
$A'[1] \to F(A)$ is zero as it is a negative ext class
between objects of $\mathcal{A}$. Hence $\gamma = c_{A'}[1]$
and we conclude the middle square commutes too which is what we
wanted to show.

\medskip\noindent
To finish the proof we are going to argue by induction on $w$
that there exist isomorphisms $c_X : X \to F(X)$ for all
$X$ with $w(X) \leq w$ compatible with all morphisms between
such objects. The base case $w = 1$ was shown above. Assume
we know the result for some $w \geq 1$.

\medskip\noindent
Let $X$ be an object with $w(X) = w + 1$. Pick $a \in \mathbf{Z}$ with
$H^i(X) = 0$ for $i \not \in [a, a + w]$. Set $b = a + w$ so that
$H^b(X)$ is nonzero. Choose $N[-b] \to X$ as in Lemma \ref{lemma-good-map}.
Choose a distinguished diagram
$$
N[-b] \to X \to Y \to N[-b + 1]
$$
Computing the long exact cohomology sequence we find
$w(Y) \leq w$. Hence by induction we find the solid arrows
in the following diagram
$$
\xymatrix{
N[-b] \ar[r] \ar[d]_{c_N[-b]} &
X \ar[r] \ar@{..>}[d]_{c_{N[-b] \to X}} &
Y \ar[r] \ar[d]^{c_Y} &
N[-b + 1] \ar[d]^{c_N[-b + 1]} \\
F(N)[-b] \ar[r] &
F(X) \ar[r] &
F(Y) \ar[r] &
F(N)[-b + 1]
}
$$
We obtain the dotted arrow $c_{N[-b] \to X}$.
By Derived Categories, Lemma \ref{derived-lemma-uniqueness-third-arrow}
the dotted arrow is unique because $\Hom(X, F(N)[-b]) \cong \Hom(X, N[-b]) = 0$
by our choice of $N$. In fact, $c_{N[-b] \to X}$ is the unique dotted
arrow making the square with vertices $X, Y, F(X), F(Y)$ commute.

\medskip\noindent
Let $N'[-b] \to X$ be another map as in Lemma \ref{lemma-good-map}
and let us prove that $c_{N[-b] \to X} = c_{N'[-b] \to X}$.
Observe that the map $(N \oplus N')[-b] \to X$ also satisfies the
conditions of Lemma \ref{lemma-good-map}.
Thus we may assume $N'[-b] \to X$ factors
as $N'[-b] \to N[-b] \to X$ for some morphism $N' \to N$.
Choose distinguished triangles $N[-b] \to X \to Y \to N[-b + 1]$ and
$N'[-b] \to X \to Y' \to N'[-b + 1]$. By axiom TR3 we can find
a morphism $g : Y' \to Y$ which joint with $\text{id}_X$ and $N' \to N$
forms a morphism of triangles. Since we have
(\ref{equation-to-show}) for $g$ we conclude that
$$
(F(X) \to F(Y)) \circ c_{N'[-b] \to X} = (F(X) \to F(Y)) \circ c_{N[-b] \to X}
$$
The uniqueness of $c_{N[-b] \to X}$ pointed out in the construction
above now shows that $c_{N'[-b] \to X} = c_{N[-b] \to X}$.

\medskip\noindent
Thus we can now define for $X$ of width $w + 1$ the isomorphism
$c_X : X \to F(X)$ as the common value of the maps $c_{N[-b] \to X}$
where $N[-b] \to X$ is as in Lemma \ref{lemma-good-map}. To finish
the proof, we have to show that the diagrams (\ref{equation-to-show})
commute for all morphisms $f : X \to X'$ between objects with $w(X) \leq w + 1$
and $w(X') \leq w + 1$. Choose $a \leq b \leq a + w$ such that
$H^i(X) = 0$ for $i \not \in [a, b]$ and
$a' \leq b' \leq a' + w$ such that $H^i(X') = 0$ for
$i \not \in [a', b']$. We will use induction on
$(b' - a') + (b - a)$ to show the claim. (The base case
is when this number is zero which is OK because $w \geq 1$.)
We distinguish two cases.

\medskip\noindent
Case I: $b' < b$. In this case, by Lemma \ref{lemma-good-map-zero}
we may choose $N[-b] \to X$ as in Lemma \ref{lemma-good-map}
such that the composition $N[-b] \to X \to X'$ is zero.
Choose a distuiguished triangle $N[-b] \to X \to Y \to N[-b + 1]$. Since
$N[-b] \to X'$ is zero, we find that $f$ factors
as $X \to Y \to X'$. Since $H^i(Y)$ is nonzero only for $i \in [a, b - 1]$
we see by induction that (\ref{equation-to-show}) commutes for
$Y \to X'$. The diagram (\ref{equation-to-show}) commutes for
$X \to Y$ by construction if $w(X) = w + 1$ and by our first
induction hypothesis if $w(X) \leq w$.
Hence (\ref{equation-to-show}) commutes for $f$.

\medskip\noindent
Case II: $b' \geq b$. In this case we choose $N'[-b'] \to X'$
as in Lemma \ref{lemma-good-map}.
We may also assume that $\Hom(H^{b'}(X), N') = 0$ (this is
relevant only if $b' = b$), for example because we can
replace $N'$ by an object $N''$ which surjects onto $N' \oplus H^{b'}(X)$
and such that $\Hom(N' \oplus H^{b'}(X), N'') = 0$.
We choose a distinguished triangle
$N'[-b'] \to X' \to Y' \to N'[-b' + 1]$. Since
$\Hom(X, X') \to \Hom(X, Y')$ is injective by our choice of $N'$
(details omitted) the same is true for
$\Hom(X, F(X')) \to \Hom(X, F(Y'))$.
Hence it suffices in this case to check that
(\ref{equation-to-show}) commutes for the composition $X \to Y'$
of the morphisms $X \to X' \to Y'$.
Since $H^i(Y')$ is nonzero only for $i \in [a', b' - 1]$
we conclude by induction hypothesis.
\end{proof}










\section{Deducing fully faithfulness}
\label{section-get-fully-faithful}

\noindent
It will be useful for us to know when a functor is fully faithful
we offer the following variant of \cite[Lemma 2.15]{Orlov-K3}.

\begin{lemma}
\label{lemma-get-fully-faithful}
\begin{reference}
Variant of \cite[Lemma 2.15]{Orlov-K3}
\end{reference}
Let $F : \mathcal{D} \to \mathcal{D}'$ be an exact functor of
triangulated categories. Let $S \subset \Ob(\mathcal{D})$ be
a set of objects. Assume
\begin{enumerate}
\item $F$ has both right and left adjoints,
\item for $K \in \mathcal{D}$ if $\Hom(E, K[i]) = 0$ for all
$E \in S$ and $i \in \mathbf{Z}$ then $K = 0$,
\item for $K \in \mathcal{D}$ if $\Hom(K, E[i]) = 0$ for all
$E \in S$ and $i \in \mathbf{Z}$ then $K = 0$,
\item the map $\Hom(E, E'[i]) \to \Hom(F(E), F(E')[i])$ induced by $F$
is bijective for all $E, E' \in S$ and $i \in \mathbf{Z}$.
\end{enumerate}
Then $F$ is fully faithful.
\end{lemma}

\begin{proof}
Denote $F_r$ and $F_l$ the right and left adjoints of $F$. For
$E \in S$ choose a distinguished triangle
$$
E \to F_r(F(E)) \to C \to E[1]
$$
where the first arrow is the unit of the adjunction. For $E' \in S$ we have
$$
\Hom(E', F_r(F(E))[i]) = \Hom(F(E'), F(E)[i]) = \Hom(E', E[i])
$$
The last equality holds by assumption (4).
Hence applying the homological functor $\Hom(E', -)$
(Derived Categories, Lemma \ref{derived-lemma-representable-homological})
to the distinguished triangle above we conclude that $\Hom(E', C[i]) = 0$
for all $i \in \mathbf{Z}$ and $E' \in S$. By assumption (2) we conclude
that $C = 0$ and $E = F_r(F(E))$.

\medskip\noindent
For $K \in \Ob(\mathcal{D})$ choose a distinguished triangle
$$
F_l(F(K)) \to K \to C \to F_l(F(K))[1]
$$
where the first arrow is the counit of the adjunction. For $E \in S$
we have
$$
\Hom(F_l(F(K)), E[i]) = \Hom(F(K), F(E)[i]) =
\Hom(K, F_r(F(E))[i]) = \Hom(K, E[i])
$$
where the last equality holds by the result of the first paragraph.
Thus we conclude as before that $\Hom(C, E[i]) = 0$ for all $E \in S$
and $i \in \mathbf{Z}$. Hence $C = 0$ by assumption (3).
Thus $F$ is fully faithful by Categories, Lemma
\ref{categories-lemma-adjoint-fully-faithful}.
\end{proof}

\begin{lemma}
\label{lemma-duality-at-point}
Let $k$ be a field. Let $X$ be a scheme of finite type over $k$ which
is regular. Let $x \in X$ be a closed point. For a coherent
$\mathcal{O}_X$-module $\mathcal{F}$ supported at $x$ choose
a coherent $\mathcal{O}_X$-module $\mathcal{F}'$ supported at $x$
such that $\mathcal{F}_x$ and $\mathcal{F}'_x$ are Matlis dual.
Then there is an isomorphism
$$
\Hom_X(\mathcal{F}, M) =
H^0(X, M \otimes_{\mathcal{O}_X}^\mathbf{L} \mathcal{F}'[-d_x])
$$
where $d_x = \dim(\mathcal{O}_{X, x})$
functorial in $M$ in $D_{perf}(\mathcal{O}_X)$.
\end{lemma}

\begin{proof}
Since $\mathcal{F}$ is supported at $x$ we have
$$
\Hom_X(\mathcal{F}, M) =
\Hom_{\mathcal{O}_{X, x}}(\mathcal{F}_x, M_x)
$$
and similarly we have
$$
H^0(X, M \otimes_{\mathcal{O}_X}^\mathbf{L} \mathcal{F}'[-d_x]) =
\text{Tor}^{\mathcal{O}_{X, x}}_{d_x}(M_x, \mathcal{F}'_x)
$$
Thus it suffices to show that given a Noetherian regular local ring $A$
of dimension $d$ and a finite length $A$-module $N$, if
$N'$ is the Matlis dual to $N$, then there exists a functorial isomorphism
$$
\Hom_A(N, K) = \text{Tor}^A_d(K, N')
$$
for $K$ in $D_{perf}(A)$. We can write the left hand side as
$H^0(R\Hom_A(N, A) \otimes_A^\mathbf{L} K)$ by
More on Algebra, Lemma \ref{more-algebra-lemma-dual-perfect-complex}
and the fact that $N$ determines a perfect object of $D(A)$.
Hence the formula holds because
$$
R\Hom_A(N, A) = R\Hom_A(N, A[d])[-d] = N'[-d]
$$
by Dualizing Complexes, Lemma \ref{dualizing-lemma-dualizing-finite-length}
and the fact that $A[d]$ is a normalized dualizing complex over $A$
($A$ is Gorenstein by
Dualizing Complexes, Lemma \ref{dualizing-lemma-regular-gorenstein}).
\end{proof}

\begin{lemma}
\label{lemma-orthogonal-point-sheaf}
Let $k$ be a field. Let $X$ be a scheme of finite type over $k$ which
is regular. Let $x \in X$ be a closed point and denote $\mathcal{O}_x$
the skyscraper sheaf at $x$ with value $\kappa(x)$. Let $K$ in
$D_{perf}(\mathcal{O}_X)$.
\begin{enumerate}
\item If $\Ext^i_X(\mathcal{O}_x, K) = 0$ then there exists an open
neighbourhood $U$ of $x$ such that $H^{i - d_x}(K)|_U = 0$ where
$d_x = \dim(\mathcal{O}_{X, x})$.
\item If $\Hom_X(\mathcal{O}_x, K[i]) = 0$ for all
$i \in \mathbf{Z}$, then $K$ is zero in an open neighbourhood of $x$.
\item If $\Ext^i_X(K, \mathcal{O}_x) = 0$ then there exists an open
neighbourhood $U$ of $x$ such that $H^i(K^\vee)|_U = 0$.
\item If $\Hom_X(K, \mathcal{O}_x[i]) = 0$ for all
$i \in \mathbf{Z}$, then $K$ is zero in an open neighbourhood of $x$.
\item If $H^i(X, K \otimes_{\mathcal{O}_X}^\mathbf{L} \mathcal{O}_x) = 0$
then there exists an open neighbourhood $U$ of $x$ such that
$H^i(K)|_U = 0$.
\item If $H^i(X, K \otimes_{\mathcal{O}_X}^\mathbf{L} \mathcal{O}_x) = 0$
for $i \in \mathbf{Z}$ then $K$ is zero in an
open neighbourhood of $x$.
\end{enumerate}
\end{lemma}

\begin{proof}
Observe that $H^i(X, K \otimes_{\mathcal{O}_X}^\mathbf{L} \mathcal{O}_x)$
is equal to $K_x  \otimes_{\mathcal{O}_{X, x}}^\mathbf{L} \kappa(x)$.
Hence part (5) follows from More on Algebra, Lemma
\ref{more-algebra-lemma-cut-complex-in-two}.
Part (6) follows from part (5).
Part (1) follows from part (5), Lemma \ref{lemma-duality-at-point}, and the
fact that the Matlis dual of $\kappa(x)$ is $\kappa(x)$.
Part (2) follows from part (1).
Part (3) follows from part (5) and the fact that
$\Ext^i(K, \mathcal{O}_x) =
H^i(X, K^\vee \otimes_{\mathcal{O}_X}^\mathbf{L} \mathcal{O}_x)$ by
Cohomology, Lemma \ref{cohomology-lemma-dual-perfect-complex}.
Part (4) follows from part (3) and the fact that $K \cong (K^\vee)^\vee$
by the lemma just cited.
\end{proof}

\begin{lemma}
\label{lemma-hom-into-point-sheaf}
Let $X$ be a Noetherian scheme. Let $x \in X$ be a closed point and
denote $\mathcal{O}_x$ the skyscraper sheaf at $x$ with value $\kappa(x)$.
Let $K$ in $D^b_{\textit{Coh}}(\mathcal{O}_X)$. Let $b \in \mathbf{Z}$.
The following are equivalent
\begin{enumerate}
\item $H^i(K)_x = 0$ for all $i > b$ and
\item $\Hom_X(K, \mathcal{O}_x[-i]) = 0$ for all $i > b$.
\end{enumerate}
\end{lemma}

\begin{proof}
Consider the complex $K_x$ in $D^b_{\textit{Coh}}(\mathcal{O}_{X, x})$.
There exist an integer $b_x \in \mathbf{Z}$ such that $K_x$
can be represented by a bounded above complex
$$
\ldots \to
\mathcal{O}_{X, x}^{\oplus n_{b_x - 2}} \to
\mathcal{O}_{X, x}^{\oplus n_{b_x - 1}} \to
\mathcal{O}_{X, x}^{\oplus n_{b_x}} \to 0 \to \ldots
$$
with $\mathcal{O}_{X, x}^{\oplus n_i}$ sitting in degree $i$
where all the transition maps are given by matrices whose
coefficients are in $\mathfrak m_x$. See
More on Algebra, Lemma
\ref{more-algebra-lemma-lift-pseudo-coherent-from-residue-field}.
The result follows easily from this (and the equivalent
conditions hold if and only if $b \geq b_x$).
\end{proof}

\begin{lemma}
\label{lemma-get-fully-faithful-geometric}
Let $k$ be a field. Let $X$ and $Y$ be proper schemes over $k$.
Assume $X$ is regular. Then a $k$-linear exact functor
$F : D_{perf}(\mathcal{O}_X) \to D_{perf}(\mathcal{O}_Y)$
is fully faithful if and only if
for any closed points $x, x' \in X$ the maps
$$
F : \Ext^i_X(\mathcal{O}_x, \mathcal{O}_{x'})
\longrightarrow
\Ext^i_Y(F(\mathcal{O}_x), F(\mathcal{O}_{x'}))
$$
are isomorphisms for all $i \in \mathbf{Z}$.
Here $\mathcal{O}_x$ is the skyscraper sheaf at $x$ with value $\kappa(x)$.
\end{lemma}

\begin{proof}
By Lemma \ref{lemma-always-right-adjoints} the functor $F$
has both a left and a right adjoint. Thus we may apply the criterion
of Lemma \ref{lemma-get-fully-faithful}
because assumptions (2) and (3) of that lemma
follow from Lemma \ref{lemma-orthogonal-point-sheaf}.
\end{proof}

\begin{lemma}
\label{lemma-noah-pre}
\begin{reference}
Email from Noah Olander of Jun 9, 2020
\end{reference}
Let $k$ be a field. Let $X$ be a proper scheme over $k$ which is regular.
Let $F : D_{perf}(\mathcal{O}_X) \to D_{perf}(\mathcal{O}_X)$
be a $k$-linear exact functor. Assume for every coherent
$\mathcal{O}_X$-module $\mathcal{F}$ with $\dim(\text{Supp}(\mathcal{F})) = 0$
there is an isomorphism $\mathcal{F} \cong F(\mathcal{F})$.
Then $F$ is fully faithful.
\end{lemma}

\begin{proof}
By Lemma \ref{lemma-get-fully-faithful-geometric} it suffices to show
that the maps
$$
F : \Ext^i_X(\mathcal{O}_x, \mathcal{O}_{x'})
\longrightarrow
\Ext^i_X(F(\mathcal{O}_x), F(\mathcal{O}_{x'}))
$$
are isomorphisms for all $i \in \mathbf{Z}$ and all closed points
$x, x' \in X$. By assumption, the source and the target are isomorphic.
If $x \not = x'$, then both sides are zero and the result is true.
If $x = x'$, then it suffices to prove that the map is either injective
or surjective. For $i < 0$ both sides are zero and the result is true.
For $i = 0$ any nonzero map $\alpha : \mathcal{O}_x \to \mathcal{O}_x$ of
$\mathcal{O}_X$-modules is an isomorphism. Hence $F(\alpha)$ is an
isomorphism too and so $F(\alpha)$ is nonzero. Thus the result for $i = 0$.
For $i = 1$ a nonzero element $\xi$ in $\Ext^1(\mathcal{O}_x, \mathcal{O}_x)$
corresponds to a nonsplit short exact sequence
$$
0 \to \mathcal{O}_x \to \mathcal{F} \to \mathcal{O}_x \to 0
$$
Since $F(\mathcal{F}) \cong \mathcal{F}$ we see that $F(\mathcal{F})$
is a nonsplit extension of $\mathcal{O}_x$ by $\mathcal{O}_x$ as well.
Since $\mathcal{O}_x \cong F(\mathcal{O}_x)$ is a simple
$\mathcal{O}_X$-module and $\mathcal{F} \cong F(\mathcal{F})$ has
length $2$, we see that in the distinguished triangle
$$
F(\mathcal{O}_x) \to F(\mathcal{F}) \to F(\mathcal{O}_x)
\xrightarrow{F(\xi)} F(\mathcal{O}_x)[1]
$$
the first two arrows must form a short exact sequence which must be
isomorphic to the above short exact sequence and hence is nonsplit.
It follows that $F(\xi)$ is nonzero and we conclude for $i = 1$.
For $i > 1$ composition of ext classes defines a surjection
$$
\Ext^1(F(\mathcal{O}_x), F(\mathcal{O}_x)) \otimes \ldots \otimes
\Ext^1(F(\mathcal{O}_x), F(\mathcal{O}_x))
\longrightarrow
\Ext^i(F(\mathcal{O}_x), F(\mathcal{O}_x))
$$
See Duality for Schemes, Lemma \ref{duality-lemma-regular-ideal-ext}.
Hence surjectivity in degree $1$ implies surjectivity for $i > 0$.
This finishes the proof.
\end{proof}










\section{Special functors}
\label{section-special-functors}

\noindent
In this section we prove some results on functors of a special type
that we will use later in this chapter.

\begin{definition}
\label{definition-siblings-geometric}
Let $k$ be a field. Let $X$, $Y$ be finite type schemes over $k$.
Recall that
$D^b_{\textit{Coh}}(\mathcal{O}_X) = D^b(\textit{Coh}(\mathcal{O}_X))$
by Derived Categories of Schemes, Proposition \ref{perfect-proposition-DCoh}.
We say two $k$-linear exact functors
$$
F, F' :
D^b_{\textit{Coh}}(\mathcal{O}_X) = D^b(\textit{Coh}(\mathcal{O}_X))
\longrightarrow
D^b_{\textit{Coh}}(\mathcal{O}_Y)
$$
are {\it siblings}, or we say $F'$ is a {\it sibling} of $F$ if $F$ and $F'$
are siblings in the sense of Definition \ref{definition-siblings}
with abelian category being $\textit{Coh}(\mathcal{O}_X)$.
If $X$ is regular then
$D_{perf}(\mathcal{O}_X) = D^b_{\textit{Coh}}(\mathcal{O}_X)$ by
Derived Categories of Schemes, Lemma \ref{perfect-lemma-perfect-on-noetherian}
and we use the same terminology for $k$-linear exact functors
$F, F' : D_{perf}(\mathcal{O}_X) \to D_{perf}(\mathcal{O}_Y)$.
\end{definition}

\begin{lemma}
\label{lemma-exact-functor-preserving-Coh}
Let $k$ be a field. Let $X$, $Y$ be finite type schemes over $k$ with
$X$ separated. Let
$F : D^b_{\textit{Coh}}(\mathcal{O}_X) \to D^b_{\textit{Coh}}(\mathcal{O}_Y)$
be a $k$-linear exact functor sending
$\textit{Coh}(\mathcal{O}_X) \subset D^b_{\textit{Coh}}(\mathcal{O}_X)$
into
$\textit{Coh}(\mathcal{O}_Y) \subset D^b_{\textit{Coh}}(\mathcal{O}_Y)$.
Then there exists a Fourier-Mukai functor
$F' : D^b_{\textit{Coh}}(\mathcal{O}_X) \to D^b_{\textit{Coh}}(\mathcal{O}_Y)$
whose kernel is a coherent $\mathcal{O}_{X \times Y}$-module $\mathcal{K}$
flat over $X$ and with support finite over $Y$ which is a sibling of $F$.
\end{lemma}

\begin{proof}
Denote $H : \textit{Coh}(\mathcal{O}_X) \to \textit{Coh}(\mathcal{O}_Y)$
the restriction of $F$. Since $F$ is an exact functor of triangulated
categories, we see that $H$ is an exact functor of abelian categories.
Of course $H$ is $k$-linear as $F$ is. By
Functors and Morphisms, Lemma \ref{functors-lemma-functor-coherent-over-field}
we obtain a coherent $\mathcal{O}_{X \times Y}$-module
$\mathcal{K}$ which is flat over $X$ and has support finite over $Y$.
Let $F'$ be the Fourier-Mukai functor defined using $\mathcal{K}$
so that $F'$ restricts to $H$ on $ \textit{Coh}(\mathcal{O}_X)$.
The functor $F'$ sends $D^b_{\textit{Coh}}(\mathcal{O}_X)$
into $D^b_{\textit{Coh}}(\mathcal{O}_Y)$ by
Lemma \ref{lemma-fourier-mukai-Coh}.
Observe that $F$ and $F'$ satisfy the first and second
condition of Lemma \ref{lemma-sibling-fully-faithful} and hence are siblings.
\end{proof}

\begin{remark}
\label{remark-difficult}
If $F, F' : D^b_{\textit{Coh}}(\mathcal{O}_X) \to \mathcal{D}$ are siblings, $F$
is fully faithful, and $X$ is reduced and projective over $k$ then
$F \cong F'$; this follows from
Proposition \ref{proposition-siblings-isomorphic} via the argument
given in the proof of Theorem \ref{theorem-fully-faithful}.
However, in general we do not know whether siblings are isomorphic.
Even in the situation of Lemma \ref{lemma-exact-functor-preserving-Coh}
it seems difficult to prove that the siblings $F$ and $F'$
are isomorphic functors. If $X$ is smooth and proper over $k$
and $F$ is fully faithful, then $F \cong F'$ as is shown in
\cite{Noah}.
If you have a proof or a counter example in more general situations,
please email
\href{mailto:stacks.project@gmail.com}{stacks.project@gmail.com}.
\end{remark}

\begin{lemma}
\label{lemma-two-functors-pre}
Let $k$ be a field. Let $X$, $Y$ be proper schemes over $k$. Assume
$X$ is regular. Let
$F, G : D_{perf}(\mathcal{O}_X) \to D_{perf}(\mathcal{O}_Y)$
be $k$-linear exact functors such that
\begin{enumerate}
\item $F(\mathcal{F}) \cong G(\mathcal{F})$ for any coherent
$\mathcal{O}_X$-module $\mathcal{F}$ with $\dim(\text{Supp}(\mathcal{F})) = 0$,
\item $F$ is fully faithful.
\end{enumerate}
Then the essential image of $G$ is contained in the essential
image of $F$.
\end{lemma}

\begin{proof}
Recall that $F$ and $G$ have both adjoints, see
Lemma \ref{lemma-always-right-adjoints}. In particular
the essential image $\mathcal{A} \subset D_{perf}(\mathcal{O}_Y)$ of $F$
satisfies the equivalent conditions of
Derived Categories, Lemma \ref{derived-lemma-right-adjoint}.
We claim that $G$ factors through $\mathcal{A}$.
Since $\mathcal{A} = {}^\perp(\mathcal{A}^\perp)$ by
Derived Categories, Lemma \ref{derived-lemma-right-adjoint}
it suffices to show that $\Hom_Y(G(M), N) = 0$ for
all $M$ in $D_{perf}(\mathcal{O}_X)$ and $N \in \mathcal{A}^\perp$.
We have
$$
\Hom_Y(G(M), N) = \Hom_X(M, G_r(N))
$$
where $G_r$ is the right adjoint to $G$. Thus it suffices to prove
that $G_r(N) = 0$. Since
$G(\mathcal{F}) \cong F(\mathcal{F})$ for $\mathcal{F}$ as in (1)
we see that
$$
\Hom_X(\mathcal{F}, G_r(N)) =
\Hom_Y(G(\mathcal{F}), N) =
\Hom_Y(F(\mathcal{F}), N) = 0
$$
as $N$ is in the right orthogonal to the essential image $\mathcal{A}$ of $F$.
Of course, the same vanishing holds for $\Hom_X(\mathcal{F}, G_r(N)[i])$
for any $i \in \mathbf{Z}$. Thus $G_r(N) = 0$ by
Lemma \ref{lemma-orthogonal-point-sheaf} and we win.
\end{proof}

\begin{lemma}
\label{lemma-noah}
\begin{reference}
Email from Noah Olander of Jun 8, 2020
\end{reference}
Let $k$ be a field. Let $X$ be a proper scheme over $k$ which is regular.
Let $F : D_{perf}(\mathcal{O}_X) \to D_{perf}(\mathcal{O}_X)$
be a $k$-linear exact functor. Assume for every coherent
$\mathcal{O}_X$-module $\mathcal{F}$ with $\dim(\text{Supp}(\mathcal{F})) = 0$
there is an isomorphism $\mathcal{F} \cong F(\mathcal{F})$.
Then there exists an automorphism $f : X \to X$ over $k$
which induces the identity on the
underlying topological space\footnote{This often forces $f$
to be the identity, see Varieties, Lemma \ref{varieties-lemma-automorphism}.}
and an invertible $\mathcal{O}_X$-module $\mathcal{L}$
such that $F$ and $F'(M) = f^*M \otimes_{\mathcal{O}_X}^\mathbf{L} \mathcal{L}$
are siblings.
\end{lemma}

\begin{proof}
By Lemma \ref{lemma-noah-pre} the functor $F$ is fully faithful.
By Lemma \ref{lemma-two-functors-pre} the essential image of
the identity functor is contained in the essential image of $F$, i.e.,
we see that $F$ is essentially surjective. Thus $F$ is an equivalence.
Observe that the quasi-inverse $F^{-1}$ satisfies the same assumptions
as $F$.

\medskip\noindent
Let $M \in D_{perf}(\mathcal{O}_X)$ and say $H^i(M) = 0$ for $i > b$.
Since $F$ is fully faithful, we see that
$$
\Hom_X(M, \mathcal{O}_x[-i]) =
\Hom_X(F(M), F(\mathcal{O}_x)[-i]) \cong
\Hom_X(F(M), \mathcal{O}_x[-i])
$$
for any $i \in \mathbf{Z}$ for any closed point $x$ of $X$.
Thus by Lemma \ref{lemma-hom-into-point-sheaf} we see that $F(M)$
has vanishing cohomology sheaves in degrees $> b$.

\medskip\noindent
Let $\mathcal{F}$ be a coherent $\mathcal{O}_X$-module. By
the above $F(\mathcal{F})$ has nonzero cohomology sheaves
only in degrees $\leq 0$.
Set $\mathcal{G} = H^0(F(\mathcal{F}))$. Choose a distinguished
triangle
$$
K \to F(\mathcal{F}) \to \mathcal{G} \to K[1]
$$
Then $K$ has nonvanishing cohomology sheaves only in
degrees $\leq -1$.
Applying $F^{-1}$ we obtain a distinguished triangle
$$
F^{-1}(K) \to \mathcal{F} \to F^{-1}(\mathcal{G}) \to F^{-1}(K')[1]
$$
Since $F^{-1}(K)$ has nonvanishing cohomology sheaves only
in degrees $\leq -1$ (by the previous paragraph applied to $F^{-1}$)
we see that the arrow $F^{-1}(K) \to \mathcal{F}$ is zero
(Derived Categories, Lemma \ref{derived-lemma-negative-exts}).
Hence $K \to F(\mathcal{F})$ is zero, which implies
that $F(\mathcal{F}) = \mathcal{G}$ by our choice of the
first distinguished triangle.

\medskip\noindent
From the preceding paragraph, we deduce that $F$ preserves
$\textit{Coh}(\mathcal{O}_X)$ and indeed defines an equivalence
$H : \textit{Coh}(\mathcal{O}_X) \to \textit{Coh}(\mathcal{O}_X)$.
By Functors and Morphisms, Lemma
\ref{functors-lemma-equivalence-coherent-over-field}
we get an automorphism $f : X \to X$ over $k$
and an invertible $\mathcal{O}_X$-module $\mathcal{L}$
such that $H(\mathcal{F}) = f^*\mathcal{F} \otimes \mathcal{L}$.
Set $F'(M) = f^*M \otimes_{\mathcal{O}_X}^\mathbf{L} \mathcal{L}$.
Using Lemma \ref{lemma-sibling-fully-faithful}
we see that $F$ and $F'$ are siblings.
To see that $f$ is the identity on the underlying topological
space of $X$, we use that $F(\mathcal{O}_x) \cong \mathcal{O}_x$
and that the support of $\mathcal{O}_x$ is $\{x\}$.
This finishes the proof.
\end{proof}

\begin{lemma}
\label{lemma-two-functors}
Let $k$ be a field. Let $X$, $Y$ be proper schemes over $k$.
Assume $X$ regular.
Let $F, G : D_{perf}(\mathcal{O}_X) \to D_{perf}(\mathcal{O}_Y)$
be $k$-linear exact functors such that
\begin{enumerate}
\item $F(\mathcal{F}) \cong G(\mathcal{F})$ for any coherent
$\mathcal{O}_X$-module $\mathcal{F}$ with $\dim(\text{Supp}(\mathcal{F})) = 0$,
\item $F$ is fully faithful, and
\item $G$ is a Fourier-Mukai functor whose kernel is in
$D_{perf}(\mathcal{O}_{X \times Y})$.
\end{enumerate}
Then there exists a Fourier-Mukai functor
$F' : D_{perf}(\mathcal{O}_X) \to D_{perf}(\mathcal{O}_Y)$
whose kernel is in $D_{perf}(\mathcal{O}_{X \times Y})$
such that $F$ and $F'$ are siblings.
\end{lemma}

\begin{proof}
The essential image of $G$ is contained in the essential
image of $F$ by Lemma \ref{lemma-two-functors-pre}.
Consider the functor $H = F^{-1} \circ G$
which makes sense as $F$ is fully faithful.
By Lemma \ref{lemma-noah} we obtain an automorphism $f : X \to X$
and an invertible $\mathcal{O}_X$-module $\mathcal{L}$ such that
the functor $H' : K \mapsto f^*K \otimes \mathcal{L}$
is a sibling of $H$. In particular
$H$ is an auto-equivalence by Lemma \ref{lemma-sibling-faithful}
and $H$ induces an auto-equivalence of
$\textit{Coh}(\mathcal{O}_X)$ (as this is true for its sibling functor $H'$).
Thus the quasi-inverses $H^{-1}$ and $(H')^{-1}$ exist, are siblings
(small detail omitted), and $(H')^{-1}$ sends $M$ to
$(f^{-1})^*(M \otimes_{\mathcal{O}_X}^\mathbf{L} \mathcal{L}^{\otimes -1})$
which is a Fourier-Mukai functor (details omitted).
Then of course $F = G \circ H^{-1}$ is a sibling of
$G \circ (H')^{-1}$. Since compositions of Fourier-Mukai
functors are Fourier-Mukai by
Lemma \ref{lemma-compose-fourier-mukai}
we conclude.
\end{proof}





\section{Fully faithful functors}
\label{section-fully-faithful}

\noindent
Our goal is to prove fully faithful functors between derived categories
are siblings of Fourier-Mukai functors, following
\cite{Orlov-K3} and \cite{Ballard}.

\begin{situation}
\label{situation-fully-faithful}
Here $k$ is a field. We have proper smooth schemes $X$ and $Y$ over $k$.
We have a $k$-linear, exact, fully faithful functor
$F : D_{perf}(\mathcal{O}_X) \to D_{perf}(\mathcal{O}_Y)$.
\end{situation}

\noindent
Before reading on, it makes sense to read at least some of
Derived Categories, Section \ref{derived-section-postnikov}.

\medskip\noindent
Recall that $X$ is regular and hence has the resolution property
(Varieties, Lemma \ref{varieties-lemma-smooth-regular} and
Derived Categories of Schemes, Lemma
\ref{perfect-lemma-regular-resolution-property}). Thus
on $X \times X$ we may choose a resolution
$$
\ldots \to
\mathcal{E}_2 \boxtimes \mathcal{G}_2 \to
\mathcal{E}_1 \boxtimes \mathcal{G}_1 \to
\mathcal{E}_0 \boxtimes \mathcal{G}_0 \to
\mathcal{O}_\Delta \to 0
$$
where each $\mathcal{E}_i$ and $\mathcal{G}_i$ is a finite locally
free $\mathcal{O}_X$-module, see Lemma \ref{lemma-diagonal-resolution}.
Using the complex
\begin{equation}
\label{equation-original-complex}
\ldots \to
\mathcal{E}_2 \boxtimes \mathcal{G}_2 \to
\mathcal{E}_1 \boxtimes \mathcal{G}_1 \to
\mathcal{E}_0 \boxtimes \mathcal{G}_0
\end{equation}
in $D_{perf}(\mathcal{O}_{X \times X})$ as in
Derived Categories, Example \ref{derived-example-key-postnikov}
if for each $n$ we denote
$$
M_n = (\mathcal{E}_n \boxtimes \mathcal{G}_n \to \ldots \to
\mathcal{E}_0 \boxtimes \mathcal{G}_0)[-n]
$$
we obtain an infinite Postnikov system for the complex
(\ref{equation-original-complex}). This means
the morphisms $M_0 \to M_1[1] \to M_2[2] \to \ldots$ and
$M_n \to \mathcal{E}_n \boxtimes \mathcal{G}_n$ and
$\mathcal{E}_n \boxtimes \mathcal{G}_n \to M_{n - 1}$
satisfy certain conditions documented in
Derived Categories, Definition \ref{derived-definition-postnikov-system}.
Set
$$
\mathcal{F}_n = \Ker(\mathcal{E}_n \boxtimes \mathcal{G}_n \to
\mathcal{E}_{n - 1} \boxtimes \mathcal{G}_{n - 1})
$$
Observe that since $\mathcal{O}_\Delta$ is flat over $X$ via $\text{pr}_1$
the same is true for $\mathcal{F}_n$ for all $n$ (this is a convenient though
not essential observation). We have
$$
H^q(M_n[n]) = \left\{
\begin{matrix}
\mathcal{O}_\Delta & \text{if} & q = 0 \\
\mathcal{F}_n & \text{if} & q = -n \\
0 & \text{if} & q \not = 0, -n
\end{matrix}
\right.
$$
Thus for $n \geq \dim(X \times X)$ we have
$$
M_n[n] \cong \mathcal{O}_\Delta \oplus \mathcal{F}_n[n]
$$
in $D_{perf}(\mathcal{O}_{X \times X})$ by
Lemma \ref{lemma-split-complex-regular}.

\medskip\noindent
We are interested in the complex
\begin{equation}
\label{equation-complex}
\ldots \to
\mathcal{E}_2 \boxtimes F(\mathcal{G}_2) \to
\mathcal{E}_1 \boxtimes F(\mathcal{G}_1) \to
\mathcal{E}_0 \boxtimes F(\mathcal{G}_0)
\end{equation}
in $D_{perf}(\mathcal{O}_{X \times Y})$
as the ``totalization'' of this complex should
give us the kernel of the Fourier-Mukai functor we are trying to construct.
For all $i, j \geq 0$ we have
\begin{align*}
\Ext^q_{X \times Y}(\mathcal{E}_i \boxtimes F(\mathcal{G}_i),
\mathcal{E}_j \boxtimes F(\mathcal{G}_j))
& =
\bigoplus\nolimits_p
\Ext^{q + p}_X(\mathcal{E}_i, \mathcal{E}_j) \otimes_k
\Ext^{-p}_Y(F(\mathcal{G}_i), F(\mathcal{G}_j)) \\
& =
\bigoplus\nolimits_p
\Ext^{q + p}_X(\mathcal{E}_i, \mathcal{E}_j) \otimes_k
\Ext^{-p}_X(\mathcal{G}_i, \mathcal{G}_j)
\end{align*}
The second equality holds because $F$ is
fully faithful and the first by
Derived Categories of Schemes, Lemma \ref{perfect-lemma-kunneth-Ext}.
We find these $\Ext^q$ are zero for $q < 0$.
Hence by
Derived Categories, Lemma \ref{derived-lemma-existence-postnikov-system}
we can build an infinite Postnikov system $K_0, K_1, K_2, \ldots$
in $D_{perf}(\mathcal{O}_{X \times Y})$ for the complex
(\ref{equation-complex}).
Parallel to what happens with $M_0, M_1, M_2, \ldots$ this means we
obtain morphisms
$K_0 \to K_1[1] \to K_2[2] \to \ldots$ and
$K_n \to \mathcal{E}_n \boxtimes F(\mathcal{G}_n)$ and
$\mathcal{E}_n \boxtimes F(\mathcal{G}_n) \to K_{n - 1}$
in $D_{perf}(\mathcal{O}_{X \times Y})$
satisfying certain conditions documented in
Derived Categories, Definition \ref{derived-definition-postnikov-system}.

\medskip\noindent
Let $\mathcal{F}$ be a coherent $\mathcal{O}_X$-module whose support
has a finite number of points, i.e., with $\dim(\text{Supp}(\mathcal{F})) = 0$.
Consider the exact functor of triangulated categories
$$
D_{perf}(\mathcal{O}_{X \times Y})
\longrightarrow
D_{perf}(\mathcal{O}_Y),\quad
N \longmapsto R\text{pr}_{2, *}(\text{pr}_1^*\mathcal{F}
\otimes^\mathbf{L}_{\mathcal{O}_{X \times Y}} N)
$$
It follows that the objects $R\text{pr}_{2, *}(\text{pr}_1^*\mathcal{F}
\otimes^\mathbf{L}_{\mathcal{O}_{X \times Y}} K_i)$
form a Postnikov system for the complex in
$D_{perf}(\mathcal{O}_Y)$ with terms
$$
R\text{pr}_{2, *}(
(\mathcal{F} \otimes \mathcal{E}_i) \boxtimes F(\mathcal{G}_i)) =
\Gamma(X, \mathcal{F} \otimes \mathcal{E}_i) \otimes_k F(\mathcal{G}_i) =
F(\Gamma(X, \mathcal{F} \otimes \mathcal{E}_i) \otimes_k \mathcal{G}_i)
$$
Here we have used that $\mathcal{F} \otimes \mathcal{E}_i$ has
vanishing higher cohomology as its support has dimension $0$.
On the other hand, applying the exact functor
$$
D_{perf}(\mathcal{O}_{X \times X})
\longrightarrow
D_{perf}(\mathcal{O}_Y),\quad
N \longmapsto F(R\text{pr}_{2, *}(\text{pr}_1^*\mathcal{F}
\otimes^\mathbf{L}_{\mathcal{O}_{X \times X}} N))
$$
we find that the objects
$F(R\text{pr}_{2, *}(\text{pr}_1^*\mathcal{F}
\otimes^\mathbf{L}_{\mathcal{O}_{X \times X}} M_n))$
form a second infinite Postnikov system
for the complex in $D_{perf}(\mathcal{O}_Y)$ with terms
$$
F(R\text{pr}_{2, *}(
(\mathcal{F} \otimes \mathcal{E}_i) \boxtimes \mathcal{G}_i)) =
F(\Gamma(X, \mathcal{F} \otimes \mathcal{E}_i) \otimes_k \mathcal{G}_i)
$$
This is the same as before! By uniqueness of Postnikov systems
(Derived Categories, Lemma \ref{derived-lemma-existence-postnikov-system})
which applies because
$$
\Ext^q_Y(
F(\Gamma(X, \mathcal{F} \otimes \mathcal{E}_i) \otimes_k \mathcal{G}_i),
F(\Gamma(X, \mathcal{F} \otimes \mathcal{E}_j) \otimes_k \mathcal{G}_j)) = 0,
\quad q < 0
$$
as $F$ is fully faithful, we find a system of isomorphisms
$$
F(R\text{pr}_{2, *}(\text{pr}_1^*\mathcal{F}
\otimes^\mathbf{L}_{\mathcal{O}_{X \times X}} M_n[n]))
\cong
R\text{pr}_{2, *}(\text{pr}_1^*\mathcal{F}
\otimes^\mathbf{L}_{\mathcal{O}_{X \times Y}} K_n[n])
$$
in $D_{perf}(\mathcal{O}_Y)$ compatible with the morphisms in
$D_{perf}(\mathcal{O}_Y)$ induced by the morphisms
$$
M_{n - 1}[n - 1] \to M_n[n]
\quad\text{and}\quad
K_{n - 1}[n - 1] \to K_n[n]
$$
$$
M_n \to \mathcal{E}_n \boxtimes \mathcal{G}_n
\quad\text{and}\quad
K_n \to \mathcal{E}_n \boxtimes F(\mathcal{G}_n)
$$
$$
\mathcal{E}_n \boxtimes \mathcal{G}_n \to M_{n - 1}
\quad\text{and}\quad
\mathcal{E}_n \boxtimes F(\mathcal{G}_n) \to K_{n - 1}
$$
which are part of the structure of Postnikov systems.
For $n$ sufficiently large we obtain a direct sum decomposition
$$
F(R\text{pr}_{2, *}(\text{pr}_1^*\mathcal{F}
\otimes^\mathbf{L}_{\mathcal{O}_{X \times X}} M_n[n]))
=
F(\mathcal{F}) \oplus
F(R\text{pr}_{2, *}(
\text{pr}_1^*\mathcal{F} \otimes_{\mathcal{O}_{X \times Y}} \mathcal{F}_n
))[n]
$$
corresponding to the direct sum decomposition of $M_n$ constructed above
(we are using the flatness of $\mathcal{F}_n$ over $X$ via $\text{pr}_1$
to write a usual tensor product in the formula above, but this isn't
essential for the argument).
By Lemma \ref{lemma-boundedness} we find there exists an integer $m \geq 0$
such that the first summand in this direct sum decomposition has nonzero
cohomology sheaves only in the interval $[-m, m]$ and the
second summand in this direct sum decomposition has nonzero cohomology
sheaves only in the interval $[-m - n, m + \dim(X) - n]$.
We conclude the system $K_0 \to K_1[1] \to K_2[2] \to \ldots$
in $D_{perf}(\mathcal{O}_{X \times Y})$ satisfies the assumptions of
Lemma \ref{lemma-bounded-fibres} after possibly replacing $m$ by
a larger integer. We conclude we can write
$$
K_n[n] = K \oplus C_n
$$
for $n \gg 0$ compatible with transition maps and with $C_n$
having nonzero cohomology sheaves only in the range $[-m - n, m - n]$.
Denote $G$ the Fourier-Mukai functor corresponding to $K$.
Putting everything together we find
$$
\begin{matrix}
G(\mathcal{F}) \oplus
R\text{pr}_{2, *}(
\text{pr}_1^*\mathcal{F} \otimes_{\mathcal{O}_{X \times Y}}^\mathbf{L} C_n)
\cong \\
R\text{pr}_{2, *}(\text{pr}_1^*\mathcal{F}
\otimes^\mathbf{L}_{\mathcal{O}_{X \times Y}} K_n[n]) \cong \\
F(R\text{pr}_{2, *}(\text{pr}_1^*\mathcal{F}
\otimes^\mathbf{L}_{\mathcal{O}_{X \times X}} M_n[n]))
\cong \\
F(\mathcal{F}) \oplus
F(R\text{pr}_{2, *}(
\text{pr}_1^*\mathcal{F} \otimes_{\mathcal{O}_{X \times Y}} \mathcal{F}_n
))[n]
\end{matrix}
$$
Looking at the degrees that objects live in we conclude that for $n \gg m$
we obtain an isomorphism
$$
F(\mathcal{F}) \cong G(\mathcal{F})
$$
Moreover, recall that this holds for every coherent $\mathcal{F}$ on $X$
whose support has dimension $0$.

\begin{lemma}
\label{lemma-fully-faithful}
Let $k$ be a field. Let $X$ and $Y$ be smooth proper schemes over $k$.
Given a $k$-linear, exact, fully faithful functor
$F : D_{perf}(\mathcal{O}_X) \to D_{perf}(\mathcal{O}_Y)$
there exists a Fourier-Mukai functor
$F' : D_{perf}(\mathcal{O}_X) \to D_{perf}(\mathcal{O}_Y)$ whose kernel
is in $D_{perf}(\mathcal{O}_{X \times Y})$ which is a sibling to $F$.
\end{lemma}

\begin{proof}
Apply Lemma \ref{lemma-two-functors} to $F$ and the functor
$G$ constructed above.
\end{proof}

\noindent
The following theorem is also true without assuming $X$ is projective,
see \cite{Noah}.

\begin{theorem}[Orlov]
\label{theorem-fully-faithful}
\begin{reference}
\cite[Theorem 2.2]{Orlov-K3}; this is shown in \cite{Noah}
without the assumption that $X$ be projective
\end{reference}
Let $k$ be a field. Let $X$ and $Y$ be smooth proper schemes over $k$
with $X$ projective over $k$. Any $k$-linear fully faithful exact 
functor $F : D_{perf}(\mathcal{O}_X) \to D_{perf}(\mathcal{O}_Y)$
is a Fourier-Mukai functor for some kernel in
$D_{perf}(\mathcal{O}_{X \times Y})$.
\end{theorem}

\begin{proof}
Let $F'$ be the Fourier-Mukai functor which is a sibling of $F$
as in Lemma \ref{lemma-fully-faithful}.
By Proposition \ref{proposition-siblings-isomorphic} we have $F \cong F'$
provided we can show that $\textit{Coh}(\mathcal{O}_X)$ has enough
negative objects. However, if $X = \Spec(k)$ for example, then
this isn't true. Thus we first decompose $X = \coprod X_i$
into its connected (and irreducible) components and we
argue that it suffices to prove the result for each of the
(fully faithful) composition functors
$$
F_i :
D_{perf}(\mathcal{O}_{X_i}) \to
D_{perf}(\mathcal{O}_X) \to
D_{perf}(\mathcal{O}_Y)
$$
Details omitted. Thus we may assume $X$ is irreducible.

\medskip\noindent
The case $\dim(X) = 0$. Here $X$ is the spectrum of a finite (separable)
extension $k'/k$ and hence $D_{perf}(\mathcal{O}_X)$
is equivalent to the category
of graded $k'$-vector spaces such that $\mathcal{O}_X$ corresponds to the
trivial $1$-dimensional vector space in degree $0$.
It is straightforward to see that any two
siblings $F, F' : D_{perf}(\mathcal{O}_X) \to D_{perf}(\mathcal{O}_Y)$
are isomorphic. Namely, we are given an isomorphism
$F(\mathcal{O}_X) \cong F'(\mathcal{O}_X)$
compatible the action of the $k$-algebra
$k' = \text{End}_{D_{perf}(\mathcal{O}_X)}(\mathcal{O}_X)$
which extends canonically to an isomorphism on any graded $k'$-vector space.

\medskip\noindent
The case $\dim(X) > 0$. Here $X$ is a projective smooth
variety of dimension $> 1$. Let $\mathcal{F}$ be a coherent
$\mathcal{O}_X$-module. We have to show there exists a
coherent module $\mathcal{N}$ such that
\begin{enumerate}
\item there is a surjection $\mathcal{N} \to \mathcal{F}$ and
\item $\Hom(\mathcal{F}, \mathcal{N}) = 0$.
\end{enumerate}
Choose an ample invertible $\mathcal{O}_X$-module $\mathcal{L}$.
We claim that $\mathcal{N} = (\mathcal{L}^{\otimes n})^{\oplus r}$
will work for $n \ll 0$ and $r$ large enough.
Condition (1) follows from
Properties, Proposition \ref{properties-proposition-characterize-ample}.
Finally, we have
$$
\Hom(\mathcal{F}, \mathcal{L}^{\otimes n}) =
H^0(X, \SheafHom(\mathcal{F}, \mathcal{L}^{\otimes n})) =
H^0(X, \SheafHom(\mathcal{F}, \mathcal{O}_X) \otimes \mathcal{L}^{\otimes n})
$$
Since the dual $\SheafHom(\mathcal{F}, \mathcal{O}_X)$ is torsion free, this
vanishes for $n \ll 0$ by Varieties, Lemma 
\ref{varieties-lemma-vanishin-h0-negative}. This finishes the proof.
\end{proof}

\begin{proposition}
\label{proposition-equivalence}
Let $k$ be a field. Let $X$ and $Y$ be smooth proper schemes over $k$.
If $F : D_{perf}(\mathcal{O}_X) \to D_{perf}(\mathcal{O}_Y)$
is a $k$-linear exact equivalence of triangulated categories then
there exists a Fourier-Mukai functor
$F' : D_{perf}(\mathcal{O}_X) \to D_{perf}(\mathcal{O}_Y)$ whose
kernel is in $D_{perf}(\mathcal{O}_{X \times Y})$
which is an equivalence and a sibling of $F$.
\end{proposition}

\begin{proof}
The functor $F'$ of Lemma \ref{lemma-fully-faithful}
is an equivalence by Lemma \ref{lemma-sibling-faithful}.
\end{proof}

\begin{lemma}
\label{lemma-uniqueness}
Let $k$ be a field. Let $X$ be a smooth proper scheme over $k$.
Let $K \in D_{perf}(\mathcal{O}_{X \times X})$. If the Fourier-Mukai
functor $\Phi_K : D_{perf}(\mathcal{O}_X) \to D_{perf}(\mathcal{O}_X)$
is isomorphic to the identity functor, then
$K \cong \Delta_*\mathcal{O}_X$ in $_{perf}(\mathcal{O}_{X \times X})$.
\end{lemma}

\begin{proof}
Let $i$ be the minimal integer such that the cohomology sheaf $H^i(K)$ is
nonzero. Let $\mathcal{E}$ and $\mathcal{G}$ be finite locally free
$\mathcal{O}_X$-modules. Then
\begin{align*}
H^i(X \times X, K \otimes_{\mathcal{O}_{X \times X}}^\mathbf{L}
(\mathcal{E} \boxtimes \mathcal{G}))
& =
H^i(X, R\text{pr}_{2, *}(K \otimes_{\mathcal{O}_{X \times X}}^\mathbf{L}
(\mathcal{E} \boxtimes \mathcal{G}))) \\
& =
H^i(X, \Phi_K(\mathcal{E}) \otimes_{\mathcal{O}_X}^\mathbf{L} \mathcal{G}) \\
& \cong
H^i(X, \mathcal{E} \otimes \mathcal{G})
\end{align*}
which is zero if $i < 0$. On the other hand, we can choose
$\mathcal{E}$ and $\mathcal{G}$ such that there is a surjection
$\mathcal{E}^\vee \boxtimes \mathcal{G}^\vee \to H^i(K)$
by Lemma \ref{lemma-on-product}.
In this case the left hand side of the equalities is nonzero.
Hence we conclude that $H^i(K) = 0$ for $i < 0$.

\medskip\noindent
Let $i$ be the maximal integer such that $H^i(K)$ is nonzero.
The same argument with $\mathcal{E}$ and $\mathcal{G}$
support of dimension $0$ shows that $i \leq 0$.
Hence we conclude that $K$ is given by a single coherent
$\mathcal{O}_{X \times X}$-module $\mathcal{K}$ sitting in degree $0$.

\medskip\noindent
Since $R\text{pr}_{2, *}(\text{pr}_1^*\mathcal{F} \otimes \mathcal{K})$
is $\mathcal{F}$, by taking $\mathcal{F}$ supported at closed points
we see that the support of $\mathcal{K}$ is finite over $X$ via
$\text{pr}_2$. Since $R\text{pr}_{2, *}(\mathcal{K}) \cong \mathcal{O}_X$
we conclude by Functors and Morphisms, Lemma
\ref{functors-lemma-pushforward-invertible-pre}
that $\mathcal{K} = s_*\mathcal{O}_X$ for some section $s : X \to X \times X$
of the second projection. Then $\Phi_K(M) = f^*M$ where
$f = \text{pr}_1 \circ s$ and this can happen only if $s$
is the diagonal morphism as desired.
\end{proof}








\section{A category of Fourier-Mukai kernels}
\label{section-category-Fourier-Mukai-kernels}

\noindent
Let $S$ be a scheme. We claim there is a category
with
\begin{enumerate}
\item Objects are proper smooth schemes over $S$.
\item Morphisms from $X$ to $Y$ are isomorphism classes
of objects of $D_{perf}(\mathcal{O}_{X \times_S Y})$.
\item Composition of the isomorphism class of
$K \in D_{perf}(\mathcal{O}_{X \times_S Y})$
and the isomorphism class of $K'$ in $D_{perf}(\mathcal{O}_{Y \times_S Z})$
is the isomorphism class of
$$
R\text{pr}_{13, *}(
L\text{pr}_{12}^*K
\otimes_{\mathcal{O}_{X \times_S Y \times_S Z}}^\mathbf{L}
L\text{pr}_{23}^*K')
$$
which is in $D_{perf}(\mathcal{O}_{X \times_S Z})$ by
Derived Categories of Schemes, Lemma
\ref{perfect-lemma-flat-proper-perfect-direct-image-general}.
\item The identity morphism from $X$ to $X$ is the
isomorphism class of $\Delta_{X/S, *}\mathcal{O}_X$
which is in $D_{perf}(\mathcal{O}_{X \times_S X})$
by More on Morphisms, Lemma
\ref{more-morphisms-lemma-perfect-closed-immersion-perfect-direct-image}
and the fact that $\Delta_{X/S}$ is a perfect morphism by
Divisors, Lemma
\ref{divisors-lemma-immersion-smooth-into-smooth-regular-immersion} and
More on Morphisms, Lemma \ref{more-morphisms-lemma-regular-immersion-perfect}.
\end{enumerate}
Let us check that associativity of composition
of morphisms holds; we omit verifying that the identity
morphisms are indeed identities. To see this suppose we have
$X, Y, Z, W$ and
$c \in D_{perf}(\mathcal{O}_{X \times_S Y})$,
$c' \in D_{perf}(\mathcal{O}_{Y \times_S Z})$, and
$c'' \in D_{perf}(\mathcal{O}_{Z \times_S W})$. Then we have
\begin{align*}
c'' \circ (c' \circ c)
& \cong
\text{pr}^{134}_{14, *}(
\text{pr}^{134, *}_{13}
\text{pr}^{123}_{13, *}(\text{pr}^{123, *}_{12}c \otimes
\text{pr}^{123, *}_{23}c')
\otimes \text{pr}^{134, *}_{34}c'') \\
& \cong
\text{pr}^{134}_{14, *}(
\text{pr}^{1234}_{134, *}
\text{pr}^{1234, *}_{123}(\text{pr}^{123, *}_{12}c \otimes
\text{pr}^{123, *}_{23}c')
\otimes \text{pr}^{134, *}_{34}c'') \\
& \cong
\text{pr}^{134}_{14, *}(
\text{pr}^{1234}_{134, *}
(\text{pr}^{1234, *}_{12}c \otimes
\text{pr}^{1234, *}_{23}c')
\otimes \text{pr}^{134, *}_{34}c'') \\
& \cong
\text{pr}^{134}_{14, *}
\text{pr}^{1234}_{134, *}
((\text{pr}^{1234, *}_{12}c \otimes
\text{pr}^{1234, *}_{23}c')
\otimes \text{pr}^{1234, *}_{34}c'') \\
& \cong
\text{pr}^{1234}_{14, *}(
(\text{pr}^{1234, *}_{12}c \otimes
\text{pr}^{1234, *}_{23}c') \otimes
\text{pr}^{1234, *}_{34}c'')
\end{align*}
Here we use the notation
$$
p^{1234}_{134} : X \times_S Y \times_S Z \times_S W
\to X \times_S Z \times_S W
\quad\text{and}\quad
p^{134}_{14} : X \times_S Z \times_S W \to X \times_S W
$$
the projections and similarly for other indices.
We also write $\text{pr}_*$ instead of $R\text{pr}_*$ and
$\text{pr}^*$ instead of $L\text{pr}^*$ and we drop
all super and sub scripts on $\otimes$.
The first equality is the definition of the composition.
The second equality holds because
$\text{pr}^{134, *}_{13} \text{pr}^{123}_{13, *} =
\text{pr}^{1234}_{134, *} \text{pr}^{1234, *}_{123}$
by base change (Derived Categories of Schemes, Lemma
\ref{perfect-lemma-compare-base-change}).
The third equality holds because pullbacks compose
correctly and pass through tensor products, see
Cohomology, Lemmas \ref{cohomology-lemma-derived-pullback-composition} and
\ref{cohomology-lemma-pullback-tensor-product}.
The fourth equality follows from the ``projection formula'' for
$p^{1234}_{134}$, see Derived Categories of Schemes, Lemma
\ref{perfect-lemma-cohomology-base-change}.
The fifth equality is that proper pushforward is compatible
with composition, see
Cohomology, Lemma \ref{cohomology-lemma-derived-pushforward-composition}.
Since tensor product is associative
this concludes the proof of associativity of composition.

\begin{lemma}
\label{lemma-base-change-is-functor}
Let $S' \to S$ be a morphism of schemes.
The rule which sends
\begin{enumerate}
\item a smooth proper scheme $X$ over $S$ to $X' =  S' \times_S X$, and
\item the isomorphism class of an object $K$
of $D_{perf}(\mathcal{O}_{X \times_S Y})$ to the isomorphism class of
$L(X' \times_{S'} Y' \to X \times_S Y)^*K$
in $D_{perf}(\mathcal{O}_{X' \times_{S'} Y'})$
\end{enumerate}
is a functor from the category defined for $S$ to the category
defined for $S'$.
\end{lemma}

\begin{proof}
To see this suppose we have $X, Y, Z$ and
$K \in D_{perf}(\mathcal{O}_{X \times_S Y})$ and
$M \in D_{perf}(\mathcal{O}_{Y \times_S Z})$.
Denote
$K' \in D_{perf}(\mathcal{O}_{X' \times_{S'} Y'})$ and
$M' \in D_{perf}(\mathcal{O}_{Y' \times_{S'} Z'})$
their pullbacks as in the statement of the lemma.
The diagram
$$
\xymatrix{
X' \times_{S'} Y' \times_{S'} Z' \ar[r] \ar[d]_{\text{pr}'_{13}} &
X \times_S Y \times_S Z \ar[d]^{\text{pr}_{13}} \\
X' \times_{S'} Z' \ar[r] &
X \times_S Z
}
$$
is cartesian and $\text{pr}_{13}$ is proper and smooth.
By Derived Categories of Schemes, Lemma
\ref{perfect-lemma-flat-proper-perfect-direct-image-general}
we see that the derived pullback by the lower horizontal
arrow of the composition
$$
R\text{pr}_{13, *}(
L\text{pr}_{12}^*K
\otimes_{\mathcal{O}_{X \times_S Y \times_S Z}}^\mathbf{L}
L\text{pr}_{23}^*M)
$$
indeed is (canonically) isomorphic to
$$
R\text{pr}'_{13, *}(
L(\text{pr}'_{12})^*K'
\otimes_{\mathcal{O}_{X' \times_{S'} Y' \times_{S'} Z'}}^\mathbf{L}
L(\text{pr}'_{23})^*M')
$$
as desired. Some details omitted.
\end{proof}





\section{Relative equivalences}
\label{section-relative-equivalences}

\noindent
In this section we prove some lemmas about the following concept.

\begin{definition}
\label{definition-relative-equivalence-kernel}
Let $S$ be a scheme. Let $X \to S$ and $Y \to S$ be smooth proper morphisms.
An object $K \in D_{perf}(\mathcal{O}_{X \times_S Y})$
is said to be {\it the Fourier-Mukai kernel of a relative equivalence
from $X$ to $Y$ over $S$}
if there exist an object $K' \in D_{perf}(\mathcal{O}_{X \times_S Y})$
such that
$$
\Delta_{X/S, *}\mathcal{O}_X \cong
R\text{pr}_{13, *}(L\text{pr}_{12}^*K
\otimes_{\mathcal{O}_{X \times_S Y \times_S X}}^\mathbf{L}
L\text{pr}_{23}^*K')
$$
in $D(\mathcal{O}_{X \times_S X})$ and
$$
\Delta_{Y/S, *}\mathcal{O}_Y \cong
R\text{pr}_{13, *}(L\text{pr}_{12}^*K'
\otimes_{\mathcal{O}_{Y \times_S X \times_S Y}}^\mathbf{L}
L\text{pr}_{23}^*K)
$$
in $D(\mathcal{O}_{Y \times_S Y})$. In other words, the isomorphism class
of $K$ defines an invertible arrow in the category defined in
Section \ref{section-category-Fourier-Mukai-kernels}.
\end{definition}

\noindent
The language is intentionally cumbersome.

\begin{lemma}
\label{lemma-equivalences-rek}
With notation as in Definition \ref{definition-relative-equivalence-kernel}
let $K$ be the Fourier-Mukai kernel of a relative equivalence from $X$
to $Y$ over $S$. Then the corresponding Fourier-Mukai functors
$\Phi_K : D_\QCoh(\mathcal{O}_X) \to D_\QCoh(\mathcal{O}_Y)$
(Lemma \ref{lemma-fourier-Mukai-QCoh})
and $\Phi_K : D_{perf}(\mathcal{O}_X) \to D_{perf}(\mathcal{O}_Y)$
(Lemma \ref{lemma-fourier-mukai})
are equivalences.
\end{lemma}

\begin{proof}
Immediate from Lemma \ref{lemma-compose-fourier-mukai} and
Example \ref{example-diagonal-fourier-mukai}.
\end{proof}

\begin{lemma}
\label{lemma-base-change-rek}
With notation as in Definition \ref{definition-relative-equivalence-kernel}
let $K$ be the Fourier-Mukai kernel of a relative equivalence from $X$
to $Y$ over $S$. Let $S_1 \to S$ be a morphism of schemes. Let
$X_1 = S_1 \times_S X$ and $Y_1 = S_1 \times_S Y$. Then the pullback
$K_1 = L(X_1 \times_{S_1} Y_1 \to X \times_S Y)^*K$ is
the Fourier-Mukai kernel of a relative equivalence from $X_1$
to $Y_1$ over $S_1$.
\end{lemma}

\begin{proof}
Let $K' \in D_{perf}(\mathcal{O}_{Y \times_S X})$ be the object assumed to
exist in Definition \ref{definition-relative-equivalence-kernel}.
Denote $K'_1$ the pullback of $K'$ by
$Y_1 \times_{S_1} X_1 \to Y \times_S X$.
Then it suffices to prove that we have
$$
\Delta_{X_1/S_1, *}\mathcal{O}_X \cong
R\text{pr}_{13, *}(L\text{pr}_{12}^*K_1
\otimes_{\mathcal{O}_{X_1 \times_{S_1} Y_1 \times_{S_1} X_1}}^\mathbf{L}
L\text{pr}_{23}^*K_1')
$$
in $D(\mathcal{O}_{X_1 \times_{S_1} X_1})$ and similarly for the other
condition. Since
$$
\xymatrix{
X_1 \times_{S_1} Y_1 \times_{S_1} X_1 \ar[r] \ar[d]_{\text{pr}_{13}} &
X \times_S Y \times_S X \ar[d]^{\text{pr}_{13}} \\
X_1 \times_{S_1} X_1 \ar[r] &
X \times_S X
}
$$
is cartesian it suffices by Derived Categories of Schemes, Lemma
\ref{perfect-lemma-flat-proper-perfect-direct-image-general}
to prove that
$$
\Delta_{X_1/S_1, *}\mathcal{O}_{X_1}
\cong
L(X_1 \times_{S_1} X_1 \to X \times_S X)^*\Delta_{X/S, *}\mathcal{O}_X 
$$
This in turn will be true if $X$ and $X_1 \times_{S_1} X_1$ are tor
independent over $X \times_S X$, see
Derived Categories of Schemes, Lemma \ref{perfect-lemma-compare-base-change}.
This tor independence can be seen directly but also follows from
the more general More on Morphisms, Lemma
\ref{more-morphisms-lemma-case-of-tor-independence} applied to the square
with corners $X, X, X, S$ and its base change by $S_1 \to S$.
\end{proof}

\begin{lemma}
\label{lemma-descend-rek}
Let $S = \lim_{i \in I} S_i$ be a limit of a directed system of schemes
with affine transition morphisms $g_{i'i} : S_{i'} \to S_i$.
We assume that $S_i$ is quasi-compact and quasi-separated for all $i \in I$.
Let $0 \in I$. Let $X_0 \to S_0$ and $Y_0 \to S_0$ be smooth proper morphisms.
We set $X_i = S_i \times_{S_0} X_0$ for $i \geq 0$
and $X = S \times_{S_0} X_0$ and similarly for $Y_0$. If $K$ is the
Fourier-Mukai kernel of a relative equivalence from $X$ to $Y$ over $S$
then for some $i \geq 0$ there exists a
Fourier-Mukai kernel of a relative equivalence from $X_i$ to $Y_i$ over $S_i$.
\end{lemma}

\begin{proof}
Let $K' \in D_{perf}(\mathcal{O}_{Y \times_S X})$ be the object assumed to
exist in Definition \ref{definition-relative-equivalence-kernel}.
Since $X \times_S Y = \lim X_i \times_{S_i} Y_i$ there exists an
$i$ and objects $K_i$ and $K'_i$ in
$D_{perf}(\mathcal{O}_{Y_i \times_{S_i} X_i})$
whose pullbacks to $Y \times_S X$ give $K$ and $K'$.
See Derived Categories of Schemes, Lemma \ref{perfect-lemma-descend-perfect}.
By Derived Categories of Schemes, Lemma
\ref{perfect-lemma-flat-proper-perfect-direct-image-general}
the object
$$
R\text{pr}_{13, *}(L\text{pr}_{12}^*K_i
\otimes_{\mathcal{O}_{X_i \times_{S_i} Y_i \times_{S_i} X_i}}^\mathbf{L}
L\text{pr}_{23}^*K_i')
$$
is perfect and its pullback to $X \times_S X$ is equal to
$$
R\text{pr}_{13, *}(L\text{pr}_{12}^*K
\otimes_{\mathcal{O}_{X \times_S Y \times_S X}}^\mathbf{L}
L\text{pr}_{23}^*K') \cong \Delta_{X/S, *}\mathcal{O}_X
$$
See proof of Lemma \ref{lemma-base-change-rek}.
On the other hand, since $X_i \to S$ is smooth and separated the
object
$$
\Delta_{i, *}\mathcal{O}_{X_i}
$$
of $D(\mathcal{O}_{X_i \times_{S_i} X_i})$ is also perfect
(by More on Morphisms, Lemmas
\ref{more-morphisms-lemma-smooth-diagonal-perfect} and
\ref{more-morphisms-lemma-perfect-proper-perfect-direct-image}) and
its pullback to $X \times_S X$ is equal to
$$
\Delta_{X/S, *}\mathcal{O}_X
$$
See proof of Lemma \ref{lemma-base-change-rek}. Thus by
Derived Categories of Schemes, Lemma \ref{perfect-lemma-descend-perfect}
after increasing $i$ we may assume that
$$
\Delta_{i, *}\mathcal{O}_{X_i} \cong
R\text{pr}_{13, *}(L\text{pr}_{12}^*K_i
\otimes_{\mathcal{O}_{X_i \times_{S_i} Y_i \times_{S_i} X_i}}^\mathbf{L}
L\text{pr}_{23}^*K_i')
$$
as desired. The same works for the roles of $K$ and $K'$ reversed.
\end{proof}








\section{No deformations}
\label{section-no-deformations}

\noindent
The title of this section refers to Lemma \ref{lemma-no-deformations}

\begin{lemma}
\label{lemma-deform-koszul}
Let $(R, \mathfrak m, \kappa) \to (A, \mathfrak n, \lambda)$
be a flat local ring homorphism of local rings
which is essentially of finite presentation.
Let $\overline{f}_1, \ldots, \overline{f}_r \in \mathfrak n/\mathfrak m A
\subset A/\mathfrak m A$ be a regular sequence. Let $K \in D(A)$. Assume
\begin{enumerate}
\item $K$ is perfect,
\item $K \otimes_A^\mathbf{L} A/\mathfrak m A$ is isomorphic in
$D(A/\mathfrak m A)$ to the
Koszul complex on $\overline{f}_1, \ldots, \overline{f}_r$.
\end{enumerate}
Then $K$ is isomorphic in $D(A)$ to a Koszul complex on a regular sequence
$f_1, \ldots, f_r \in A$ lifting the given elements
$\overline{f}_1, \ldots, \overline{f}_r$. Moreover, $A/(f_1, \ldots, f_r)$
is flat over $R$.
\end{lemma}

\begin{proof}
Let us use chain complexes in the proof of this lemma.
The Koszul complex $K_\bullet(\overline{f}_1, \ldots, \overline{f}_r)$
is defined in More on Algebra, Definition
\ref{more-algebra-definition-koszul-complex}.
By More on Algebra, Lemma \ref{more-algebra-lemma-lift-complex-stably-frees}
we can represent $K$ by a complex
$$
K_\bullet :
A \to A^{\oplus r} \to \ldots \to A^{\oplus r} \to A
$$
whose tensor product with $A/\mathfrak mA$ is equal (!)
to $K_\bullet(\overline{f}_1, \ldots, \overline{f}_r)$.
Denote $f_1, \ldots, f_r \in A$ the components of the
arrow $A^{\oplus r} \to A$. These $f_i$ are lifts of the
$\overline{f}_i$. By Algebra, Lemma
\ref{algebra-lemma-grothendieck-regular-sequence-general}
$f_1, \ldots, f_r$ form a regular sequence in $A$ and $A/(f_1, \ldots, f_r)$
is flat over $R$. Let $J = (f_1, \ldots, f_r) \subset A$.
Consider the diagram
$$
\xymatrix{
K_\bullet \ar[rd] \ar@{..>}[rr]_{\varphi_\bullet} & &
K_\bullet(f_1, \ldots, f_r) \ar[ld] \\
& A/J
}
$$
Since $f_1, \ldots, f_r$ is a regular sequence the south-west arrow
is a quasi-isomorphism (see
More on Algebra, Lemma \ref{more-algebra-lemma-regular-koszul-regular}).
Hence we can find the dotted arrow making the
diagram commute for example by
Algebra, Lemma \ref{algebra-lemma-compare-resolutions}.
Reducing modulo $\mathfrak m$ we obtain a commutative diagram
$$
\xymatrix{
K_\bullet(\overline{f}_1, \ldots, \overline{f}_r)
\ar[rd] \ar[rr]_{\overline{\varphi}_\bullet} & &
K_\bullet(\overline{f}_1, \ldots, \overline{f}_r) \ar[ld] \\
& (A/\mathfrak m A)/(\overline{f}_1, \ldots, \overline{f}_r)
}
$$
by our choice of $K_\bullet$. Thus $\overline{\varphi}$ is an isomorphism
in the derived category $D(A/\mathfrak m A)$. It follows that
$\overline{\varphi} \otimes_{A/\mathfrak m A}^\mathbf{L} \lambda$
is an isomorphism. Since $\overline{f}_i \in \mathfrak n / \mathfrak m A$
we see that
$$
\text{Tor}_i^{A/\mathfrak m A}(
K_\bullet(\overline{f}_1, \ldots, \overline{f}_r), \lambda)
=
K_i(\overline{f}_1, \ldots, \overline{f}_r) \otimes_{A/\mathfrak m A} \lambda
$$
Hence $\varphi_i \bmod \mathfrak n$ is invertible.
Since $A$ is local this means that $\varphi_i$ is an
isomorphism and the proof is complete.
\end{proof}

\begin{lemma}
\label{lemma-limit-arguments}
Let $R \to S$ be a finite type flat ring map of Noetherian rings.
Let $\mathfrak q \subset S$ be a prime ideal lying over
$\mathfrak p \subset R$. Let $K \in D(S)$ be perfect.
Let $f_1, \ldots, f_r \in \mathfrak q S_\mathfrak q$
be a regular sequence such that $S_\mathfrak q/(f_1, \ldots, f_r)$
is flat over $R$ and such that
$K \otimes_S^\mathbf{L} S_\mathfrak q$ is isomorphic to the
Koszul complex on $f_1, \ldots, f_r$. Then there exists a
$g \in S$, $g \not \in \mathfrak q$ such that
\begin{enumerate}
\item $f_1, \ldots, f_r$ are the images of
$f'_1, \ldots, f'_r \in S_g$,
\item $f'_1, \ldots, f'_r$ form a regular sequence in $S_g$,
\item $S_g/(f'_1, \ldots, f'_r)$ is flat over $R$,
\item $K \otimes_S^\mathbf{L} S_g$ is isomorphic to the
Koszul complex on $f_1, \ldots, f_r$.
\end{enumerate}
\end{lemma}

\begin{proof}
We can find $g \in S$, $g \not \in \mathfrak q$ with property (1) by
the definition of localizations. After replacing $g$ by
$gg'$ for some $g' \in S$, $g' \not \in \mathfrak q$
we may assume (2) holds, see
Algebra, Lemma \ref{algebra-lemma-regular-sequence-in-neighbourhood}.
By Algebra, Theorem \ref{algebra-theorem-openness-flatness}
we find that $S_g/(f'_1, \ldots, f'_r)$ is flat over $R$
in an open neighbourhood of $\mathfrak q$.
Hence after once more replacing $g$ by $gg'$ for some
$g' \in S$, $g' \not \in \mathfrak q$ we may assume (3) holds as well.
Finally, we get (4) for a further replacement by
More on Algebra, Lemma \ref{more-algebra-lemma-colimit-perfect-complexes}.
\end{proof}

\noindent
For a generalization of the following lemma, please see
More on Morphisms of Spaces, Lemma
\ref{spaces-more-morphisms-lemma-where-isomorphism}.

\begin{lemma}
\label{lemma-isomorphism-in-neighbourhood}
Let $S$ be a Noetherian scheme. Let $s \in S$.
Let $p : X \to Y$ be a morphism of schemes over $S$.
Assume
\begin{enumerate}
\item $Y \to S$ and $X \to S$ proper,
\item $X$ is flat over $S$,
\item $X_s \to Y_s$ an isomorphism.
\end{enumerate}
Then there exists an open neighbourhood $U \subset S$ of $s$
such that the base change $X_U \to Y_U$ is an isomorphism.
\end{lemma}

\begin{proof}
The morphism $p$ is proper by Morphisms, Lemma
\ref{morphisms-lemma-closed-immersion-proper}.
By Cohomology of Schemes, Lemma
\ref{coherent-lemma-proper-finite-fibre-finite-in-neighbourhood}
there is an open $Y_s \subset V \subset Y$ such that
$p|_{p^{-1}(V)} : p^{-1}(V) \to V$ is finite.
By More on Morphisms, Theorem
\ref{more-morphisms-theorem-criterion-flatness-fibre-Noetherian}
there is an open $X_s \subset U \subset X$ such that
$p|_U : U \to Y$ is flat. After removing the images of
$X \setminus U$ and $Y \setminus V$ (which are closed subsets
not containing $s$) we may assume $p$ is flat and finite.
Then $p$ is open (Morphisms, Lemma \ref{morphisms-lemma-fppf-open})
and $Y_s \subset p(X) \subset Y$ hence after shrinking $S$
we may assume $p$ is surjective.
As $p_s : X_s \to Y_s$ is an isomorphism, the map
$$
p^\sharp : \mathcal{O}_Y \longrightarrow p_*\mathcal{O}_X
$$
of coherent $\mathcal{O}_Y$-modules ($p$ is finite)
becomes an isomorphism after pullback by $i : Y_s \to Y$
(by Cohomology of Schemes, Lemma
\ref{coherent-lemma-affine-base-change} for example).
By Nakayama's lemma, this implies that
$\mathcal{O}_{Y, y} \to (p_*\mathcal{O}_X)_y$ is surjective
for all $y \in Y_s$. Hence there is an open $Y_s \subset V \subset Y$
such that $p^\sharp|_V$ is surjective
(Modules, Lemma \ref{modules-lemma-finite-type-surjective-on-stalk}).
Hence after shrinking $S$ once more we may assume
$p^\sharp$ is surjective which means that $p$ is a closed
immersion (as $p$ is already finite).
Thus now $p$ is a surjective flat closed immersion
of Noetherian schemes and hence an isomorphism, see
Morphisms, Section \ref{morphisms-section-flat-closed-immersions}.
\end{proof}

\begin{lemma}
\label{lemma-no-deformations}
Let $k$ be a field. Let $S$ be a finite type scheme over $k$
with $k$-rational point $s$. Let $Y \to S$ be a smooth proper morphism.
Let $X = Y_s \times S \to S$ be the constant family with fibre
$Y_s$. Let $K$ be the Fourier-Mukai kernel of a relative equivalence
from $X$ to $Y$ over $S$. Assume the restriction
$$
L(Y_s \times_S Y_s \to X \times_S Y)^*K \cong 
\Delta_{Y_s/k, *} \mathcal{O}_{Y_s}
$$
in $D(\mathcal{O}_{Y_s \times Y_s})$. Then there is an open neighbourhood
$s \in U \subset S$ such that $Y|_U$ is isomorphic to $Y_s \times U$ over $U$.
\end{lemma}

\begin{proof}
Denote $i : Y_s \times Y_s = X_s \times Y_s \to X \times_S Y$
the natural closed immersion. (We will write $Y_s$ and not $X_s$
for the fibre of $X$ over $s$ from now on.) Let
$z \in Y_s \times Y_s = (X \times_S Y)_s \subset X \times_S Y$
be a closed point. As indicated we think of $z$ both as a closed point
of $Y_s \times Y_s$ as well as a closed point of $X \times_S Y$.

\medskip\noindent
Case I: $z \not \in \Delta_{Y_s/k}(Y_s)$. Denote $\mathcal{O}_z$
the coherent $\mathcal{O}_{Y_s \times Y_s}$-module supported at $z$
whose value is $\kappa(z)$. Then $i_*\mathcal{O}_z$ is the
coherent $\mathcal{O}_{X \times_S Y}$-module supported at $z$
whose value is $\kappa(z)$. Our assumption means that
$$
K \otimes_{\mathcal{O}_{X \times_S Y}}^\mathbf{L} i_*\mathcal{O}_z =
Li^*K \otimes_{\mathcal{O}_{Y_s \times Y_s}}^\mathbf{L} \mathcal{O}_z = 0
$$
Hence by Lemma \ref{lemma-orthogonal-point-sheaf}
we find an open neighbourhood $U(z) \subset X \times_S Y$ of $z$
such that $K|_{U(z)} = 0$. In this case we set $Z(z) = \emptyset$
as closed subscheme of $U(z)$.

\medskip\noindent
Case II: $z \in \Delta_{Y_s/k}(Y_s)$. Since $Y_s$ is smooth over $k$
we know that $\Delta_{Y_s/k} : Y_s \to Y_s \times Y_s$ is a
regular immersion, see More on Morphisms, Lemma
\ref{more-morphisms-lemma-smooth-diagonal-perfect}.
Choose a regular sequence $\overline{f}_1, \ldots, \overline{f}_r \in
\mathcal{O}_{Y_s \times Y_s, z}$ cutting out the ideal sheaf of
$\Delta_{Y_s/k}(Y_s)$. Since a regular sequence is Koszul-regular
(More on Algebra, Lemma \ref{more-algebra-lemma-regular-koszul-regular})
our assumption means that
$$
K_z \otimes_{\mathcal{O}_{X \times_S Y, z}}^\mathbf{L}
\mathcal{O}_{Y_s \times Y_s, z}
\in D(\mathcal{O}_{Y_s \times Y_s, z})
$$
is represented by the Koszul complex on
$\overline{f}_1, \ldots, \overline{f}_r$ over
$\mathcal{O}_{Y_s \times Y_s, z}$.
By Lemma \ref{lemma-deform-koszul} applied to
$\mathcal{O}_{S, s} \to \mathcal{O}_{X \times_S Y, z}$
we conclude that $K_z \in D(\mathcal{O}_{X \times_S Y, z})$ is
represented by the Koszul complex on a regular sequence
$f_1, \ldots, f_r \in \mathcal{O}_{X \times_S Y, z}$
lifting the regular sequence
$\overline{f}_1, \ldots, \overline{f}_r$
such that moreover $\mathcal{O}_{X \times_S Y}/(f_1, \ldots, f_r)$
is flat over $\mathcal{O}_{S, s}$.
By some limit arguments (Lemma \ref{lemma-limit-arguments})
we conclude that there exists an affine open neighbourhood
$U(z) \subset X \times_S Y$ of $z$ and a closed subscheme
$Z(z) \subset U(z)$ such that
\begin{enumerate}
\item $Z(z) \to U(z)$ is a regular closed immersion,
\item $K|_{U(z)}$ is quasi-isomorphic to $\mathcal{O}_{Z(z)}$,
\item $Z(z) \to S$ is flat,
\item $Z(z)_s = \Delta_{Y_s/k}(Y_s) \cap U(z)_s$
as closed subschemes of $U(z)_s$.
\end{enumerate}

\noindent
By property (2), for $z, z' \in Y_s \times Y_s$, we
find that $Z(z) \cap U(z') = Z(z') \cap U(z)$ as closed subschemes.
Hence we obtain an open neighbourhood
$$
U = \bigcup\nolimits_{z \in Y_s \times Y_s\text{ closed}} U(z)
$$
of $Y_s \times Y_s$ in $X \times_S Y$ and a closed subscheme $Z \subset U$
such that (1) $Z \to U$ is a regular closed immersion,
(2) $Z \to S$ is flat, and (3) $Z_s = \Delta_{Y_s/k}(Y_s)$.
Since $X \times_S Y \to S$ is proper, after replacing $S$
by an open neighbourhood of $s$ we may assume $U = X \times_S Y$.
Since the projections $Z_s \to Y_s$ and $Z_s \to X_s$
are isomorphisms, we conclude that after shrinking $S$
we may assume $Z \to Y$ and $Z \to X$ are isomorphisms, see
Lemma \ref{lemma-isomorphism-in-neighbourhood}.
This finishes the proof.
\end{proof}

\begin{lemma}
\label{lemma-no-deformations-better}
Let $k$ be an algebraically closed field. Let $X$
be a smooth proper scheme over $k$.
Let $f : Y \to S$ be a smooth proper morphism with $S$ of finite type over $k$.
Let $K$ be the Fourier-Mukai kernel of a relative equivalence
from $X \times S$ to $Y$ over $S$. Then $S$ can be covered by
open subschemes $U$ such that there is a $U$-isomorphism
$f^{-1}(U) \cong Y_0 \times U$ for some $Y_0$ proper and smooth over $k$.
\end{lemma}

\begin{proof}
Choose a closed point $s \in S$. Since $k$ is algebraically closed
this is a $k$-rational point. Set $Y_0 = Y_s$. The restriction
$K_0$ of $K$ to $X \times Y_0$ is the Fourier-Mukai kernel of a
relative equivalence from $X$ to $Y_0$ over $\Spec(k)$ by
Lemma \ref{lemma-base-change-rek}. Let $K'_0$ in
$D_{perf}(\mathcal{O}_{Y_0 \times X})$ be the 
object assumed to
exist in Definition \ref{definition-relative-equivalence-kernel}.
Then $K'_0$ is the Fourier-Mukai kernel of a
relative equivalence from $Y_0$ to $X$ over $\Spec(k)$
by the symmetry inherent in
Definition \ref{definition-relative-equivalence-kernel}.
Hence by
Lemma \ref{lemma-base-change-rek}
we see that the pullback
$$
M = (Y_0 \times X \times S \to Y_0 \times X)^*K'_0
$$
on $(Y_0 \times S) \times_S (X \times S) = Y_0 \times X \times S$
is the Fourier-Mukai kernel of a
relative equivalence from $Y_0 \times S$ to $X \times S$ over $S$.
Now consider the kernel
$$
K_{new} =
R\text{pr}_{13, *}(L\text{pr}_{12}^*M
\otimes_{\mathcal{O}_{(Y_0 \times S) \times_S (X \times S)
\times_S Y}}^\mathbf{L}
L\text{pr}_{23}^*K)
$$
on $(Y_0 \times S) \times_S Y$. This is the Fourier-Mukai kernel of a
relative equivalence from $Y_0 \times S$ to $Y$ over $S$ since it is
the composition of two invertible arrows in
the category constructed in
Section \ref{section-category-Fourier-Mukai-kernels}.
Moreover, this composition passes through base change
(Lemma \ref{lemma-base-change-is-functor}).
Hence we see that the pullback of $K_{new}$ to
$((Y_0 \times S) \times_S Y)_s = Y_0 \times Y_0$
is equal to the composition of $K_0$ and $K'_0$
and hence equal to the identity in this category.
In other words, we have
$$
L(Y_0 \times Y_0 \to (Y_0 \times S) \times_S Y)^*K_{new}
\cong
\Delta_{Y_0/k, *}\mathcal{O}_{Y_0}
$$
Thus by Lemma \ref{lemma-no-deformations} we conclude that $Y \to S$
is isomorphic to $Y_0 \times S$ in an open neighbourhood of $s$.
This finishes the proof.
\end{proof}






\section{Countability}
\label{section-countability}

\noindent
In this section we prove some elementary lemmas about countability
of certain sets. Let $\mathcal{C}$ be a category. In this section
we will say that $\mathcal{C}$ is {\it countable} if
\begin{enumerate}
\item for any $X, Y \in \Ob(\mathcal{C})$ the set
$\Mor_\mathcal{C}(X, Y)$ is countable, and
\item the set of isomorphism classes of objects of $\mathcal{C}$
is countable.
\end{enumerate}

\begin{lemma}
\label{lemma-countable-finite-type}
Let $R$ be a countable Noetherian ring. Then the category of schemes of finite
type over $R$ is countable.
\end{lemma}

\begin{proof}
Omitted.
\end{proof}

\begin{lemma}
\label{lemma-countable-abelian}
Let $\mathcal{A}$ be a countable abelian category.
Then $D^b(\mathcal{A})$ is countable.
\end{lemma}

\begin{proof}
It suffices to prove the statement for $D(\mathcal{A})$ as the others
are full subcategories of this one. Since every object in $D(\mathcal{A})$
is a complex of objects of $\mathcal{A}$ it is immediate that the set of
isomorphism classes of objects of $D^b(\mathcal{A})$ is countable.
Moreover, for bounded complexes $A^\bullet$ and $B^\bullet$ of $\mathcal{A}$
it is clear that $\Hom_{K^b(\mathcal{A})}(A^\bullet, B^\bullet)$ is countable.
We have
$$
\Hom_{D^b(\mathcal{A})}(A^\bullet, B^\bullet) =
\colim_{s : (A')^\bullet \to A^\bullet
\text{ qis and }(A')^\bullet\text{ bounded}}
\Hom_{K^b(\mathcal{A})}((A')^\bullet, B^\bullet)
$$
by Derived Categories, Lemma \ref{derived-lemma-bounded-derived}.
Thus this is a countable set as a countable colimit of
\end{proof}

\begin{lemma}
\label{lemma-countable-perfect}
Let $X$ be a scheme of finite type over a countable Noetherian ring.
Then the categories $D_{perf}(\mathcal{O}_X)$ and
$D^b_{\textit{Coh}}(\mathcal{O}_X)$ are countable.
\end{lemma}

\begin{proof}
Observe that $X$ is Noetherian by
Morphisms, Lemma \ref{morphisms-lemma-finite-type-noetherian}.
Hence $D_{perf}(\mathcal{O}_X)$ is a full subcategory of
$D^b_{\textit{Coh}}(\mathcal{O}_X)$ by
Derived Categories of Schemes, Lemma \ref{perfect-lemma-perfect-on-noetherian}.
Thus it suffices to prove
the result for $D^b_{\textit{Coh}}(\mathcal{O}_X)$.
Recall that
$D^b_{\textit{Coh}}(\mathcal{O}_X) = D^b(\textit{Coh}(\mathcal{O}_X))$
by
Derived Categories of Schemes, Proposition \ref{perfect-proposition-DCoh}.
Hence by Lemma \ref{lemma-countable-abelian}
it suffices to prove that $\textit{Coh}(\mathcal{O}_X)$ is
countable. This we omit.
\end{proof}

\begin{lemma}
\label{lemma-countable-isos}
Let $K$ be an algebraically closed field.
Let $S$ be a finite type scheme over $K$.
Let $X \to S$ and $Y \to S$ be finite type morphisms.
There exists a countable set $I$ and for $i \in I$ a pair
$(S_i \to S, h_i)$ with the following properties
\begin{enumerate}
\item $S_i \to S$ is a morphism of finite type, set
$X_i = X \times_S S_i$ and $Y_i = Y \times_S S_i$,
\item $h_i : X_i \to Y_i$ is an isomorphism over $S_i$, and
\item for any closed point $s \in S(K)$ if $X_s \cong Y_s$
over $K = \kappa(s)$ then $s$ is in the image of $S_i \to S$
for some $i$.
\end{enumerate}
\end{lemma}

\begin{proof}
The field $K$ is the filtered union of its countable subfields.
Dually, $\Spec(K)$ is the cofiltered limit of the spectra
of the countable subfields of $K$.
Hence Limits, Lemma \ref{limits-lemma-descend-finite-presentation}
guarantees that we can find a countable subfield
$k$ and morphisms $X_0 \to S_0$ and $Y_0 \to S_0$
of schemes of finite type over $k$ such that
$X \to S$ and $Y \to S$ are the base changes of these.

\medskip\noindent
By Lemma \ref{lemma-countable-finite-type} there is a countable set $I$ and
pairs $(S_{0, i} \to S_0, h_{0, i})$ such that
\begin{enumerate}
\item $S_{0, i} \to S_0$ is a morphism of finite type, set
$X_{0, i} = X_0 \times_{S_0} S_{0, i}$ and
$Y_{0, i} = Y_0 \times_{S_0} S_{0, i}$,
\item $h_{0, i} : X_{0, i} \to Y_{0, i}$ is an isomorphism over $S_{0, i}$.
\end{enumerate}
such that every pair $(T \to S_0, h_T)$ with $T \to S_0$ of finite type
and $h_T : X_0 \times_{S_0} T \to Y_0 \times_{S_0} T$ an isomorphism
is isomorphic to one of these.
Denote $(S_i \to S, h_i)$ the base change of $(S_{0, i} \to S_0, h_{0, i})$
by $\Spec(K) \to \Spec(k)$.
We claim this works.

\medskip\noindent
Let $s \in S(K)$ and let $h_s : X_s \to Y_s$ be an isomorphism over
$K = \kappa(s)$. We can write $K$ as the filtered union of its
finitely generated $k$-subalgebras. Hence by
Limits, Proposition
\ref{limits-proposition-characterize-locally-finite-presentation} and
Lemma \ref{limits-lemma-descend-finite-presentation}
we can find such a finitely generated $k$-subalgebra
$K \supset A \supset k$ such that
\begin{enumerate}
\item there is a commutative diagram
$$
\xymatrix{
\Spec(K) \ar[d]_s \ar[r] &
\Spec(A) \ar[d]^{s'} \\
S \ar[r] &
S_0}
$$
for some morphism $s' : \Spec(A) \to S_0$ over $k$,
\item $h_s$ is the base change of an isomorphism
$h_{s'} : X_0 \times_{S_0, s'} \Spec(A) \to
X_0 \times_{S_0, s'} \Spec(A)$ over $A$.
\end{enumerate}
Of course, then $(s' : \Spec(A) \to S_0, h_{s'})$ is isomorphic
to the pair $(S_{0, i} \to S_0, h_{0, i})$ for some $i \in I$.
This concludes the proof because the commutative diagram
in (1) shows that $s$ is in the image of
the base change of $s'$ to $\Spec(K)$.
\end{proof}

\begin{lemma}
\label{lemma-countable-equivs}
Let $K$ be an algebraically closed field. There exists a countable set $I$
and for $i \in I$ a pair $(S_i/K, X_i \to S_i, Y_i \to S_i, M_i)$
with the following properties
\begin{enumerate}
\item $S_i$ is a scheme of finite type over $K$,
\item $X_i \to S_i$ and $Y_i \to S_i$ are proper smooth
morphisms of schemes,
\item $M_i \in D_{perf}(\mathcal{O}_{X_i \times_{S_i} Y_i})$
is the Fourier-Mukai kernel of a relative equivalence from
$X_i$ to $Y_i$ over $S_i$, and
\item for any smooth proper schemes $X$ and $Y$ over $K$
such that there is a $K$-linear exact equivalence
$D_{perf}(\mathcal{O}_X) \to D_{perf}(\mathcal{O}_Y)$
there exists an $i \in I$ and a $s \in S_i(K)$
such that $X \cong (X_i)_s$ and $Y \cong (Y_i)_s$.
\end{enumerate}
\end{lemma}

\begin{proof}
Choose a countable subfield $k \subset K$ for example the prime field.
By Lemmas \ref{lemma-countable-finite-type} and \ref{lemma-countable-perfect}
there exists a countable set of isomorphism classes of systems
over $k$ satisfying parts (1), (2), (3) of the lemma.
Thus we can choose a countable set
$I$ and for each $i \in I$ such a system
$$
(S_{0, i}/k, X_{0, i} \to S_{0, i}, Y_{0, i} \to S_{0, i}, M_{0, i})
$$
over $k$ such that each isomorphism class occurs at least once.
Denote $(S_i/K, X_i \to S_i, Y_i \to S_i, M_i)$ the base change
of the displayed system to $K$. This system has properties (1), (2), (3),
see Lemma \ref{lemma-base-change-rek}. Let us prove property (4).

\medskip\noindent
Consider smooth proper schemes $X$ and $Y$ over $K$
such that there is a $K$-linear exact equivalence
$F : D_{perf}(\mathcal{O}_X) \to D_{perf}(\mathcal{O}_Y)$.
By Proposition \ref{proposition-equivalence}
we may assume that there exists an object
$M \in D_{perf}(\mathcal{O}_{X \times Y})$
such that $F = \Phi_M$ is the corresponding Fourier-Mukai functor.
By Lemma \ref{lemma-fourier-mukai-flat-proper-over-noetherian}
there is an $M'$ in $D_{perf}(\mathcal{O}_{Y \times X})$
such that $\Phi_{M'}$ is the right adjoint to $\Phi_M$.
Since $\Phi_M$ is an equivalence, this means that
$\Phi_{M'}$ is the quasi-inverse to $\Phi_M$.
By Lemma \ref{lemma-fourier-mukai-flat-proper-over-noetherian}
we see that the Fourier-Mukai functors defined by the objects
$$
A = R\text{pr}_{13, *}(
L\text{pr}_{12}^*M
\otimes_{\mathcal{O}_{X \times Y \times X}}^\mathbf{L}
L\text{pr}_{23}^*M')
$$
in $D_{perf}(\mathcal{O}_{X \times X})$ and
$$
B = R\text{pr}_{13, *}(
L\text{pr}_{12}^*M'
\otimes_{\mathcal{O}_{Y \times X \times Y}}^\mathbf{L}
L\text{pr}_{23}^*M)
$$
in $D_{perf}(\mathcal{O}_{Y \times Y})$ 
are isomorphic to
$\text{id} : D_{perf}(\mathcal{O}_X) \to D_{perf}(\mathcal{O}_X)$
and
$\text{id} : D_{perf}(\mathcal{O}_Y) \to D_{perf}(\mathcal{O}_Y)$
Hence
$A \cong \Delta_{X/K, *}\mathcal{O}_X$ and
$B \cong \Delta_{Y/K, *}\mathcal{O}_Y$
by Lemma \ref{lemma-uniqueness}. Hence we see that $M$ is the
Fourier-Mukai kernel of a relative equivalence from $X$ to $Y$
over $K$ by definition.

\medskip\noindent
We can write $K$ as the filtered colimit of its finite type
$k$-subalgebras $A \subset K$. By
Limits, Lemma \ref{limits-lemma-descend-finite-presentation}
we can find $X_0, Y_0$ of finite type over $A$ whose
base changes to $K$ produces $X$ and $Y$.
By Limits, Lemmas
\ref{limits-lemma-eventually-proper} and \ref{limits-lemma-descend-smooth}
after enlarging $A$ we may assume $X_0$ and $Y_0$
are smooth and proper over $A$.
By Lemma \ref{lemma-descend-rek}
after enlarging $A$ we may assume $M$ is the pullback of
some $M_0  \in D_{perf}(\mathcal{O}_{X_0 \times_{\Spec(A)} Y_0})$
which is the Fourier-Mukai kernel of a relative equivalence
from $X_0$ to $Y_0$ over $\Spec(A)$.
Thus we see that $(S_0/k, X_0 \to S_0, Y_0 \to S_0, M_0)$
is isomorphic to
$(S_{0, i}/k, X_{0, i} \to S_{0, i}, Y_{0, i} \to S_{0, i}, M_{0, i})$
for some $i \in I$.
Since $S_i = S_{0, i} \times_{\Spec(k)} \Spec(K)$
we conclude that (4) is true with $s : \Spec(K) \to S_i$
induced by the morphism $\Spec(K) \to \Spec(A) \cong S_{0, i}$
we get from $A \subset K$.
\end{proof}








\section{Countability of derived equivalent varieties}
\label{section-countable-derived-equivalent}

\noindent
In this section we prove a result of Anel and To\"en, see \cite{AT}.

\begin{definition}
\label{definition-derived-equivalent}
Let $k$ be a field. Let $X$ and $Y$ be smooth projective schemes over $k$.
We say $X$ and $Y$ are {\it derived equivalent} if there exists a $k$-linear
exact equivalence
$D_{perf}(\mathcal{O}_X) \to D_{perf}(\mathcal{O}_Y)$.
\end{definition}

\noindent
Here is the result

\begin{theorem}
\label{theorem-countable}
\begin{reference}
Slight improvement of \cite{AT}
\end{reference}
Let $K$ be an algebraically closed field. Let $\mathbf{X}$ be a smooth proper
scheme over $K$. There are at most countably many isomorphism classes
of smooth proper schemes $\mathbf{Y}$ over $K$ which are derived
equivalent to $\mathbf{X}$.
\end{theorem}

\begin{proof}
Choose a countable set $I$ and for $i \in I$ systems
$(S_i/K, X_i \to S_i, Y_i \to S_i, M_i)$ satisfying properties
(1), (2), (3), and (4) of Lemma \ref{lemma-countable-equivs}.
Pick $i \in I$ and set $S = S_i$, $X = X_i$, $Y = Y_i$, and
$M = M_i$. Clearly it suffice to show that
the set of isomorphism classes of fibres $Y_s$ for $s \in S(K)$
such that $X_s \cong \mathbf{X}$ is countable.
This we prove in the next paragraph.

\medskip\noindent
Let $S$ be a finite type scheme over $K$, let $X \to S$ and $Y \to S$
be proper smooth morphisms, and let $M \in D_{perf}(\mathcal{O}_{X \times_S Y})$
be the Fourier-Mukai kernel of a relative equivalence from $X$
to $Y$ over $S$. We will show
the set of isomorphism classes of fibres $Y_s$ for $s \in S(K)$
such that $X_s \cong \mathbf{X}$ is countable.
By Lemma \ref{lemma-countable-isos} applied
to the families $\mathbf{X} \times S \to S$ and $X \to S$
there exists a countable set $I$ and for $i \in I$ a pair
$(S_i \to S, h_i)$ with the following properties
\begin{enumerate}
\item $S_i \to S$ is a morphism of finite type, set
$X_i = X \times_S S_i$,
\item $h_i : \mathbf{X} \times S_i \to X_i$
is an isomorphism over $S_i$, and
\item for any closed point $s \in S(K)$ if $\mathbf{X} \cong X_s$
over $K = \kappa(s)$ then $s$ is in the image of $S_i \to S$
for some $i$.
\end{enumerate}
Set $Y_i = Y \times_S S_i$. Denote
$M_i \in D_{perf}(\mathcal{O}_{X_i \times_{S_i} Y_i})$
the pullback of $M$. By Lemma \ref{lemma-base-change-rek}
$M_i$ is the Fourier-Mukai kernel of a relative equivalence from
$X_i$ to $Y_i$ over $S_i$. Since $I$ is countable, by
property (3) it suffices to prove that
the set of isomorphism classes of fibres $Y_{i, s}$ for $s \in S_i(K)$
is countable.
In fact, this number is finite by
Lemma \ref{lemma-no-deformations-better}
and the proof is complete.
\end{proof}







\begin{multicols}{2}[\section{Other chapters}]
\noindent
Preliminaries
\begin{enumerate}
\item \hyperref[introduction-section-phantom]{Introduction}
\item \hyperref[conventions-section-phantom]{Conventions}
\item \hyperref[sets-section-phantom]{Set Theory}
\item \hyperref[categories-section-phantom]{Categories}
\item \hyperref[topology-section-phantom]{Topology}
\item \hyperref[sheaves-section-phantom]{Sheaves on Spaces}
\item \hyperref[sites-section-phantom]{Sites and Sheaves}
\item \hyperref[stacks-section-phantom]{Stacks}
\item \hyperref[fields-section-phantom]{Fields}
\item \hyperref[algebra-section-phantom]{Commutative Algebra}
\item \hyperref[brauer-section-phantom]{Brauer Groups}
\item \hyperref[homology-section-phantom]{Homological Algebra}
\item \hyperref[derived-section-phantom]{Derived Categories}
\item \hyperref[simplicial-section-phantom]{Simplicial Methods}
\item \hyperref[more-algebra-section-phantom]{More on Algebra}
\item \hyperref[smoothing-section-phantom]{Smoothing Ring Maps}
\item \hyperref[modules-section-phantom]{Sheaves of Modules}
\item \hyperref[sites-modules-section-phantom]{Modules on Sites}
\item \hyperref[injectives-section-phantom]{Injectives}
\item \hyperref[cohomology-section-phantom]{Cohomology of Sheaves}
\item \hyperref[sites-cohomology-section-phantom]{Cohomology on Sites}
\item \hyperref[dga-section-phantom]{Differential Graded Algebra}
\item \hyperref[dpa-section-phantom]{Divided Power Algebra}
\item \hyperref[sdga-section-phantom]{Differential Graded Sheaves}
\item \hyperref[hypercovering-section-phantom]{Hypercoverings}
\end{enumerate}
Schemes
\begin{enumerate}
\setcounter{enumi}{25}
\item \hyperref[schemes-section-phantom]{Schemes}
\item \hyperref[constructions-section-phantom]{Constructions of Schemes}
\item \hyperref[properties-section-phantom]{Properties of Schemes}
\item \hyperref[morphisms-section-phantom]{Morphisms of Schemes}
\item \hyperref[coherent-section-phantom]{Cohomology of Schemes}
\item \hyperref[divisors-section-phantom]{Divisors}
\item \hyperref[limits-section-phantom]{Limits of Schemes}
\item \hyperref[varieties-section-phantom]{Varieties}
\item \hyperref[topologies-section-phantom]{Topologies on Schemes}
\item \hyperref[descent-section-phantom]{Descent}
\item \hyperref[perfect-section-phantom]{Derived Categories of Schemes}
\item \hyperref[more-morphisms-section-phantom]{More on Morphisms}
\item \hyperref[flat-section-phantom]{More on Flatness}
\item \hyperref[groupoids-section-phantom]{Groupoid Schemes}
\item \hyperref[more-groupoids-section-phantom]{More on Groupoid Schemes}
\item \hyperref[etale-section-phantom]{\'Etale Morphisms of Schemes}
\end{enumerate}
Topics in Scheme Theory
\begin{enumerate}
\setcounter{enumi}{41}
\item \hyperref[chow-section-phantom]{Chow Homology}
\item \hyperref[intersection-section-phantom]{Intersection Theory}
\item \hyperref[pic-section-phantom]{Picard Schemes of Curves}
\item \hyperref[weil-section-phantom]{Weil Cohomology Theories}
\item \hyperref[adequate-section-phantom]{Adequate Modules}
\item \hyperref[dualizing-section-phantom]{Dualizing Complexes}
\item \hyperref[duality-section-phantom]{Duality for Schemes}
\item \hyperref[discriminant-section-phantom]{Discriminants and Differents}
\item \hyperref[derham-section-phantom]{de Rham Cohomology}
\item \hyperref[local-cohomology-section-phantom]{Local Cohomology}
\item \hyperref[algebraization-section-phantom]{Algebraic and Formal Geometry}
\item \hyperref[curves-section-phantom]{Algebraic Curves}
\item \hyperref[resolve-section-phantom]{Resolution of Surfaces}
\item \hyperref[models-section-phantom]{Semistable Reduction}
\item \hyperref[functors-section-phantom]{Functors and Morphisms}
\item \hyperref[equiv-section-phantom]{Derived Categories of Varieties}
\item \hyperref[pione-section-phantom]{Fundamental Groups of Schemes}
\item \hyperref[etale-cohomology-section-phantom]{\'Etale Cohomology}
\item \hyperref[crystalline-section-phantom]{Crystalline Cohomology}
\item \hyperref[proetale-section-phantom]{Pro-\'etale Cohomology}
\item \hyperref[relative-cycles-section-phantom]{Relative Cycles}
\item \hyperref[more-etale-section-phantom]{More \'Etale Cohomology}
\item \hyperref[trace-section-phantom]{The Trace Formula}
\end{enumerate}
Algebraic Spaces
\begin{enumerate}
\setcounter{enumi}{64}
\item \hyperref[spaces-section-phantom]{Algebraic Spaces}
\item \hyperref[spaces-properties-section-phantom]{Properties of Algebraic Spaces}
\item \hyperref[spaces-morphisms-section-phantom]{Morphisms of Algebraic Spaces}
\item \hyperref[decent-spaces-section-phantom]{Decent Algebraic Spaces}
\item \hyperref[spaces-cohomology-section-phantom]{Cohomology of Algebraic Spaces}
\item \hyperref[spaces-limits-section-phantom]{Limits of Algebraic Spaces}
\item \hyperref[spaces-divisors-section-phantom]{Divisors on Algebraic Spaces}
\item \hyperref[spaces-over-fields-section-phantom]{Algebraic Spaces over Fields}
\item \hyperref[spaces-topologies-section-phantom]{Topologies on Algebraic Spaces}
\item \hyperref[spaces-descent-section-phantom]{Descent and Algebraic Spaces}
\item \hyperref[spaces-perfect-section-phantom]{Derived Categories of Spaces}
\item \hyperref[spaces-more-morphisms-section-phantom]{More on Morphisms of Spaces}
\item \hyperref[spaces-flat-section-phantom]{Flatness on Algebraic Spaces}
\item \hyperref[spaces-groupoids-section-phantom]{Groupoids in Algebraic Spaces}
\item \hyperref[spaces-more-groupoids-section-phantom]{More on Groupoids in Spaces}
\item \hyperref[bootstrap-section-phantom]{Bootstrap}
\item \hyperref[spaces-pushouts-section-phantom]{Pushouts of Algebraic Spaces}
\end{enumerate}
Topics in Geometry
\begin{enumerate}
\setcounter{enumi}{81}
\item \hyperref[spaces-chow-section-phantom]{Chow Groups of Spaces}
\item \hyperref[groupoids-quotients-section-phantom]{Quotients of Groupoids}
\item \hyperref[spaces-more-cohomology-section-phantom]{More on Cohomology of Spaces}
\item \hyperref[spaces-simplicial-section-phantom]{Simplicial Spaces}
\item \hyperref[spaces-duality-section-phantom]{Duality for Spaces}
\item \hyperref[formal-spaces-section-phantom]{Formal Algebraic Spaces}
\item \hyperref[restricted-section-phantom]{Algebraization of Formal Spaces}
\item \hyperref[spaces-resolve-section-phantom]{Resolution of Surfaces Revisited}
\end{enumerate}
Deformation Theory
\begin{enumerate}
\setcounter{enumi}{89}
\item \hyperref[formal-defos-section-phantom]{Formal Deformation Theory}
\item \hyperref[defos-section-phantom]{Deformation Theory}
\item \hyperref[cotangent-section-phantom]{The Cotangent Complex}
\item \hyperref[examples-defos-section-phantom]{Deformation Problems}
\end{enumerate}
Algebraic Stacks
\begin{enumerate}
\setcounter{enumi}{93}
\item \hyperref[algebraic-section-phantom]{Algebraic Stacks}
\item \hyperref[examples-stacks-section-phantom]{Examples of Stacks}
\item \hyperref[stacks-sheaves-section-phantom]{Sheaves on Algebraic Stacks}
\item \hyperref[criteria-section-phantom]{Criteria for Representability}
\item \hyperref[artin-section-phantom]{Artin's Axioms}
\item \hyperref[quot-section-phantom]{Quot and Hilbert Spaces}
\item \hyperref[stacks-properties-section-phantom]{Properties of Algebraic Stacks}
\item \hyperref[stacks-morphisms-section-phantom]{Morphisms of Algebraic Stacks}
\item \hyperref[stacks-limits-section-phantom]{Limits of Algebraic Stacks}
\item \hyperref[stacks-cohomology-section-phantom]{Cohomology of Algebraic Stacks}
\item \hyperref[stacks-perfect-section-phantom]{Derived Categories of Stacks}
\item \hyperref[stacks-introduction-section-phantom]{Introducing Algebraic Stacks}
\item \hyperref[stacks-more-morphisms-section-phantom]{More on Morphisms of Stacks}
\item \hyperref[stacks-geometry-section-phantom]{The Geometry of Stacks}
\end{enumerate}
Topics in Moduli Theory
\begin{enumerate}
\setcounter{enumi}{107}
\item \hyperref[moduli-section-phantom]{Moduli Stacks}
\item \hyperref[moduli-curves-section-phantom]{Moduli of Curves}
\end{enumerate}
Miscellany
\begin{enumerate}
\setcounter{enumi}{109}
\item \hyperref[examples-section-phantom]{Examples}
\item \hyperref[exercises-section-phantom]{Exercises}
\item \hyperref[guide-section-phantom]{Guide to Literature}
\item \hyperref[desirables-section-phantom]{Desirables}
\item \hyperref[coding-section-phantom]{Coding Style}
\item \hyperref[obsolete-section-phantom]{Obsolete}
\item \hyperref[fdl-section-phantom]{GNU Free Documentation License}
\item \hyperref[index-section-phantom]{Auto Generated Index}
\end{enumerate}
\end{multicols}


\bibliography{my}
\bibliographystyle{amsalpha}

\end{document}

