\IfFileExists{stacks-project.cls}{%
\documentclass{stacks-project}
}{%
\documentclass{amsart}
}

% For dealing with references we use the comment environment
\usepackage{verbatim}
\newenvironment{reference}{\comment}{\endcomment}
%\newenvironment{reference}{}{}
\newenvironment{slogan}{\comment}{\endcomment}
\newenvironment{history}{\comment}{\endcomment}

% For commutative diagrams we use Xy-pic
\usepackage[all]{xy}

% We use 2cell for 2-commutative diagrams.
\xyoption{2cell}
\UseAllTwocells

% We use multicol for the list of chapters between chapters
\usepackage{multicol}

% This is generall recommended for better output
\usepackage{lmodern}
\usepackage[T1]{fontenc}

% For cross-file-references
\usepackage{xr-hyper}

% Package for hypertext links:
\usepackage{hyperref}

% For any local file, say "hello.tex" you want to link to please
% use \externaldocument[hello-]{hello}
\externaldocument[introduction-]{introduction}
\externaldocument[conventions-]{conventions}
\externaldocument[sets-]{sets}
\externaldocument[categories-]{categories}
\externaldocument[topology-]{topology}
\externaldocument[sheaves-]{sheaves}
\externaldocument[sites-]{sites}
\externaldocument[stacks-]{stacks}
\externaldocument[fields-]{fields}
\externaldocument[algebra-]{algebra}
\externaldocument[brauer-]{brauer}
\externaldocument[homology-]{homology}
\externaldocument[derived-]{derived}
\externaldocument[simplicial-]{simplicial}
\externaldocument[more-algebra-]{more-algebra}
\externaldocument[smoothing-]{smoothing}
\externaldocument[modules-]{modules}
\externaldocument[sites-modules-]{sites-modules}
\externaldocument[injectives-]{injectives}
\externaldocument[cohomology-]{cohomology}
\externaldocument[sites-cohomology-]{sites-cohomology}
\externaldocument[dga-]{dga}
\externaldocument[dpa-]{dpa}
\externaldocument[sdga-]{sdga}
\externaldocument[hypercovering-]{hypercovering}
\externaldocument[schemes-]{schemes}
\externaldocument[constructions-]{constructions}
\externaldocument[properties-]{properties}
\externaldocument[morphisms-]{morphisms}
\externaldocument[coherent-]{coherent}
\externaldocument[divisors-]{divisors}
\externaldocument[limits-]{limits}
\externaldocument[varieties-]{varieties}
\externaldocument[topologies-]{topologies}
\externaldocument[descent-]{descent}
\externaldocument[perfect-]{perfect}
\externaldocument[more-morphisms-]{more-morphisms}
\externaldocument[flat-]{flat}
\externaldocument[groupoids-]{groupoids}
\externaldocument[more-groupoids-]{more-groupoids}
\externaldocument[etale-]{etale}
\externaldocument[chow-]{chow}
\externaldocument[intersection-]{intersection}
\externaldocument[pic-]{pic}
\externaldocument[weil-]{weil}
\externaldocument[adequate-]{adequate}
\externaldocument[dualizing-]{dualizing}
\externaldocument[duality-]{duality}
\externaldocument[discriminant-]{discriminant}
\externaldocument[derham-]{derham}
\externaldocument[local-cohomology-]{local-cohomology}
\externaldocument[algebraization-]{algebraization}
\externaldocument[curves-]{curves}
\externaldocument[resolve-]{resolve}
\externaldocument[models-]{models}
\externaldocument[functors-]{functors}
\externaldocument[equiv-]{equiv}
\externaldocument[pione-]{pione}
\externaldocument[etale-cohomology-]{etale-cohomology}
\externaldocument[proetale-]{proetale}
\externaldocument[relative-cycles-]{relative-cycles}
\externaldocument[more-etale-]{more-etale}
\externaldocument[trace-]{trace}
\externaldocument[crystalline-]{crystalline}
\externaldocument[spaces-]{spaces}
\externaldocument[spaces-properties-]{spaces-properties}
\externaldocument[spaces-morphisms-]{spaces-morphisms}
\externaldocument[decent-spaces-]{decent-spaces}
\externaldocument[spaces-cohomology-]{spaces-cohomology}
\externaldocument[spaces-limits-]{spaces-limits}
\externaldocument[spaces-divisors-]{spaces-divisors}
\externaldocument[spaces-over-fields-]{spaces-over-fields}
\externaldocument[spaces-topologies-]{spaces-topologies}
\externaldocument[spaces-descent-]{spaces-descent}
\externaldocument[spaces-perfect-]{spaces-perfect}
\externaldocument[spaces-more-morphisms-]{spaces-more-morphisms}
\externaldocument[spaces-flat-]{spaces-flat}
\externaldocument[spaces-groupoids-]{spaces-groupoids}
\externaldocument[spaces-more-groupoids-]{spaces-more-groupoids}
\externaldocument[bootstrap-]{bootstrap}
\externaldocument[spaces-pushouts-]{spaces-pushouts}
\externaldocument[spaces-chow-]{spaces-chow}
\externaldocument[groupoids-quotients-]{groupoids-quotients}
\externaldocument[spaces-more-cohomology-]{spaces-more-cohomology}
\externaldocument[spaces-simplicial-]{spaces-simplicial}
\externaldocument[spaces-duality-]{spaces-duality}
\externaldocument[formal-spaces-]{formal-spaces}
\externaldocument[restricted-]{restricted}
\externaldocument[spaces-resolve-]{spaces-resolve}
\externaldocument[formal-defos-]{formal-defos}
\externaldocument[defos-]{defos}
\externaldocument[cotangent-]{cotangent}
\externaldocument[examples-defos-]{examples-defos}
\externaldocument[algebraic-]{algebraic}
\externaldocument[examples-stacks-]{examples-stacks}
\externaldocument[stacks-sheaves-]{stacks-sheaves}
\externaldocument[criteria-]{criteria}
\externaldocument[artin-]{artin}
\externaldocument[quot-]{quot}
\externaldocument[stacks-properties-]{stacks-properties}
\externaldocument[stacks-morphisms-]{stacks-morphisms}
\externaldocument[stacks-limits-]{stacks-limits}
\externaldocument[stacks-cohomology-]{stacks-cohomology}
\externaldocument[stacks-perfect-]{stacks-perfect}
\externaldocument[stacks-introduction-]{stacks-introduction}
\externaldocument[stacks-more-morphisms-]{stacks-more-morphisms}
\externaldocument[stacks-geometry-]{stacks-geometry}
\externaldocument[moduli-]{moduli}
\externaldocument[moduli-curves-]{moduli-curves}
\externaldocument[examples-]{examples}
\externaldocument[exercises-]{exercises}
\externaldocument[guide-]{guide}
\externaldocument[desirables-]{desirables}
\externaldocument[coding-]{coding}
\externaldocument[obsolete-]{obsolete}
\externaldocument[fdl-]{fdl}
\externaldocument[index-]{index}

% Theorem environments.
%
\theoremstyle{plain}
\newtheorem{theorem}[subsection]{Theorem}
\newtheorem{proposition}[subsection]{Proposition}
\newtheorem{lemma}[subsection]{Lemma}

\theoremstyle{definition}
\newtheorem{definition}[subsection]{Definition}
\newtheorem{example}[subsection]{Example}
\newtheorem{exercise}[subsection]{Exercise}
\newtheorem{situation}[subsection]{Situation}

\theoremstyle{remark}
\newtheorem{remark}[subsection]{Remark}
\newtheorem{remarks}[subsection]{Remarks}

\numberwithin{equation}{subsection}

% Macros
%
\def\lim{\mathop{\mathrm{lim}}\nolimits}
\def\colim{\mathop{\mathrm{colim}}\nolimits}
\def\Spec{\mathop{\mathrm{Spec}}}
\def\Hom{\mathop{\mathrm{Hom}}\nolimits}
\def\Ext{\mathop{\mathrm{Ext}}\nolimits}
\def\SheafHom{\mathop{\mathcal{H}\!\mathit{om}}\nolimits}
\def\SheafExt{\mathop{\mathcal{E}\!\mathit{xt}}\nolimits}
\def\Sch{\mathit{Sch}}
\def\Mor{\mathop{\mathrm{Mor}}\nolimits}
\def\Ob{\mathop{\mathrm{Ob}}\nolimits}
\def\Sh{\mathop{\mathit{Sh}}\nolimits}
\def\NL{\mathop{N\!L}\nolimits}
\def\CH{\mathop{\mathrm{CH}}\nolimits}
\def\proetale{{pro\text{-}\acute{e}tale}}
\def\etale{{\acute{e}tale}}
\def\QCoh{\mathit{QCoh}}
\def\Ker{\mathop{\mathrm{Ker}}}
\def\Im{\mathop{\mathrm{Im}}}
\def\Coker{\mathop{\mathrm{Coker}}}
\def\Coim{\mathop{\mathrm{Coim}}}

% Boxtimes
%
\DeclareMathSymbol{\boxtimes}{\mathbin}{AMSa}{"02}

%
% Macros for moduli stacks/spaces
%
\def\QCohstack{\mathcal{QC}\!\mathit{oh}}
\def\Cohstack{\mathcal{C}\!\mathit{oh}}
\def\Spacesstack{\mathcal{S}\!\mathit{paces}}
\def\Quotfunctor{\mathrm{Quot}}
\def\Hilbfunctor{\mathrm{Hilb}}
\def\Curvesstack{\mathcal{C}\!\mathit{urves}}
\def\Polarizedstack{\mathcal{P}\!\mathit{olarized}}
\def\Complexesstack{\mathcal{C}\!\mathit{omplexes}}
% \Pic is the operator that assigns to X its picard group, usage \Pic(X)
% \Picardstack_{X/B} denotes the Picard stack of X over B
% \Picardfunctor_{X/B} denotes the Picard functor of X over B
\def\Pic{\mathop{\mathrm{Pic}}\nolimits}
\def\Picardstack{\mathcal{P}\!\mathit{ic}}
\def\Picardfunctor{\mathrm{Pic}}
\def\Deformationcategory{\mathcal{D}\!\mathit{ef}}


% OK, start here.
%
\begin{document}

\title{The Geometry of Algebraic Stacks}


\maketitle

\phantomsection
\label{section-phantom}

\tableofcontents

\section{Introduction}
\label{section-introduction}

\noindent
This chapter discusses a few geometric properties of algebraic stacks.
The initial versions of Sections \ref{section-multiplicities} and
\ref{section-dimension-of-algebraic-stacks}
were written by Matthew Emerton and Toby Gee and can be found
in their original form in \cite{Emerton-Gee-dim}.







\section{Versal rings}
\label{section-versal}

\noindent
In this section we elucidate the relationship between deformation rings
and local rings on algebraic stacks of finite type over a locally
Noetherian base.

\begin{situation}
\label{situation-versal}
Here $\mathcal{X}$ is an algebraic stack locally of finite type
over a locally Noetherian scheme $S$.
\end{situation}

\noindent
Here is the definition.

\begin{definition}
\label{definition-versal-ring-at-x}
In Situation \ref{situation-versal} let $x_0 : \Spec(k) \to \mathcal{X}$
be a morphism, where $k$ is a finite type field over $S$.
A {\it versal ring to $\mathcal{X}$ at $x_0$} is a complete
Noetherian local $S$-algebra $A$ with residue field $k$
such that there exists a versal formal object
$(A, \xi_n, f_n)$ as in Artin's Axioms, Definition
\ref{artin-definition-versal-formal-object}
with $\xi_1 \cong x_0$ (a $2$-isomorphism).
\end{definition}

\noindent
We want to prove that versal rings exist and are unique up to
smooth factors. To do this, we will use the predeformation categories of
Artin's Axioms, Section \ref{artin-section-predeformation-categories}.
These are always deformation categories in our situation.

\begin{lemma}
\label{lemma-deformation-category}
In Situation \ref{situation-versal} let $x_0 : \Spec(k) \to \mathcal{X}$
be a morphism, where $k$ is a finite type field over $S$.
Then $\mathcal{F}_{\mathcal{X}, k, x_0}$
is a deformation category and $T\mathcal{F}_{\mathcal{X}, k, x_0}$
and $\text{Inf}(\mathcal{F}_{\mathcal{X}, k, x_0})$
are finite dimensional $k$-vector spaces.
\end{lemma}

\begin{proof}
Choose an affine open $\Spec(\Lambda) \subset S$ such that
$\Spec(k) \to S$ factors through it.
By Artin's Axioms, Section \ref{artin-section-predeformation-categories}
we obtain a predeformation category
$\mathcal{F}_{\mathcal{X}, k, x_0}$
over the category $\mathcal{C}_\Lambda$.
(As pointed out in locus citatus this category only depends
on the morphism $\Spec(k) \to S$ and not on the choice of
$\Lambda$.) By Artin's Axioms, Lemmas
\ref{artin-lemma-deformation-category} and
\ref{artin-lemma-algebraic-stack-RS}
$\mathcal{F}_{\mathcal{X}, k, x_0}$ is actually a deformation category.
By Artin's Axioms, Lemma \ref{artin-lemma-finite-dimension}
we find that $T\mathcal{F}_{\mathcal{X}, k, x_0}$
and $\text{Inf}(\mathcal{F}_{\mathcal{X}, k, x_0})$
are finite dimensional $k$-vector spaces.
\end{proof}

\begin{lemma}
\label{lemma-versal-ring}
In Situation \ref{situation-versal} let $x_0 : \Spec(k) \to \mathcal{X}$
be a morphism, where $k$ is a finite type field over $S$.
Then a versal ring to $\mathcal{X}$ at $x_0$ exists. Given a pair
$A$, $A'$ of these, then $A \cong A'[[t_1, \ldots, t_r]]$
or $A' \cong A[[t_1, \ldots, t_r]]$ as $S$-algebras
for some $r$.
\end{lemma}

\begin{proof}
By Lemma \ref{lemma-deformation-category} and
Formal Deformation Theory, Lemma
\ref{formal-defos-lemma-versal-object-existence}
(note that the assumptions of this lemma hold by
Formal Deformation Theory, Lemmas
\ref{formal-defos-lemma-RS-implies-S1-S2} and
Definition \ref{formal-defos-definition-deformation-category}).
By the uniquness result of
Formal Deformation Theory, Lemma \ref{formal-defos-lemma-minimal-versal}
there exists a ``minimal'' versal ring $A$ of $\mathcal{X}$ at $x_0$
such that any other versal ring of $\mathcal{X}$ at $x_0$ is
isomorphic to $A[[t_1, \ldots, t_r]]$ for some $r$.
This clearly implies the second statement.
\end{proof}

\begin{lemma}
\label{lemma-versal-ring-field-extension}
In Situation \ref{situation-versal} let $x_0 : \Spec(k) \to \mathcal{X}$
be a morphism, where $k$ is a finite type field over $S$.
Let $l/k$ be a finite extension of fields and denote
$x_{l, 0} : \Spec(l) \to \mathcal{X}$ the induced morphism.
Given a versal ring $A$ to $\mathcal{X}$ at $x_0$ there exists
a versal ring $A'$ to $\mathcal{X}$ at $x_{l, 0}$ such that
there is a $S$-algebra map $A \to A'$ which induces the given
field extension $l/k$ and is formally smooth in the $\mathfrak m_{A'}$-adic
topology.
\end{lemma}

\begin{proof}
Follows immediately from
Artin's Axioms, Lemma \ref{artin-lemma-change-of-field}
and
Formal Deformation Theory, Lemma
\ref{formal-defos-lemma-change-of-field-versal-ring}.
(We also use that $\mathcal{X}$ satisfies (RS) by
Artin's Axioms, Lemma \ref{artin-lemma-algebraic-stack-RS}.)
\end{proof}

\begin{lemma}
\label{lemma-compare-versal-ring-completion}
In Situation \ref{situation-versal} let $x : U \to \mathcal{X}$ be a
morphism where $U$ is a scheme locally of finite type over $S$.
Let $u_0 \in U$ be a finite type point.
Set $k = \kappa(u_0)$ and denote $x_0 : \Spec(k) \to \mathcal{X}$
the induced map. The following are equivalent
\begin{enumerate}
\item $x$ is versal at $u_0$
(Artin's Axioms, Definition \ref{artin-definition-versal}),
\item $\hat x : \mathcal{F}_{U, k, u_0} \to \mathcal{F}_{\mathcal{X}, k, x_0}$
is smooth,
\item the formal object associated to
$x|_{\Spec(\mathcal{O}_{U, u_0}^\wedge)}$ is versal, and
\item there is an open neighbourhood $U' \subset U$ of $x$ such that
$x|_{U'} : U' \to \mathcal{X}$ is smooth.
\end{enumerate}
Moreover, in this case the completion $\mathcal{O}_{U, u_0}^\wedge$
is a versal ring to $\mathcal{X}$ at $x_0$.
\end{lemma}

\begin{proof}
Since $U \to S$ is locally of finite type (as a composition of such morphisms),
we see that $\Spec(k) \to S$ is of finite type (again as a composition).
Thus the statement makes sense. The equivalence of (1) and (2)
is the definition of $x$ being versal at $u_0$.
The equivalence of (1) and (3) is
Artin's Axioms, Lemma \ref{artin-lemma-versality-matches}.
Thus (1), (2), and (3) are equivalent.

\medskip\noindent
If $x|_{U'}$ is smooth, then the functor
$\hat x : \mathcal{F}_{U, k, u_0} \to \mathcal{F}_{\mathcal{X}, k, x_0}$
is smooth by Artin's Axioms, Lemma
\ref{artin-lemma-formally-smooth-on-deformation-categories}.
Thus (4) implies (1), (2), and (3).
For the converse, assume $x$ is versal at $u_0$.
Choose a surjective smooth morphism $y : V \to \mathcal{X}$ where $V$
is a scheme. Set $Z = V \times_\mathcal{X} U$ and pick a finite type
point $z_0 \in |Z|$ lying over $u_0$ (this is possible by
Morphisms of Spaces, Lemma
\ref{spaces-morphisms-lemma-finite-type-points-surjective-morphism}).
By Artin's Axioms, Lemma \ref{artin-lemma-base-change-versal}
the morphism $Z \to V$ is smooth at $z_0$.
By definition we can find an open neighbourhood $W \subset Z$
of $z_0$ such that $W \to V$ is smooth. Since $Z \to U$ is open,
let $U' \subset U$ be the image of $W$. Then we see that
$U' \to \mathcal{X}$ is smooth by our definition of smooth morphisms
of stacks.

\medskip\noindent
The final statement follows from the definitions as
$\mathcal{O}_{U, u_0}^\wedge$
prorepresents $\mathcal{F}_{U, k, u_0}$.
\end{proof}

\begin{lemma}
\label{lemma-characterize-smoothness}
In Situation \ref{situation-versal}. Let $x_0 : \Spec(k) \to \mathcal{X}$
be a morphism such that $\Spec(k) \to S$ is of finite type with image $s$.
Let $A$ be a versal ring to $\mathcal{X}$ at $x_0$. The following
are equivalent
\begin{enumerate}
\item $x_0$ is in the smooth locus of $\mathcal{X} \to S$
(Morphisms of Stacks, Lemma \ref{stacks-morphisms-lemma-where-smooth}),
\item $\mathcal{O}_{S, s} \to A$ is formally smooth in the
$\mathfrak m_A$-adic topology, and
\item $\mathcal{F}_{\mathcal{X}, k, x_0}$ is unobstructed.
\end{enumerate}
\end{lemma}

\begin{proof}
The equivalence of (2) and (3) follows immediately from
Formal Deformation Theory, Lemma
\ref{formal-defos-lemma-smooth-power-series-classical}.

\medskip\noindent
Note that $\mathcal{O}_{S, s} \to A$ is formally smooth in the
$\mathfrak m_A$-adic topology if and only if
$\mathcal{O}_{S, s} \to A' = A[[t_1, \ldots, t_r]]$
is formally smooth in the $\mathfrak m_{A'}$-adic topology.
Hence (2) does not depend on the choice
of our versal ring by Lemma \ref{lemma-versal-ring}.
Next, let $l/k$ be a finite extension and choose
$A \to A'$ as in Lemma \ref{lemma-versal-ring-field-extension}.
If $\mathcal{O}_{S, s} \to A$ is formally smooth in the
$\mathfrak m_A$-adic topology, then $\mathcal{O}_{S, s} \to A'$
is formally smooth in the $\mathfrak m_{A'}$-adic topology, see
More on Algebra, Lemma \ref{more-algebra-lemma-compose-formally-smooth}.
Conversely, if $\mathcal{O}_{S, s} \to A'$
is formally smooth in the $\mathfrak m_{A'}$-adic topology,
then $\mathcal{O}_{S, s}^\wedge \to A'$ and $A \to A'$ are regular
(More on Algebra, Proposition \ref{more-algebra-proposition-fs-regular})
and hence $\mathcal{O}_{S, s}^\wedge \to A$ is regular
(More on Algebra, Lemma \ref{more-algebra-lemma-regular-permanence}),
hence $\mathcal{O}_{S, s} \to A$ is formally smooth in the
$\mathfrak m_A$-adic topology (same lemma as before).
Thus the equivalence of (2) and (1) holds for
$k$ and $x_0$ if and only if it holds for $l$ and $x_{0, l}$.

\medskip\noindent
Choose a scheme $U$ and a smooth morphism $U \to \mathcal{X}$ such
that $\Spec(k) \times_\mathcal{X} U$ is nonempty. Choose a finite
extension $l/k$ and a point $w_0 : \Spec(l) \to \Spec(k) \times_\mathcal{X} U$.
Let $u_0 \in U$ be the image of $w_0$.
We may apply the above to $l/k$ and to $l/\kappa(u_0)$
to see that we can reduce to $u_0$. Thus we may assume
$A = \mathcal{O}_{U, u_0}^\wedge$, see
Lemma \ref{lemma-compare-versal-ring-completion}.
Observe that $x_0$ is in the smooth locus of $\mathcal{X} \to S$
if and only if $u_0$ is in the smooth locus of $U \to S$, see for example
Morphisms of Stacks, Lemma \ref{stacks-morphisms-lemma-where-smooth}.
Thus the equivalence of (1) and (2) follows from
More on Algebra, Lemma \ref{more-algebra-lemma-formally-smooth-finite-type}.
\end{proof}

\noindent
We recall a consequence of Artin approximation.

\begin{lemma}
\label{lemma-Artin-approximation-by-smooth-morphism}
In Situation \ref{situation-versal}. Let $x_0 : \Spec(k) \to \mathcal{X}$
be a morphism such that $\Spec(k) \to S$ is of finite type with image $s$.
Let $A$ be a versal ring to $\mathcal{X}$ at $x_0$.
If $\mathcal{O}_{S, s}$ is a G-ring, then we may find a smooth morphism
$U \to \mathcal{X}$ whose source is a scheme and a point
$u_0 \in U$ with residue field $k$, such that
\begin{enumerate}
\item $\Spec(k) \to U \to \mathcal{X}$ coincides with the given morphism $x_0$,
\item there is an isomorphism $\mathcal{O}_{U, u_0}^\wedge \cong A$.
\end{enumerate}
\end{lemma}

\begin{proof}
Let $(\xi_n, f_n)$ be the versal formal object over $A$.
By Artin's Axioms, Lemma \ref{artin-lemma-effective}
we know that $\xi = (A, \xi_n, f_n)$ is effective.
By assumption $\mathcal{X}$ is locally of finite presentation over $S$
(use Morphisms of Stacks, Lemma
\ref{stacks-morphisms-lemma-noetherian-finite-type-finite-presentation}),
and hence limit preserving by Limits of Stacks, Proposition
\ref{stacks-limits-proposition-characterize-locally-finite-presentation}.
Thus Artin approximation as in
Artin's Axioms, Lemma \ref{artin-lemma-approximate-versal}
shows that we may find a morphism $U \to \mathcal{X}$ with
source a finite type $S$-scheme, containing a point $u_0 \in U$
of residue field $k$ satisfying (1) and (2) such that $U \to \mathcal{X}$
is versal at $u_0$. By Lemma \ref{lemma-compare-versal-ring-completion}
after shrinking $U$ we may assume $U \to \mathcal{X}$ is smooth.
\end{proof}

\begin{remark}[Upgrading versal rings]
\label{remark-upgrade}
In Situation \ref{situation-versal} let $x_0 : \Spec(k) \to \mathcal{X}$
be a morphism, where $k$ is a finite type field over $S$.
Let $A$ be a versal ring to $\mathcal{X}$ at $x_0$. By Artin's Axioms,
Lemma \ref{artin-lemma-effective} our versal formal object
in fact comes from a morphism
$$
\Spec(A) \longrightarrow \mathcal{X}
$$
over $S$. Moreover, the results above each can be upgraded to be compatible
with this morphism. Here is a list:
\begin{enumerate}
\item in Lemma \ref{lemma-versal-ring} the isomorphism
$A \cong A'[[t_1, \ldots, t_r]]$ or $A' \cong A[[t_1, \ldots, t_r]]$
may be chosen compatible with these morphisms,
\item in Lemma \ref{lemma-versal-ring-field-extension} the
homomorphism $A \to A'$ may be chosen compatible with these morphisms,
\item in Lemma \ref{lemma-compare-versal-ring-completion}
the morphism $\Spec(\mathcal{O}_{U, u_0}^\wedge) \to \mathcal{X}$
is the composition of the canonical map
$\Spec(\mathcal{O}_{U, u_0}^\wedge) \to U$ and the given map
$U \to \mathcal{X}$,
\item in Lemma \ref{lemma-Artin-approximation-by-smooth-morphism}
the isomorphism $\mathcal{O}_{U, u_0}^\wedge \cong A$ may
be chosen so $\Spec(A) \to \mathcal{X}$ corresponds to the canonical map
in the item above.
\end{enumerate}
In each case the statement follows from the fact that our maps are
compatible with versal formal elements; we note however that the
implied diagrams are $2$-commutative only up to a (noncanonical)
choice of a $2$-arrow. Still, this means that the implied map $A' \to A$
or $A \to A'$ in (1) is well defined up to formal homotopy, see
Formal Deformation Theory, Lemma
\ref{formal-defos-lemma-versal-unique-up-to-homotopy}.
\end{remark}

\begin{lemma}
\label{lemma-versal-ring-flat}
In Situation \ref{situation-versal} let $x_0 : \Spec(k) \to \mathcal{X}$
be a morphism, where $k$ is a finite type field over $S$.
Let $A$ be a versal ring to $\mathcal{X}$ at $x_0$.
Then the morphism $\Spec(A) \to \mathcal{X}$ of
Remark \ref{remark-upgrade} is flat.
\end{lemma}

\begin{proof}
If the local ring of $S$ at the image point is a G-ring, then this
follows immediately from
Lemma \ref{lemma-Artin-approximation-by-smooth-morphism}
and the fact that the map from a Noetherian local ring to its
completion is flat. In general we prove it as follows.

\medskip\noindent
Step I. If $A$ and $A'$ are two versal rings to $\mathcal{X}$ at $x_0$,
then the result is true for $A$ if and only if it is true for $A'$.
Namely, after possible swapping $A$ and $A'$, we may assume there is
a formally smooth map $\varphi : A \to A'$ such that the composition
$$
\Spec(A') \to \Spec(A) \to \mathcal{X}
$$
is the morphism $\Spec(A') \to \mathcal{X}$, see
Lemma \ref{lemma-versal-ring} and
Remark \ref{remark-upgrade}.
Since $A \to A'$ is faithfully flat we obtain the equivalence from
Morphisms of Stacks, Lemmas \ref{stacks-morphisms-lemma-composition-flat} and
\ref{stacks-morphisms-lemma-flat-permanence}.

\medskip\noindent
Step II. Let $l/k$ be a finite extension of fields. Let
$x_{l, 0} : \Spec(l) \to \mathcal{X}$ be the induced morphism.
Let $A$ be a versal ring to $\mathcal{X}$ at $x_0$ and
let $A \to A'$ be as in Lemma \ref{lemma-versal-ring-field-extension}.
Then again the composition
$$
\Spec(A') \to \Spec(A) \to \mathcal{X}
$$
is the morphism $\Spec(A') \to \mathcal{X}$, see Remark \ref{remark-upgrade}.
Arguing as before and using step I to see choice of versal rings
is irrelevant, we see that the lemma holds for $x_0$ if and only
if it holds for $x_{l, 0}$.

\medskip\noindent
Step III. Choose a scheme $U$ and a surjective smooth morphism
$U \to \mathcal{X}$. Then we can choose a finite type point
$z_0$ on $Z = U \times_\mathcal{X} x_0$ (this is a nonempty
algebraic space). Let $u_0 \in U$ be the image of $z_0$ in $U$.
Choose a scheme and a surjective \'etale map $W \to Z$
such that $z_0$ is the image of a closed point $w_0 \in W$
(see Morphisms of Spaces, Section
\ref{spaces-morphisms-section-points-finite-type}).
Since $W \to \Spec(k)$ and $W \to U$ are of finite type,
we see that $\kappa(w_0)/k$ and $\kappa(w_0)/\kappa(u_0)$
are finite extensions of fields
(see Morphisms, Section \ref{morphisms-section-points-finite-type}).
Applying Step II twice we may replace $x_0$ by
$u_0 \to U \to \mathcal{X}$. Then we see our morphism is the composition
$$
\Spec(\mathcal{O}_{U, u_0}^\wedge) \to U \to \mathcal{X}
$$
The first arrow is flat because completion of Noetherian local rings
are flat (Algebra, Lemma \ref{algebra-lemma-completion-flat})
and the second arrow is flat as
a smooth morphism is flat. The composition is flat as composition
preserves flatness.
\end{proof}

\begin{remark}
\label{remark-groupoid-defo}
In Situation \ref{situation-versal} let $x_0 : \Spec(k) \to \mathcal{X}$
be a morphism, where $k$ is a finite type field over $S$.
By Lemma \ref{lemma-deformation-category} and
Formal Deformation Theory, Theorem
\ref{formal-defos-theorem-presentation-deformation-groupoid}
we know that $\mathcal{F}_{\mathcal{X}, k, x_0}$ has a
presentation by a smooth prorepresentable groupoid in
functors on $\mathcal{C}_\Lambda$.
Unwinding the definitions, this means we can choose
\begin{enumerate}
\item a Noetherian complete local $\Lambda$-algebra $A$
with residue field $k$ and a versal formal object $\xi$
of $\mathcal{F}_{\mathcal{X}, k, x_0}$ over $A$,
\item a Noetherian complete local $\Lambda$-algebra $B$
with residue field $k$ and an isomorphism
$$
\underline{B}|_{\mathcal{C}_\Lambda}
\longrightarrow
\underline{A}|_{\mathcal{C}_\Lambda}
\times_{\underline{\xi}, \mathcal{F}_{\mathcal{X}, k, x_0}, \underline{\xi}}
\underline{A}|_{\mathcal{C}_\Lambda}
$$
\end{enumerate}
The projections correspond to formally smooth maps
$t : A \to B$ and $s : A \to B$ (because $\xi$ is versal).
There is a map $c : B \to B \widehat{\otimes}_{s, A, t} B$
which turns $(A, B, s, t, c)$ into a cogroupoid in the category
of Noetherian complete local $\Lambda$-algebras with residue field $k$
(on prorepresentable functors this map is constructed in
Formal Deformation Theory, Lemma
\ref{formal-defos-lemma-presentation-construction}).
Finally, the cited theorem tells us that $\xi$ induces
an equivalence
$$
[\underline{A}|_{\mathcal{C}_\Lambda} / \underline{B}|_{\mathcal{C}_\Lambda}]
\longrightarrow
\mathcal{F}_{\mathcal{X}, k, x_0}
$$
of groupoids cofibred over $\mathcal{C}_\Lambda$. In fact, we also
get an equivalence
$$
[\underline{A}/\underline{B}]
\longrightarrow
\widehat{\mathcal{F}}_{\mathcal{X}, k, x_0}
$$
of groupoids cofibred over the completed category
$\widehat{\mathcal{C}}_\Lambda$ (see discussion in
Formal Deformation Theory, Section
\ref{formal-defos-section-prorepresentable-groupoids-in-functors}
as to why this works). Of course $A$ is a versal ring to
$\mathcal{X}$ at $x_0$.
\end{remark}








\section{Multiplicities of components of algebraic stacks}
\label{section-multiplicities}

\noindent
If $X$ is a locally Noetherian scheme, then we may write $X$ (thought
of simply as a topological space) as a union of irreducible components,
say $X = \bigcup T_i$.  Each irreducible component is the closure of a
unique generic point $\xi_i$, and the local ring $\mathcal O_{X,\xi_i}$
is a local Artin ring. We may define the {\it multiplicity of $X$ along $T_i$}
or the {\it multiplicity of $T_i$ in $X$} by
$$
m_{T_i, X} = \text{length}_{\mathcal O_{X, \xi_i}} \mathcal O_{X, \xi_i}
$$
In other words, it is the length of the local Artinian ring. Please
compare with
Chow Homology, Section \ref{chow-section-cycle-of-closed-subscheme}.

\medskip\noindent
Our goal here is to generalise this definition to locally
Noetherian algebraic stacks. If $\mathcal{X}$ is a stack,
then its topological space $|\mathcal{X}|$
(see Properties of Stacks, Definition
\ref{stacks-properties-definition-topological-space})
is locally Noetherian
(Morphisms of Stacks, Lemma \ref{stacks-morphisms-lemma-Noetherian-topology}).
The irreducible components of $|\mathcal{X}|$ are sometimes
referred to as the irreducible components of $\mathcal{X}$.
If $\mathcal{X}$ is quasi-separated, then $|\mathcal{X}|$
is sober (Morphisms of Stacks, Lemma
\ref{stacks-morphisms-lemma-sober-qs}),
but it need not be in the
non-quasi-separated case. Consider for example the non-quasi-separated
algebraic space $X = \mathbf{A}^1_\mathbf{C}/\mathbf{Z}$.
Furthermore, there is no structure sheaf
on $|\mathcal{X}|$ whose stalks can be used to define multiplicities.

\begin{lemma}
\label{lemma-map-of-components}
Let $f : U \to \mathcal{X}$ be a smooth morphism from a scheme
to a locally Noetherian algebraic stack. The closure of the image of any
irreducible component of $|U|$ is an irreducible component of $|\mathcal{X}|$.
If $U \to \mathcal{X}$ is surjective, then all irreducible components of
$|\mathcal{X}|$ are obtained in this way.
\end{lemma}

\begin{proof}
The map $|U| \to |\mathcal{X}|$ is continuous and open by
Properties of Stacks, Lemma \ref{stacks-properties-lemma-topology-points}.
Let $T \subset |U|$ be an irreducible component. Since $U$ is locally
Noetherian, we can find a nonempty affine open $W \subset U$ contained in $T$.
Then $f(T) \subset |\mathcal{X}|$ is irreducible and contains the
nonempty open subset $f(W)$. Thus the closure of $f(T)$ is irreducible and
contains a nonempty open. It follows that this closure is an irreducible
component.

\medskip\noindent
Assume $U \to \mathcal{X}$ is surjective and let $Z \subset |\mathcal{X}|$
be an irreducible component. Choose a Noetherian open subset $V$
of $|\mathcal{X}|$ meeting $Z$. After removing the other irreducible
components from $V$ we may assume that $V \subset Z$.
Take an irreducible component of the nonempty
open $f^{-1}(V) \subset |U|$ and let $T \subset |U|$ be its closure.
This is an irreducible component of $|U|$ and the closure of $f(T)$
must agree with $Z$ by our choice of $T$.
\end{proof}

\noindent
The preceding lemma applies in particular in the case of smooth morphisms
between locally Noetherian schemes. This particular case is
implicitly invoked in the statement of the following lemma.

\begin{lemma}
\label{lemma-multiplicities}
Let $U \to X$ be a smooth morphism of locally Noetherian schemes.
Let $T'$ is an irreducible component of $U$. Let $T$ be the
irreducible component of $X$ obtained as the closure of the
image of $T'$. Then $m_{T', U} = m_{T, X}$.
\end{lemma}

\begin{proof}
Write $\xi'$ for the generic point of $T'$, and $\xi$ for the
generic point of $T$. Let $A = \mathcal{O}_{X, \xi}$ and
$B = \mathcal{O}_{U, \xi'}$. We need to show that
$\text{length}_A A = \text{length}_B B$. Since
$A \to B$ is a flat local homomorphism of rings
(since smooth morphisms are flat), we have
$$
\text{length}_A(A) \text{length}_B(B/\mathfrak m_A B) =
\text{length}_B(B)
$$
by Algebra, Lemma \ref{algebra-lemma-pullback-module}. Thus it suffices
to show $\mathfrak m_A B = \mathfrak m_B$, or equivalently, that
$B/\mathfrak m_A B$ is reduced. Since $U \to X$ is smooth,
so is its base change $U_{\xi} \to \Spec \kappa(\xi)$. As $U_{\xi}$ is a
smooth scheme over a field, it is reduced, and thus so its local ring
at any point
(Varieties, Lemma \ref{varieties-lemma-smooth-geometrically-normal}).
In particular,
$$
B/\mathfrak m_A B =
\mathcal{O}_{U, \xi'}/\mathfrak m_{X, \xi}\mathcal{O}_{U, \xi'} =
\mathcal{O}_{U_\xi, \xi'}
$$
is reduced, as required.
\end{proof}

\noindent
Using this result, we may show that there exists a good notion
of multiplicity by looking smooth locally.

\begin{lemma}
\label{lemma-multiplicity}
Let $U_1 \to \mathcal{X}$ and $U_2 \to \mathcal{X}$ be two smooth
morphisms from schemes to a locally Noetherian algebraic stack $\mathcal{X}$.
Let $T_1'$ and $T_2'$ be irreducible components of $|U_1|$
and $|U_2|$ respectively. Assume the closures of the images of
$T_1'$ and $T_2'$ are the same irreducible component $T$ of $|\mathcal{X}|$.
Then $m_{T_1', U_1} = m_{T_2', U_2}$.
\end{lemma}

\begin{proof}
Let $V_1$ and $V_2$ be dense subsets of $T_1'$ and $T'_2$, respectively,
that are open in $U_1$ and $U_2$ respectively (see proof of
Lemma \ref{lemma-map-of-components}).
The images of $|V_1|$ and $|V_2|$ in $|\mathcal{X}|$ are non-empty open
subsets of the irreducible subset $T$, and therefore have non-empty
intersection. By
Properties of Stacks, Lemma \ref{stacks-properties-lemma-points-cartesian},
the map $|V_1 \times_\mathcal{X} V_2| \to |V_1| \times_{|\mathcal{X}|} |V_2|$
is surjective. Consequently $V_1 \times_\mathcal{X} V_2$
is a non-empty algebraic space; we may therefore choose an
\'etale surjection $V \to V_1 \times_\mathcal{X} V_2$
whose source is a (non-empty) scheme.
If we let $T'$ be any irreducible component of $V$,
then Lemma \ref{lemma-map-of-components} shows that the closure of
the image of $T'$ in $U_1$ (respectively $U_2$) is equal to $T'_1$
(respectively $T'_2$).

\medskip\noindent
Applying Lemma \ref{lemma-multiplicities} twice we find
that
$$
m_{T_1', U_1} = m_{T', V} = m_{T_2', U_2},
$$
as required.
\end{proof}

\noindent
At this point we have done enough work to show the following
definition makes sense.

\begin{definition}
\label{definition-multiplicity}
Let $\mathcal{X}$ be a locally Noetherian algebraic stack. Let
$T \subset |\mathcal{X}|$ be an irreducible component.
The {\it multiplicity} of $T$ in $\mathcal{X}$ is defined as
$m_{T, \mathcal{X}} = m_{T', U}$ where $f : U \to \mathcal{X}$
is a smooth morphism from a scheme and $T' \subset |U|$
is an irreducible component with $f(T') \subset T$.
\end{definition}

\noindent
This is independent of the choice of $f : U \to \mathcal{X}$
and the choice of the irreducible component $T'$ mapping to $T$
by Lemmas \ref{lemma-map-of-components} and \ref{lemma-multiplicity}.

\medskip\noindent
As a closing remark, we note that it is sometimes convenient to think
of an irreducible component of $\mathcal{X}$ as a closed substack.
To this end, if $T \subset |\mathcal{X}|$ is an irreducible component,
then we may consider the unique reduced closed substack
$\mathcal{T} \subset \mathcal{X}$ with $|\mathcal{T}| = T$, see
Properties of Stacks, Definition
\ref{stacks-properties-definition-reduced-induced-stack}.
If $\mathcal{X}$ is quasi-separated, then
an irreducible component is an integral stack; see
Morphisms of Stacks, Section \ref{stacks-morphisms-section-integral-stacks}
for further discussion.



\section{Formal branches and multiplicities}
\label{section-formal-branches}

\noindent
It will be convenient to have a comparison between the notion of multiplicity
of an irreducible component given by Definition \ref{definition-multiplicity}
and the related notion of multiplicities of irreducible components of
(the spectra of) versal rings of $\mathcal{X}$ at finite type points.

\medskip\noindent
In Situation \ref{situation-versal} let $x_0 : \Spec(k) \to \mathcal{X}$
be a morphism, where $k$ is a finite type field over $S$.
Let $A$, $A'$ be versal rings to $\mathcal{X}$ at $x_0$.
After possibly swapping $A$ and $A'$, we know there is a
formally smooth\footnote{In the sense that
$A'$ becomes isomorphic to a power series ring over $A$.}
map $\varphi : A \to A'$ compatible with versal formal objects, see
Lemma \ref{lemma-versal-ring} and
Remark \ref{remark-upgrade}.
Moreover, $\varphi$ is well defined up to formal homotopy, see
Formal Deformation Theory, Lemma
\ref{formal-defos-lemma-versal-unique-up-to-homotopy}.
In particular, we find that $\varphi(\mathfrak p)A'$ is a well
defined ideal of $A'$ by
Formal Deformation Theory, Lemma
\ref{formal-defos-lemma-homotopic-minimal-prime}.
Since $A \to A'$ is formally smooth, in fact
$\varphi(\mathfrak p)A'$ is a minimal prime of $A'$
and every minimal prime of $A'$ is of this form for
a unique minimal prime $\mathfrak p \subset A$ (all of this
is easy to prove by writing $A'$ as a power series ring over $A$).
Therefore, recalling that minimal primes correspond
to irreducible components, the following definition makes sense.

\begin{definition}
\label{definition-formal-branches}
Let $\mathcal{X}$ be an algebraic stack locally of finite type
over a locally Noetherian scheme $S$. Let $x_0 : \Spec(k) \to \mathcal{X}$
is a morphism where $k$ is a field of finite type over $S$.
The {\it formal branches of $\mathcal{X}$ through $x_0$}
is the set of irreducible components of $\Spec(A)$
for any choice of versal ring to $\mathcal{X}$ at $x_0$
identified for different choices of $A$ by the procedure described above.
\end{definition}

\noindent
Suppose in the situation of Definition \ref{definition-formal-branches}
we are given a finite extension $l/k$. Set
$x_{l, 0} : \Spec(l) \to \mathcal{X}$ equal to the composition of
$\Spec(l) \to \Spec(k)$ with $x_0$. Let
$A \to A'$ be as in Lemma \ref{lemma-versal-ring-field-extension}.
Since $A \to A'$ is faithfully flat, the morphism
$$
\Spec(A') \to \Spec(A)
$$
sends (generic points of) irreducible components to
(generic points of) irreducible components.
This will be a surjective map, but
in general this map will not be a bijection.
In other words, we obtain a surjective map
$$
\text{formal branches of }\mathcal{X}\text{ through }x_{l, 0}
\longrightarrow
\text{formal branches of }\mathcal{X}\text{ through }x_0
$$
It turns out that if $l/k$ is purely inseparable, then
the map is injective as well (we'll add a precise statement
and proof here if we ever need this).

\begin{lemma}
\label{lemma-branches}
In the situation of Definition \ref{definition-formal-branches}
there is a canonical surjection from the set of formal branches of
$\mathcal{X}$ through $x_0$ to the set of irreducible components of
$|\mathcal{X}|$ containing $x_0$ in $|\mathcal{X}|$.
\end{lemma}

\begin{proof}
Let $A$ be as in Definition \ref{definition-formal-branches} and
let $\Spec(A) \to \mathcal{X}$ be as in Remark \ref{remark-upgrade}.
We claim that the generic point of an irreducible component of
$\Spec(A)$ maps to a generic point of an irreducible component
of $|\mathcal{X}|$. Choose a scheme $U$ and a surjective
smooth morphism $U \to \mathcal{X}$. Consider the diagram
$$
\xymatrix{
\Spec(A) \times_\mathcal{X} U \ar[d]_p \ar[r]_-q & U \ar[d]^f \\
\Spec(A) \ar[r]^j & \mathcal{X}
}
$$
By Lemma \ref{lemma-versal-ring-flat} we see that $j$ is flat.
Hence $q$ is flat. On the other hand, $f$ is surjective smooth
hence $p$ is surjective smooth. This implies that any generic point
$\eta \in \Spec(A)$ of an irreducible component is the image of a
codimension $0$ point $\eta'$ of the algebraic space
$\Spec(A) \times_\mathcal{X} U$ (see
Properties of Spaces, Section \ref{spaces-properties-section-generic-points}
for notation and use going down on \'etale local rings).
Since $q$ is flat, $q(\eta')$ is a codimension $0$ point of $U$
(same argument).
Since $U$ is a scheme, $q(\eta')$ is the generic point
of an irreducible component of $U$. Thus the closure of the image
of $q(\eta')$ in $|\mathcal{X}|$ is an irreducible component
by Lemma \ref{lemma-map-of-components} as claimed.

\medskip\noindent
Clearly the claim provides a mechanism for defining the desired map.
To see that it is surjective, we choose $u_0 \in U$ mapping to
$x_0$ in $|\mathcal{X}|$. Choose an affine open $U' \subset U$
neighbourhood of $u_0$. After shrinking $U'$ we may assume every
irreducible component of $U'$ passes through $u_0$. Then we may
replace $\mathcal{X}$ by the open substack corresponding to the
image of $|U'| \to |\mathcal{X}|$. Thus we may assume $U$ is affine
has a point $u_0$ mapping to $x_0 \in |\mathcal{X}|$
and every irreducible component of $U$ passes through $u_0$.
By Properties of Stacks, Lemma
\ref{stacks-properties-lemma-points-cartesian}
there is a point $t \in |\Spec(A) \times_\mathcal{X} U|$
mapping to the closed point of $\Spec(A)$ and to $u_0$.
Using going down for the flat local ring homomorphisms
$$
A \longrightarrow
\mathcal{O}_{\Spec(A) \times_\mathcal{X} U, \overline{t}}
\longleftarrow \mathcal{O}_{U, u_0}
$$
we see that every minimal prime of $\mathcal{O}_{U, u_0}$
is the image of a minimal prime of the local ring in the middle
and such a minimal prime maps to a minimal prime of $A$.
This proves the surjectivity. Some details omitted.
\end{proof}

\noindent
Let $A$ be a Noetherian complete local ring. Then the irreducible
components of $\Spec(A)$ have multiplicities, see
introduction to Section \ref{section-multiplicities}.
If $A' = A[[t_1, \ldots, t_r]]$, then the morphism
$\Spec(A') \to \Spec(A)$ induces a bijection on irreducible
components preserving multiplicities (we omit the easy proof).
This and the discussion preceding
Definition \ref{definition-formal-branches}
mean that the following definition makes sense.

\begin{definition}
\label{definition-multiplicity-formal-branches}
Let $\mathcal{X}$ be an algebraic stack locally of finite type
over a locally Noetherian scheme $S$. Let $x_0 : \Spec(k) \to \mathcal{X}$
is a morphism where $k$ is a field of finite type over $S$.
The {\it multiplicity of a formal branch of $\mathcal{X}$ through $x_0$}
is the multiplicity of the corresponding irreducible component of
$\Spec(A)$ for any choice of versal ring to $\mathcal{X}$ at $x_0$
(see discussion above).
\end{definition}

\begin{lemma}
\label{lemma-branches-multiplicity}
Let $\mathcal{X}$ be an algebraic stack locally of finite type
over a locally Noetherian scheme $S$. Let $x_0 : \Spec(k) \to \mathcal{X}$
is a morphism where $k$ is a field of finite type over $S$ with
image $s \in S$. If $\mathcal{O}_{S, s}$ is a G-ring, then
the map of Lemma \ref{lemma-branches} preserves multiplicities.
\end{lemma}

\begin{proof}
By Lemma \ref{lemma-Artin-approximation-by-smooth-morphism} we
may assume there is a smooth morphism $U \to \mathcal{X}$
where $U$ is a scheme and a $k$-valued point $u_0$ of $U$
such that $\mathcal{O}_{U, u_0}^\wedge$ is a versal ring to
$\mathcal{X}$ at $x_0$. By construction of our map in
the proof of Lemma \ref{lemma-branches} (which simplifies
greatly because $A = \mathcal{O}_{U, u_0}^\wedge$) we find
that it suffices to show: the multiplicity of an irreducible
component of $U$ passing through $u_0$ is the same as the
multiplicity of any irreducible component of
$\Spec(\mathcal{O}_{U, u_0}^\wedge)$ mapping into it.

\medskip\noindent
Translated into commutative algebra we find the following:
Let $C = \mathcal{O}_{U, u_0}$. This is essentially of finite type
over $\mathcal{O}_{S, s}$ and hence is a G-ring (More on Algebra,
Proposition \ref{more-algebra-proposition-finite-type-over-G-ring}).
Then $A = C^\wedge$. Therefore $C \to A$ is a regular ring map.
Let $\mathfrak q \subset C$ be a minimal prime and let $\mathfrak p \subset A$
be a minimal prime lying over $\mathfrak q$. Then
$$
R = C_\mathfrak p \longrightarrow A_\mathfrak p = R'
$$
is a regular ring map of Artinian local rings. For such a ring map
it is always the case that
$$
\text{length}_R R = \text{length}_{R'} R'
$$
This is what we have to show because the left hand side
is the multiplicity of our component on $U$ and the right hand
side is the multiplicity of our component on $\Spec(A)$.
To see the equality, first we use that
$$
\text{length}_R(R) \text{length}_{R'}(R'/\mathfrak m_R R') =
\text{length}_{R'}(R')
$$
by Algebra, Lemma \ref{algebra-lemma-pullback-module}. Thus it suffices
to show $\mathfrak m_R R' = \mathfrak m_{R'}$, which is a consequence
of being a regular homomorphism of zero dimensional local rings.
\end{proof}













\section{Dimension theory of algebraic stacks}
\label{section-dimension-of-algebraic-stacks}

\noindent
The main results on the dimension theory of algebraic stacks in the
literature that we are aware of are those of \cite{Osserman}, which
makes a study of the notions of codimension and relative dimension. We
make a more detailed examination of the notion of the dimension of an
algebraic stack at a point, and prove various results
relating the dimension of the fibres of a morphism at a point in the source
to the dimension of its source and target.  We also prove a result
(Lemma \ref{lemma-dimension-formula} below) which
allow us (under suitable hypotheses) to compute the dimension of
an algebraic stack at a point in terms of a versal ring.

\medskip\noindent
While we haven't always tried to optimise our results, we have
largely tried to avoid making unnecessary hypotheses.  However, in some
of our results, in which we compare certain properties of an algebraic
stack to the properties of a versal ring to this
stack at a point, we have restricted our attention
to the case of algebraic stacks that are locally finitely presented
over a locally Noetherian scheme base, all of whose local rings are
$G$-rings. This gives us the convenience of having Artin approximation
available to compare the geometry of the versal ring to the geometry
of the stack itself.  However, this restrictive hypothesis
may not be necessary for the truth
of all of the various statements that we prove.
Since it is satisfied in the applications that we have in mind,
though, we have been content to make it when it helps.

\medskip\noindent
If $X$ is a scheme, then we define the dimension $\dim(X)$ of $X$
to be the Krull dimension of the
topological space underlying $X$,
while if $x$ is a point of $X$,
then we define the dimension $\dim_x (X)$ of $X$ at $x$ to be the
minimum of the dimensions of the open subsets $U$ of $X$ containing
$x$, see
Properties, Definition \ref{properties-definition-dimension}.
One has the relation $\dim(X) = \sup_{x \in X} \dim_x(X)$, see
Properties, Lemma \ref{properties-lemma-dimension}.
If $X$ is locally Noetherian, then $\dim_x(X)$ coincides with the supremum
of the dimensions at $x$
of the irreducible components of $X$ passing through $x$.

\medskip\noindent
If $X$ is an algebraic space and $x \in |X|$,
then we define $\dim_x X = \dim_u U,$ where $U$ is any scheme
admitting an \'etale surjection $U \to X$,
and $u\in U$ is any point lying over $x$, see
Properties of Spaces, Definition
\ref{spaces-properties-definition-dimension-at-point}.
We set $\dim(X) = \sup_{x \in |X|} \dim_x(X)$, see
Properties of Spaces, Definition \ref{spaces-properties-definition-dimension}.

\begin{remark}
\label{remark-dimension-algebraic-space}
In general, the dimension of the algebraic space $X$ at a point $x$
may not coincide with the dimension of the underlying topological space
$|X|$ at $x$.  E.g.\ if $k$ is a field of characteristic zero and
$X =  \mathbf{A}^1_k / \mathbf{Z}$, then $X$ has dimension $1$ (the dimension
of $\mathbf{A}^1_k$) at each of its points,
while $|X|$ has the indiscrete topology, and hence is of Krull
dimension zero. On the other hand, in
Algebraic Spaces, Example \ref{spaces-example-infinite-product}
there is given an example of an algebraic space
which is of dimension $0$ at each of its points, while $|X|$ is
irreducible of Krull dimension $1$, and admits a generic point (so that the
dimension of $|X|$ at any of its points is $1$); see also the discussion
of this example in
Properties of Spaces, Section \ref{spaces-properties-section-dimension}.

\medskip\noindent
On the other hand, if $X$ is a {\it decent} algebraic space, in the sense of
Decent Spaces, Definition \ref{decent-spaces-definition-very-reasonable}
(in particular, if $X$ is quasi-separated; see
Decent Spaces, Section \ref{decent-spaces-section-reasonable-decent})
then in fact the dimension of $X$ at $x$ does coincide with the dimension
of $|X|$ at $x$; see
Decent Spaces, Lemma \ref{decent-spaces-lemma-dimension-decent-space}.
\end{remark}

\noindent
In order to  define the dimension of an algebraic stack,
it will be useful to first have the notion of the relative dimension,
at a point in the source,
of a morphism whose source is an algebraic space,
and whose target is an algebraic stack.  The definition is slightly
involved, just because (unlike in the case of schemes) the points of an
algebraic stack, or an algebraic
space, are not describable as morphisms from the spectrum of a field,
but only as equivalence classes of such.

\begin{definition}
\label{definition-relative-dimension}
If $f : T \to \mathcal{X}$ is a locally of finite type morphism from an
algebraic space to an algebraic stack,
and if $t \in |T|$ is a point with image $x \in | \mathcal{X}|$, then we define
{\it the relative dimension} of $f$ at $t$, denoted
$\dim_t(T_x),$
as follows:
choose a morphism $\Spec k \to \mathcal{X}$, with source the spectrum of
a field, which represents $x$, and choose a point
$t' \in |T \times_{\mathcal{X}} \Spec k|$
mapping to $t$ under the projection to $|T|$
(such a point $t'$ exists, by
Properties of Stacks, Lemma \ref{stacks-properties-lemma-points-cartesian});
then
$$
\dim_t(T_x) = \dim_{t'}(T \times_{\mathcal{X}} \Spec k ).
$$
\end{definition}

\noindent
Note that since $T$ is an algebraic space and $\mathcal{X}$ is an
algebraic stack, the fibre product $T \times_{\mathcal{X}} \Spec k$
is an algebraic space, and so the quantity on the right hand side of
this proposed definition is in fact defined (see discussion above).

\begin{remark}
\label{remark-relative-dimension}
(1)
One easily verifies (for example, by using the invariance
of the relative dimension of locally of finite type morphisms of schemes
under base-change; see for example
Morphisms, Lemma \ref{morphisms-lemma-dimension-fibre-after-base-change})
that $\dim_t(T_x)$ is well-defined, independently of the choices
used to compute it.

\medskip\noindent
(2)
In the case that $\mathcal{X}$ is also an algebraic space,
it is straightforward to confirm that this definition agrees with
the definition of relative dimension given in
Morphisms of Spaces, Definition
\ref{spaces-morphisms-definition-dimension-fibre}.
\end{remark}

\noindent
We next recall the following lemma, on which our study of
the dimension of a locally Noetherian algebraic stack is founded.

\begin{lemma}
\label{lemma-behaviour-of-dimensions-wrt-smooth-morphisms}
If $f: U \to X$ is a smooth morphism of locally Noetherian algebraic
spaces, and
if $u \in |U|$ with image $x \in |X|$, then
$$
\dim_u (U) = \dim_x(X) + \dim_{u} (U_x)
$$
where $\dim_u (U_x)$ is defined via
Definition \ref{definition-relative-dimension}.
\end{lemma}

\begin{proof}
See Morphisms of Spaces, Lemma
\ref{spaces-morphisms-lemma-smoothness-dimension-spaces}
noting that the definition of $\dim_u (U_x)$ used here coincides with
the definition used there, by Remark \ref{remark-relative-dimension} (2).
\end{proof}

\begin{lemma}
\label{lemma-dimension-for-stacks}
If $\mathcal{X}$ is a locally Noetherian algebraic stack and
$x \in |\mathcal{X}|$. Let $U \to \mathcal{X}$ be a smooth morphism
from an algebraic space to $\mathcal{X}$, let $u$ be any point of $|U|$
mapping to $x$. Then we have
$$
\dim_x(\mathcal{X}) =  \dim_u(U) - \dim_{u}(U_x)
$$
where the relative dimension $\dim_u(U_x)$ is defined
by Definition \ref{definition-relative-dimension} and the
dimension of $\mathcal{X}$ at $x$ is as in
Properties of Stacks, Definition
\ref{stacks-properties-definition-dimension-at-point}.
\end{lemma}

\begin{proof}
Lemma \ref{lemma-behaviour-of-dimensions-wrt-smooth-morphisms}
can be used to verify that the right hand side
$\dim_u(U) + \dim_u(U_x)$ is independent of the choice
of the smooth morphism $U \to \mathcal{X}$ and $u \in |U|$.
We omit the details. In particular, we may assume $U$ is a scheme.
In this case we can compute $\dim_u(U_x)$
by choosing the representative of $x$
to be the composite $\Spec \kappa(u) \to U \to \mathcal{X}$, where
the first morphism is the canonical one with image $u \in U$.
Then, if we write $R = U \times_{\mathcal{X}} U$, and let
$e : U \to R$ denote the diagonal morphism, the invariance of
relative dimension under base-change shows that
$\dim_u(U_x) = \dim_{e(u)}(R_u)$. Thus we see that
the right hand side is equal to
$\dim_u (U) - \dim_{e(u)}(R_u) = \dim_x(\mathcal{X})$ as desired.
\end{proof}

\begin{remark}
\label{remark-dimension-DM}
For Deligne--Mumford stacks which are suitably decent
(e.g.\ quasi-separated),
it will again be the case that $\dim_x(\mathcal{X})$ coincides with the
topologically
defined quantity $\dim_x |\mathcal{X}|$.  However, for more general Artin
stacks,
this will typically not be the case.  For example, if
$\mathcal{X} = [\mathbf{A}^1/\mathbf{G}_m]$
(over some field, with the quotient being taken with
respect to the usual multiplication action of $\mathbf{G}_m$ on $\mathbf{A}^1$),
then  $|\mathcal{X}|$ has two points, one the specialisation of the other
(corresponding
to the two orbits of $\mathbf{G}_m$ on $\mathbf{A}^1$), and hence is of
dimension $1$ as
a topological space; but $\dim_x (\mathcal{X}) = 0$ for both points
$x \in |\mathcal{X}|$.
(An even more extreme example is given by the classifying space
$[\Spec k/\mathbf{G}_m]$, whose dimension at its unique point
is equal to $-1$.)
\end{remark}

\noindent
We can now extend Definition \ref{definition-relative-dimension}
to the context of (locally finite type)
morphisms between (locally Noetherian) algebraic stacks.

\begin{definition}
\label{definition-relative-dimension-for-stacks}
If $f : \mathcal{T} \to \mathcal{X}$
is a locally of finite type morphism between
locally Noetherian algebraic stacks, and if
$t \in |\mathcal{T}|$ is a point with image $x \in |\mathcal{X}|$, then
we define the {\it relative dimension} of $f$ at $t$, denoted
$\dim_t(\mathcal{T}_x),$ as follows:
choose a morphism $\Spec k \to \mathcal{X}$, with source the spectrum of
a field, which represents $x$, and choose a point
$t' \in |\mathcal{T} \times_{\mathcal{X}} \Spec k|$
mapping to $t$ under the projection to $|\mathcal{T}|$
(such a point $t'$ exists, by
Properties of Stacks, Lemma
\ref{stacks-properties-lemma-points-cartesian}; then
$$
\dim_t(\mathcal{T}_x) = \dim_{t'}(\mathcal{T} \times_{\mathcal{X}} \Spec k ).
$$
\end{definition}

\noindent
Note that since $\mathcal{T}$ is an algebraic stack and $\mathcal{X}$ is an
algebraic stack, the fibre product $\mathcal{T}\times_{\mathcal{X}} \Spec k$
is an algebraic stack, which is locally Noetherian by
Morphisms of Stacks, Lemma
\ref{stacks-morphisms-lemma-locally-finite-type-locally-noetherian}.
Thus the quantity on the right side of this proposed definition
is defined by Properties of Stacks,
Definition \ref{stacks-properties-definition-dimension-at-point}.

\begin{remark}
\label{remark-dimension-tangent-space-well-defined}
Standard manipulations show that $\dim_t(\mathcal{T}_x)$ is well-defined,
independently of the choices made to compute it.
\end{remark}

\noindent
We now establish some basic properties of relative dimension, which
are obvious generalisations of the corresponding statements in the
case of morphisms of schemes.

\begin{lemma}
\label{lemma-base-change-invariance-of-relative-dimension}
Suppose given
a Cartesian square of morphisms of locally Noetherian stacks
$$
\xymatrix{
\mathcal{T}' \ar[d]\ar[r] & \mathcal{T} \ar[d] \\
\mathcal{X}' \ar[r] & \mathcal{X}
}
$$
in which the vertical morphisms are locally of finite type.
If $t' \in |\mathcal{T}'|$,
with images $t$, $x'$, and $x$ in $|\mathcal{T}|$, $|\mathcal{X}'|$, and
$|\mathcal{X}|$
respectively, then $\dim_{t'}(\mathcal{T}'_{x'}) = \dim_{t}(\mathcal{T}_x).$
\end{lemma}

\begin{proof}
Both sides can (by definition) be computed as the
dimension of the same fibre product.
\end{proof}

\begin{lemma}
\label{lemma-behaviour-of-dimensions-wrt-smooth-morphisms-stacky}
If $f: \mathcal{U} \to \mathcal{X}$ is a smooth morphism of locally Noetherian
algebraic stacks, and
if $u \in |\mathcal{U}|$ with image $x \in |\mathcal{X}|$,
then
$$
\dim_u (\mathcal{U}) = \dim_x(\mathcal{X}) + \dim_{u} (\mathcal{U}_x).
$$
\end{lemma}

\begin{proof}
Choose a smooth surjective morphism $V \to \mathcal{U}$ whose source
is a scheme, and let $v\in |V|$ be a point mapping to $u$.
Then the composite $V \to \mathcal{U} \to \mathcal{X}$ is also smooth,
and by Lemma \ref{lemma-behaviour-of-dimensions-wrt-smooth-morphisms}
we have $\dim_x(\mathcal{X}) = \dim_v(V) - \dim_v(V_x),$
while $\dim_u(\mathcal{U}) = \dim_v(V) - \dim_v(V_u).$
Thus
$$
\dim_u(\mathcal{U}) - \dim_x(\mathcal{X}) = \dim_v (V_x) - \dim_v (V_u).
$$

\medskip\noindent
Choose a representative $\Spec k \to \mathcal{X}$ of $x$
and choose a point $v' \in | V \times_{\mathcal{X}} \Spec k|$ lying over
$v$, with image $u'$ in $|\mathcal{U}\times_{\mathcal{X}} \Spec k|$;
then by definition
$\dim_u(\mathcal{U}_x) = \dim_{u'}(\mathcal{U}\times_{\mathcal{X}} \Spec k),$
and
$\dim_v(V_x) = \dim_{v'}(V\times_{\mathcal{X}} \Spec k).$

\medskip\noindent
Now $V\times_{\mathcal{X}} \Spec k \to \mathcal{U}\times_{\mathcal{X}}\Spec k$
is a smooth surjective morphism (being the base-change
of such a morphism) whose source is an algebraic space
(since $V$ and $\Spec k$ are schemes, and $\mathcal{X}$
is an algebraic stack).  Thus, again by definition,
we have
\begin{align*}
\dim_{u'}(\mathcal{U}\times_{\mathcal{X}} \Spec k)
& =
\dim_{v'}(V\times_{\mathcal{X}} \Spec k) -
\dim_{v'}(V \times_{\mathcal{X}} \Spec k)_{u'}) \\
& = \dim_v(V_x) -
\dim_{v'}( (V\times_{\mathcal{X}} \Spec k)_{u'}).
\end{align*}
Now $V\times_{\mathcal{X}} \Spec k \cong
V\times_{\mathcal{U}} (\mathcal{U}\times_{\mathcal{X}} \Spec k),$
and so
Lemma \ref{lemma-base-change-invariance-of-relative-dimension}
shows that
$\dim_{v'}((V\times_{\mathcal{X}} \Spec k)_{u'})  = \dim_v(V_u).$
Putting everything together, we find that
$$
\dim_u(\mathcal{U}) - \dim_x(\mathcal{X}) =
\dim_u(\mathcal{U}_x),
$$
as required.
\end{proof}

\begin{lemma}
\label{lemma-relative-dimension-is-semi-continuous}
Let $f: \mathcal{T} \to \mathcal{X}$ be a locally of finite type morphism of
algebraic stacks.
\begin{enumerate}
\item
The function $t \mapsto \dim_t(\mathcal{T}_{f(t)})$ is upper semi-continuous
on $|\mathcal{T}|$.
\item If $f$ is smooth, then
the function $t \mapsto \dim_t(\mathcal{T}_{f(t)})$ is locally constant
on $|\mathcal{T}|$.
\end{enumerate}
\end{lemma}

\begin{proof}
Suppose to begin with that $\mathcal{T}$ is a scheme $T$,
let $U \to \mathcal{X}$ be a smooth surjective morphism whose source
is a scheme, and let $T' = T \times_{\mathcal{X}} U$.
Let $f': T' \to U$ be the pull-back of $f$ over $U$,
and let $g: T' \to T$ be the projection.

\medskip\noindent
Lemma \ref{lemma-base-change-invariance-of-relative-dimension}
shows that $\dim_{t'}(T'_{f'(t')}) = \dim_{g(t')}(T_{f(g(t'))}),$
for $t' \in T'$, while,
since $g$ is smooth and surjective (being the base-change
of a smooth surjective morphism) the map induced by $g$ on underlying
topological spaces is continuous and open
(by
Properties of Spaces, Lemma \ref{spaces-properties-lemma-topology-points}), and
surjective. Thus it suffices to note that part (1) for the morphism $f'$
follows from
Morphisms of Spaces, Lemma
\ref{spaces-morphisms-lemma-openness-bounded-dimension-fibres}, and part (2)
from either of Morphisms, Lemma
\ref{morphisms-lemma-flat-finite-presentation-CM-fibres-relative-dimension}
or
Morphisms, Lemma \ref{morphisms-lemma-smooth-omega-finite-locally-free}
(each of which gives the result for schemes, from which
the analogous results for algebraic spaces can
be deduced exactly as in
Morphisms of Spaces, Lemma
\ref{spaces-morphisms-lemma-openness-bounded-dimension-fibres}.

\medskip\noindent
Now return to the general case,
and choose a smooth surjective morphism
$h:V \to \mathcal{T}$ whose source is a scheme.
If $v \in V$, then, essentially by definition,
we have
$$
\dim_{h(v)}(\mathcal{T}_{f(h(v))}) =
\dim_{v}(V_{f(h(v))}) - \dim_{v}(V_{h(v)}).
$$
Since $V$ is a scheme, we have proved that the first
of the terms on the right hand side of this equality
is upper semi-continuous (and even locally
constant if $f$ is smooth), while the second term is
in fact locally constant.
Thus their difference is upper semi-continuous
(and locally constant if $f$ is smooth),
and hence the function
$\dim_{h(v)}(\mathcal{T}_{f(h(v))})$
is upper semi-continuous on $|V|$ (and locally
constant if $f$ is smooth).
Since the morphism $|V| \to |\mathcal{T}|$ is open and surjective,
the lemma follows.
\end{proof}

\noindent
Before continuing with our development,
we prove two lemmas related to the dimension theory of schemes.

\medskip\noindent
To put the first lemma in context,
we note that if $X$ is a finite dimensional scheme, then since $\dim X$
is defined to equal the supremum of the dimensions $\dim_x X$,
there exists a point $x \in X$ such that $\dim_x X = \dim X$.
The following lemma shows that we may furthermore take the point
$x$ to be of finite type.

\begin{lemma}
\label{lemma-dimension-achieved-by-finite-type-point}
If $X$ is a finite dimensional scheme,
then there exists a closed (and hence finite type) point $x \in X$
such that $\dim_x X = \dim X$.
\end{lemma}

\begin{proof}
Let $d = \dim X$,
and choose a maximal strictly decreasing
chain of irreducible closed subsets of $X$,
say
\begin{equation}
\label{equation-maximal-chain}
Z_0 \supset Z_1 \supset \ldots \supset Z_d.
\end{equation}
The subset $Z_d$ is a minimal irreducible closed subset of $X$,
and thus any point of $Z_d$ is a generic point of $Z_d$.
Since the underlying topological space of the scheme $X$ is sober,
we conclude that $Z_d$ is a singleton, consisting of a single
closed point $x \in X$.
If $U$ is
any neighbourhood of $x$, then
the chain
$$
U\cap Z_0 \supset U\cap Z_1 \supset \ldots \supset U\cap Z_d = Z_d =
\{x\}
$$
is then a strictly descending chain of irreducible
closed subsets of $U$, showing that $\dim U \geq d$.
Thus we find that $\dim_x X \geq d$.  The other inequality
being obvious, the lemma is proved.
\end{proof}

\noindent
The next lemma shows that $\dim_x X$ is a {\it constant} function
on an irreducible scheme satisfying some mild additional hypotheses.

\begin{lemma}
\label{lemma-constancy-of-dimension}
If $X$ is an irreducible, Jacobson, catenary, and locally Noetherian
scheme of finite dimension,
then $\dim U = \dim X$ for every
non-empty open subset $U$ of $X$.
Equivalently, $\dim_x X$ is a constant function on $X$.
\end{lemma}

\begin{proof}
The equivalence of the two claims follows directly from the
definitions.   Suppose, then, that $U\subset X$ is a non-empty open
subset.
Certainly $\dim U \leq \dim X$, and we have to show
that $\dim U \geq \dim X.$
Write $d = \dim X$, and choose a maximal strictly
decreasing chain of irreducible closed subsets
of $X$, say
$$
X = Z_0 \supset Z_1 \supset \ldots \supset Z_d.
$$
Since $X$ is Jacobson, the minimal irreducible closed
subset $Z_d$ is equal to $\{x\}$ for some closed
point $x$.

\medskip\noindent
If $x \in U,$ then
$$
U = U \cap Z_0  \supset U\cap Z_1 \supset \ldots \supset
U\cap Z_d = \{x\}
$$
is a strictly decreasing chain of irreducible closed
subsets of $U$, and so we conclude that $\dim U \geq d$,
as required.  Thus we may suppose that $x \not\in U.$

\medskip\noindent
Consider the flat morphism $\Spec \mathcal{O}_{X,x} \to X$.
The non-empty (and hence dense) open subset $U$ of $X$
pulls back to an open subset $V \subset \Spec \mathcal{O}_{X,x}$.
Replacing $U$ by a non-empty quasi-compact, and hence
Noetherian, open subset, we may assume that the inclusion
$U \to X$ is a quasi-compact morphism.  Since the
formation of scheme-theoretic images of quasi-compact
morphisms commutes with flat base-change
Morphisms, Lemma
\ref{morphisms-lemma-flat-base-change-scheme-theoretic-image}
we see that $V$ is dense in $\Spec \mathcal{O}_{X,x}$,
and so in particular non-empty,
and of course $x \not\in V.$  (Here we use $x$ also to denote
the closed point of $\Spec \mathcal{O}_{X,x}$, since its image
is equal to the given point $x \in X$.)
Now $\Spec \mathcal{O}_{X,x} \setminus \{x\}$ is Jacobson
Properties, Lemma
\ref{properties-lemma-complement-closed-point-Jacobson}
and hence $V$ contains a closed point $z$
of $\Spec \mathcal{O}_{X,x} \setminus \{x\}$.  The closure
in $X$ of the image of $z$ is then an irreducible
closed subset $Z$ of $X$ containing $x$, whose intersection
with $U$ is non-empty, and
for which there is no irreducible closed
subset properly contained in $Z$
and properly containing $\{x\}$
(because pull-back to $\Spec \mathcal{O}_{X,x}$ induces
a bijection between irreducible closed subsets of $X$
containing $x$ and irreducible closed subsets of $\Spec
\mathcal{O}_{X,x}$).
Since $U \cap Z$ is a non-empty closed subset of $U$,
it contains a point $u$ that is closed in $X$ (since
$X$ is Jacobson), and since $U\cap Z$
is a non-empty (and hence dense) open subset of the irreducible set $Z$
(which contains a point not lying in $U$, namely $x$),
the inclusion $\{u\} \subset U\cap Z$ is proper.

\medskip\noindent
As $X$ is catenary, the chain
$$
X = Z_0 \supset Z \supset \{x\} = Z_d
$$
can be refined to a chain of length $d+1$, which must then
be of the form
$$
X = Z_0 \supset W_1 \supset \ldots \supset W_{d-1} = Z \supset \{x\} = Z_d.
$$
Since $U\cap Z$ is non-empty, we then find that
$$
U = U \cap Z_0 \supset U \cap W_1\supset \ldots \supset U\cap W_{d-1}
= U\cap Z \supset \{u\}
$$
is a strictly decreasing chain of irreducible closed subsets
of $U$ of length $d+1$, showing that $\dim U \geq d$,
as required.
\end{proof}

\noindent
We will prove a stack-theoretic analogue
of Lemma \ref{lemma-constancy-of-dimension}
in Lemma \ref{lemma-irreducible-implies-equidimensional} below,
but before doing so, we have to introduce an additional definition,
necessitated by the fact that the notion of a scheme being catenary
is not an \'etale local one
(see the example of
Algebra, Remark \ref{algebra-remark-universally-catenary-does-not-descend}
which makes it difficult to define what it means for an algebraic
space or algebraic stack to be catenary
(see the discussion of \cite[page 3]{Osserman}).
For certain aspects of dimension theory, the following
definition seems to provide a good substitute for the missing
notion of a catenary algebraic stack.

\begin{definition}
\label{definition-pseudo-catenary}
We say that a locally Noetherian algebraic stack $\mathcal{X}$
is {\it pseudo-catenary} if there exists a smooth
and surjective morphism $U \to \mathcal{X}$ whose source is
a universally catenary scheme.
\end{definition}

\begin{example}
\label{example-pseudo-catenary}
If $\mathcal{X}$ is locally of finite type over a universally
catenary locally Noetherian scheme $S$,
and $U\to \mathcal{X}$ is a smooth surjective morphism
whose source is a scheme, then the composite
$U \to \mathcal{X} \to S$ is locally of finite type,
and so $U$ is universally catenary
Morphisms, Lemma
\ref{morphisms-lemma-universally-catenary-local}.
Thus $\mathcal{X}$ is pseudo-catenary.
\end{example}

\noindent
The following lemma shows that the property of being pseudo-catenary
passes through finite-type morphisms.

\begin{lemma}
\label{lemma-catenary-covers}
If $\mathcal{X}$ is a pseudo-catenary locally Noetherian algebraic
stack, and if $\mathcal{Y} \to \mathcal{X}$ is a locally of finite type
morphism,
then there exists a smooth surjective morphism $V \to \mathcal{Y}$
whose source is a universally catenary scheme; thus
$\mathcal{Y}$ is again pseudo-catenary.
\end{lemma}

\begin{proof}
By assumption we may find a smooth surjective morphism
$U \to \mathcal{X}$ whose source is a universally catenary scheme.
The base-change $U\times_{\mathcal{X}} \mathcal{Y}$ is then an algebraic
stack; let $V \to U\times_{\mathcal{X}} \mathcal{Y}$ be a smooth
surjective morphism whose source is a scheme.
The composite $V \to U\times_{\mathcal{X}} \mathcal{Y} \to \mathcal{Y}$ is then
smooth and surjective (being a composite of smooth and
surjective morphisms), while the morphism $V \to U\times_{\mathcal{X}}
\mathcal{Y} \to U$ is locally of finite type (being a composite
of morphisms that are locally finite type).  Since $U$
is universally catenary, we see that $V$ is universally catenary
(by Morphisms, Lemma
\ref{morphisms-lemma-universally-catenary-local}),
as claimed.
\end{proof}

\noindent
We now study the behaviour of the function $\dim_x(\mathcal{X})$ on
$|\mathcal{X}|$
(for some locally Noetherian stack $\mathcal{X}$) with respect to the
irreducible
components of $|\mathcal{X}|$, as well as various
related topics.

\begin{lemma}
\label{lemma-irreducible-implies-equidimensional}
If $\mathcal{X}$ is
a Jacobson, pseudo-catenary, and locally Noetherian  algebraic stack
for which $|\mathcal{X}|$ is irreducible,
then $\dim_x(\mathcal{X})$ is a constant function on $|\mathcal{X}|$.
\end{lemma}

\begin{proof}
It suffices to show that $\dim_x(\mathcal{X})$ is locally constant on
$|\mathcal{X}|$,
since it will then necessarily be constant (as $|\mathcal{X}|$ is connected,
being irreducible).  Since $\mathcal{X}$ is pseudo-catenary,
we may find a smooth surjective morphism $U \to \mathcal{X}$ with $U$
being a universally catenary scheme.  If $\{U_i\}$ is an
cover of $U$ by quasi-compact open subschemes, we may replace
$U$ by $\coprod U_i,$, and
it suffices to show that
the function $u \mapsto \dim_{f(u)}(\mathcal{X})$ is locally constant on $U_i$.
Since we check this for one $U_i$ at a time, we now drop the subscript,
and write simply $U$ rather than $U_i$.
Since $U$ is quasi-compact, it
is the union of a finite number of irreducible components,
say $T_1 \cup \ldots \cup T_n$.  Note that each $T_i$ is Jacobson,
catenary, and locally Noetherian,
being a closed subscheme of the Jacobson, catenary, and locally Noetherian
scheme $U$.

\medskip\noindent
By Lemma \ref{lemma-behaviour-of-dimensions-wrt-smooth-morphisms}, we have
$\dim_{f(u)}(\mathcal{X}) = \dim_{u}(U) - \dim_{u}(U_{f(u)}).$
Lemma \ref{lemma-relative-dimension-is-semi-continuous} (2)
shows that the second term in the right hand expression is locally
constant on $U$, as $f$ is smooth,
and hence we must show that $\dim_u(U)$
is locally constant on $U$.  Since $\dim_u(U)$ is the maximum
of the dimensions $\dim_u T_i$, as $T_i$ ranges over the components
of $U$ containing $u$, it suffices to show
that if a point $u$ lies on two distinct components,
say $T_i$ and $T_j$ (with $i \neq j$),
then $\dim_u T_i = \dim_u T_j$,
and then to note that $t\mapsto \dim_t T$ is a constant
function on an irreducible Jacobson,
catenary, and locally Noetherian scheme $T$
(as follows from Lemma \ref{lemma-constancy-of-dimension}).

\medskip\noindent
Let $V = T_i \setminus (\bigcup_{i' \neq i} T_{i'})$
and $W = T_j \setminus (\bigcup_{i' \neq j} T_{i'})$.
Then each of $V$ and $W$ is a non-empty open subset of $U$,
and so each has non-empty open image in $|\mathcal{X}|$.  As $|\mathcal{X}|$ is
irreducible,
these two non-empty open subsets of $|\mathcal{X}|$ have a non-empty
intersection.
Let $x$ be a point lying in this intersection, and let $v \in V$ and
$w\in W$ be points mapping to $x$.
We then find that
$$
\dim T_i = \dim V = \dim_v (U) = \dim_x (\mathcal{X}) + \dim_v (U_x)
$$
and similarly that
$$
\dim T_j = \dim W = \dim_w (U) = \dim_x (\mathcal{X}) + \dim_w (U_x).
$$
Since $u \mapsto \dim_u (U_{f(u)})$ is locally constant on $U$,
and since $T_i \cup T_j$ is connected (being the union of two irreducible,
hence connected, sets that have non-empty intersection),
we see that $\dim_v (U_x) = \dim_w(U_x)$,
and hence, comparing the preceding two equations,
that $\dim T_i = \dim T_j$, as required.
\end{proof}

\begin{lemma}
\label{lemma-closed-immersions}
If $\mathcal{Z} \hookrightarrow \mathcal{X}$ is a closed immersion
of locally Noetherian algebraic stacks,
and if $z \in |\mathcal{Z}|$ has image $x \in |\mathcal{X}|$,
then $\dim_z (\mathcal{Z}) \leq \dim_x(\mathcal{X})$.
\end{lemma}

\begin{proof}
Choose a smooth surjective morphism
$U\to \mathcal{X}$ whose source is a scheme;
the base-changed morphism
$V = U\times_{\mathcal{X}} \mathcal{Z} \to \mathcal{Z}$
is then also smooth and surjective, and the projection
$V \to U$ is a closed immersion.
If $v \in |V|$ maps to $z \in |\mathcal{Z}|$, and
if we let $u$ denote the image of $v$ in $|U|$,
then clearly
$\dim_v(V) \leq \dim_u(U)$,
while
$\dim_v (V_z) = \dim_u(U_x)$,
by Lemma \ref{lemma-base-change-invariance-of-relative-dimension}.
Thus
$$
\dim_z(\mathcal{Z})  = \dim_v(V) - \dim_v(V_z)
\leq \dim_u(U) - \dim_u(U_x) = \dim_x(\mathcal{X}),
$$
as claimed.
\end{proof}

\begin{lemma}
\label{lemma-dimension-via-components}
If $\mathcal{X}$ is a locally Noetherian algebraic stack, and if
$x \in |\mathcal{X}|$,
then $\dim_x(\mathcal{X}) = \sup_{\mathcal{T}} \{ \dim_x(\mathcal{T}) \} $,
where $\mathcal{T}$ runs over all the irreducible components
of $|\mathcal{X}|$ passing through $x$ (endowed with their
induced reduced structure).
\end{lemma}

\begin{proof}
Lemma \ref{lemma-closed-immersions}
shows that
$\dim_x (\mathcal{T}) \leq \dim_x(\mathcal{X})$ for each
irreducible component $\mathcal{T}$ passing through
the point $x$.   Thus to prove the lemma,
it suffices to show that
\begin{equation}
\label{equation-desired-inequality}
\dim_x(\mathcal{X}) \leq
\sup_{\mathcal{T}} \{\dim_x(\mathcal{T})\}.
\end{equation}
Let $U\to\mathcal{X}$ be a smooth cover by a scheme. If $T$ is an irreducible
component of $U$ then we let $\mathcal{T}$ denote the closure of its image
in $\mathcal{X}$, which is an irreducible component of $\mathcal{X}$. Let
$u \in U$ be
a point mapping to $x$. Then we have
$\dim_x(\mathcal{X})=\dim_uU-\dim_uU_x=\sup_T\dim_uT-\dim_uU_x$, where the
supremum is over the irreducible components of $U$ passing
through $u$. Choose a component $T$ for which the supremum
is achieved, and note that
$\dim_x(\mathcal{T})=\dim_uT-\dim_u T_x$.
The desired inequality (\ref{equation-desired-inequality})
now follows from the evident inequality $\dim_u T_x \leq \dim_u U_x.$
(Note that if $\Spec k \to \mathcal{X}$ is a representative of $x$,
then $T\times_{\mathcal{X}} \Spec k$ is a closed subspace of
$U\times_{\mathcal{X}}
\Spec k$.)
\end{proof}

\begin{lemma}
\label{lemma-dimension-at-finite-type-point}
If $\mathcal{X}$ is a locally Noetherian algebraic stack, and if
$x \in |\mathcal{X}|$, then
for any open substack $\mathcal{V}$ of $\mathcal{X}$ containing $x$,
there is a finite type point $x_0 \in |\mathcal{V}|$ such that
$\dim_{x_0}(\mathcal{X}) = \dim_x(\mathcal{V})$.
\end{lemma}

\begin{proof}
Choose a smooth surjective
morphism  $f:U \to \mathcal{X}$ whose source is a scheme, and consider the
function $u \mapsto \dim_{f(u)}(\mathcal{X});$
since the morphism $|U| \to |\mathcal{X}|$ induced by $f$ is open (as $f$
is smooth) as well as  surjective (by assumption),
and takes finite type points to finite type points (by the very definition
of the finite type points of $|\mathcal{X}|$),
it suffices to show that for any $u \in U$, and any open neighbourhood of $u$,
there is a finite type point $u_0$ in this neighbourhood such that
$\dim_{f(u_0)}(\mathcal{X}) = \dim_{f(u)}(\mathcal{X}).$
Since, with this reformulation
of the problem,  the surjectivity of $f$ is no longer required,
we may replace $U$ by the open neighbourhood of the point $u$ in question,
and thus reduce to the problem of showing that for each $u \in U$,
there is a finite type point $u_0 \in U$ such that
$\dim_{f(u_0)}(\mathcal{X}) = \dim_{f(u)}(\mathcal{X}).$
By Lemma \ref{lemma-behaviour-of-dimensions-wrt-smooth-morphisms}
$\dim_{f(u)}(\mathcal{X}) = \dim_u(U) - \dim_u(U_{f(u)}),$
while
$\dim_{f(u_0)}(\mathcal{X}) = \dim_{u_0}(U) - \dim_{u_0}(U_{f(u_0)}).$
Since $f$ is smooth, the expression $\dim_{u_0}(U_{f(u_0)})$ is locally
constant as $u_0$ varies over $U$ (by
Lemma \ref{lemma-relative-dimension-is-semi-continuous} (2)),
and so shrinking $U$ further around
$u$ if necessary, we may assume it is constant.  Thus the problem
becomes to show that we may find a finite type point $u_0 \in U$
for which $\dim_{u_0}(U) = \dim_u(U)$.
Since by definition $\dim_u U$ is the minimum of the dimensions
$\dim V$, as $V$ ranges over the open neighbourhoods $V$ of $u$
in $U$, we may shrink $U$ down further around $u$ so that
$\dim_u U = \dim U$.
The existence of desired point $u_0$ then follows from
Lemma \ref{lemma-dimension-achieved-by-finite-type-point}.
\end{proof}

\begin{lemma}
\label{lemma-monomorphing-a-component-in-of-the-right-dimension}
Let $\mathcal{T} \hookrightarrow \mathcal{X}$ be a locally
of finite type monomorphism of algebraic stacks,
with $\mathcal{X}$ (and thus also $\mathcal{T}$)
being Jacobson, pseudo-catenary, and locally Noetherian.
Suppose further that $\mathcal{T}$ is irreducible
of some (finite) dimension $d$, and that $\mathcal{X}$ is reduced
and of dimension less
than or equal to $d$.
Then there is a non-empty open substack $\mathcal{V}$ of $\mathcal{T}$ such
that the induced
monomorphism $\mathcal{V} \hookrightarrow \mathcal{X}$ is an open immersion
which identifies
$\mathcal{V}$ with an open subset of an irreducible component of $\mathcal{X}$.
\end{lemma}

\begin{proof}
Choose a smooth surjective morphism $f:U \to \mathcal{X}$ with source a scheme,
necessarily reduced since $\mathcal{X}$ is,
and write $U' = \mathcal{T}\times_{\mathcal{X}} U$.  The base-changed morphism
$U' \to U$ is a monomorphism of algebraic spaces, locally of finite
type, and thus representable
Morphisms of Spaces, Lemma
\ref{spaces-morphisms-lemma-locally-quasi-finite-separated-representable} and
\ref{spaces-morphisms-lemma-monomorphism-loc-finite-type-loc-quasi-finite};
since $U$ is a scheme, so is $U'$.
The projection $f': U' \to \mathcal{T}$ is again a smooth surjection.
Let $u' \in U'$, with image $u \in U$.
Lemma \ref{lemma-base-change-invariance-of-relative-dimension}
shows that $\dim_{u'}(U'_{f(u')}) = \dim_u(U_{f(u)}),$
while $\dim_{f'(u')}(\mathcal{T}) =d
\geq \dim_{f(u)}(\mathcal{X})$ by
Lemma \ref{lemma-irreducible-implies-equidimensional}
and our assumptions on $\mathcal{T}$ and $\mathcal{X}$.
Thus we see that
\begin{equation}
\label{equation-dim-inequality}
\dim_{u'} (U') = \dim_{u'} (U'_{f(u')}) + \dim_{f'(u')}(\mathcal{T})
\\
\geq \dim_u (U_{f(u)}) + \dim_{f(u)}(\mathcal{X}) = \dim_u (U).
\end{equation}
Since $U' \to U$ is a monomorphism, locally of finite type,
it is in particular unramified,
and so by the \'etale local structure of unramified morphisms
\'Etale Morphisms, Lemma \ref{etale-lemma-finite-unramified-etale-local},
we may find a commutative diagram
$$
\xymatrix{
V' \ar[r]\ar[d] & V \ar[d] \\
U' \ar[r] & U
}
$$
in which the scheme $V'$ is non-empty,
the vertical arrows are \'etale,
and the upper horizontal arrow is a closed immersion.
Replacing $V$ by a quasi-compact open subset
whose image has non-empty intersection with the image of $U'$,
and replacing $V'$ by the preimage of $V$, we may further
assume that $V$ (and thus $V'$) is quasi-compact.
Since $V$ is also locally Noetherian,
it is thus Noetherian, and so is the union of finitely many irreducible
components.

\medskip\noindent
Since \'etale morphisms preserve pointwise dimension
Descent, Lemma \ref{descent-lemma-dimension-at-point-local}
we deduce from (\ref{equation-dim-inequality})
that for any point $v' \in V'$,
with image $v \in V$, we have
$\dim_{v'}( V') \geq \dim_v(V)$.
In particular, the image of $V'$ can't be contained in the intersection
of two distinct irreducible components of $V$, and so we may find
at least one irreducible open subset of $V$ which has non-empty intersection
with $V'$; replacing $V$ by this subset, we may assume that $V$ is integral
(being both reduced and irreducible).  From the preceding inequality
on dimensions, we conclude that the closed immersion $V' \hookrightarrow V$
is in fact an isomorphism.
If we let $W$ denote the image of $V'$
in $U'$, then $W$ is a non-empty
open subset of $U'$ (as \'etale morphisms are open),
and the induced monomorphism $W \to U$ is \'etale
(since it is so \'etale locally on the source, i.e.\ after pulling back
to $V'$), and hence is an open immersion (being an \'etale monomorphism).
Thus, if we let $\mathcal{V}$ denote the image of $W$ in $\mathcal{T}$,
then $\mathcal{V}$ is a dense (equivalently, non-empty) open substack of
$\mathcal{T}$,
whose image is dense in an irreducible component of $\mathcal{X}$.
Finally,
we note that the morphism is $\mathcal{V} \to \mathcal{X}$ is smooth
(since its composite
with the smooth morphism $W\to \mathcal{V}$ is smooth),
and also a monomorphism, and thus is an open immersion.
\end{proof}

\begin{lemma}
\label{lemma-dims-of-images}
Let $f: \mathcal{T} \to \mathcal{X}$ be a locally of finite type
morphism of Jacobson, pseudo-catenary, and locally Noetherian
algebraic stacks,
whose source is irreducible and whose target is quasi-separated,
and let $\mathcal{Z} \hookrightarrow \mathcal{X}$ denote the scheme-theoretic
image of $\mathcal{T}$.
Then for all $t \in |T|$,
we have that
$\dim_t( \mathcal{T}_{f(t)}) \geq \dim \mathcal{T}  - \dim \mathcal{Z}$,
and there is a non-empty (equivalently, dense)
open subset of $|\mathcal{T}|$ over which equality holds.
\end{lemma}

\begin{proof}
Replacing $\mathcal{X}$ by $\mathcal{Z}$, we may and do assume that $f$ is
scheme theoretically dominant,
and also that $\mathcal{X}$ is irreducible.
By the upper semi-continuity of fibre dimensions
(Lemma \ref{lemma-relative-dimension-is-semi-continuous} (1)),
it suffices to prove that the equality
$\dim_t( \mathcal{T}_{f(t)}) =\dim \mathcal{T}  - \dim \mathcal{Z}$
holds for $t$ lying in
some non-empty open substack of $\mathcal{T}$.
For this reason, in the argument we are always free
to replace $\mathcal{T}$ by a non-empty open substack.

\medskip\noindent
Let $T' \to \mathcal{T}$ be a smooth surjective morphism whose source
is a scheme, and let $T$ be a non-empty quasi-compact open subset
of $T'$.  Since $\mathcal{Y}$ is quasi-separated, we find
that $T \to  \mathcal{Y}$ is quasi-compact
(by Morphisms of Stacks, Lemma
\ref{stacks-morphisms-lemma-quasi-compact-permanence}, applied to the morphisms
$T \to \mathcal{Y} \to \Spec \mathbf{Z}$).
Thus, if we replace $\mathcal{T}$ by the image of $T$ in $\mathcal{T}$,
then we may assume (appealing to
Morphisms of Stacks, Lemma
\ref{stacks-morphisms-lemma-surjection-from-quasi-compact}
that the morphism $f:\mathcal{T} \to \mathcal{X}$ is quasi-compact.

\medskip\noindent
If we choose a smooth surjection $U \to \mathcal{X}$ with $U$ a scheme,
then Lemma \ref{lemma-map-of-components} ensures that
we may find an irreducible open subset $V$ of $U$ such
that $V \to \mathcal{X}$ is smooth and scheme-theoretically dominant.
Since scheme-theoretic dominance for quasi-compact morphisms
is preserved by flat base-change,
the base-change $\mathcal{T} \times_{\mathcal{X}} V \to V$
of the scheme-theoretically
dominant morphism $f$ is again
scheme-theoretically dominant.   We let $Z$ denote a scheme
admitting a smooth surjection onto this fibre product;
then $Z \to \mathcal{T} \times_{\mathcal{X}} V \to V$
is again scheme-theoretically dominant.
Thus we may find an irreducible
component $C$ of $Z$ which scheme-theoretically
dominates $V$.
Since the composite  $Z \to \mathcal{T}\times_{\mathcal{X}} V \to \mathcal{T}$
is smooth,
and since $\mathcal{T}$ is irreducible,
Lemma \ref{lemma-map-of-components} shows that any irreducible
component of the source has dense image in $|\mathcal{T}|$.
We now replace
$C$ by a non-empty open subset $W$ which is disjoint from every other
irreducible component of $Z$, and
then replace $\mathcal{T}$ and $\mathcal{X}$ by the images of $W$
and $V$
(and apply Lemma \ref{lemma-irreducible-implies-equidimensional}
to see that this
doesn't change the dimension of either $\mathcal{T}$ or $\mathcal{X}$).
If we let $\mathcal{W}$ denote the image of the morphism
$W \to \mathcal{T}\times_{\mathcal{X}} V$,
then $\mathcal{W}$ is open in $\mathcal{T}\times_{\mathcal{X}} V$ (since the
morphism $W \to \mathcal{T}\times_{\mathcal{X}} V$ is smooth),
and is irreducible (being the image of an irreducible
scheme).  Thus we end up with a commutative diagram
$$
\xymatrix{
W \ar[dr] \ar[r]  & \mathcal{W} \ar[r] \ar[d] & V \ar[d] \\
& \mathcal{T} \ar[r] & \mathcal{X}
}
$$
in which $W$ and $V$ are schemes,
the vertical arrows are smooth and surjective,
the diagonal arrows and the left-hand
upper horizontal arroware smooth,
and the induced morphism $\mathcal{W} \to \mathcal{T}\times_{\mathcal{X}} V$ is
an open immersion. Using this diagram, together with the definitions
of the various dimensions involved in
the statement of the lemma, we will reduce our verification
of the lemma to the case of schemes, where it is known.

\medskip\noindent
Fix $w \in |W|$ with image $w' \in |\mathcal{W}|$,
image $t \in |\mathcal{T}|$, image $v$ in $|V|$,
and image $x$ in $|\mathcal{X}|$.
Essentially by definition (using the
fact that $\mathcal{W}$ is open in $\mathcal{T}\times_{\mathcal{X}} V$, and that
the fibre of a base-change is the base-change of the fibre),
we obtain the equalities
$$
\dim_v V_x = \dim_{w'} \mathcal{W}_t
$$
and
$$
\dim_t \mathcal{T}_x = \dim_{w'} \mathcal{W}_v.
$$
By Lemma \ref{lemma-behaviour-of-dimensions-wrt-smooth-morphisms}
(the diagonal arrow and right-hand vertical
arrow in our diagram realise $W$ and $V$ as smooth covers by
schemes of the stacks $\mathcal{T}$ and $\mathcal{X}$), we find that
$$
\dim_t \mathcal{T} = \dim_w W - \dim_w W_t
$$
and
$$
\dim_x \mathcal{X} = \dim_v V - \dim_v V_x.
$$
Combining the equalities, we find that
$$
\dim_t \mathcal{T}_x - \dim_t \mathcal{T} + \dim_x \mathcal{X}
= \dim_{w'} \mathcal{W}_v - \dim_w W + \dim_w W_t + \dim_v V -
\dim_{w'} \mathcal{W}_t
$$
Since $W \to \mathcal{W}$ is a smooth surjection, the same is true
if we base-change over the morphism $\Spec \kappa(v) \to V$
(thinking of $W \to \mathcal{W}$ as a morphism over $V$),
and from this smooth morphism we obtain the first of the following
two equalities
$$
\dim_w W_v - \dim_{w'} \mathcal{W}_v = \dim_w (W_v)_{w'} = \dim_w W_{w'};
$$
the second equality follows via a direct comparison of the
two fibres involved.
Similarly, if we think of $W \to \mathcal{W}$ as a morphism of schemes
over $\mathcal{T}$, and base-change over some representative of the point
$t \in |\mathcal{T}|$, we obtain the equalities
$$
\dim_w W_t - \dim_{w'} \mathcal{W}_t = \dim_w (W_t)_{w'} = \dim_w W_{w'}.
$$
Putting everything together, we find that
$$
\dim_t \mathcal{T}_x - \dim_t \mathcal{T} + \dim_x \mathcal{X}
=  \dim_w W_v - \dim_w W + \dim_v V.
$$
Our goal is to show that the left-hand side of this equality
vanishes for a non-empty open subset
of $t$.  As $w$ varies over a non-empty open subset of $W$,
its image $t \in |\mathcal{T}|$ varies over a non-empty open
subset of $|\mathcal{T}|$ (as $W \to \mathcal{T}$ is smooth).

\medskip\noindent
We are therefore reduced to showing that if $W\to V$ is a
scheme-theoretically dominant morphism of irreducible locally
Noetherian schemes that is locally of finite type,
then there is a non-empty open subset of
points $w\in W$ such that $\dim_w W_v =\dim_w W - \dim_v V$
(where $v$ denotes the image of $w$ in $V$).
This is a standard fact,
whose proof we recall for the convenience of the reader.

\medskip\noindent
We may replace $W$ and $V$ by their underlying reduced subschemes
without altering the validity (or not) of this equation,
and thus we may assume that they are in fact integral schemes.
Since $\dim_w W_v$ is locally constant on $W,$ replacing $W$
by a non-empty open subset if necessary, we may assume that $\dim_w W_v$
is constant, say equal to $d$.  Choosing this open subset to be affine,
we may also assume that the morphism $W\to V$ is in fact of finite type.
Replacing $V$ by a non-empty open subset if necessary
(and then pulling back $W$ over this open subset; the resulting pull-back
is non-empty, since the flat base-change of a quasi-compact
and scheme-theoretically
dominant morphism remains scheme-theoretically dominant),
we may furthermore assume that $W$ is flat over $V$.
The morphism $W\to V$ is thus of relative dimension $d$
in the sense of
Morphisms, Definition
\ref{morphisms-definition-relative-dimension-d}
and it follows from
Morphisms, Lemma \ref{morphisms-lemma-rel-dimension-dimension}
that $\dim_w(W) = \dim_v(V) + d,$ as required.
\end{proof}

\begin{remark}
\label{remark-negative-dimension}
We note that in the context of the preceding lemma,
it need not be that $\dim \mathcal{T} \geq \dim \mathcal{Z}$; this does
not contradict the inequality in the statement of the lemma, because
the fibres of the morphism $f$ are again algebraic stacks, and
so may have negative dimension.  This is illustrated by taking
$k$ to be a field, and applying the lemma to the morphism
$[\Spec k/\mathbf{G}_m] \to \Spec k$.

\medskip\noindent
If the morphism $f$ in the statement of the lemma is assumed
to be quasi-DM (in the sense of
Morphisms of Stacks, Definition
\ref{stacks-morphisms-definition-separated}; e.g.\ morphisms that are
representable by algebraic spaces are quasi-DM),
then the fibres of the morphism over points of the target
are quasi-DM algebraic stacks, and hence are of non-negative
dimension.  In this case, the lemma implies
that indeed $\dim \mathcal{T} \geq \dim \mathcal{Z}$.  In fact, we obtain
the following more general result.
\end{remark}

\begin{lemma}
\label{lemma-dims-of-images-two}
Let $f: \mathcal{T} \to \mathcal{X}$ be a locally of finite type
morphism of Jacobson, pseudo-catenary, and locally Noetherian
algebraic stacks
which is quasi-DM,
whose source is irreducible and whose target is quasi-separated,
and let $\mathcal{Z} \hookrightarrow \mathcal{X}$ denote the scheme-theoretic
image of $\mathcal{T}$.
Then $\dim \mathcal{Z} \leq \dim \mathcal{T}$,
and furthermore, exactly one of the following two conditions holds:
\begin{enumerate}
\item for every finite type point $t \in |T|,$
we have
$\dim_t(\mathcal{T}_{f(t)}) > 0,$ in which
case $\dim \mathcal{Z} < \dim \mathcal{T}$; or
\item   $\mathcal{T}$ and $\mathcal{Z}$
are of the same dimension.
\end{enumerate}
\end{lemma}

\begin{proof}
As was observed in the preceding remark,
the dimension of a quasi-DM stack is always non-negative,
from which we conclude that $\dim_t \mathcal{T}_{f(t)} \geq 0$
for all $t \in |\mathcal{T}|$, with the equality
$$
\dim_t \mathcal{T}_{f(t)} = \dim_t \mathcal{T} - \dim_{f(t)} \mathcal{Z}
$$
holding
for a dense open subset of points $t\in |\mathcal{T}|$.
\end{proof}





\section{The dimension of the local ring}
\label{section-dimension-local-ring}

\noindent
An algebraic stack doesn't really have local rings in the usual sense,
but we can define the dimension of the local ring as follows.

\begin{lemma}
\label{lemma-dimension-local-ring-pre}
Let $\mathcal{X}$ be a locally Noetherian algebraic stack.
Let $U \to \mathcal{X}$ be a smooth morphism and let $u \in U$.
Then
$$
\dim(\mathcal{O}_{U, \overline{u}}) -
\dim(\mathcal{O}_{R_u, e(\overline{u})}) =
2\dim(\mathcal{O}_{U, \overline{u}}) -
\dim(\mathcal{O}_{R, e(\overline{u})})
$$
Here $R = U \times_\mathcal{X} U$ with projections $s, t : R \to U$ and
diagonal $e : U \to R$ and $R_u$ is the fibre of $s : R \to U$ over $u$.
\end{lemma}

\begin{proof}
This is true because
$s : \mathcal{O}_{U, \overline{u}} \to \mathcal{O}_{R, e(\overline{u})}$
is a flat local homomorphism of Noetherian local rings and hence
$$
\dim(\mathcal{O}_{R, e(\overline{u})}) =
\dim(\mathcal{O}_{U, \overline{u}}) +
\dim(\mathcal{O}_{R_u, e(\overline{u})})
$$
by
Algebra, Lemma \ref{algebra-lemma-dimension-base-fibre-equals-total}.
\end{proof}

\begin{lemma}
\label{lemma-dimension-local-ring}
Let $\mathcal{X}$ be a locally Noetherian algebraic stack.
Let $x \in |\mathcal{X}|$ be a finite type point 
Morphisms of Stacks, Definition
\ref{stacks-morphisms-definition-finite-type-point}).
Let $d \in \mathbf{Z}$.
The following are equivalent
\begin{enumerate}
\item there exists a scheme $U$, a smooth morphism $U \to \mathcal{X}$,
and a finite type point $u \in U$ mapping to $x$ such that
$2\dim(\mathcal{O}_{U, \overline{u}}) -
\dim(\mathcal{O}_{R, e(\overline{u})}) = d$, and
\item for any scheme $U$, a smooth morphism $U \to \mathcal{X}$,
and finite type point $u \in U$ mapping to $x$ we have
$2\dim(\mathcal{O}_{U, \overline{u}}) -
\dim(\mathcal{O}_{R, e(\overline{u})}) = d$.
\end{enumerate}
Here $R = U \times_\mathcal{X} U$ with projections $s, t : R \to U$ and
diagonal $e : U \to R$ and $R_u$ is the fibre of $s : R \to U$ over $u$.
\end{lemma}

\begin{proof}
Suppose we have two smooth neighbourhoods $(U, u)$ and $(U', u')$
of $x$ with $u$ and $u'$ finite type points. After shrinking $U$
and $U'$ we may assume that $u$ and $u'$ are closed points
(by definition of finite type points). Then we choose
a surjective \'etale morphism $W \to U \times_\mathcal{X} U'$.
Let $W_u$ be the fibre of $W \to U$ over $u$ and
let $W_{u'}$ be the fibre of $W \to U'$ over $u'$.
Since $u$ and $u'$ map to the same point of $|\mathcal{X}|$
we see that $W_u \cap W_{u'}$ is nonempty.
Hence we may choose a closed point $w \in W$ mapping to
both $u$ and $u'$. This reduces us to the discussion in the
next paragraph.

\medskip\noindent
Assume $(U', u') \to (U, u)$ is a smooth morphism of smooth
neightbourhoods of $x$ with $u$ and $u'$ closed points.
Goal: prove the invariant defined for $(U, u)$ is the same
as the invariant defined for $(U', u')$.
To see this observe that $\mathcal{O}_{U, u} \to \mathcal{O}_{U', u'}$
is a flat local homomorphism of Noetherian local rings and hence
$$
\dim(\mathcal{O}_{U', \overline{u}'}) =
\dim(\mathcal{O}_{U, \overline{u}}) +
\dim(\mathcal{O}_{U'_u, \overline{u}'})
$$
by Algebra, Lemma \ref{algebra-lemma-dimension-base-fibre-equals-total}.
(We omit working through all the steps to relate properties of local
rings and their strict henselizations, see
More on Algebra, Section \ref{more-algebra-section-permanence-henselization}).
On the other hand we have
$$
R' = U' \times_{U, t} R \times_{s, U} U'
$$
Thus we see that
$$
\dim(\mathcal{O}_{R', e(\overline{u}')}) =
\dim(\mathcal{O}_{R, e(\overline{u})}) +
\dim(\mathcal{O}_{U'_u \times_u U'_u, (\overline{u}', \overline{u}')})
$$
To prove the lemma it suffices to show that
$$
\dim(\mathcal{O}_{U'_u \times_u U'_u, (\overline{u}', \overline{u}')}) =
2\dim(\mathcal{O}_{U'_u, \overline{u}'})
$$
Observe that this isn't always true (example: if $U'_u$ is a curve
and $u'$ is the generic point of this curve). However, we know
that $u'$ is a closed point of the algebraic space $U'_u$ locally
of finite type over $u$. In this case the equality holds because, first
$\dim_{(u', u')}(U'_u \times_u U'_u) = 2\dim_{u'}(U'_u)$ by
Varieties, Lemma \ref{varieties-lemma-dimension-product-locally-algebraic}
and second the agreement of dimension with dimension of local rings
in closed points of locally algebraic schemes, see
Varieties, Lemma \ref{varieties-lemma-dimension-locally-algebraic}.
We omit the translation of these results for schemes into the
language of algebraic spaces.
\end{proof}

\begin{definition}
\label{definition-dimension-local-ring}
Let $\mathcal{X}$ be a locally Noetherian algebraic stack.
Let $x \in |\mathcal{X}|$ be a finite type point.
The {\it dimension of the local ring of $\mathcal{X}$ at $x$}
is $d \in \mathbf{Z}$ if the equivalent conditions of
Lemma \ref{lemma-dimension-local-ring} are satisfied.
\end{definition}

\noindent
To be sure, this is motivated by Lemma \ref{lemma-dimension-local-ring-pre}
and Properties of Stacks, Definition
\ref{stacks-properties-definition-dimension-at-point}.
We close this section by establishing a formula allowing us to
compute $\dim_x(\mathcal{X})$ in terms of properties of the versal ring
to $\mathcal{X}$ at $x$.

\begin{lemma}
\label{lemma-dimension-formula}
Suppose that $\mathcal{X}$ is an algebraic stack, locally of finite type
over a locally Noetherian scheme $S$. Let $x_0 : \Spec(k) \to \mathcal{X}$
be a morphism where $k$ is a field of finite type over $S$. Represent
$\mathcal{F}_{\mathcal{X}, k, x_0}$ as in Remark \ref{remark-groupoid-defo}
by a cogroupoid $(A, B, s, t, c)$ of Noetherian complete local $S$-algebras
with residue field $k$. Then
$$
\text{the dimension of the local ring of }\mathcal{X}\text{ at }x_0 =
2\dim A - \dim B
$$
\end{lemma}

\begin{proof}
Let $s \in S$ be the image of $x_0$. If $\mathcal{O}_{S, s}$
is a G-ring (a condition that is almost always satisfied in practice),
then we can prove the lemma as follows.
By Lemma \ref{lemma-Artin-approximation-by-smooth-morphism},
we may find a smooth morphism $U \to \mathcal{X}$, whose source is a scheme,
containing a point $u_0 \in U$ of residue field $k$, such that induced
morphism $\Spec(k) \to U \to \mathcal{X}$ coincides with $x_0$
and such that $A = \mathcal{O}_{U, u_0}^\wedge$.
Write $R = U \times_\mathcal{X} U$. Then we may identify
$\mathcal{O}_{R, e(u_0)}^\wedge$ with $B$.
Hence the equality follows from the definitions.

\medskip\noindent
In the rest of this proof we explain how to prove the lemma in
general, but we urge the reader to skip this.

\medskip\noindent
First let us show that the right hand side is independent of the choice
of $(A, B, s, t, c)$. Namely, suppose that $(A', B', s', t', c')$
is a second choice. Since $A$ and $A'$ are versal rings to $\mathcal{X}$
at $x_0$, we can choose, after possibly switching $A$ and $A'$,
a formally smooth map $A \to A'$ compatible with the given versal
formal objects $\xi$ and $\xi'$ over $A$ and $A'$.
Recall that $\widehat{\mathcal{C}}_\Lambda$ has
coproducts and that these are given by completed tensor product
over $\Lambda$, see Formal Deformation Theory, Lemma
\ref{formal-defos-lemma-CLambdahat-coproducts}.
Then $B$ prorepresents the functor of isomorphisms between the
two pushforwards of $\xi$ to $A \widehat{\otimes}_\Lambda A$.
Similarly for $B'$. We conclude that
$$
B' =
B \otimes_{(A \widehat{\otimes}_\Lambda A)}
(A' \widehat{\otimes}_\Lambda A')
$$
It is straightforward to see that
$$
A \widehat{\otimes}_\Lambda A \longrightarrow
A \widehat{\otimes}_\Lambda A' \longrightarrow
A' \widehat{\otimes}_\Lambda A'
$$
is formally smooth of relative dimension equal to $2$ times the
relative dimension of the formally smooth map $A \to A'$.
(This follows from general principles, but also because
in this particular case $A'$ is a power series ring over $A$
in $r$ variables.) Hence $B \to B'$ is formally smooth of
relative dimension $2(\dim(A') - \dim(A))$ as desired.

\medskip\noindent
Next, let $l/k$ be a finite extension. let
$x_{l, 0} : \Spec(l) \to \mathcal{X}$ be
the induced point. We claim that the right hand side of the formula
is the same for $x_0$ as it is for $x_{l, 0}$.
This can be shown by choosing $A \to A'$ as in
Lemma \ref{lemma-versal-ring-field-extension}
and arguing exactly as in the preceding paragraph.
We omit the details.

\medskip\noindent
Finally, arguing as in the proof of Lemma \ref{lemma-versal-ring-flat}
we can use the compatibilities in the previous two paragraphs
to reduce to the case (discussed in the first paragraph)
where $A$ is the complete local ring of $U$ at $u_0$ for some
scheme smooth over $\mathcal{X}$ and finite type point $u_0$.
Details omitted.
\end{proof}






\begin{multicols}{2}[\section{Other chapters}]
\noindent
Preliminaries
\begin{enumerate}
\item \hyperref[introduction-section-phantom]{Introduction}
\item \hyperref[conventions-section-phantom]{Conventions}
\item \hyperref[sets-section-phantom]{Set Theory}
\item \hyperref[categories-section-phantom]{Categories}
\item \hyperref[topology-section-phantom]{Topology}
\item \hyperref[sheaves-section-phantom]{Sheaves on Spaces}
\item \hyperref[sites-section-phantom]{Sites and Sheaves}
\item \hyperref[stacks-section-phantom]{Stacks}
\item \hyperref[fields-section-phantom]{Fields}
\item \hyperref[algebra-section-phantom]{Commutative Algebra}
\item \hyperref[brauer-section-phantom]{Brauer Groups}
\item \hyperref[homology-section-phantom]{Homological Algebra}
\item \hyperref[derived-section-phantom]{Derived Categories}
\item \hyperref[simplicial-section-phantom]{Simplicial Methods}
\item \hyperref[more-algebra-section-phantom]{More on Algebra}
\item \hyperref[smoothing-section-phantom]{Smoothing Ring Maps}
\item \hyperref[modules-section-phantom]{Sheaves of Modules}
\item \hyperref[sites-modules-section-phantom]{Modules on Sites}
\item \hyperref[injectives-section-phantom]{Injectives}
\item \hyperref[cohomology-section-phantom]{Cohomology of Sheaves}
\item \hyperref[sites-cohomology-section-phantom]{Cohomology on Sites}
\item \hyperref[dga-section-phantom]{Differential Graded Algebra}
\item \hyperref[dpa-section-phantom]{Divided Power Algebra}
\item \hyperref[sdga-section-phantom]{Differential Graded Sheaves}
\item \hyperref[hypercovering-section-phantom]{Hypercoverings}
\end{enumerate}
Schemes
\begin{enumerate}
\setcounter{enumi}{25}
\item \hyperref[schemes-section-phantom]{Schemes}
\item \hyperref[constructions-section-phantom]{Constructions of Schemes}
\item \hyperref[properties-section-phantom]{Properties of Schemes}
\item \hyperref[morphisms-section-phantom]{Morphisms of Schemes}
\item \hyperref[coherent-section-phantom]{Cohomology of Schemes}
\item \hyperref[divisors-section-phantom]{Divisors}
\item \hyperref[limits-section-phantom]{Limits of Schemes}
\item \hyperref[varieties-section-phantom]{Varieties}
\item \hyperref[topologies-section-phantom]{Topologies on Schemes}
\item \hyperref[descent-section-phantom]{Descent}
\item \hyperref[perfect-section-phantom]{Derived Categories of Schemes}
\item \hyperref[more-morphisms-section-phantom]{More on Morphisms}
\item \hyperref[flat-section-phantom]{More on Flatness}
\item \hyperref[groupoids-section-phantom]{Groupoid Schemes}
\item \hyperref[more-groupoids-section-phantom]{More on Groupoid Schemes}
\item \hyperref[etale-section-phantom]{\'Etale Morphisms of Schemes}
\end{enumerate}
Topics in Scheme Theory
\begin{enumerate}
\setcounter{enumi}{41}
\item \hyperref[chow-section-phantom]{Chow Homology}
\item \hyperref[intersection-section-phantom]{Intersection Theory}
\item \hyperref[pic-section-phantom]{Picard Schemes of Curves}
\item \hyperref[weil-section-phantom]{Weil Cohomology Theories}
\item \hyperref[adequate-section-phantom]{Adequate Modules}
\item \hyperref[dualizing-section-phantom]{Dualizing Complexes}
\item \hyperref[duality-section-phantom]{Duality for Schemes}
\item \hyperref[discriminant-section-phantom]{Discriminants and Differents}
\item \hyperref[derham-section-phantom]{de Rham Cohomology}
\item \hyperref[local-cohomology-section-phantom]{Local Cohomology}
\item \hyperref[algebraization-section-phantom]{Algebraic and Formal Geometry}
\item \hyperref[curves-section-phantom]{Algebraic Curves}
\item \hyperref[resolve-section-phantom]{Resolution of Surfaces}
\item \hyperref[models-section-phantom]{Semistable Reduction}
\item \hyperref[functors-section-phantom]{Functors and Morphisms}
\item \hyperref[equiv-section-phantom]{Derived Categories of Varieties}
\item \hyperref[pione-section-phantom]{Fundamental Groups of Schemes}
\item \hyperref[etale-cohomology-section-phantom]{\'Etale Cohomology}
\item \hyperref[crystalline-section-phantom]{Crystalline Cohomology}
\item \hyperref[proetale-section-phantom]{Pro-\'etale Cohomology}
\item \hyperref[relative-cycles-section-phantom]{Relative Cycles}
\item \hyperref[more-etale-section-phantom]{More \'Etale Cohomology}
\item \hyperref[trace-section-phantom]{The Trace Formula}
\end{enumerate}
Algebraic Spaces
\begin{enumerate}
\setcounter{enumi}{64}
\item \hyperref[spaces-section-phantom]{Algebraic Spaces}
\item \hyperref[spaces-properties-section-phantom]{Properties of Algebraic Spaces}
\item \hyperref[spaces-morphisms-section-phantom]{Morphisms of Algebraic Spaces}
\item \hyperref[decent-spaces-section-phantom]{Decent Algebraic Spaces}
\item \hyperref[spaces-cohomology-section-phantom]{Cohomology of Algebraic Spaces}
\item \hyperref[spaces-limits-section-phantom]{Limits of Algebraic Spaces}
\item \hyperref[spaces-divisors-section-phantom]{Divisors on Algebraic Spaces}
\item \hyperref[spaces-over-fields-section-phantom]{Algebraic Spaces over Fields}
\item \hyperref[spaces-topologies-section-phantom]{Topologies on Algebraic Spaces}
\item \hyperref[spaces-descent-section-phantom]{Descent and Algebraic Spaces}
\item \hyperref[spaces-perfect-section-phantom]{Derived Categories of Spaces}
\item \hyperref[spaces-more-morphisms-section-phantom]{More on Morphisms of Spaces}
\item \hyperref[spaces-flat-section-phantom]{Flatness on Algebraic Spaces}
\item \hyperref[spaces-groupoids-section-phantom]{Groupoids in Algebraic Spaces}
\item \hyperref[spaces-more-groupoids-section-phantom]{More on Groupoids in Spaces}
\item \hyperref[bootstrap-section-phantom]{Bootstrap}
\item \hyperref[spaces-pushouts-section-phantom]{Pushouts of Algebraic Spaces}
\end{enumerate}
Topics in Geometry
\begin{enumerate}
\setcounter{enumi}{81}
\item \hyperref[spaces-chow-section-phantom]{Chow Groups of Spaces}
\item \hyperref[groupoids-quotients-section-phantom]{Quotients of Groupoids}
\item \hyperref[spaces-more-cohomology-section-phantom]{More on Cohomology of Spaces}
\item \hyperref[spaces-simplicial-section-phantom]{Simplicial Spaces}
\item \hyperref[spaces-duality-section-phantom]{Duality for Spaces}
\item \hyperref[formal-spaces-section-phantom]{Formal Algebraic Spaces}
\item \hyperref[restricted-section-phantom]{Algebraization of Formal Spaces}
\item \hyperref[spaces-resolve-section-phantom]{Resolution of Surfaces Revisited}
\end{enumerate}
Deformation Theory
\begin{enumerate}
\setcounter{enumi}{89}
\item \hyperref[formal-defos-section-phantom]{Formal Deformation Theory}
\item \hyperref[defos-section-phantom]{Deformation Theory}
\item \hyperref[cotangent-section-phantom]{The Cotangent Complex}
\item \hyperref[examples-defos-section-phantom]{Deformation Problems}
\end{enumerate}
Algebraic Stacks
\begin{enumerate}
\setcounter{enumi}{93}
\item \hyperref[algebraic-section-phantom]{Algebraic Stacks}
\item \hyperref[examples-stacks-section-phantom]{Examples of Stacks}
\item \hyperref[stacks-sheaves-section-phantom]{Sheaves on Algebraic Stacks}
\item \hyperref[criteria-section-phantom]{Criteria for Representability}
\item \hyperref[artin-section-phantom]{Artin's Axioms}
\item \hyperref[quot-section-phantom]{Quot and Hilbert Spaces}
\item \hyperref[stacks-properties-section-phantom]{Properties of Algebraic Stacks}
\item \hyperref[stacks-morphisms-section-phantom]{Morphisms of Algebraic Stacks}
\item \hyperref[stacks-limits-section-phantom]{Limits of Algebraic Stacks}
\item \hyperref[stacks-cohomology-section-phantom]{Cohomology of Algebraic Stacks}
\item \hyperref[stacks-perfect-section-phantom]{Derived Categories of Stacks}
\item \hyperref[stacks-introduction-section-phantom]{Introducing Algebraic Stacks}
\item \hyperref[stacks-more-morphisms-section-phantom]{More on Morphisms of Stacks}
\item \hyperref[stacks-geometry-section-phantom]{The Geometry of Stacks}
\end{enumerate}
Topics in Moduli Theory
\begin{enumerate}
\setcounter{enumi}{107}
\item \hyperref[moduli-section-phantom]{Moduli Stacks}
\item \hyperref[moduli-curves-section-phantom]{Moduli of Curves}
\end{enumerate}
Miscellany
\begin{enumerate}
\setcounter{enumi}{109}
\item \hyperref[examples-section-phantom]{Examples}
\item \hyperref[exercises-section-phantom]{Exercises}
\item \hyperref[guide-section-phantom]{Guide to Literature}
\item \hyperref[desirables-section-phantom]{Desirables}
\item \hyperref[coding-section-phantom]{Coding Style}
\item \hyperref[obsolete-section-phantom]{Obsolete}
\item \hyperref[fdl-section-phantom]{GNU Free Documentation License}
\item \hyperref[index-section-phantom]{Auto Generated Index}
\end{enumerate}
\end{multicols}


\bibliography{my}
\bibliographystyle{amsalpha}

\end{document}
