\IfFileExists{stacks-project.cls}{%
\documentclass{stacks-project}
}{%
\documentclass{amsart}
}

% For dealing with references we use the comment environment
\usepackage{verbatim}
\newenvironment{reference}{\comment}{\endcomment}
%\newenvironment{reference}{}{}
\newenvironment{slogan}{\comment}{\endcomment}
\newenvironment{history}{\comment}{\endcomment}

% For commutative diagrams we use Xy-pic
\usepackage[all]{xy}

% We use 2cell for 2-commutative diagrams.
\xyoption{2cell}
\UseAllTwocells

% We use multicol for the list of chapters between chapters
\usepackage{multicol}

% This is generall recommended for better output
\usepackage{lmodern}
\usepackage[T1]{fontenc}

% For cross-file-references
\usepackage{xr-hyper}

% Package for hypertext links:
\usepackage{hyperref}

% For any local file, say "hello.tex" you want to link to please
% use \externaldocument[hello-]{hello}
\externaldocument[introduction-]{introduction}
\externaldocument[conventions-]{conventions}
\externaldocument[sets-]{sets}
\externaldocument[categories-]{categories}
\externaldocument[topology-]{topology}
\externaldocument[sheaves-]{sheaves}
\externaldocument[sites-]{sites}
\externaldocument[stacks-]{stacks}
\externaldocument[fields-]{fields}
\externaldocument[algebra-]{algebra}
\externaldocument[brauer-]{brauer}
\externaldocument[homology-]{homology}
\externaldocument[derived-]{derived}
\externaldocument[simplicial-]{simplicial}
\externaldocument[more-algebra-]{more-algebra}
\externaldocument[smoothing-]{smoothing}
\externaldocument[modules-]{modules}
\externaldocument[sites-modules-]{sites-modules}
\externaldocument[injectives-]{injectives}
\externaldocument[cohomology-]{cohomology}
\externaldocument[sites-cohomology-]{sites-cohomology}
\externaldocument[dga-]{dga}
\externaldocument[dpa-]{dpa}
\externaldocument[sdga-]{sdga}
\externaldocument[hypercovering-]{hypercovering}
\externaldocument[schemes-]{schemes}
\externaldocument[constructions-]{constructions}
\externaldocument[properties-]{properties}
\externaldocument[morphisms-]{morphisms}
\externaldocument[coherent-]{coherent}
\externaldocument[divisors-]{divisors}
\externaldocument[limits-]{limits}
\externaldocument[varieties-]{varieties}
\externaldocument[topologies-]{topologies}
\externaldocument[descent-]{descent}
\externaldocument[perfect-]{perfect}
\externaldocument[more-morphisms-]{more-morphisms}
\externaldocument[flat-]{flat}
\externaldocument[groupoids-]{groupoids}
\externaldocument[more-groupoids-]{more-groupoids}
\externaldocument[etale-]{etale}
\externaldocument[chow-]{chow}
\externaldocument[intersection-]{intersection}
\externaldocument[pic-]{pic}
\externaldocument[weil-]{weil}
\externaldocument[adequate-]{adequate}
\externaldocument[dualizing-]{dualizing}
\externaldocument[duality-]{duality}
\externaldocument[discriminant-]{discriminant}
\externaldocument[derham-]{derham}
\externaldocument[local-cohomology-]{local-cohomology}
\externaldocument[algebraization-]{algebraization}
\externaldocument[curves-]{curves}
\externaldocument[resolve-]{resolve}
\externaldocument[models-]{models}
\externaldocument[functors-]{functors}
\externaldocument[equiv-]{equiv}
\externaldocument[pione-]{pione}
\externaldocument[etale-cohomology-]{etale-cohomology}
\externaldocument[proetale-]{proetale}
\externaldocument[relative-cycles-]{relative-cycles}
\externaldocument[more-etale-]{more-etale}
\externaldocument[trace-]{trace}
\externaldocument[crystalline-]{crystalline}
\externaldocument[spaces-]{spaces}
\externaldocument[spaces-properties-]{spaces-properties}
\externaldocument[spaces-morphisms-]{spaces-morphisms}
\externaldocument[decent-spaces-]{decent-spaces}
\externaldocument[spaces-cohomology-]{spaces-cohomology}
\externaldocument[spaces-limits-]{spaces-limits}
\externaldocument[spaces-divisors-]{spaces-divisors}
\externaldocument[spaces-over-fields-]{spaces-over-fields}
\externaldocument[spaces-topologies-]{spaces-topologies}
\externaldocument[spaces-descent-]{spaces-descent}
\externaldocument[spaces-perfect-]{spaces-perfect}
\externaldocument[spaces-more-morphisms-]{spaces-more-morphisms}
\externaldocument[spaces-flat-]{spaces-flat}
\externaldocument[spaces-groupoids-]{spaces-groupoids}
\externaldocument[spaces-more-groupoids-]{spaces-more-groupoids}
\externaldocument[bootstrap-]{bootstrap}
\externaldocument[spaces-pushouts-]{spaces-pushouts}
\externaldocument[spaces-chow-]{spaces-chow}
\externaldocument[groupoids-quotients-]{groupoids-quotients}
\externaldocument[spaces-more-cohomology-]{spaces-more-cohomology}
\externaldocument[spaces-simplicial-]{spaces-simplicial}
\externaldocument[spaces-duality-]{spaces-duality}
\externaldocument[formal-spaces-]{formal-spaces}
\externaldocument[restricted-]{restricted}
\externaldocument[spaces-resolve-]{spaces-resolve}
\externaldocument[formal-defos-]{formal-defos}
\externaldocument[defos-]{defos}
\externaldocument[cotangent-]{cotangent}
\externaldocument[examples-defos-]{examples-defos}
\externaldocument[algebraic-]{algebraic}
\externaldocument[examples-stacks-]{examples-stacks}
\externaldocument[stacks-sheaves-]{stacks-sheaves}
\externaldocument[criteria-]{criteria}
\externaldocument[artin-]{artin}
\externaldocument[quot-]{quot}
\externaldocument[stacks-properties-]{stacks-properties}
\externaldocument[stacks-morphisms-]{stacks-morphisms}
\externaldocument[stacks-limits-]{stacks-limits}
\externaldocument[stacks-cohomology-]{stacks-cohomology}
\externaldocument[stacks-perfect-]{stacks-perfect}
\externaldocument[stacks-introduction-]{stacks-introduction}
\externaldocument[stacks-more-morphisms-]{stacks-more-morphisms}
\externaldocument[stacks-geometry-]{stacks-geometry}
\externaldocument[moduli-]{moduli}
\externaldocument[moduli-curves-]{moduli-curves}
\externaldocument[examples-]{examples}
\externaldocument[exercises-]{exercises}
\externaldocument[guide-]{guide}
\externaldocument[desirables-]{desirables}
\externaldocument[coding-]{coding}
\externaldocument[obsolete-]{obsolete}
\externaldocument[fdl-]{fdl}
\externaldocument[index-]{index}

% Theorem environments.
%
\theoremstyle{plain}
\newtheorem{theorem}[subsection]{Theorem}
\newtheorem{proposition}[subsection]{Proposition}
\newtheorem{lemma}[subsection]{Lemma}

\theoremstyle{definition}
\newtheorem{definition}[subsection]{Definition}
\newtheorem{example}[subsection]{Example}
\newtheorem{exercise}[subsection]{Exercise}
\newtheorem{situation}[subsection]{Situation}

\theoremstyle{remark}
\newtheorem{remark}[subsection]{Remark}
\newtheorem{remarks}[subsection]{Remarks}

\numberwithin{equation}{subsection}

% Macros
%
\def\lim{\mathop{\mathrm{lim}}\nolimits}
\def\colim{\mathop{\mathrm{colim}}\nolimits}
\def\Spec{\mathop{\mathrm{Spec}}}
\def\Hom{\mathop{\mathrm{Hom}}\nolimits}
\def\Ext{\mathop{\mathrm{Ext}}\nolimits}
\def\SheafHom{\mathop{\mathcal{H}\!\mathit{om}}\nolimits}
\def\SheafExt{\mathop{\mathcal{E}\!\mathit{xt}}\nolimits}
\def\Sch{\mathit{Sch}}
\def\Mor{\mathop{\mathrm{Mor}}\nolimits}
\def\Ob{\mathop{\mathrm{Ob}}\nolimits}
\def\Sh{\mathop{\mathit{Sh}}\nolimits}
\def\NL{\mathop{N\!L}\nolimits}
\def\CH{\mathop{\mathrm{CH}}\nolimits}
\def\proetale{{pro\text{-}\acute{e}tale}}
\def\etale{{\acute{e}tale}}
\def\QCoh{\mathit{QCoh}}
\def\Ker{\mathop{\mathrm{Ker}}}
\def\Im{\mathop{\mathrm{Im}}}
\def\Coker{\mathop{\mathrm{Coker}}}
\def\Coim{\mathop{\mathrm{Coim}}}

% Boxtimes
%
\DeclareMathSymbol{\boxtimes}{\mathbin}{AMSa}{"02}

%
% Macros for moduli stacks/spaces
%
\def\QCohstack{\mathcal{QC}\!\mathit{oh}}
\def\Cohstack{\mathcal{C}\!\mathit{oh}}
\def\Spacesstack{\mathcal{S}\!\mathit{paces}}
\def\Quotfunctor{\mathrm{Quot}}
\def\Hilbfunctor{\mathrm{Hilb}}
\def\Curvesstack{\mathcal{C}\!\mathit{urves}}
\def\Polarizedstack{\mathcal{P}\!\mathit{olarized}}
\def\Complexesstack{\mathcal{C}\!\mathit{omplexes}}
% \Pic is the operator that assigns to X its picard group, usage \Pic(X)
% \Picardstack_{X/B} denotes the Picard stack of X over B
% \Picardfunctor_{X/B} denotes the Picard functor of X over B
\def\Pic{\mathop{\mathrm{Pic}}\nolimits}
\def\Picardstack{\mathcal{P}\!\mathit{ic}}
\def\Picardfunctor{\mathrm{Pic}}
\def\Deformationcategory{\mathcal{D}\!\mathit{ef}}


% OK, start here.
%
\begin{document}

\title{Limits of Algebraic Spaces}


\maketitle

\phantomsection
\label{section-phantom}

\tableofcontents

\section{Introduction}
\label{section-introduction}

\noindent
In this chapter we put material related to limits of algebraic spaces.
A first topic is the characterization of algebraic spaces $F$ locally
of finite presentation over the base $S$ as limit preserving functors.
We continue with a study of limits of inverse systems over
directed sets (Categories, Definition \ref{categories-definition-directed-set})
with affine transition maps. We discuss absolute
Noetherian approximation for quasi-compact and quasi-separated algebraic
spaces following \cite{CLO}. Another approach is due to David Rydh (see
\cite{rydh_approx}) whose results also cover absolute Noetherian
approximation for certain algebraic stacks.


\section{Conventions}
\label{section-conventions}

\noindent
The standing assumption is that all schemes are contained in
a big fppf site $\Sch_{fppf}$. And all rings $A$ considered
have the property that $\Spec(A)$ is (isomorphic) to an
object of this big site.

\medskip\noindent
Let $S$ be a scheme and let $X$ be an algebraic space over $S$.
In this chapter and the following we will write $X \times_S X$
for the product of $X$ with itself (in the category of algebraic
spaces over $S$), instead of $X \times X$.











\section{Morphisms of finite presentation}
\label{section-finite-presentation}

\noindent
In this section we generalize
Limits, Proposition
\ref{limits-proposition-characterize-locally-finite-presentation}
to morphisms of algebraic spaces.
The motivation for the following definition comes from
the proposition just cited.

\begin{definition}
\label{definition-locally-finite-presentation}
Let $S$ be a scheme.
\begin{enumerate}
\item A functor $F : (\Sch/S)_{fppf}^{opp} \to \textit{Sets}$
is said to be {\it limit preserving} or {\it locally of finite presentation} if
for every affine scheme $T$ over $S$ which is a limit $T = \lim T_i$
of a directed inverse system of affine schemes $T_i$ over $S$, we have
$$
F(T) = \colim F(T_i).
$$
We sometimes say that $F$ is {\it locally of finite presentation over $S$}.
\item Let $F, G : (\Sch/S)_{fppf}^{opp} \to \textit{Sets}$.
A transformation of functors $a : F \to G$
is {\it limit preserving} or {\it locally of finite presentation}
if for every scheme $T$ over $S$ and every $y \in G(T)$ the functor
$$
F_y : (\Sch/T)_{fppf}^{opp} \longrightarrow \textit{Sets}, \quad
T'/T \longmapsto \{x \in F(T') \mid a(x) = y|_{T'}\}
$$
is locally of finite presentation over $T$\footnote{The characterization (2) in
Lemma \ref{lemma-characterize-relative-limit-preserving}
may be easier to parse.}. We sometimes say that
$F$ is {\it relatively limit preserving} over $G$.
\end{enumerate}
\end{definition}

\noindent
The functor $F_y$ is in some sense the fiber of
$a : F \to G$ over $y$, except that it is a presheaf on the big fppf
site of $T$. A formula for this functor is:
\begin{equation}
\label{equation-fibre-map-functors}
F_y =
F|_{(\Sch/T)_{fppf}}
{\times}_{G|_{(\Sch/T)_{fppf}}}
*
\end{equation}
Here $*$ is the final object in the category of (pre)sheaves
on $(\Sch/T)_{fppf}$ (see
Sites, Example \ref{sites-example-singleton-sheaf})
and the map $* \to G|_{(\Sch/T)_{fppf}}$ is given by $y$.
Note that if $j : (\Sch/T)_{fppf} \to (\Sch/S)_{fppf}$
is the localization functor, then the formula above becomes
$F_y = j^{-1}F \times_{j^{-1}G} *$ and $j_!F_y$ is just the fiber product
$F \times_{G, y} T$. (See
Sites, Section \ref{sites-section-localize},
for information on localization, and especially
Sites, Remark \ref{sites-remark-localize-presheaves}
for information on $j_!$ for presheaves.)

\medskip\noindent
At this point we temporarily have two definitions of what it means
for a morphism $X \to Y$ of algebraic spaces over $S$ to be locally of finite
presentation. Namely, one by
Morphisms of Spaces,
Definition \ref{spaces-morphisms-definition-locally-finite-presentation}
and one using that $X \to Y$ is a transformation of functors so that
Definition \ref{definition-locally-finite-presentation}
applies (we will use the terminology ``limit preserving''
for this notion as much as possible). We will show in
Proposition \ref{proposition-characterize-locally-finite-presentation}
that these two definitions agree.

\begin{lemma}
\label{lemma-characterize-relative-limit-preserving}
Let $S$ be a scheme. Let $a : F \to G$ be a transformation of functors
$(\Sch/S)_{fppf}^{opp} \to \textit{Sets}$.
The following are equivalent
\begin{enumerate}
\item $a : F \to G$ is limit preserving, and
\item for every affine scheme $T$ over $S$ which is a
limit $T = \lim T_i$ of a directed inverse system of affine
schemes $T_i$ over $S$ the diagram of sets
$$
\xymatrix{
\colim_i F(T_i) \ar[r] \ar[d]_a & F(T) \ar[d]^a \\
\colim_i G(T_i) \ar[r] & G(T)
}
$$
is a fibre product diagram.
\end{enumerate}
\end{lemma}

\begin{proof}
Assume (1). Consider $T = \lim_{i \in I} T_i$ as in (2). Let
$(y, x_T)$ be an element of the fibre product
$\colim_i G(T_i) \times_{G(T)} F(T)$.
Then $y$ comes from $y_i \in G(T_i)$ for some $i$.
Consider the functor $F_{y_i}$ on $(\Sch/T_i)_{fppf}$ as in
Definition \ref{definition-locally-finite-presentation}.
We see that $x_T \in F_{y_i}(T)$. Moreover $T = \lim_{i' \geq i} T_{i'}$
is a directed system of affine schemes over $T_i$. Hence (1) implies
that $x_T$ the image of a unique element $x$ of
$\colim_{i' \geq i} F_{y_i}(T_{i'})$. Thus $x$ is the unique
element of $\colim F(T_i)$ which maps to the pair $(y, x_T)$.
This proves that (2) holds.

\medskip\noindent
Assume (2). Let $T$ be a scheme and $y_T \in G(T)$. We have to show that
$F_{y_T}$ is limit preserving. Let $T' = \lim_{i \in I} T'_i$ be an
affine scheme over $T$ which is the directed limit of affine scheme $T'_i$
over $T$. Let $x_{T'} \in F_{y_T}$. Pick $i \in I$ which is possible as
$I$ is a directed set. Denote $y_i \in F(T'_i)$ the
image of $y_{T'}$. Then we see that $(y_i, x_{T'})$ is an
element of the fibre product
$\colim_i G(T'_i) \times_{G(T')} F(T')$.
Hence by (2) we get a unique element $x$ of $\colim_i F(T'_i)$
mapping to $(y_i, x_{T'})$. It is clear that $x$ defines an element
of $\colim_i F_y(T'_i)$ mapping to $x_{T'}$ and we win.
\end{proof}

\begin{lemma}
\label{lemma-composition-locally-finite-presentation}
Let $S$ be a scheme contained in $\Sch_{fppf}$.
Let $F, G, H : (\Sch/S)_{fppf}^{opp} \to \textit{Sets}$.
Let $a : F \to G$, $b : G \to H$ be transformations of functors.
If $a$ and $b$ are limit preserving, then
$$
b \circ a : F \longrightarrow H
$$
is limit preserving.
\end{lemma}

\begin{proof}
Let $T = \lim_{i \in I} T_i$ as in characterization (2) of
Lemma \ref{lemma-characterize-relative-limit-preserving}.
Consider the diagram of sets
$$
\xymatrix{
\colim_i F(T_i) \ar[r] \ar[d]_a & F(T) \ar[d]^a \\
\colim_i G(T_i) \ar[r] \ar[d]_b & G(T) \ar[d]^b \\
\colim_i H(T_i) \ar[r] & H(T)
}
$$
By assumption the two squares are fibre product squares. Hence the
outer rectangle is a fibre product diagram too which proves the lemma.
\end{proof}

\begin{lemma}
\label{lemma-locally-finite-presentation-permanence}
Let $S$ be a scheme contained in $\Sch_{fppf}$.
Let $F, G, H : (\Sch/S)_{fppf}^{opp} \to \textit{Sets}$.
Let $a : F \to G$, $b : G \to H$ be transformations of functors.
If $b \circ a$ and $b$ are limit preserving, then $a$
is limit preserving.
\end{lemma}

\begin{proof}
Let $T = \lim_{i \in I} T_i$ as in characterization (2) of
Lemma \ref{lemma-characterize-relative-limit-preserving}.
Consider the diagram of sets
$$
\xymatrix{
\colim_i F(T_i) \ar[r] \ar[d]_a & F(T) \ar[d]^a \\
\colim_i G(T_i) \ar[r] \ar[d]_b & G(T) \ar[d]^b \\
\colim_i H(T_i) \ar[r] & H(T)
}
$$
By assumption the lower square and the outer rectangle
are fibre products of sets. Hence the upper square
is a fibre product square too which proves the lemma.
\end{proof}

\begin{lemma}
\label{lemma-base-change-locally-finite-presentation}
Let $S$ be a scheme contained in $\Sch_{fppf}$.
Let $F, G, H : (\Sch/S)_{fppf}^{opp} \to \textit{Sets}$.
Let $a : F \to G$, $b : H \to G$ be transformations of functors.
Consider the fibre product diagram
$$
\xymatrix{
H \times_{b, G, a} F \ar[r]_-{b'} \ar[d]_{a'} & F \ar[d]^a \\
H \ar[r]^b & G
}
$$
If $a$ is limit preserving, then the base change $a'$ is limit preserving.
\end{lemma}

\begin{proof}
Omitted. Hint: This is formal.
\end{proof}

\begin{lemma}
\label{lemma-fibre-product-locally-finite-presentation}
Let $S$ be a scheme contained in $\Sch_{fppf}$.
Let $E, F, G, H : (\Sch/S)_{fppf}^{opp} \to \textit{Sets}$.
Let $a : F \to G$, $b : H \to G$, and $c : G \to E$
be transformations of functors. If $c$, $c \circ a$, and $c \circ b$
are limit preserving, then $F \times_G H \to E$ is too.
\end{lemma}

\begin{proof}
Let $T = \lim_{i \in I} T_i$ as in characterization (2) of
Lemma \ref{lemma-characterize-relative-limit-preserving}.
Then we have
$$
\colim (F \times_G H)(T_i) =
\colim F(T_i) \times_{\colim G(T_i)} \colim H(T_i)
$$
as filtered colimits commute with finite products. Our goal is thus to
show that
$$
\xymatrix{
\colim F(T_i) \times_{\colim G(T_i)} \colim H(T_i) \ar[r] \ar[d] &
F(T) \times_{G(T)} H(T) \ar[d] \\
\colim_i E(T_i) \ar[r] & E(T)
}
$$
is a fibre product diagram. This follows from the observation that
given maps of sets $E' \to E$, $F \to G$, $H \to G$, and $G \to E$
we have
$$
E' \times_E (F \times_G H) =
(E' \times_E F) \times_{(E' \times_E G)} (E' \times_E H)
$$
Some details omitted.
\end{proof}

\begin{lemma}
\label{lemma-sheafify-finite-presentation}
Let $S$ be a scheme contained in $\Sch_{fppf}$.
Let $F : (\Sch/S)_{fppf}^{opp} \to \textit{Sets}$ be a functor.
If $F$ is limit preserving then its sheafification $F^\#$ is limit preserving.
\end{lemma}

\begin{proof}
Assume $F$ is limit preserving.
It suffices to show that $F^+$ is limit preserving, since
$F^\# = (F^+)^+$, see
Sites, Theorem \ref{sites-theorem-plus}.
Let $T$ be an affine scheme over $S$, and let $T = \lim T_i$ be written
as the directed limit of an inverse system of affine $S$ schemes.
Recall that $F^+(T)$ is the colimit of $\check H^0(\mathcal{V}, F)$
where the limit is over all coverings of $T$ in $(\Sch/S)_{fppf}$.
Any fppf covering of an affine scheme can be refined by a standard
fppf covering, see
Topologies, Lemma \ref{topologies-lemma-fppf-affine}.
Hence we can write
$$
F^+(T)
=
\colim_{\mathcal{V}\text{ standard covering }T}
\check H^0(\mathcal{V}, F).
$$
Any $\mathcal{V} = \{T_k \to T\}_{k = 1, \ldots, n}$
in the colimit may be written as
$V_i \times_{T_i} T$ for some $i$ and some standard fppf covering
$\mathcal{V}_i = \{T_{i, k} \to T_i\}_{k = 1, \ldots, n}$ of $T_i$.
Denote $\mathcal{V}_{i'} = \{T_{i', k} \to T_{i'}\}_{k = 1, \ldots, n}$
the base change for $i' \geq i$. Then we see that
\begin{align*}
\colim_{i' \geq i} \check H^0(\mathcal{V}_i, F)
& =
\colim_{i' \geq i}
\text{Equalizer}(
\xymatrix{
\prod F(T_{i', k})
\ar@<1ex>[r] \ar@<-1ex>[r] &
\prod F(T_{i', k} \times_{T_{i'}} T_{i', l})
}
\\
& =
\text{Equalizer}(
\xymatrix{
\colim_{i' \geq i}
\prod F(T_{i', k})
\ar@<1ex>[r] \ar@<-1ex>[r] &
\colim_{k' \geq k}
\prod F(T_{i', k} \times_{T_{i'}} T_{i', l})
}
\\
& =
\text{Equalizer}(
\xymatrix{
\prod F(T_k)
\ar@<1ex>[r] \ar@<-1ex>[r] &
\prod F(T_k \times_T T_l)
}
\\
& =
\check H^0(\mathcal{V}, F)
\end{align*}
Here the second equality holds because filtered colimits are exact.
The third equality holds because $F$ is limit preserving and because
$\lim_{i' \geq i} T_{i', k} = T_k$ and
$\lim_{i' \geq i} T_{i', k} \times_{T_{i'}} T_{i', l} = T_k \times_T T_l$
by Limits, Lemma \ref{limits-lemma-scheme-over-limit}.
If we use this for all coverings at the same time we obtain
\begin{align*}
F^+(T)
& =
\colim_{\mathcal{V}\text{ standard covering }T} \check H^0(\mathcal{V}, F) \\
& =
\colim_{i \in I}
\colim_{\mathcal{V}_i\text{ standard covering }T_i}
\check H^0(T \times_{T_i}\mathcal{V}_i, F) \\
& =
\colim_{i \in I} F^+(T_i)
\end{align*}
The switch of the order of the colimits is allowed by
Categories, Lemma \ref{categories-lemma-colimits-commute}.
\end{proof}

\begin{lemma}
\label{lemma-sheaf-finite-presentation}
Let $S$ be a scheme.
Let $F : (\Sch/S)_{fppf}^{opp} \to \textit{Sets}$ be a functor.
Assume that
\begin{enumerate}
\item $F$ is a sheaf, and
\item there exists an fppf covering $\{U_j \to S\}_{j \in J}$ such that
$F|_{(\Sch/U_j)_{fppf}}$ is limit preserving.
\end{enumerate}
Then $F$ is limit preserving.
\end{lemma}

\begin{proof}
Let $T$ be an affine scheme over $S$.
Let $I$ be a directed set, and let
$T_i$ be an inverse system of affine schemes over $S$ such that
$T = \lim T_i$. We have to show that the canonical
map $\colim F(T_i) \to F(T)$ is bijective.

\medskip\noindent
Choose some $0 \in I$ and choose a standard fppf covering
$\{V_{0, k} \to T_{0}\}_{k = 1, \ldots, m}$ which refines
the pullback $\{U_j \times_S T_0 \to T_0\}$ of the given fppf covering of $S$.
For each $i \geq 0$ we set $V_{i, k} = T_i \times_{T_0} V_{0, k}$, and
we set $V_k = T \times_{T_0} V_{0, k}$. Note that
$V_k = \lim_{i \geq 0} V_{i, k}$, see
Limits, Lemma \ref{limits-lemma-scheme-over-limit}.

\medskip\noindent
Suppose that $x, x' \in \colim F(T_i)$ map to the same
element of $F(T)$. Say $x, x'$ are given by elements $x_i, x'_i \in F(T_i)$
for some $i \in I$ (we may choose the same $i$ for both as $I$ is directed).
By assumption (2) and the fact that $x_i, x'_i$ map to the same element
of $F(T)$ this implies that
$$
x_i|_{V_{i', k}} = x'_i|_{V_{i', k}}
$$
for some suitably large $i' \in I$. We can choose the same $i'$ for each
$k$ as $k \in \{1, \ldots, m\}$ ranges over a finite set.
Since $\{V_{i', k} \to T_{i'}\}$
is an fppf covering and $F$ is a sheaf this implies that
$x_i|_{T_{i'}} = x'_i|_{T_{i'}}$ as desired. This proves that the map
$\colim F(T_i) \to F(T)$ is injective.

\medskip\noindent
To show surjectivity we argue in a similar fashion.
Let $x \in F(T)$. By assumption (2) for each $k$ we
can choose a $i$ such that $x|_{V_k}$ comes from an
element $x_{i, k} \in F(V_{i, k})$. As before we may choose a
single $i$ which works for all $k$. By the injectivity
proved above we see that
$$
x_{i, k}|_{V_{i', k} \times_{T_{i'}} V_{i', l}}
=
x_{i, l}|_{V_{i', k} \times_{T_{i'}} V_{i', l}}
$$
for some large enough $i'$. Hence by the sheaf condition of $F$
the elements $x_{i, k}|_{V_{i', k}}$ glue to an element $x_{i'} \in F(T_{i'})$
as desired.
\end{proof}

\begin{lemma}
\label{lemma-sheafify-finite-presentation-map}
Let $S$ be a scheme contained in $\Sch_{fppf}$.
Let $F, G : (\Sch/S)_{fppf}^{opp} \to \textit{Sets}$ be functors.
If $a : F \to G$ is a transformation which is limit preserving,
then the induced transformation of sheaves
$F^\# \to G^\#$ is limit preserving.
\end{lemma}

\begin{proof}
Suppose that $T$ is a scheme and $y \in G^\#(T)$.
We have to show the functor
$F^\#_y : (\Sch/T)_{fppf}^{opp} \to \textit{Sets}$
constructed from $F^\# \to G^\#$ and $y$ as in
Definition \ref{definition-locally-finite-presentation}
is limit preserving.
By Equation (\ref{equation-fibre-map-functors})
we see that $F^\#_y$ is a sheaf. Choose an fppf covering
$\{V_j \to T\}_{j \in J}$ such that $y|_{V_j}$ comes from
an element $y_j \in F(V_j)$.
Note that the restriction of $F^\#$ to $(\Sch/V_j)_{fppf}$
is just $F^\#_{y_j}$. If we can show that $F^\#_{y_j}$ is
limit preserving then
Lemma \ref{lemma-sheaf-finite-presentation}
guarantees that $F^\#_y$ is limit preserving and
we win. This reduces us to the case $y \in G(T)$.

\medskip\noindent
Let $y \in G(T)$. In this case we claim that $F^\#_y = (F_y)^\#$.
This follows from
Equation (\ref{equation-fibre-map-functors}).
Thus this case follows from
Lemma \ref{lemma-sheafify-finite-presentation}.
\end{proof}

\begin{proposition}
\label{proposition-characterize-locally-finite-presentation}
Let $S$ be a scheme. Let $f : X \to Y$ be a morphism of algebraic
spaces over $S$. The following are equivalent:
\begin{enumerate}
\item The morphism $f$ is a morphism of algebraic spaces which is
locally of finite presentation, see
Morphisms of Spaces,
Definition \ref{spaces-morphisms-definition-locally-finite-presentation}.
\item The morphism $f : X \to Y$ is limit preserving as
a transformation of functors, see
Definition \ref{definition-locally-finite-presentation}.
\end{enumerate}
\end{proposition}

\begin{proof}
Assume (1). Let $T$ be a scheme and let $y \in Y(T)$. We have to show that
$T \times_Y X$ is limit preserving over $T$ in the sense of
Definition \ref{definition-locally-finite-presentation}.
Hence we are reduced to proving that if $X$ is an algebraic space which
is locally of finite presentation over $S$ as an algebraic space, then it
is limit preserving as a functor
$X : (\Sch/S)_{fppf}^{opp} \to \textit{Sets}$.
To see this choose a presentation $X = U/R$, see
Spaces, Definition \ref{spaces-definition-presentation}.
It follows from
Morphisms of Spaces,
Definition \ref{spaces-morphisms-definition-locally-finite-presentation}
that both $U$ and $R$ are schemes which are locally of finite presentation
over $S$. Hence by
Limits, Proposition
\ref{limits-proposition-characterize-locally-finite-presentation}
we have
$$
U(T) = \colim U(T_i), \quad
R(T) = \colim R(T_i)
$$
whenever $T = \lim_i T_i$ in $(\Sch/S)_{fppf}$. It follows
that the presheaf
$$
(\Sch/S)_{fppf}^{opp} \longrightarrow \textit{Sets}, \quad
W \longmapsto U(W)/R(W)
$$
is limit preserving. Hence by
Lemma \ref{lemma-sheafify-finite-presentation}
its sheafification $X = U/R$ is limit preserving too.

\medskip\noindent
Assume (2). Choose a scheme $V$ and a surjective \'etale morphism
$V \to Y$. Next, choose a scheme $U$ and a surjective \'etale morphism
$U \to V \times_Y X$. By
Lemma \ref{lemma-base-change-locally-finite-presentation}
the transformation of functors $V \times_Y X \to V$ is limit preserving. By
Morphisms of Spaces,
Lemma \ref{spaces-morphisms-lemma-etale-locally-finite-presentation}
the morphism of algebraic spaces $U \to V \times_Y X$ is locally
of finite presentation, hence limit preserving as
a transformation of functors by the first part of the proof. By
Lemma \ref{lemma-composition-locally-finite-presentation}
the composition $U \to V \times_Y X \to V$ is limit preserving
as a transformation of functors. Hence
the morphism of schemes $U \to V$ is locally of finite presentation by
Limits, Proposition
\ref{limits-proposition-characterize-locally-finite-presentation}
(modulo a set theoretic remark, see last paragraph of the proof).
This means, by definition, that (1) holds.

\medskip\noindent
Set theoretic remark. Let $U \to V$ be a morphism of
$(\Sch/S)_{fppf}$. In the statement of
Limits, Proposition
\ref{limits-proposition-characterize-locally-finite-presentation}
we characterize $U \to V$ as being locally of finite presentation
if for {\it all} directed inverse systems $(T_i, f_{ii'})$ of affine schemes
over $V$ we have $U(T) = \colim V(T_i)$, but in the current setting
we may only consider affine schemes $T_i$ over $V$ which are (isomorphic to)
an object of $(\Sch/S)_{fppf}$. So we have to make sure that there
are enough affines in $(\Sch/S)_{fppf}$ to make the proof work.
Inspecting the proof of (2) $\Rightarrow$ (1) of
Limits, Proposition
\ref{limits-proposition-characterize-locally-finite-presentation}
we see that the question reduces to the case that $U$ and $V$ are affine.
Say $U = \Spec(A)$ and $V = \Spec(B)$. By construction
of $(\Sch/S)_{fppf}$ the spectrum of any ring of cardinality
$\leq |B|$ is isomorphic to an object of $(\Sch/S)_{fppf}$.
Hence it suffices to observe that in the "only if" part of the proof of
Algebra, Lemma \ref{algebra-lemma-characterize-finite-presentation}
only $A$-algebras of cardinality $\leq |B|$ are used.
\end{proof}

\begin{remark}
\label{remark-limit-preserving}
Here is an important special case of
Proposition \ref{proposition-characterize-locally-finite-presentation}.
Let $S$ be a scheme. Let $X$ be an algebraic space over $S$.
Then $X$ is locally of finite presentation over $S$ if and only
if $X$, as a functor $(\Sch/S)^{opp} \to \textit{Sets}$,
is limit preserving. Compare with
Limits, Remark \ref{limits-remark-limit-preserving}.
In fact, we will see in Lemma \ref{lemma-surjection-is-enough}
below that it suffices if the map
$$
\colim X(T_i) \longrightarrow X(T)
$$
is surjective whenever $T = \lim T_i$ is a directed limit of
affine schemes over $S$.
\end{remark}

\begin{lemma}
\label{lemma-surjection-is-enough}
Let $S$ be a scheme.
Let $f : X \to Y$ be a morphism of algebraic spaces over $S$.
If for every directed limit $T = \lim_{i \in I} T_i$
of affine schemes over $S$ the map
$$
\colim X(T_i) \longrightarrow X(T) \times_{Y(T)} \colim Y(T_i)
$$
is surjective, then $f$ is locally of finite presentation.
In other words, in
Proposition \ref{proposition-characterize-locally-finite-presentation}
part (2) it suffices to check surjectivity in the criterion of
Lemma \ref{lemma-characterize-relative-limit-preserving}.
\end{lemma}

\begin{proof}
Choose a scheme $V$ and a surjective \'etale morphism $g : V \to Y$.
Next, choose a scheme $U$ and a surjective \'etale morphism
$h : U \to V \times_Y X$. It suffices to show for $T = \lim T_i$
as in the lemma that the map
$$
\colim U(T_i) \longrightarrow U(T) \times_{V(T)} \colim V(T_i)
$$
is surjective, because then $U \to V$ will be locally of finite
presentation by Limits, Lemma \ref{limits-lemma-surjection-is-enough}
(modulo a set theoretic remark exactly as in the proof of
Proposition \ref{proposition-characterize-locally-finite-presentation}).
Thus we take $a : T \to U$ and $b_i : T_i \to V$ which determine
the same morphism $T \to V$. Picture
$$
\xymatrix{
T \ar[d]_a \ar[rr]_{p_i} & & T_i \ar[d]^{b_i} \ar@{..>}[ld] \\
U \ar[r]^-h & X \times_Y V \ar[d] \ar[r] & V \ar[d]^g \\
& X \ar[r]^f & Y
}
$$
By the assumption of the lemma after increasing $i$
we can find a morphism $c_i : T_i \to X$ such that
$h \circ a = (b_i, c_i) \circ p_i : T_i \to V \times_Y X$
and such that $f \circ c_i = g \circ b_i$.
Since $h$ is an \'etale morphism of algebraic spaces
(and hence locally of finite presentation), we have the surjectivity of
$$
\colim U(T_i) \longrightarrow U(T) \times_{(X \times_Y V)(T)}
\colim (X \times_Y V)(T_i)
$$
by Proposition \ref{proposition-characterize-locally-finite-presentation}.
Hence after increasing $i$ again we can find the desired
morphism $a_i : T_i \to U$ with $a = a_i \circ p_i$ and
$b_i = (U \to V) \circ a_i$.
\end{proof}















\section{Limits of algebraic spaces}
\label{section-limits}

\noindent
The following lemma explains how we think of limits of algebraic
spaces in this chapter. We will use (without further mention) that the
base change of an affine morphism of algebraic spaces is affine (see
Morphisms of Spaces, Lemma \ref{spaces-morphisms-lemma-base-change-affine}).

\begin{lemma}
\label{lemma-directed-inverse-system-has-limit}
Let $S$ be a scheme. Let $I$ be a directed set.
Let $(X_i, f_{ii'})$ be an inverse system over $I$
in the category of algebraic spaces over $S$.
If the morphisms $f_{ii'} : X_i \to X_{i'}$ are affine, then the
limit $X = \lim_i X_i$ (as an fppf sheaf) is an algebraic space.
Moreover,
\begin{enumerate}
\item each of the morphisms $f_i : X \to X_i$ is affine,
\item for any $i \in I$ and any morphism of algebraic spaces
$T \to X_i$ we have
$$
X \times_{X_i} T = \lim_{i' \geq i} X_{i'} \times_{X_i} T.
$$
as algebraic spaces over $S$.
\end{enumerate}
\end{lemma}

\begin{proof}
Part (2) is a formal consequence of the existence of the
limit $X = \lim X_i$ as an algebraic space over $S$.
Choose an element $0 \in I$ (this is possible as a directed set is nonempty).
Choose a scheme $U_0$ and a surjective
\'etale morphism $U_0 \to X_0$. Set $R_0 = U_0 \times_{X_0} U_0$
so that $X_0 = U_0/R_0$. For $i \geq 0$ set
$U_i = X_i \times_{X_0} U_0$ and
$R_i = X_i \times_{X_0} R_0 = U_i \times_{X_i} U_i$.
By Limits, Lemma \ref{limits-lemma-directed-inverse-system-has-limit}
we see that $U = \lim_{i \geq 0} U_i$ and $R = \lim_{i \geq 0} R_i$
are schemes. Moreover, the two morphisms $s, t : R \to U$ are the base
change of the two projections $R_0 \to U_0$ by the morphism
$U \to U_0$, in particular \'etale. The morphism $R \to U \times_S U$
defines an equivalence relation as directed a limit of equivalence relations
is an equivalence relation. Hence the morphism
$R \to U \times_S U$ is an \'etale equivalence relation. We claim that
the natural map
\begin{equation}
\label{equation-isomorphism-sheaves}
U/R \longrightarrow \lim X_i
\end{equation}
is an isomorphism of fppf sheaves on the category of schemes over $S$.
The claim implies $X = \lim X_i$ is an algebraic
space by Spaces, Theorem \ref{spaces-theorem-presentation}.

\medskip\noindent
Let $Z$ be a scheme and let $a : Z \to \lim X_i$ be a morphism.
Then $a = (a_i)$ where $a_i : Z \to X_i$. Set $W_0 = Z \times_{a_0, X_0} U_0$.
Note that $W_0 = Z \times_{a_i, X_i} U_i$ for all $i \geq 0$ by our
choice of $U_i \to X_i$ above. Hence we obtain a morphism
$W_0 \to \lim_{i \geq 0} U_i = U$. Since $W_0 \to Z$ is surjective
and \'etale, we conclude that (\ref{equation-isomorphism-sheaves})
is a surjective map of sheaves. Finally, suppose that
$Z$ is a scheme and that $a, b : Z \to U/R$ are two morphisms
which are equalized by (\ref{equation-isomorphism-sheaves}).
We have to show that $a = b$.
After replacing $Z$ by the members of an fppf covering
we may assume there exist morphisms $a', b' : Z \to U$ which
give rise to $a$ and $b$. The condition that $a, b$ are
equalized by (\ref{equation-isomorphism-sheaves}) means that
for each $i \geq 0$ the compositions $a_i', b_i' : Z \to U \to U_i$
are equal as morphisms into $U_i/R_i = X_i$. Hence
$(a_i', b_i') : Z \to U_i \times_S U_i$ factors through
$R_i$, say by some morphism $c_i : Z \to R_i$. Since
$R = \lim_{i \geq 0} R_i$ we see that $c = \lim c_i : Z \to R$
is a morphism which shows that $a, b$ are equal as morphisms
of $Z$ into $U/R$.

\medskip\noindent
Part (1) follows as we have seen above that
$U_i \times_{X_i} X = U$ and $U \to U_i$ is affine by
construction.
\end{proof}

\begin{lemma}
\label{lemma-space-over-limit}
Let $S$ be a scheme. Let $I$ be a directed set.
Let $(X_i, f_{ii'})$ be an inverse system over $I$ of algebraic spaces
over $S$ with affine transition maps.
Let $X = \lim_i X_i$. Let $0 \in I$. Suppose that $T \to X_0$ is a
morphism of algebraic spaces. Then
$$
T \times_{X_0} X = \lim_{i \geq 0} T \times_{X_0} X_i
$$
as algebraic spaces over $S$.
\end{lemma}

\begin{proof}
The limit $X$ is an algebraic space by
Lemma \ref{lemma-directed-inverse-system-has-limit}.
The equality is formal, see
Categories, Lemma \ref{categories-lemma-colimits-commute}.
\end{proof}

\begin{lemma}
\label{lemma-directed-inverse-system-closed-immersions}
Let $S$ be a scheme. Let $I$ be a directed set.
Let $(X_i, f_{i'i}) \to (Y_i, g_{i'i})$ be a morphism
of inverse systems over $I$ of algebraic spaces over $S$.
Assume
\begin{enumerate}
\item the morphisms $f_{i'i} : X_{i'} \to X_i$ are affine,
\item the morphisms $g_{i'i} : Y_{i'} \to Y_i$ are affine,
\item the morphisms $X_i \to Y_i$ are closed immersions.
\end{enumerate}
Then $\lim X_i \to \lim Y_i$ is a closed immersion.
\end{lemma}

\begin{proof}
Observe that $\lim X_i$ and $\lim Y_i$ exist by
Lemma \ref{lemma-directed-inverse-system-has-limit}.
Pick $0 \in I$ and choose an affine scheme $V_0$ and an \'etale morphism
$V_0 \to Y_0$. Then the morphisms
$V_i = Y_i \times_{Y_0} V_0 \to U_i = X_i \times_{Y_0} V_0$
are closed immersions of affine schemes.
Hence the morphism $V = Y \times_{Y_0} V_0 \to U = X \times_{Y_0} V_0$
is a closed immersion because $V = \lim V_i$, $U = \lim U_i$
and because a limit of closed immersions of affine schemes is a
closed immersion: a filtered colimit of surjective ring maps
is surjective. Since the \'etale morphisms $V \to Y$ form an
\'etale covering of $Y$ as we vary our choice of $V_0 \to Y_0$
we see that the lemma is true.
\end{proof}

\begin{lemma}
\label{lemma-directed-inverse-system-reduced}
Let $S$ be a scheme. Let $I$ be a directed set.
Let $(X_i, f_{i'i})$ be an inverse systems over $I$
of algebraic spaces over $S$. If $X_i$ is reduced
for all $i$, then $X$ is reduced.
\end{lemma}

\begin{proof}
Observe that $\lim X_i$ exists by
Lemma \ref{lemma-directed-inverse-system-has-limit}.
Pick $0 \in I$ and choose an affine scheme $V_0$ and an \'etale morphism
$U_0 \to X_0$. Then the affine schemes
$U_i = X_i \times_{X_0} U_0$ are reduced.
Hence $U = X \times_{X_0} U_0$
is a reduced affine scheme as a limit of reduced affine schemes:
a filtered colimit of reduced rings is reduced.
Since the \'etale morphisms $U \to X$ form an
\'etale covering of $X$ as we vary our choice of $U_0 \to X_0$
we see that the lemma is true.
\end{proof}

\begin{lemma}
\label{lemma-better-characterize-relative-limit-preserving}
Let $S$ be a scheme. Let $X \to Y$ be a morphism of algebraic spaces
over $S$. The equivalent conditions (1) and (2) of
Proposition \ref{proposition-characterize-locally-finite-presentation}
are also equivalent to
\begin{enumerate}
\item[(3)] for every directed limit $T = \lim T_i$ of quasi-compact
and quasi-separated algebraic spaces $T_i$ over $S$ with affine
transition morphisms the diagram of sets
$$
\xymatrix{
\colim_i \Mor(T_i, X) \ar[r] \ar[d] & \Mor(T, X) \ar[d] \\
\colim_i \Mor(T_i, Y) \ar[r] & \Mor(T, Y)
}
$$
is a fibre product diagram.
\end{enumerate}
\end{lemma}

\begin{proof}
It is clear that (3) implies (2). We will assume (2) and prove (3).
The proof is rather formal and we encourage the reader to find their
own proof.

\medskip\noindent
Let us first prove that (3) holds
when $T_i$ is in addition assumed separated for all $i$.
Choose $i \in I$ and choose a surjective \'etale morphism $U_i \to T_i$
where $U_i$ is affine. Using Lemma \ref{lemma-space-over-limit}
we see that with $U = U_i \times_{T_i} T$ and
$U_{i'} = U_i \times_{T_i} T_{i'}$ we have $U = \lim_{i' \geq i} U_{i'}$.
Of course $U$ and $U_{i'}$ are affine (see
Lemma \ref{lemma-directed-inverse-system-has-limit}).
Since $T_i$ is separated, the fibre product $V_i = U_i \times_{T_i} U_i$
is an affine scheme as well and we obtain affine schemes
$V = V_i \times_{T_i} T$ and
$V_{i'} = V_i \times_{T_i} T_{i'}$ with $V = \lim_{i' \geq i} V_{i'}$.
Observe that $U \to T$ and $U_i \to T_i$ are surjective \'etale and
that $V = U \times_T U$ and $V_{i'} = U_{i'} \times_{T_{i'}} U_{i'}$.
Note that $\Mor(T, X)$ is the equalizer of the two maps
$\Mor(U, X) \to \Mor(V, X)$; this is true for example because
$X$ as a sheaf on $(\Sch/S)_{fppf}$ is the coequalizer
of the two maps $h_V \to h_u$. Similarly
$\Mor(T_{i'}, X)$ is the equalizer of the
two maps $\Mor(U_{i'}, X) \to \Mor(V_{i'}, X)$.
And of course the same thing is true with $X$ replaced with $Y$.
Condition (2) says that the diagrams of in (3) are fibre products
in the case of $U = \lim U_i$ and $V = \lim V_i$.
It follows formally that the same thing is true for $T = \lim T_i$.

\medskip\noindent
In the general case, choose an affine scheme $U$, an $i \in I$,
and a surjective \'etale morphism $U \to T_i$. Repeating the
argument of the previous paragraph we still achieve the proof:
the schemes $V_{i'}$, $V$ are no longer affine, but they are
still quasi-compact and
separated and the result of the preceding paragraph applies.
\end{proof}




\section{Descending properties}
\label{section-descent}

\noindent
This section is the analogue of Limits, Section \ref{limits-section-descent}.

\begin{lemma}
\label{lemma-inverse-limit-sets}
Let $S$ be a scheme. Let $X = \lim_{i \in I} X_i$ be the limit of a directed
inverse system of algebraic spaces over $S$ with affine transition morphisms
(Lemma \ref{lemma-directed-inverse-system-has-limit}). If each $X_i$
is decent (for example quasi-separated or locally separated)
then $|X| = \lim_i |X_i|$ as sets.
\end{lemma}

\begin{proof}
There is a canonical map $|X| \to \lim |X_i|$. Choose $0 \in I$.
If $W_0 \subset X_0$ is an open subspace, then we have
$f_0^{-1}W_0 = \lim_{i \geq 0} f_{i0}^{-1}W_0$, see
Lemma \ref{lemma-directed-inverse-system-has-limit}.
Hence, if we can prove the lemma for inverse systems where $X_0$
is quasi-compact, then the lemma follows in general. Thus we may and do
assume $X_0$ is quasi-compact.

\medskip\noindent
Choose an affine scheme $U_0$ and a surjective \'etale morphism $U_0 \to X_0$.
Set $U_i = X_i \times_{X_0} U_0$ and $U = X \times_{X_0} U_0$.
Set $R_i = U_i \times_{X_i} U_i$ and $R = U \times_X U$.
Recall that $U = \lim U_i$ and $R = \lim R_i$, see proof of
Lemma \ref{lemma-directed-inverse-system-has-limit}.
Recall that $|X| = |U|/|R|$ and $|X_i| = |U_i|/|R_i|$. By
Limits, Lemma \ref{limits-lemma-topology-limit} we have
$|U| = \lim |U_i|$ and $|R| = \lim |R_i|$.

\medskip\noindent
Surjectivity of $|X| \to \lim |X_i|$. Let $(x_i) \in \lim |X_i|$. Denote
$S_i \subset |U_i|$ the inverse image of $x_i$. This is a finite nonempty
set by the definition of decent spaces
(Decent Spaces, Definition \ref{decent-spaces-definition-very-reasonable}).
Hence $\lim S_i$ is nonempty, see
Categories, Lemma \ref{categories-lemma-nonempty-limit}.
Let $(u_i) \in \lim S_i \subset \lim |U_i|$. By the above this determines
a point $u \in |U|$ which maps to an $x \in |X|$ mapping to the given
element $(x_i)$ of $\lim |X_i|$.

\medskip\noindent
Injectivity of $|X| \to \lim |X_i|$. Suppose that $x, x' \in |X|$
map to the same point of $\lim |X_i|$. Choose lifts $u, u' \in |U|$
and denote $u_i, u'_i \in |U_i|$ the images.
For each $i$ let $T_i \subset |R_i|$ be the set of points mapping
to $(u_i, u'_i) \in |U_i| \times |U_i|$. This is a finite
set by the definition of decent spaces
(Decent Spaces, Definition \ref{decent-spaces-definition-very-reasonable}).
Moreover $T_i$ is nonempty as we've assumed that $x$ and $x'$ map to the
same point of $X_i$. Hence $\lim T_i$ is nonempty, see
Categories, Lemma \ref{categories-lemma-nonempty-limit}.
As before let $r \in |R| = \lim |R_i|$ be a point corresponding to an
element of $\lim T_i$. Then $r$ maps to $(u, u')$ in $|U| \times |U|$
by construction and we see that $x = x'$ in $|X|$ as desired.

\medskip\noindent
Parenthetical statement: A quasi-separated algebraic space is decent, see
Decent Spaces, Section \ref{decent-spaces-section-reasonable-decent}
(the key observation to this is Properties of Spaces, Lemma
\ref{spaces-properties-lemma-finite-fibres-presentation}).
A locally separated algebraic space is decent by
Decent Spaces, Lemma \ref{decent-spaces-lemma-locally-separated-decent}.
\end{proof}

\begin{lemma}
\label{lemma-topology-limit}
With same notation and assumptions as in Lemma \ref{lemma-inverse-limit-sets}
we have $|X| = \lim_i |X_i|$ as topological spaces.
\end{lemma}

\begin{proof}
We will use the criterion of
Topology, Lemma \ref{topology-lemma-characterize-limit}.
We have seen that $|X| = \lim_i |X_i|$ as sets in
Lemma \ref{lemma-inverse-limit-sets}.
The maps $f_i : X \to X_i$ are morphisms of algebraic spaces
hence determine continuous maps $|X| \to |X_i|$.
Thus $f_i^{-1}(U_i)$ is open for each
open $U_i \subset |X_i|$. Finally, 
let $x \in |X|$ and let $x \in V \subset |X|$ be an open
neighbourhood. We have to find an $i$ and an open neighbourhood
$W_i \subset |X_i|$ of the image $x$
with $f_i^{-1}(W_i) \subset V$.
Choose $0 \in I$. Choose a scheme $U_0$ and a surjective
\'etale morphism $U_0 \to X_0$. Set $U = X \times_{X_0} U_0$
and $U_i = X_i \times_{X_0} U_0$ for $i \geq 0$.
Then $U = \lim_{i \geq 0} U_i$ in the category of schemes by
Lemma \ref{lemma-directed-inverse-system-has-limit}.
Choose $u \in U$ mapping to $x$. By the result for schemes
(Limits, Lemma \ref{limits-lemma-inverse-limit-top})
we can find an $i \geq 0$ and an open neighbourhood
$E_i \subset U_i$ of the image of $u$
whose inverse image in $U$ is contained in the
inverse image of $V$ in $U$. Then we can set $W_i \subset |X_i|$
equal to the image of $E_i$. This works because $|U_i| \to |X_i|$ is open.
\end{proof}

\begin{lemma}
\label{lemma-limit-nonempty}
Let $S$ be a scheme. Let $X = \lim_{i \in I} X_i$ be the limit of a directed
inverse system of algebraic spaces over $S$ with affine transition morphisms
(Lemma \ref{lemma-directed-inverse-system-has-limit}). If each $X_i$
is quasi-compact and nonempty, then $|X|$ is nonempty.
\end{lemma}

\begin{proof}
Choose $0 \in I$.
Choose an affine scheme $U_0$ and a surjective \'etale morphism $U_0 \to X_0$.
Set $U_i = X_i \times_{X_0} U_0$ and $U = X \times_{X_0} U_0$.
Then each $U_i$ is a nonempty affine scheme. Hence $U = \lim U_i$
is nonempty (Limits, Lemma \ref{limits-lemma-limit-nonempty}) and thus
$X$ is nonempty.
\end{proof}

\begin{lemma}
\label{lemma-inverse-limit-irreducibles}
Let $S$ be a scheme. Let $X = \lim_{i \in I} X_i$ be the limit of a directed
inverse system of algebraic spaces over $S$ with affine transition morphisms
(Lemma \ref{lemma-directed-inverse-system-has-limit}).
Let $x \in |X|$ with images $x_i \in |X_i|$. If each $X_i$ is decent,
then $\overline{\{x\}} = \lim_i \overline{\{x_i\}}$ as sets
and as algebraic spaces if endowed with reduced induced scheme structure.
\end{lemma}

\begin{proof}
Set $Z = \overline{\{x\}} \subset |X|$ and
$Z_i = \overline{\{x_i\}} \subset |X_i|$.
Since $|X| \to |X_i|$ is continuous we see that $Z$ maps into $Z_i$
for each $i$. Hence we obtain an injective map $Z \to \lim Z_i$
because $|X| = \lim |X_i|$ as sets (Lemma \ref{lemma-inverse-limit-sets}).
Suppose that $x' \in |X|$ is not in $Z$.
Then there is an open subset $U \subset |X|$ with $x' \in U$
and $x \not \in U$. Since
$|X| = \lim |X_i|$ as topological spaces (Lemma \ref{lemma-topology-limit})
we can write $U = \bigcup_{j \in J} f_j^{-1}(U_j)$
for some subset $J \subset I$ and opens $U_j \subset |X_j|$, see
Topology, Lemma \ref{topology-lemma-describe-limits}.
Then we see that for some $j \in J$ we have $f_j(x') \in U_j$
and $f_j(x) \not \in U_j$. In other words, we see that $f_j(x') \not \in Z_j$.
Thus $Z = \lim Z_i$ as sets.

\medskip\noindent
Next, endow $Z$ and $Z_i$ with their reduced induced scheme structures, see
Properties of Spaces, Definition
\ref{spaces-properties-definition-reduced-induced-space}.
The transition morphisms $X_{i'} \to X_i$ induce affine
morphisms $Z_{i'} \to Z_i$ and the projections $X \to X_i$
induce compatible morphisms $Z \to Z_i$.
Hence we obtain morphisms $Z \to \lim Z_i \to X$ of algebraic spaces.
By Lemma \ref{lemma-directed-inverse-system-closed-immersions}
we see that $\lim Z_i \to X$ is a
closed immersion. By Lemma \ref{lemma-directed-inverse-system-reduced}
the algebraic space $\lim Z_i$ is reduced.
By the above $Z \to \lim Z_i$ is bijective on points.
By uniqueness of the reduced induced closed subscheme structure
we find that this morphism is an isomorphism of algebraic spaces.
\end{proof}

\begin{situation}
\label{situation-descent}
Let $S$ be a scheme. Let $X = \lim_{i \in I} X_i$ be the limit of a directed
inverse system of algebraic spaces over $S$ with affine transition morphisms
(Lemma \ref{lemma-directed-inverse-system-has-limit}).
We assume that $X_i$ is quasi-compact and quasi-separated for all $i \in I$.
We also choose an element $0 \in I$.
\end{situation}

\begin{lemma}
\label{lemma-descend-section}
Notation and assumptions as in Situation \ref{situation-descent}.
Suppose that $\mathcal{F}_0$ is a quasi-coherent sheaf on $X_0$.
Set $\mathcal{F}_i = f_{0i}^*\mathcal{F}_0$ for $i \geq 0$ and set
$\mathcal{F} = f_0^*\mathcal{F}_0$. Then
$$
\Gamma(X, \mathcal{F}) = \colim_{i \geq 0} \Gamma(X_i, \mathcal{F}_i)
$$
\end{lemma}

\begin{proof}
Choose a surjective \'etale morphism $U_0 \to X_0$ where $U_0$ is an affine
scheme (Properties of Spaces, Lemma
\ref{spaces-properties-lemma-quasi-compact-affine-cover}).
Set $U_i = X_i \times_{X_0} U_0$.
Set $R_0 = U_0 \times_{X_0} U_0$ and $R_i = R_0 \times_{X_0} X_i$.
In the proof of Lemma \ref{lemma-directed-inverse-system-has-limit} we have
seen that there exists a presentation $X = U/R$ with
$U = \lim U_i$ and $R = \lim R_i$.
Note that $U_i$ and $U$ are affine and that $R_i$ and $R$ are
quasi-compact and separated (as $X_i$ is quasi-separated). Hence
Limits, Lemma \ref{limits-lemma-descend-section}
implies that
$$
\mathcal{F}(U) = \colim \mathcal{F}_i(U_i)
\quad\text{and}\quad
\mathcal{F}(R) = \colim \mathcal{F}_i(R_i).
$$
The lemma follows as
$\Gamma(X, \mathcal{F}) = \Ker(\mathcal{F}(U) \to \mathcal{F}(R))$
and similarly
$\Gamma(X_i, \mathcal{F}_i) =
\Ker(\mathcal{F}_i(U_i) \to \mathcal{F}_i(R_i))$
\end{proof}

\begin{lemma}
\label{lemma-descend-opens}
Notation and assumptions as in Situation \ref{situation-descent}.
For any quasi-compact open subspace $U \subset X$ there exists an $i$
and a quasi-compact open $U_i \subset X_i$ whose inverse image in $X$ is $U$.
\end{lemma}

\begin{proof}
Follows formally from the construction of limits in
Lemma \ref{lemma-directed-inverse-system-has-limit}
and the corresponding result for schemes:
Limits, Lemma \ref{limits-lemma-descend-opens}.
\end{proof}

\noindent
The following lemma will be superseded by the stronger
Lemma \ref{lemma-descend-isomorphism}.

\begin{lemma}
\label{lemma-descend-equality}
Notation and assumptions as in Situation \ref{situation-descent}.
Let $f_0 : Y_0 \to Z_0$ be a morphism of algebraic spaces over $X_0$.
Assume (a) $Y_0 \to X_0$ and $Z_0 \to X_0$ are representable, (b)
$Y_0$, $Z_0$ quasi-compact and quasi-separated, (c)
$f_0$ locally of finite presentation, and
(d) $Y_0 \times_{X_0} X \to Z_0 \times_{X_0} X$ an isomorphism.
Then there exists an $i \geq 0$ such that
$Y_0 \times_{X_0} X_i \to Z_0 \times_{X_0} X_i$ is an isomorphism.
\end{lemma}

\begin{proof}
Choose an affine scheme $U_0$ and a surjective \'etale morphism $U_0 \to X_0$.
Set $U_i = U_0 \times_{X_0} X_i$ and $U = U_0 \times_{X_0} X$.
Apply Limits, Lemma \ref{limits-lemma-descend-isomorphism}
to see that $Y_0 \times_{X_0} U_i \to Z_0 \times_{X_0} U_i$
is an isomorphism of schemes for some $i \geq 0$ (details omitted).
As $U_i \to X_i$ is surjective \'etale, it follows that
$Y_0 \times_{X_0} X_i \to Z_0 \times_{X_0} X_i$ is an isomorphism
(details omitted).
\end{proof}

\begin{lemma}
\label{lemma-descend-separated}
Notation and assumptions as in Situation \ref{situation-descent}.
If $X$ is separated, then $X_i$ is separated for some $i \in I$.
\end{lemma}

\begin{proof}
Choose an affine scheme $U_0$ and a surjective \'etale morphism $U_0 \to X_0$.
For $i \geq 0$ set $U_i = U_0 \times_{X_0} X_i$ and set
$U = U_0 \times_{X_0} X$. Note that $U_i$ and $U$ are affine schemes
which come equipped with surjective \'etale morphisms $U_i \to X_i$
and $U \to X$. Set $R_i = U_i \times_{X_i} U_i$ and $R = U \times_X U$
with projections $s_i, t_i : R_i \to U_i$ and $s, t : R \to U$.
Note that $R_i$ and $R$ are quasi-compact separated schemes (as the
algebraic spaces $X_i$ and $X$ are quasi-separated). The maps
$s_i : R_i \to U_i$ and $s : R \to U$ are of finite type.
By definition $X_i$ is separated if and only if
$(t_i, s_i) : R_i \to U_i \times U_i$
is a closed immersion, and since $X$ is separated by assumption,
the morphism $(t, s) : R \to U \times U$ is a closed immersion. Since
$R \to U$ is of finite type, there exists an
$i$ such that the morphism $R \to U_i \times U$ is a closed immersion
(Limits, Lemma \ref{limits-lemma-finite-type-eventually-closed}).
Fix such an $i \in I$. Apply Limits, Lemma
\ref{limits-lemma-descend-closed-immersion-finite-presentation}
to the system of morphisms $R_{i'} \to U_i \times U_{i'}$ for $i' \geq i$
(this is permissible as indeed
$R_{i'} = R_i \times_{U_i \times U_i} U_i \times U_{i'}$)
to see that $R_{i'} \to U_i \times U_{i'}$ is a closed immersion
for $i'$ sufficiently large. This implies immediately
that $R_{i'} \to U_{i'} \times U_{i'}$ is a closed immersion
finishing the proof of the lemma.
\end{proof}

\begin{lemma}
\label{lemma-limit-is-affine}
Notation and assumptions as in Situation \ref{situation-descent}.
If $X$ is affine, then there exists an $i$ such that $X_i$ is affine.
\end{lemma}

\begin{proof}
Choose $0 \in I$. Choose an affine scheme $U_0$ and a surjective
\'etale morphism $U_0 \to X_0$. Set $U = U_0 \times_{X_0} X$
and $U_i = U_0 \times_{X_0} X_i$ for $i \geq 0$. Since the transition
morphisms are affine, the algebraic spaces $U_i$ and $U$ are affine.
Thus $U \to X$ is an \'etale morphism of affine schemes. Hence we
can write $X = \Spec(A)$, $U = \Spec(B)$ and
$$
B = A[x_1, \ldots, x_n]/(g_1, \ldots, g_n)
$$
such that $\Delta = \det(\partial g_\lambda/\partial x_\mu)$ is invertible
in $B$, see Algebra, Lemma \ref{algebra-lemma-etale-standard-smooth}.
Set $A_i = \mathcal{O}_{X_i}(X_i)$. We have $A = \colim A_i$ by
Lemma \ref{lemma-descend-section}. After increasing $0$ we may assume
we have $g_{1, i}, \ldots, g_{n, i} \in A_i[x_1, \ldots, x_n]$ mapping to
$g_1, \ldots, g_n$. Set
$$
B_i = A_i[x_1, \ldots, x_n]/(g_{1, i}, \ldots, g_{n, i})
$$
for all $i \geq 0$. Increasing $0$ if necessary we may assume that
$\Delta_i = \det(\partial g_{\lambda, i}/\partial x_\mu)$ is invertible
in $B_i$ for all $i \geq 0$. Thus $A_i \to B_i$ is an \'etale ring map.
After increasing $0$ we may assume also that
$\Spec(B_i) \to \Spec(A_i)$ is surjective, see
Limits, Lemma \ref{limits-lemma-descend-surjective}. Increasing
$0$ yet again we may choose elements
$h_{1, i}, \ldots, h_{n, i} \in \mathcal{O}_{U_i}(U_i)$ which map to the
classes of $x_1, \ldots, x_n$ in $B = \mathcal{O}_U(U)$ and such that
$g_{\lambda, i}(h_{\nu, i}) = 0$ in $\mathcal{O}_{U_i}(U_i)$. Thus
we obtain a commutative diagram
\begin{equation}
\label{equation-to-show-cartesian}
\vcenter{
\xymatrix{
X_i \ar[d] & U_i \ar[l] \ar[d] \\
\Spec(A_i) & \Spec(B_i) \ar[l]
}
}
\end{equation}
By construction $B_i = B_0 \otimes_{A_0} A_i$ and
$B = B_0 \otimes_{A_0} A$. Consider the morphism
$$
f_0 : U_0 \longrightarrow X_0 \times_{\Spec(A_0)} \Spec(B_0)
$$
This is a morphism of quasi-compact and quasi-separated algebraic spaces
representable, separated and \'etale over $X_0$. The base change of $f_0$
to $X$ is an isomorphism by our choices. Hence
Lemma \ref{lemma-descend-equality}
guarantees that there exists an $i$ such that the base change of $f_0$
to $X_i$ is an isomorphism, in other words the diagram
(\ref{equation-to-show-cartesian}) is cartesian. Thus
Descent, Lemma \ref{descent-lemma-descent-data-sheaves}
applied to the fppf covering $\{\Spec(B_i) \to \Spec(A_i)\}$
combined with Descent, Lemma \ref{descent-lemma-affine}
give that $X_i \to \Spec(A_i)$ is representable by a scheme
affine over $\Spec(A_i)$ as desired. (Of course it then also follows
that $X_i = \Spec(A_i)$ but we don't need this.)
\end{proof}

\begin{lemma}
\label{lemma-limit-is-scheme}
Notation and assumptions as in Situation \ref{situation-descent}.
If $X$ is a scheme, then there exists an $i$ such that $X_i$ is a scheme.
\end{lemma}

\begin{proof}
Choose a finite affine open covering $X = \bigcup W_j$.
By Lemma \ref{lemma-descend-opens}
we can find an $i \in I$ and open subspaces $W_{j, i} \subset X_i$
whose base change to $X$ is $W_j \to X$. By
Lemma \ref{lemma-limit-is-affine} we may assume that
each $W_{j, i}$ is an affine scheme. This means that $X_i$
is a scheme (see for example
Properties of Spaces, Section \ref{spaces-properties-section-schematic}).
\end{proof}

\begin{lemma}
\label{lemma-finite-type-eventually-closed}
Let $S$ be a scheme. Let $B$ be an algebraic space over $S$.
Let $X = \lim X_i$ be a directed limit of
algebraic spaces over $B$ with affine transition morphisms.
Let $Y \to X$ be a morphism of algebraic spaces over $B$.
\begin{enumerate}
\item If $Y \to X$ is a closed immersion, $X_i$ quasi-compact, and
$Y \to B$ locally of finite type, then $Y \to X_i$ is a closed immersion
for $i$ large enough.
\item If $Y \to X$ is an immersion, $X_i$ quasi-separated, $Y \to B$
locally of finite type, and $Y$ quasi-compact, then $Y \to X_i$ is an
immersion for $i$ large enough.
\item If $Y \to X$ is an isomorphism, $X_i$ quasi-compact,
$X_i \to B$ locally of finite type, the transition morphisms
$X_{i'} \to X_i$ are closed immersions, and $Y \to B$ is locally
of finite presentation, then $Y \to X_i$ is an isomorphism for $i$
large enough.
\item If $Y \to X$ is a monomorphism, $X_i$ quasi-separated,
$Y \to B$ locally of finite type, and $Y$ quasi-compact, then
$Y \to X_i$ is a monomorphism for $i$ large enough.
\end{enumerate}
\end{lemma}

\begin{proof}
Proof of (1). Choose $0 \in I$. As $X_0$ is quasi-compact, we can choose
an affine scheme $W$ and an \'etale morphism $W \to B$ such that the image of
$|X_0| \to |B|$ is contained in $|W| \to |B|$. 
Choose an affine scheme $U_0$ and an \'etale morphism
$U_0 \to X_0 \times_B W$ such that $U_0 \to X_0$ is surjective.
(This is possible by our choice of $W$ and the fact that $X_0$ is
quasi-compact; details omitted.)
Let $V \to Y$, resp.\ $U \to X$, resp.\ $U_i \to X_i$ be the base change
of $U_0 \to X_0$ (for $i \geq 0$). It suffices to prove that $V \to U_i$
is a closed immersion for $i$ sufficiently large. Thus we reduce
to proving the result for $V \to U = \lim U_i$ over $W$. This follows from
the case of schemes, which is
Limits, Lemma \ref{limits-lemma-finite-type-eventually-closed}.

\medskip\noindent
Proof of (2). Choose $0 \in I$. Choose a quasi-compact open subspace
$X'_0 \subset X_0$ such that $Y \to X_0$ factors through $X'_0$.
After replacing $X_i$ by the inverse image of $X'_0$ for $i \geq 0$
we may assume all $X_i'$ are quasi-compact and quasi-separated.
Let $U \subset X$ be a quasi-compact open such that $Y \to X$ factors
through a closed immersion $Y \to U$ ($U$ exists as $Y$ is quasi-compact). By
Lemma \ref{lemma-descend-opens}
we may assume that $U = \lim U_i$ with $U_i \subset X_i$ quasi-compact
open. By part (1) we see that $Y \to U_i$ is a closed immersion for some
$i$. Thus (2) holds.

\medskip\noindent
Proof of (3). Choose $0 \in I$. Choose an affine scheme $U_0$
and a surjective \'etale morphism $U_0 \to X_0$.
Set $U_i = X_i \times_{X_0} U_0$,
$U = X \times_{X_0} U_0 = Y \times_{X_0} U_0$. Then $U = \lim U_i$ is a
limit of affine schemes, the transition maps of the system are closed
immersions, and $U \to U_0$ is of finite presentation (because
$U \to B$ is locally of finite presentation and $U_0 \to B$ is locally
of finite type and
Morphisms of Spaces, Lemma
\ref{spaces-morphisms-lemma-finite-presentation-permanence}).
Thus we've reduced to the following algebra fact: If $A = \lim A_i$
is a directed colimit of $R$-algebras with surjective transition
maps and $A$ of finite presentation over $A_0$, then $A = A_i$ for
some $i$. Namely, write $A = A_0/(f_1, \ldots, f_n)$. Pick $i$ such
that $f_1, \ldots, f_n$ map to zero under the surjective map $A_0 \to A_i$.

\medskip\noindent
Proof of (4). Set $Z_i = Y \times_{X_i} Y$.
As the transition morphisms $X_{i'} \to X_i$ are affine hence separated,
the transition morphisms $Z_{i'} \to Z_i$ are closed immersions, see
Morphisms of Spaces, Lemma
\ref{spaces-morphisms-lemma-fibre-product-after-map}.
We have $\lim Z_i = Y \times_X Y = Y$ as $Y \to X$ is a monomorphism.
Choose $0 \in I$. Since $Y \to X_0$ is locally of finite type
(Morphisms of Spaces, Lemma
\ref{spaces-morphisms-lemma-permanence-finite-type})
the morphism $Y \to Z_0$ is locally of finite presentation
(Morphisms of Spaces, Lemma
\ref{spaces-morphisms-lemma-diagonal-morphism-finite-type}).
The morphisms $Z_i \to Z_0$ are locally of finite type
(they are closed immersions).
Finally, $Z_i = Y \times_{X_i} Y$ is quasi-compact as
$X_i$ is quasi-separated and $Y$ is quasi-compact.
Thus part (3) applies to $Y = \lim_{i \geq 0} Z_i$ over $Z_0$
and we conclude $Y = Z_i$ for some $i$. This proves (4) and the lemma.
\end{proof}

\begin{lemma}
\label{lemma-eventually-separated}
Let $S$ be a scheme. Let $Y$ be an algebraic space over $S$.
Let $X = \lim X_i$ be a directed limit of algebraic spaces over $Y$
with affine transition morphisms. Assume
\begin{enumerate}
\item $Y$ is quasi-separated,
\item $X_i$ is quasi-compact and quasi-separated,
\item the morphism $X \to Y$ is separated.
\end{enumerate}
Then $X_i \to Y$ is separated for all $i$ large enough.
\end{lemma}

\begin{proof}
Let $0 \in I$. Choose an affine scheme $W$ and an \'etale morphism
$W \to Y$ such that the image of $|W| \to |Y|$ contains the image of
$|X_0| \to |Y|$. This is possible as $X_0$ is quasi-compact.
It suffices to check that $W \times_Y X_i \to W$ is separated
for some $i \geq 0$ because the diagonal of $W \times_Y X_i$
over $W$ is the base change of $X_i \to X_i \times_Y X_i$ by
the surjective \'etale morphism $(X_i \times_Y X_i) \times_Y W \to
X_i \times_Y X_i$. Since $Y$ is quasi-separated the algebraic spaces
$W \times_Y X_i$ are quasi-compact (as well as quasi-separated).
Thus we may base change to $W$ and assume $Y$ is an affine scheme.
When $Y$ is an affine scheme, we have to show that $X_i$ is a
separated algebraic space for $i$ large enough and we are given that
$X$ is a separated algebraic space. Thus this case follows from
Lemma \ref{lemma-descend-separated}.
\end{proof}

\begin{lemma}
\label{lemma-eventually-affine}
Let $S$ be a scheme. Let $Y$ be an algebraic space over $S$.
Let $X = \lim X_i$ be a directed limit of algebraic spaces over $Y$
with affine transition morphisms. Assume
\begin{enumerate}
\item $Y$ quasi-compact and quasi-separated,
\item $X_i$ quasi-compact and quasi-separated,
\item $X \to Y$ affine.
\end{enumerate}
Then $X_i \to Y$ is affine for $i$ large enough.
\end{lemma}

\begin{proof}
Choose an affine scheme $W$ and a surjective \'etale morphism $W \to Y$.
Then $X \times_Y W$ is affine and it suffices to check that
$X_i \times_Y W$ is affine for some $i$ (Morphisms of Spaces,
Lemma \ref{spaces-morphisms-lemma-affine-local}).
This follows from Lemma \ref{lemma-limit-is-affine}.
\end{proof}

\begin{lemma}
\label{lemma-eventually-finite}
Let $S$ be a scheme. Let $Y$ be an algebraic space over $S$.
Let $X = \lim X_i$ be a directed limit of algebraic spaces
over $Y$ with affine transition morphisms. Assume
\begin{enumerate}
\item $Y$ quasi-compact and quasi-separated,
\item $X_i$ quasi-compact and quasi-separated,
\item the transition morphisms $X_{i'} \to X_i$ are finite,
\item $X_i \to Y$ locally of finite type
\item $X \to Y$ integral.
\end{enumerate}
Then $X_i \to Y$ is finite for $i$ large enough.
\end{lemma}

\begin{proof}
Choose an affine scheme $W$ and a surjective \'etale morphism $W \to Y$.
Then $X \times_Y W$ is finite over $W$ and it suffices to check that
$X_i \times_Y W$ is finite over $W$ for some $i$ (Morphisms of Spaces,
Lemma \ref{spaces-morphisms-lemma-integral-local}). By
Lemma \ref{lemma-limit-is-scheme} this reduces us to the case of schemes.
In the case of schemes it follows from
Limits, Lemma \ref{limits-lemma-eventually-finite}.
\end{proof}

\begin{lemma}
\label{lemma-eventually-closed-immersion}
Let $S$ be a scheme. Let $Y$ be an algebraic space over $S$.
Let $X = \lim X_i$ be a directed limit of algebraic spaces
over $Y$ with affine transition morphisms. Assume
\begin{enumerate}
\item $Y$ quasi-compact and quasi-separated,
\item $X_i$ quasi-compact and quasi-separated,
\item the transition morphisms $X_{i'} \to X_i$ are closed immersions,
\item $X_i \to Y$ locally of finite type
\item $X \to Y$ is a closed immersion.
\end{enumerate}
Then $X_i \to Y$ is a closed immersion for $i$ large enough.
\end{lemma}

\begin{proof}
Choose an affine scheme $W$ and a surjective \'etale morphism $W \to Y$.
Then $X \times_Y W$ is a closed subspace of $W$ and it suffices to check
that $X_i \times_Y W$ is a closed subspace $W$ for some $i$
(Morphisms of Spaces,
Lemma \ref{spaces-morphisms-lemma-closed-immersion-local}). By
Lemma \ref{lemma-limit-is-scheme} this reduces us to the case of schemes.
In the case of schemes it follows from
Limits, Lemma \ref{limits-lemma-eventually-closed-immersion}.
\end{proof}

















\section{Descending properties of morphisms}
\label{section-descent-of-properties}

\noindent
This section is the analogue of Section \ref{section-descent}
for properties of morphisms. We will work in the following situation.

\begin{situation}
\label{situation-descent-property}
Let $S$ be a scheme. Let $B = \lim B_i$ be a limit of a directed inverse system
of algebraic spaces over $S$ with affine transition morphisms
(Lemma \ref{lemma-directed-inverse-system-has-limit}).
Let $0 \in I$ and let $f_0 : X_0 \to Y_0$ be a morphism of algebraic spaces
over $B_0$. Assume $B_0$, $X_0$, $Y_0$ are quasi-compact and quasi-separated.
Let $f_i : X_i \to Y_i$ be the base change of $f_0$ to $B_i$ and
let $f : X \to Y$ be the base change of $f_0$ to $B$.
\end{situation}

\begin{lemma}
\label{lemma-descend-etale}
With notation and assumptions as in
Situation \ref{situation-descent-property}. If
\begin{enumerate}
\item $f$ is \'etale,
\item $f_0$ is locally of finite presentation,
\end{enumerate}
then $f_i$ is \'etale for some $i \geq 0$.
\end{lemma}

\begin{proof}
Choose an affine scheme $V_0$ and a surjective \'etale morphism
$V_0 \to Y_0$. Choose an affine scheme $U_0$ and a surjective \'etale
morphism $U_0 \to V_0 \times_{Y_0} X_0$. Diagram
$$
\xymatrix{
U_0 \ar[d] \ar[r] & V_0 \ar[d] \\
X_0 \ar[r] & Y_0
}
$$
The vertical arrows are surjective and \'etale by construction.
We can base change this diagram to $B_i$ or $B$ to get
$$
\vcenter{
\xymatrix{
U_i \ar[d] \ar[r] & V_i \ar[d] \\
X_i \ar[r] & Y_i
}
}
\quad\text{and}\quad
\vcenter{
\xymatrix{
U \ar[d] \ar[r] & V \ar[d] \\
X \ar[r] & Y
}
}
$$
Note that $U_i, V_i, U, V$ are affine schemes,
the vertical morphisms are surjective \'etale, and the limit of the
morphisms $U_i \to V_i$ is $U \to V$. Recall that $X_i \to Y_i$ is \'etale
if and only if $U_i \to V_i$ is
\'etale and similarly $X \to Y$ is \'etale if and only if
$U \to V$ is \'etale
(Morphisms of Spaces, Lemma \ref{spaces-morphisms-lemma-etale-local}).
Since $f_0$ is locally of finite
presentation, so is the morphism $U_0 \to V_0$. Hence the lemma follows
from Limits, Lemma \ref{limits-lemma-descend-etale}.
\end{proof}

\begin{lemma}
\label{lemma-descend-smooth}
With notation and assumptions as in
Situation \ref{situation-descent-property}. If
\begin{enumerate}
\item $f$ is smooth,
\item $f_0$ is locally of finite presentation,
\end{enumerate}
then $f_i$ is smooth for some $i \geq 0$.
\end{lemma}

\begin{proof}
Choose an affine scheme $V_0$ and a surjective \'etale morphism
$V_0 \to Y_0$. Choose an affine scheme $U_0$ and a surjective \'etale
morphism $U_0 \to V_0 \times_{Y_0} X_0$. Diagram
$$
\xymatrix{
U_0 \ar[d] \ar[r] & V_0 \ar[d] \\
X_0 \ar[r] & Y_0
}
$$
The vertical arrows are surjective and \'etale by construction.
We can base change this diagram to $B_i$ or $B$ to get
$$
\vcenter{
\xymatrix{
U_i \ar[d] \ar[r] & V_i \ar[d] \\
X_i \ar[r] & Y_i
}
}
\quad\text{and}\quad
\vcenter{
\xymatrix{
U \ar[d] \ar[r] & V \ar[d] \\
X \ar[r] & Y
}
}
$$
Note that $U_i, V_i, U, V$ are affine schemes,
the vertical morphisms are surjective \'etale, and the limit of the
morphisms $U_i \to V_i$ is $U \to V$. Recall that $X_i \to Y_i$ is smooth
if and only if $U_i \to V_i$ is smooth and similarly
$X \to Y$ is smooth if and only if $U \to V$ is smooth
(Morphisms of Spaces, Definition \ref{spaces-morphisms-definition-smooth}).
Since $f_0$ is locally of finite
presentation, so is the morphism $U_0 \to V_0$. Hence the lemma follows
from Limits, Lemma \ref{limits-lemma-descend-smooth}.
\end{proof}

\begin{lemma}
\label{lemma-descend-surjective}
With notation and assumptions as in
Situation \ref{situation-descent-property}. If
\begin{enumerate}
\item $f$ is surjective,
\item $f_0$ is locally of finite presentation,
\end{enumerate}
then $f_i$ is surjective for some $i \geq 0$.
\end{lemma}

\begin{proof}
Choose an affine scheme $V_0$ and a surjective \'etale morphism
$V_0 \to Y_0$. Choose an affine scheme $U_0$ and a surjective \'etale
morphism $U_0 \to V_0 \times_{Y_0} X_0$. Diagram
$$
\xymatrix{
U_0 \ar[d] \ar[r] & V_0 \ar[d] \\
X_0 \ar[r] & Y_0
}
$$
The vertical arrows are surjective and \'etale by construction.
We can base change this diagram to $B_i$ or $B$ to get
$$
\vcenter{
\xymatrix{
U_i \ar[d] \ar[r] & V_i \ar[d] \\
X_i \ar[r] & Y_i
}
}
\quad\text{and}\quad
\vcenter{
\xymatrix{
U \ar[d] \ar[r] & V \ar[d] \\
X \ar[r] & Y
}
}
$$
Note that $U_i, V_i, U, V$ are affine schemes, the vertical morphisms are
surjective \'etale, the limit of the morphisms $U_i \to V_i$ is $U \to V$,
and the morphisms $U_i \to X_i \times_{Y_i} V_i$ and
$U \to X \times_Y V$ are surjective (as base changes of
$U_0 \to X_0 \times_{Y_0} V_0$). In particular, we see that
$X_i \to Y_i$ is surjective if and only if $U_i \to V_i$ is surjective
and similarly $X \to Y$ is surjective if and only if $U \to V$ is surjective.
Since $f_0$ is locally of finite
presentation, so is the morphism $U_0 \to V_0$. Hence the lemma follows
from the case of schemes (Limits, Lemma \ref{limits-lemma-descend-surjective}).
\end{proof}

\begin{lemma}
\label{lemma-descend-universally-injective}
Notation and assumptions as in Situation \ref{situation-descent-property}. If
\begin{enumerate}
\item $f$ is universally injective,
\item $f_0$ is locally of finite type,
\end{enumerate}
then $f_i$ is universally injective for some $i \geq 0$.
\end{lemma}

\begin{proof}
Recall that a morphism $X \to Y$ is universally injective if and
only if the diagonal $X \to X \times_Y X$ is surjective
(Morphisms of Spaces, Definition
\ref{spaces-morphisms-definition-universally-injective} and
Lemma \ref{spaces-morphisms-lemma-universally-injective}).
Observe that $X_0 \to X_0 \times_{Y_0} X_0$ is of locally of finite
presentation (Morphisms of Spaces, Lemma
\ref{spaces-morphisms-lemma-diagonal-morphism-finite-type}).
Hence the lemma follows from Lemma \ref{lemma-descend-surjective}
by considering the morphism $X_0 \to X_0 \times_{Y_0} X_0$.
\end{proof}

\begin{lemma}
\label{lemma-descend-affine}
Notation and assumptions as in Situation \ref{situation-descent-property}. If
$f$ is affine, then $f_i$ is affine for some $i \geq 0$.
\end{lemma}

\begin{proof}
Choose an affine scheme $V_0$ and a surjective \'etale morphism $V_0 \to Y_0$.
Set $V_i = V_0 \times_{Y_0} Y_i$ and $V = V_0 \times_{Y_0} Y$.
Since $f$ is affine we see that $V \times_Y X = \lim V_i \times_{Y_i} X_i$
is affine. By Lemma \ref{lemma-limit-is-affine} we see that
$V_i \times_{Y_i} X_i$ is affine for some $i \geq 0$. For this $i$ the morphism
$f_i$ is affine
(Morphisms of Spaces, Lemma \ref{spaces-morphisms-lemma-affine-local}).
\end{proof}

\begin{lemma}
\label{lemma-descend-finite}
Notation and assumptions as in Situation \ref{situation-descent-property}. If
\begin{enumerate}
\item $f$ is finite,
\item $f_0$ is locally of finite type,
\end{enumerate}
then $f_i$ is finite for some $i \geq 0$.
\end{lemma}

\begin{proof}
Choose an affine scheme $V_0$ and a surjective \'etale morphism $V_0 \to Y_0$.
Set $V_i = V_0 \times_{Y_0} Y_i$ and $V = V_0 \times_{Y_0} Y$.
Since $f$ is finite we see that $V \times_Y X = \lim V_i \times_{Y_i} X_i$
is a scheme finite over $V$. By Lemma \ref{lemma-limit-is-affine} we see that
$V_i \times_{Y_i} X_i$ is affine for some $i \geq 0$. Increasing $i$ if
necessary we find that $V_i \times_{Y_i} X_i \to V_i$ is finite by
Limits, Lemma \ref{limits-lemma-descend-finite-finite-presentation}.
For this $i$ the morphism $f_i$ is finite
(Morphisms of Spaces, Lemma \ref{spaces-morphisms-lemma-integral-local}).
\end{proof}

\begin{lemma}
\label{lemma-descend-closed-immersion}
Notation and assumptions as in Situation \ref{situation-descent-property}. If
\begin{enumerate}
\item $f$ is a closed immersion,
\item $f_0$ is locally of finite type,
\end{enumerate}
then $f_i$ is a closed immersion for some $i \geq 0$.
\end{lemma}

\begin{proof}
Choose an affine scheme $V_0$ and a surjective \'etale morphism $V_0 \to Y_0$.
Set $V_i = V_0 \times_{Y_0} Y_i$ and $V = V_0 \times_{Y_0} Y$.
Since $f$ is a closed immersion we see that
$V \times_Y X = \lim V_i \times_{Y_i} X_i$
is a closed subscheme of the affine scheme $V$. By
Lemma \ref{lemma-limit-is-affine} we see that
$V_i \times_{Y_i} X_i$ is affine for some $i \geq 0$. Increasing $i$ if
necessary we find that $V_i \times_{Y_i} X_i \to V_i$ is a closed immersion by
Limits, Lemma \ref{limits-lemma-descend-closed-immersion-finite-presentation}.
For this $i$ the morphism $f_i$ is a closed immersion
(Morphisms of Spaces, Lemma \ref{spaces-morphisms-lemma-integral-local}).
\end{proof}

\begin{lemma}
\label{lemma-descend-separated-morphism}
Notation and assumptions as in Situation \ref{situation-descent-property}.
If $f$ is separated, then $f_i$ is separated for some $i \geq 0$.
\end{lemma}

\begin{proof}
Apply Lemma \ref{lemma-descend-closed-immersion}
to the diagonal morphism $\Delta_{X_0/Y_0} : X_0 \to X_0 \times_{Y_0} X_0$.
(Diagonal morphisms are locally of finite type
and the fibre product $X_0 \times_{Y_0} X_0$ is quasi-compact and
quasi-separated. Some details omitted.)
\end{proof}

\begin{lemma}
\label{lemma-descend-isomorphism}
Notation and assumptions as in Situation \ref{situation-descent-property}. If
\begin{enumerate}
\item $f$ is a isomorphism,
\item $f_0$ is locally of finite presentation,
\end{enumerate}
then $f_i$ is a isomorphism for some $i \geq 0$.
\end{lemma}

\begin{proof}
Being an isomorphism is equivalent to being \'etale, universally injective,
and surjective, see
Morphisms of Spaces, Lemma
\ref{spaces-morphisms-lemma-etale-universally-injective-open}.
Thus the lemma follows from
Lemmas \ref{lemma-descend-etale},
\ref{lemma-descend-surjective}, and
\ref{lemma-descend-universally-injective}.
\end{proof}

\begin{lemma}
\label{lemma-descend-monomorphism}
Notation and assumptions as in Situation \ref{situation-descent-property}. If
\begin{enumerate}
\item $f$ is a monomorphism,
\item $f_0$ is locally of finite type,
\end{enumerate}
then $f_i$ is a monomorphism for some $i \geq 0$.
\end{lemma}

\begin{proof}
Recall that a morphism is a monomorphism if and only if the diagonal is
an isomorphism. The morphism $X_0 \to X_0 \times_{Y_0} X_0$ is locally of
finite presentation by
Morphisms of Spaces, Lemma
\ref{spaces-morphisms-lemma-diagonal-morphism-finite-type}.
Since $X_0 \times_{Y_0} X_0$ is quasi-compact and quasi-separated
we conclude from
Lemma \ref{lemma-descend-isomorphism}
that $\Delta_i : X_i \to X_i \times_{Y_i} X_i$ is an isomorphism for
some $i \geq 0$. For this $i$ the morphism $f_i$ is a monomorphism.
\end{proof}

\begin{lemma}
\label{lemma-descend-flat}
Notation and assumptions as in Situation \ref{situation-descent-property}.
Let $\mathcal{F}_0$ be a quasi-coherent $\mathcal{O}_{X_0}$-module
and denote $\mathcal{F}_i$ the pullback to $X_i$ and $\mathcal{F}$
the pullback to $X$. If
\begin{enumerate}
\item $\mathcal{F}$ is flat over $Y$,
\item $\mathcal{F}_0$ is of finite presentation, and
\item $f_0$ is locally of finite presentation,
\end{enumerate}
then $\mathcal{F}_i$ is flat over $Y_i$ for some $i \geq 0$.
In particular, if $f_0$ is locally of finite presentation and
$f$ is flat, then $f_i$ is flat for some $i \geq 0$.
\end{lemma}

\begin{proof}
Choose an affine scheme $V_0$ and a surjective \'etale morphism
$V_0 \to Y_0$. Choose an affine scheme $U_0$ and a surjective \'etale
morphism $U_0 \to V_0 \times_{Y_0} X_0$. Diagram
$$
\xymatrix{
U_0 \ar[d] \ar[r] & V_0 \ar[d] \\
X_0 \ar[r] & Y_0
}
$$
The vertical arrows are surjective and \'etale by construction.
We can base change this diagram to $B_i$ or $B$ to get
$$
\vcenter{
\xymatrix{
U_i \ar[d] \ar[r] & V_i \ar[d] \\
X_i \ar[r] & Y_i
}
}
\quad\text{and}\quad
\vcenter{
\xymatrix{
U \ar[d] \ar[r] & V \ar[d] \\
X \ar[r] & Y
}
}
$$
Note that $U_i, V_i, U, V$ are affine schemes, the vertical morphisms are
surjective \'etale, and the limit of the morphisms $U_i \to V_i$ is
$U \to V$. Recall that $\mathcal{F}_i$ is flat over $Y_i$ if and only if
$\mathcal{F}_i|_{U_i}$ is flat over $V_i$ and similarly $\mathcal{F}$ is flat
over $Y$ if and only if $\mathcal{F}|_U$ is flat over $V$
(Morphisms of Spaces, Definition \ref{spaces-morphisms-definition-flat}).
Since $f_0$ is locally of finite presentation, so is the morphism
$U_0 \to V_0$. Hence the lemma follows
from Limits, Lemma \ref{limits-lemma-descend-module-flat-finite-presentation}.
\end{proof}

\begin{lemma}
\label{lemma-eventually-proper}
Assumptions and notation as in Situation \ref{situation-descent-property}.
If
\begin{enumerate}
\item $f$ is proper, and
\item $f_0$ is locally of finite type,
\end{enumerate}
then there exists an $i$ such that $f_i$ is proper.
\end{lemma}

\begin{proof}
Choose an affine scheme $V_0$ and a surjective \'etale morphism $V_0 \to Y_0$.
Set $V_i = Y_i \times_{Y_0} V_0$ and $V = Y \times_{Y_0} V_0$.
It suffices to prove that the base change of $f_i$ to $V_i$ is
proper, see Morphisms of Spaces, Lemma
\ref{spaces-morphisms-lemma-proper-local}.
Thus we may assume $Y_0$ is affine.

\medskip\noindent
By Lemma \ref{lemma-descend-separated-morphism} we see that
$f_i$ is separated for some $i \geq 0$. Replacing
$0$ by $i$ we may assume that $f_0$ is separated.
Observe that $f_0$ is quasi-compact. Thus $f_0$ is separated and
of finite type. By
Cohomology of Spaces, Lemma \ref{spaces-cohomology-lemma-weak-chow}
we can choose a diagram
$$
\xymatrix{
X_0 \ar[rd] & X_0' \ar[d] \ar[l]^\pi \ar[r] & \mathbf{P}^n_{Y_0} \ar[dl] \\
& Y_0 &
}
$$
where $X_0' \to \mathbf{P}^n_{Y_0}$ is an immersion, and
$\pi : X_0' \to X_0$ is proper and surjective. Introduce
$X' = X_0' \times_{Y_0} Y$ and $X_i' = X_0' \times_{Y_0} Y_i$.
By Morphisms of Spaces, Lemmas
\ref{spaces-morphisms-lemma-composition-proper} and
\ref{spaces-morphisms-lemma-base-change-proper}
we see that $X' \to Y$ is proper. Hence $X' \to \mathbf{P}^n_Y$ is
a closed immersion (Morphisms of Spaces, Lemma
\ref{spaces-morphisms-lemma-universally-closed-permanence}). By
Morphisms of Spaces, Lemma \ref{spaces-morphisms-lemma-image-proper-is-proper}
it suffices to prove that $X'_i \to Y_i$ is proper for some $i$.
By Lemma \ref{lemma-descend-closed-immersion}
we find that $X'_i \to \mathbf{P}^n_{Y_i}$ is
a closed immersion for $i$ large enough. Then $X'_i \to Y_i$
is proper and we win.
\end{proof}

\begin{lemma}
\label{lemma-eventually-relative-dimension}
Assumptions and notation as in Situation \ref{situation-descent-property}.
Let $d \geq 0$. If
\begin{enumerate}
\item $f$ has relative dimension $\leq d$
(Morphisms of Spaces, Definition
\ref{spaces-morphisms-definition-relative-dimension}), and
\item $f_0$ is locally of finite type,
\end{enumerate}
then there exists an $i$ such that $f_i$ has relative dimension $\leq d$.
\end{lemma}

\begin{proof}
Choose an affine scheme $V_0$ and a surjective \'etale morphism
$V_0 \to Y_0$. Choose an affine scheme $U_0$ and a surjective \'etale
morphism $U_0 \to V_0 \times_{Y_0} X_0$. Diagram
$$
\xymatrix{
U_0 \ar[d] \ar[r] & V_0 \ar[d] \\
X_0 \ar[r] & Y_0
}
$$
The vertical arrows are surjective and \'etale by construction.
We can base change this diagram to $B_i$ or $B$ to get
$$
\vcenter{
\xymatrix{
U_i \ar[d] \ar[r] & V_i \ar[d] \\
X_i \ar[r] & Y_i
}
}
\quad\text{and}\quad
\vcenter{
\xymatrix{
U \ar[d] \ar[r] & V \ar[d] \\
X \ar[r] & Y
}
}
$$
Note that $U_i, V_i, U, V$ are affine schemes,
the vertical morphisms are surjective \'etale, and the limit of the
morphisms $U_i \to V_i$ is $U \to V$.
In this situation $X_i \to Y_i$ has relative dimension $\leq d$
if and only if $U_i \to V_i$ has relative dimension $\leq d$
(as defined in Morphisms, Definition
\ref{morphisms-definition-relative-dimension-d}).
To see the equivalence, use that the definition for morphisms
of algebraic spaces involves Morphisms of Spaces, Definition
\ref{spaces-morphisms-definition-dimension-fibre}
which uses \'etale localization. The same is true for $X \to Y$ and $U \to V$.
Since $f_0$ is locally of finite type, so is the morphism $U_0 \to V_0$.
Hence the lemma follows from the more general
Limits, Lemma \ref{limits-lemma-limit-dimension}.
\end{proof}














\section{Descending relative objects}
\label{section-descending-relative}

\noindent
The following lemma is typical of the type of results in this section.

\begin{lemma}
\label{lemma-descend-finite-presentation}
Let $S$ be a scheme. Let $I$ be a directed set.
Let $(X_i, f_{ii'})$ be an inverse system over $I$ of algebraic spaces
over $S$. Assume
\begin{enumerate}
\item the morphisms $f_{ii'} : X_i \to X_{i'}$ are affine,
\item the spaces $X_i$ are quasi-compact and quasi-separated.
\end{enumerate}
Let $X = \lim_i X_i$. Then the category of algebraic spaces
of finite presentation over $X$ is the colimit over $I$ of the
categories of algebraic spaces of finite presentation over $X_i$.
\end{lemma}

\begin{proof}
Pick $0 \in I$. Choose a surjective \'etale morphism $U_0 \to X_0$ where
$U_0$ is an affine scheme (Properties of Spaces, Lemma
\ref{spaces-properties-lemma-quasi-compact-affine-cover}).
Set $U_i = X_i \times_{X_0} U_0$. Set $R_0 = U_0 \times_{X_0} U_0$ and
$R_i = R_0 \times_{X_0} X_i$. Denote $s_i, t_i : R_i \to U_i$ and
$s, t : R \to U$ the two projections. In the proof of
Lemma \ref{lemma-directed-inverse-system-has-limit} we have
seen that there exists a presentation $X = U/R$ with
$U = \lim U_i$ and $R = \lim R_i$. Note that $U_i$ and $U$ are affine and
that $R_i$ and $R$ are quasi-compact and separated (as $X_i$ is
quasi-separated). Let $Y$ be an algebraic space over $S$ and let
$Y \to X$ be a morphism of finite presentation. Set $V = U \times_X Y$.
This is an algebraic space of finite presentation over $U$.
Choose an affine scheme $W$ and a surjective \'etale morphism $W \to V$.
Then $W \to Y$ is surjective \'etale as well. Set $R' = W \times_Y W$
so that $Y = W/R'$ (see Spaces, Section \ref{spaces-section-presentations}).
Note that $W$ is a scheme of finite presentation over $U$ and that $R'$
is a scheme of finite presentation over $R$ (details omitted).
By Limits, Lemma \ref{limits-lemma-descend-finite-presentation}
we can find an index $i$ and a morphism of schemes $W_i \to U_i$ of
finite presentation whose base change to $U$ gives $W \to U$. Similarly
we can find, after possibly increasing $i$, a scheme $R'_i$ of finite
presentation over $R_i$ whose base change to $R$ is $R'$.
The projection morphisms $s', t' : R' \to W$ are morphisms over
the projection morphisms $s, t : R \to U$. Hence we can view $s'$,
resp.\ $t'$ as a morphism between schemes of finite presentation over
$U$ (with structure morphism $R' \to U$ given by $R' \to R$ followed
by $s$, resp.\ $t$). Hence we can apply
Limits, Lemma \ref{limits-lemma-descend-finite-presentation}
again to see that, after possibly increasing $i$, there exist
morphisms $s'_i, t'_i : R'_i \to W_i$, whose base change to $U$
is $S', t'$. By Limits, Lemmas \ref{limits-lemma-descend-etale} and
\ref{limits-lemma-descend-monomorphism}
we may assume that $s'_i, t'_i$ are \'etale and that
$j'_i : R'_i \to W_i \times_{X_i} W_i$ is a monomorphism (here we
view $j'_i$ as a morphism of schemes of finite presentation over $U_i$ via
one of the projections -- it doesn't matter which one). Setting
$Y_i = W_i/R'_i$ (see Spaces, Theorem \ref{spaces-theorem-presentation})
we obtain an algebraic space of finite presentation
over $X_i$ whose base change to $X$ is isomorphic to $Y$.

\medskip\noindent
This shows that every algebraic space of finite presentation over $X$ comes
from an algebraic space of finite presentation over some $X_i$, i.e.,
it shows that the functor of the lemma is essentially surjective. To
show that it is fully faithful, consider an index $0 \in I$ and two
algebraic spaces $Y_0, Z_0$ of finite presentation over $X_0$.
Set $Y_i = X_i \times_{X_0} Y_0$, $Y = X \times_{X_0} Y_0$,
$Z_i = X_i \times_{X_0} Z_0$, and $Z = X \times_{X_0} Z_0$. Let
$\alpha : Y \to Z$ be a morphism of algebraic spaces over $X$.
Choose a surjective \'etale morphism $V_0 \to Y_0$ where $V_0$ is
an affine scheme. Set $V_i = V_0 \times_{Y_0} Y_i$ and
$V = V_0 \times_{Y_0} Y$ which are affine schemes endowed with
surjective \'etale morphisms to $Y_i$ and $Y$. The composition
$V \to Y \to Z \to Z_0$ comes from a (essentially unique) morphism
$V_i \to Z_0$ for some $i \geq 0$ by
Proposition \ref{proposition-characterize-locally-finite-presentation}
(applied to $Z_0 \to X_0$ which is of finite presentation by assumption).
After increasing $i$ the two compositions
$$
V_i \times_{Y_i} V_i \to V_i \to Z_0
$$
are equal as this is true in the limit. Hence we obtain a (essentially unique)
morphism $Y_i \to Z_0$. Since this is a morphism over $X_0$
it induces a morphism into $Z_i = Z_0 \times_{X_0} X_i$ as desired.
\end{proof}

\begin{lemma}
\label{lemma-descend-modules-finite-presentation}
With notation and assumptions as in
Lemma \ref{lemma-descend-finite-presentation}.
The category of $\mathcal{O}_X$-modules of finite presentation is the
colimit over $I$ of the categories $\mathcal{O}_{X_i}$-modules of finite
presentation.
\end{lemma}

\begin{proof}
Choose $0 \in I$. Choose an affine scheme $U_0$ and a surjective
\'etale morphism $U_0 \to X_0$. Set $U_i = X_i \times_{X_0} U_0$.
Set $R_0 = U_0 \times_{X_0} U_0$ and $R_i = R_0 \times_{X_0} X_i$.
Denote $s_i, t_i : R_i \to U_i$ and $s, t : R \to U$ the two
projections. In the proof of
Lemma \ref{lemma-directed-inverse-system-has-limit} we have
seen that there exists a presentation $X = U/R$ with
$U = \lim U_i$ and $R = \lim R_i$. Note that $U_i$ and $U$ are affine and
that $R_i$ and $R$ are quasi-compact and separated (as $X_i$ is
quasi-separated). Moreover, it is also true that
$R \times_{s, U, t} R = \colim R_i \times_{s_i, U_i, t_i} R_i$.
Thus we know that
$\QCoh(\mathcal{O}_U) = \colim \QCoh(\mathcal{O}_{U_i})$,
$\QCoh(\mathcal{O}_R) = \colim \QCoh(\mathcal{O}_{R_i})$,
and
$\QCoh(\mathcal{O}_{R \times_{s, U, t} R}) =
\colim \QCoh(\mathcal{O}_{R_i \times_{s_i, U_i, t_i} R_i})$ by
Limits, Lemma \ref{limits-lemma-descend-modules-finite-presentation}.
We have $\QCoh(\mathcal{O}_X) = \QCoh(U, R, s, t, c)$ and
$\QCoh(\mathcal{O}_{X_i}) = \QCoh(U_i, R_i, s_i, t_i, c_i)$,
see Properties of Spaces, Proposition
\ref{spaces-properties-proposition-quasi-coherent}.
Thus the result follows formally.
\end{proof}

\begin{lemma}
\label{lemma-descend-invertible-modules}
With notation and assumptions as in
Lemma \ref{lemma-descend-finite-presentation}. Then
\begin{enumerate}
\item any finite locally free $\mathcal{O}_X$-module is the pullback
of a finite locally free $\mathcal{O}_{X_i}$-module for some $i$,
\item any invertible $\mathcal{O}_X$-module is the pullback of an invertible
$\mathcal{O}_{X_i}$-module for some $i$.
\end{enumerate}
\end{lemma}

\begin{proof}
Proof of (2).
Let $\mathcal{L}$ be an invertible $\mathcal{O}_X$-module. Since
invertible modules are of finite presentation we can find an $i$
and modules $\mathcal{L}_i$ and $\mathcal{N}_i$ of finite presentation
over $X_i$ such that $f_i^*\mathcal{L}_i \cong \mathcal{L}$ and
$f_i^*\mathcal{N}_i \cong \mathcal{L}^{\otimes -1}$, see
Lemma \ref{lemma-descend-modules-finite-presentation}.
Since pullback commutes with tensor product we see that
$f_i^*(\mathcal{L}_i \otimes_{\mathcal{O}_{X_i}} \mathcal{N}_i)$
is isomorphic to $\mathcal{O}_X$. Since the tensor product of
finitely presented modules is finitely presented, the same
lemma implies that
$f_{i'i}^*\mathcal{L}_i
\otimes_{\mathcal{O}_{X_{i'}}} f_{i'i}^*\mathcal{N}_i$
is isomorphic to $\mathcal{O}_{X_{i'}}$ for some $i' \geq i$.
It follows that $f_{i'i}^*\mathcal{L}_i$ is invertible
(Modules on Sites, Lemma \ref{sites-modules-lemma-invertible})
and the proof is complete.

\medskip\noindent
Proof of (1). Omitted. Hint: argue as in the proof of (2)
using that a module (on a locally ringed site) is finite locally free
if and only if it has a dual, see
Modules on Sites, Section \ref{sites-modules-section-duals}.
Alternatively, argue as in the proof for schemes, see
Limits, Lemma \ref{limits-lemma-descend-invertible-modules}.
\end{proof}














\section{Absolute Noetherian approximation}
\label{section-approximation}

\noindent
The following result is \cite[Theorem 1.2.2]{CLO}.
A key ingredient in the proof is
Decent Spaces, Lemma
\ref{decent-spaces-lemma-filter-quasi-compact-quasi-separated}.

\begin{proposition}
\label{proposition-approximate}
\begin{reference}
Our proof follows closely the proof given in \cite[Theorem 1.2.2]{CLO}.
\end{reference}
Let $X$ be a quasi-compact and quasi-separated algebraic space over
$\Spec(\mathbf{Z})$. There exist a directed set $I$
and an inverse system of algebraic spaces $(X_i, f_{ii'})$ over $I$
such that
\begin{enumerate}
\item the transition morphisms $f_{ii'}$ are affine
\item each $X_i$ is quasi-separated and of finite type over
$\mathbf{Z}$, and
\item $X = \lim X_i$.
\end{enumerate}
\end{proposition}

\begin{proof}
We apply Decent Spaces, Lemma
\ref{decent-spaces-lemma-filter-quasi-compact-quasi-separated}
to get open subspaces $U_p \subset X$, schemes $V_p$, and morphisms
$f_p : V_p \to U_p$ with properties as stated. Note that
$f_n : V_n \to U_n$ is an \'etale morphism of algebraic spaces
whose restriction to the inverse image of $T_n = (V_n)_{red}$ is an
isomorphism. Hence $f_n$ is an isomorphism, for example by
Morphisms of Spaces, Lemma
\ref{spaces-morphisms-lemma-etale-universally-injective-open}.
In particular $U_n$ is a quasi-compact and separated scheme.
Thus we can write $U_n = \lim U_{n, i}$ as a directed limit
of schemes of finite type over $\mathbf{Z}$ with affine transition
morphisms, see Limits, Proposition \ref{limits-proposition-approximate}.
Thus, applying descending induction on $p$, we see that we have reduced
to the problem posed in the following paragraph.

\medskip\noindent
Here we have $U \subset X$, $U = \lim U_i$, $Z \subset X$, and
$f : V \to X$ with the following properties
\begin{enumerate}
\item $X$ is a quasi-compact and quasi-separated algebraic space,
\item $V$ is a quasi-compact and separated scheme,
\item $U \subset X$ is a quasi-compact open subspace,
\item $(U_i, g_{ii'})$ is a directed inverse system of
quasi-separated algebraic spaces
of finite type over $\mathbf{Z}$ with affine transition morphisms
whose limit is $U$,
\item $Z \subset X$ is a closed subspace such that $|X| = |U| \amalg |Z|$,
\item $f : V \to X$ is a surjective \'etale morphism such that
$f^{-1}(Z) \to Z$ is an isomorphism.
\end{enumerate}
Problem: Show that the conclusion of the proposition holds for $X$.

\medskip\noindent
Note that $W = f^{-1}(U) \subset V$ is a quasi-compact open subscheme
\'etale over $U$. Hence we may apply
Lemmas \ref{lemma-descend-finite-presentation} and \ref{lemma-descend-etale}
to find an index $0 \in I$ and an \'etale morphism $W_0 \to U_0$
of finite presentation whose base change to $U$ produces $W$. Setting
$W_i = W_0 \times_{U_0} U_i$ we see that $W = \lim_{i \geq 0} W_i$. After
increasing $0$ we may assume the $W_i$ are schemes, see
Lemma \ref{lemma-limit-is-scheme}.
Moreover, $W_i$ is of finite type over $\mathbf{Z}$.

\medskip\noindent
Apply Limits, Lemma \ref{limits-lemma-approximate} to
$W = \lim_{i \geq 0} W_i$ and the inclusion $W \subset V$. Replace $I$
by the directed set $J$ found in that lemma. This allows us
to write $V$ as a directed limit $V = \lim V_i$ of finite type schemes over
$\mathbf{Z}$ with affine transition maps such that each $V_i$ contains
$W_i$ as an open subscheme (compatible with transition morphisms).
For each $i$ we can form the push out
$$
\xymatrix{
W_i \ar[r] \ar[d]_\Delta & V_i \ar[d] \\
W_i \times_{U_i} W_i \ar[r] & R_i
}
$$
in the category of schemes. Namely, the left vertical and upper horizontal
arrows are open immersions of schemes. In other words, we can construct
$R_i$ as the glueing of $V_i$ and $W_i \times_{U_i} W_i$ along the common open
$W_i$ (see Schemes, Section \ref{schemes-section-glueing-schemes}). Note that
the \'etale projection maps $W_i \times_{U_i} W_i \to W_i$ extend
to \'etale morphisms $s_i, t_i : R_i \to V_i$. It is clear that the
morphism $j_i = (t_i, s_i) : R_i \to V_i \times V_i$ is an \'etale
equivalence relation on $V_i$. Note that $W_i \times_{U_i} W_i$ is
quasi-compact (as $U_i$ is quasi-separated and $W_i$ quasi-compact)
and $V_i$ is quasi-compact, hence $R_i$ is quasi-compact. For
$i \geq i'$ the diagram
\begin{equation}
\label{equation-cartesian}
\vcenter{
\xymatrix{
R_i \ar[r] \ar[d]_{s_i} & R_{i'} \ar[d]^{s_{i'}} \\
V_i \ar[r] & V_{i'}
}
}
\end{equation}
is cartesian because
$$
(W_{i'} \times_{U_{i'}} W_{i'}) \times_{U_{i'}} U_i =
W_{i'} \times_{U_{i'}} U_i \times_{U_i} U_i \times_{U_{i'}} W_{i'} =
W_i \times_{U_i} W_i.
$$
Consider the algebraic space $X_i = V_i/R_i$ (see
Spaces, Theorem \ref{spaces-theorem-presentation}).
As $V_i$ is of finite type over $\mathbf{Z}$ and $R_i$ is quasi-compact
we see that $X_i$ is quasi-separated and of finite type over $\mathbf{Z}$
(see
Properties of Spaces, Lemma \ref{spaces-properties-lemma-quasi-separated}
and
Morphisms of Spaces, Lemmas
\ref{spaces-morphisms-lemma-surjection-from-quasi-compact} and
\ref{spaces-morphisms-lemma-finite-type-local}).
As the construction of $R_i$ above is compatible
with transition morphisms, we obtain morphisms of algebraic spaces
$X_i \to X_{i'}$ for $i \geq i'$. The commutative diagrams
$$
\xymatrix{
V_i \ar[r] \ar[d] & V_{i'} \ar[d] \\
X_i \ar[r] & X_{i'}
}
$$
are cartesian as (\ref{equation-cartesian}) is cartesian, see
Groupoids, Lemma \ref{groupoids-lemma-criterion-fibre-product}.
Since $V_i \to V_{i'}$ is affine, this implies that $X_i \to X_{i'}$
is affine, see
Morphisms of Spaces, Lemma \ref{spaces-morphisms-lemma-affine-local}.
Thus we can form the limit $X' = \lim X_i$ by
Lemma \ref{lemma-directed-inverse-system-has-limit}.
We claim that $X \cong X'$ which finishes the proof of the proposition.

\medskip\noindent
Proof of the claim. Set $R = \lim R_i$.
By construction the algebraic space $X'$ comes
equipped with a surjective \'etale morphism $V \to X'$ such that
$$
V \times_{X'} V \cong R
$$
(use Lemma \ref{lemma-directed-inverse-system-has-limit}).
By construction $\lim W_i \times_{U_i} W_i = W \times_U W$ and $V = \lim V_i$
so that $R$ is the union of $W \times_U W$ and $V$ glued along $W$.
Property (6) implies the projections $V \times_X V \to V$ are isomorphisms
over $f^{-1}(Z) \subset V$. Hence the scheme $V \times_X V$ is the union
of the opens $\Delta_{V/X}(V)$ and $W \times_U W$ which intersect
along $\Delta_{W/X}(W)$. We conclude that there exists a unique isomorphism
$R \cong V \times_X V$ compatible with the projections to $V$.
Since $V \to X$ and $V \to X'$ are surjective \'etale we see that
$$
X = V/ V \times_X V = V/R = V/V \times_{X'} V = X'
$$
by Spaces, Lemma \ref{spaces-lemma-space-presentation} and we win.
\end{proof}




\section{Applications}
\label{section-applications}

\noindent
The following lemma can also be deduced directly from
Decent Spaces, Lemma
\ref{decent-spaces-lemma-filter-quasi-compact-quasi-separated}
without passing through absolute Noetherian approximation.

\begin{lemma}
\label{lemma-colimit-finitely-presented}
Let $S$ be a scheme. Let $X$ be a quasi-compact and quasi-separated algebraic
space over $S$. Every quasi-coherent $\mathcal{O}_X$-module is a
filtered colimit of finitely presented $\mathcal{O}_X$-modules.
\end{lemma}

\begin{proof}
We may view $X$ as an algebraic space over $\Spec(\mathbf{Z})$, see
Spaces, Definition \ref{spaces-definition-base-change} and
Properties of Spaces, Definition \ref{spaces-properties-definition-separated}.
Thus we may apply Proposition \ref{proposition-approximate}
and write $X = \lim X_i$ with $X_i$ of finite presentation over $\mathbf{Z}$.
Thus $X_i$ is a Noetherian algebraic space, see
Morphisms of Spaces, Lemma
\ref{spaces-morphisms-lemma-finite-presentation-noetherian}.
The morphism $X \to X_i$ is affine, see
Lemma \ref{lemma-directed-inverse-system-has-limit}.
Conclusion by
Cohomology of Spaces, Lemma
\ref{spaces-cohomology-lemma-direct-colimit-finite-presentation}.
\end{proof}

\noindent
The rest of this section consists of straightforward
applications of Lemma \ref{lemma-colimit-finitely-presented}.

\begin{lemma}
\label{lemma-directed-colimit-finite-type}
Let $S$ be a scheme. Let $X$ be a quasi-compact and quasi-separated
algebraic space over $S$.
Let $\mathcal{F}$ be a quasi-coherent $\mathcal{O}_X$-module.
Then $\mathcal{F}$ is the directed colimit of its finite type
quasi-coherent submodules.
\end{lemma}

\begin{proof}
If $\mathcal{G}, \mathcal{H} \subset \mathcal{F}$ are finite type
quasi-coherent $\mathcal{O}_X$-submodules then the image
of $\mathcal{G} \oplus \mathcal{H} \to \mathcal{F}$ is another
finite type quasi-coherent $\mathcal{O}_X$-submodule which contains
both of them. In this way we see that the system is directed.
To show that $\mathcal{F}$ is the colimit of this system, write
$\mathcal{F} = \colim_i \mathcal{F}_i$ as a directed
colimit of finitely presented quasi-coherent sheaves as in
Lemma \ref{lemma-colimit-finitely-presented}.
Then the images $\mathcal{G}_i = \Im(\mathcal{F}_i \to \mathcal{F})$ are
finite type quasi-coherent subsheaves of $\mathcal{F}$. Since
$\mathcal{F}$ is the colimit of these the result follows.
\end{proof}

\begin{lemma}
\label{lemma-finite-directed-colimit-surjective-maps}
Let $S$ be a scheme. Let $X$ be a quasi-compact and quasi-separated
algebraic space over $S$. Let $\mathcal{F}$ be a finite type
quasi-coherent $\mathcal{O}_X$-module. Then we can write
$\mathcal{F} = \lim \mathcal{F}_i$ where each $\mathcal{F}_i$ is an
$\mathcal{O}_X$-module of finite presentation and all transition maps
$\mathcal{F}_i \to \mathcal{F}_{i'}$ surjective.
\end{lemma}

\begin{proof}
Write $\mathcal{F} = \colim \mathcal{G}_i$ as a filtered colimit of
finitely presented $\mathcal{O}_X$-modules
(Lemma \ref{lemma-colimit-finitely-presented}).
We claim that $\mathcal{G}_i \to \mathcal{F}$ is surjective for some $i$.
Namely, choose an \'etale surjection $U \to X$ where $U$ is an affine scheme.
Choose finitely many sections $s_k \in \mathcal{F}(U)$ generating
$\mathcal{F}|_U$. Since $U$ is affine we see that $s_k$ is in the image
of $\mathcal{G}_i \to \mathcal{F}$ for $i$ large enough. Hence
$\mathcal{G}_i \to \mathcal{F}$ is surjective for $i$ large enough.
Choose such an $i$ and let $\mathcal{K} \subset \mathcal{G}_i$ be the
kernel of the map $\mathcal{G}_i \to \mathcal{F}$. Write
$\mathcal{K} = \colim \mathcal{K}_a$
as the filtered colimit of its finite type quasi-coherent submodules
(Lemma \ref{lemma-directed-colimit-finite-type}). Then
$\mathcal{F} = \colim \mathcal{G}_i/\mathcal{K}_a$ is a solution
to the problem posed by the lemma.
\end{proof}

\noindent
Let $X$ be an algebraic space. In the following lemma we use the notion
of a {\it finitely presented quasi-coherent $\mathcal{O}_X$-algebra
$\mathcal{A}$}. This means that for every affine
$U = \Spec(R)$ \'etale over $X$ we have $\mathcal{A}|_U = \widetilde{A}$
where $A$ is a (commutative) $R$-algebra which is of finite presentation
as an $R$-algebra.

\begin{lemma}
\label{lemma-algebra-directed-colimit-finite-presentation}
Let $S$ be a scheme. Let $X$ be a quasi-compact and quasi-separated
algebraic space over $S$.
Let $\mathcal{A}$ be a quasi-coherent $\mathcal{O}_X$-algebra.
Then $\mathcal{A}$ is a directed colimit of finitely presented
quasi-coherent $\mathcal{O}_X$-algebras.
\end{lemma}

\begin{proof}
First we write $\mathcal{A} = \colim_i \mathcal{F}_i$ as a directed
colimit of finitely presented quasi-coherent sheaves as in
Lemma \ref{lemma-colimit-finitely-presented}.
For each $i$ let $\mathcal{B}_i = \text{Sym}(\mathcal{F}_i)$ be the
symmetric algebra on $\mathcal{F}_i$ over $\mathcal{O}_X$. Write
$\mathcal{I}_i = \Ker(\mathcal{B}_i \to \mathcal{A})$. Write
$\mathcal{I}_i = \colim_j \mathcal{F}_{i, j}$ where
$\mathcal{F}_{i, j}$ is a finite type quasi-coherent submodule of
$\mathcal{I}_i$, see
Lemma \ref{lemma-directed-colimit-finite-type}.
Set $\mathcal{I}_{i, j} \subset \mathcal{I}_i$
equal to the $\mathcal{B}_i$-ideal generated by $\mathcal{F}_{i, j}$.
Set $\mathcal{A}_{i, j} = \mathcal{B}_i/\mathcal{I}_{i, j}$.
Then $\mathcal{A}_{i, j}$ is a quasi-coherent finitely presented
$\mathcal{O}_X$-algebra. Define $(i, j) \leq (i', j')$ if
$i \leq i'$ and the map $\mathcal{B}_i \to \mathcal{B}_{i'}$
maps the ideal $\mathcal{I}_{i, j}$ into the ideal $\mathcal{I}_{i', j'}$.
Then it is clear that $\mathcal{A} = \colim_{i, j} \mathcal{A}_{i, j}$.
\end{proof}

\noindent
Let $X$ be an algebraic space. In the following lemma we use the notion
of a {\it quasi-coherent $\mathcal{O}_X$-algebra $\mathcal{A}$
of finite type}. This means that for every affine
$U = \Spec(R)$ \'etale over $X$ we have $\mathcal{A}|_U = \widetilde{A}$
where $A$ is a (commutative) $R$-algebra which is of finite type
as an $R$-algebra.

\begin{lemma}
\label{lemma-algebra-directed-colimit-finite-type}
Let $S$ be a scheme. Let $X$ be a quasi-compact and quasi-separated algebraic
space over $S$. Let $\mathcal{A}$ be a quasi-coherent $\mathcal{O}_X$-algebra.
Then $\mathcal{A}$ is the directed colimit of its finite type
quasi-coherent $\mathcal{O}_X$-subalgebras.
\end{lemma}

\begin{proof}
Omitted. Hint: Compare with the proof of
Lemma \ref{lemma-directed-colimit-finite-type}.
\end{proof}

\noindent
Let $X$ be an algebraic space. In the following lemma we use the notion
of a {\it finite (resp.\ integral) quasi-coherent
$\mathcal{O}_X$-algebra $\mathcal{A}$}. This means that for every
affine $U = \Spec(R)$ \'etale over $X$ we have
$\mathcal{A}|_U = \widetilde{A}$ where $A$ is a (commutative) $R$-algebra
which is finite (resp.\ integral) as an $R$-algebra.

\begin{lemma}
\label{lemma-finite-algebra-directed-colimit-finite-finitely-presented}
Let $S$ be a scheme. Let $X$ be a quasi-compact and quasi-separated
algebraic space over $S$. Let $\mathcal{A}$ be a finite quasi-coherent
$\mathcal{O}_X$-algebra. Then $\mathcal{A} = \colim \mathcal{A}_i$
is a directed colimit of finite and finitely presented quasi-coherent
$\mathcal{O}_X$-algebras with surjective transition maps.
\end{lemma}

\begin{proof}
By Lemma \ref{lemma-finite-directed-colimit-surjective-maps}
there exists a finitely presented $\mathcal{O}_X$-module
$\mathcal{F}$ and a surjection $\mathcal{F} \to \mathcal{A}$.
Using the algebra structure we obtain a surjection
$$
\text{Sym}^*_{\mathcal{O}_X}(\mathcal{F}) \longrightarrow \mathcal{A}
$$
Denote $\mathcal{J}$ the kernel. Write $\mathcal{J} = \colim \mathcal{E}_i$
as a filtered colimit of finite type $\mathcal{O}_X$-submodules
$\mathcal{E}_i$ (Lemma \ref{lemma-directed-colimit-finite-type}). Set
$$
\mathcal{A}_i = \text{Sym}^*_{\mathcal{O}_X}(\mathcal{F})/(\mathcal{E}_i)
$$
where $(\mathcal{E}_i)$ indicates the ideal sheaf generated by
the image of $\mathcal{E}_i \to \text{Sym}^*_{\mathcal{O}_X}(\mathcal{F})$.
Then each $\mathcal{A}_i$ is a finitely presented $\mathcal{O}_X$-algebra,
the transition maps are surjective, and $\mathcal{A} = \colim \mathcal{A}_i$.
To finish the proof we still
have to show that $\mathcal{A}_i$ is a finite $\mathcal{O}_X$-algebra
for $i$ sufficiently large. To do this we choose an \'etale surjective
map $U \to X$ where $U$ is an affine scheme. Take generators
$f_1, \ldots, f_m \in \Gamma(U, \mathcal{F})$.
As $\mathcal{A}(U)$ is a finite $\mathcal{O}_X(U)$-algebra we
see that for each $j$ there exists a monic polynomial
$P_j \in \mathcal{O}(U)[T]$ such that $P_j(f_j)$ is zero in $\mathcal{A}(U)$.
Since $\mathcal{A} = \colim \mathcal{A}_i$ by construction, we
have $P_j(f_j) = 0$ in $\mathcal{A}_i(U)$ for all sufficiently large $i$.
For such $i$ the algebras $\mathcal{A}_i$ are finite.
\end{proof}

\begin{lemma}
\label{lemma-integral-algebra-directed-colimit-finite}
Let $S$ be a scheme. Let $X$ be a quasi-compact and quasi-separated
algebraic space over $S$. Let $\mathcal{A}$ be an integral quasi-coherent
$\mathcal{O}_X$-algebra. Then
\begin{enumerate}
\item $\mathcal{A}$ is the directed colimit of its finite
quasi-coherent $\mathcal{O}_X$-subalgebras, and
\item $\mathcal{A}$ is a directed colimit of finite and finitely presented
$\mathcal{O}_X$-algebras.
\end{enumerate}
\end{lemma}

\begin{proof}
By Lemma \ref{lemma-algebra-directed-colimit-finite-type} we have
$\mathcal{A} = \colim \mathcal{A}_i$ where
$\mathcal{A}_i \subset \mathcal{A}$ runs through the
quasi-coherent $\mathcal{O}_X$-sub algebras of finite type.
Any finite type quasi-coherent $\mathcal{O}_X$-subalgebra
of $\mathcal{A}$ is finite (use Algebra, Lemma
\ref{algebra-lemma-characterize-finite-in-terms-of-integral}
on affine schemes \'etale over $X$). This proves (1).

\medskip\noindent
To prove (2), write $\mathcal{A} = \colim \mathcal{F}_i$
as a colimit of finitely presented $\mathcal{O}_X$-modules using
Lemma \ref{lemma-colimit-finitely-presented}.
For each $i$, let $\mathcal{J}_i$ be the kernel of the map
$$
\text{Sym}^*_{\mathcal{O}_X}(\mathcal{F}_i) \longrightarrow \mathcal{A}
$$
For $i' \geq i$ there is an induced map $\mathcal{J}_i \to \mathcal{J}_{i'}$
and we have $\mathcal{A} =
\colim \text{Sym}^*_{\mathcal{O}_X}(\mathcal{F}_i)/\mathcal{J}_i$.
Moreover, the quasi-coherent $\mathcal{O}_X$-algebras
$\text{Sym}^*_{\mathcal{O}_X}(\mathcal{F}_i)/\mathcal{J}_i$
are finite (see above). Write $\mathcal{J}_i = \colim \mathcal{E}_{ik}$
as a colimit of finitely presented $\mathcal{O}_X$-modules.
Given $i' \geq i$ and $k$ there exists a $k'$ such that we
have a map $\mathcal{E}_{ik} \to \mathcal{E}_{i'k'}$
making
$$
\xymatrix{
\mathcal{J}_i \ar[r] & \mathcal{J}_{i'} \\
\mathcal{E}_{ik} \ar[u] \ar[r] & \mathcal{E}_{i'k'} \ar[u]
}
$$
commute. This follows from
Cohomology of Spaces, Lemma
\ref{spaces-cohomology-lemma-finite-presentation-quasi-compact-colimit}.
This induces a map
$$
\mathcal{A}_{ik} =
\text{Sym}^*_{\mathcal{O}_X}(\mathcal{F}_i)/(\mathcal{E}_{ik})
\longrightarrow
\text{Sym}^*_{\mathcal{O}_X}(\mathcal{F}_{i'})/(\mathcal{E}_{i'k'}) =
\mathcal{A}_{i'k'}
$$
where $(\mathcal{E}_{ik})$ denotes the ideal generated by $\mathcal{E}_{ik}$.
The quasi-coherent $\mathcal{O}_X$-algebras $\mathcal{A}_{ki}$
are of finite presentation and finite for $k$ large enough
(see proof of
Lemma \ref{lemma-finite-algebra-directed-colimit-finite-finitely-presented}).
Finally, we have
$$
\colim \mathcal{A}_{ik} = \colim \mathcal{A}_i = \mathcal{A}
$$
Namely, the first equality was shown in the proof of
Lemma \ref{lemma-finite-algebra-directed-colimit-finite-finitely-presented}
and the second equality because $\mathcal{A}$ is the colimit of
the modules $\mathcal{F}_i$.
\end{proof}

\begin{lemma}
\label{lemma-extend}
Let $S$ be a scheme.
Let $X$ be a quasi-compact and quasi-separated algebraic space over $S$.
Let $U \subset X$ be a quasi-compact open.
Let $\mathcal{F}$ be a quasi-coherent $\mathcal{O}_X$-module.
Let $\mathcal{G} \subset \mathcal{F}|_U$ be a quasi-coherent
$\mathcal{O}_U$-submodule which is of finite type. Then
there exists a quasi-coherent submodule $\mathcal{G}' \subset \mathcal{F}$
which is of finite type such that $\mathcal{G}'|_U = \mathcal{G}$.
\end{lemma}

\begin{proof}
Denote $j : U \to X$ the inclusion morphism. As $X$ is quasi-separated
and $U$ quasi-compact, the morphism $j$ is quasi-compact. Hence
$j_*\mathcal{G} \subset j_*\mathcal{F}|_U$ are quasi-coherent modules
on $X$ (Morphisms of Spaces, Lemma
\ref{spaces-morphisms-lemma-pushforward}).
Let $\mathcal{H} =
\Ker(j_*\mathcal{G} \oplus \mathcal{F} \to j_*\mathcal{F}|_U)$.
Then $\mathcal{H}|_U = \mathcal{G}$. By
Lemma \ref{lemma-directed-colimit-finite-type}
we can find a finite type quasi-coherent submodule
$\mathcal{H}' \subset \mathcal{H}$ such that
$\mathcal{H}'|_U = \mathcal{H}|_U = \mathcal{G}$.
Set $\mathcal{G}' = \Im(\mathcal{H}' \to \mathcal{F})$
to conclude.
\end{proof}









\section{Relative approximation}
\label{section-relative-approximation}

\noindent
We discuss variants of Proposition \ref{proposition-approximate}
over a base.

\begin{lemma}
\label{lemma-approximate-morphism}
Let $f : X \to Y$ be a morphism of quasi-compact and quasi-separated
algebraic spaces over $\mathbf{Z}$.
Then there exists a direct set $I$ and an inverse system $(f_i : X_i \to Y_i)$
of morphisms algebraic spaces over $I$, such that the transition morphisms
$X_i \to X_{i'}$ and $Y_i \to Y_{i'}$ are affine, such that $X_i$
and $Y_i$ are quasi-separated and of finite type over $\mathbf{Z}$, and
such that $(X \to Y) = \lim (X_i \to Y_i)$.
\end{lemma}

\begin{proof}
Write $X = \lim_{a \in A} X_a$ and $Y = \lim_{b \in B} Y_b$ as in
Proposition \ref{proposition-approximate}, i.e., with $X_a$ and $Y_b$
quasi-separated and of finite type over $\mathbf{Z}$ and
with affine transition morphisms.

\medskip\noindent
Fix $b \in B$.
By Lemma \ref{lemma-better-characterize-relative-limit-preserving}
applied to $Y_b$ and $X = \lim X_a$ over $\mathbf{Z}$
we find there exists an $a \in A$ and a morphism
$f_{a, b} : X_a \to Y_b$ making the diagram
$$
\xymatrix{
X \ar[d] \ar[r] & Y \ar[d] \\
X_a \ar[r] & Y_b
}
$$
commute. Let $I$ be the set of triples $(a, b, f_{a, b})$ we
obtain in this manner.

\medskip\noindent
Let $(a, b, f_{a, b})$ and $(a', b', f_{a', b'})$ be in $I$.
Let $b'' \leq \min(b, b')$. By
Lemma \ref{lemma-better-characterize-relative-limit-preserving}
again, there exists an $a'' \geq \max(a, a')$ such that
the compositions $X_{a''} \to X_a \to Y_b \to Y_{b''}$ and
$X_{a''} \to X_{a'} \to Y_{b'} \to Y_{b''}$ are equal.
We endow $I$ with the preorder
$$
(a, b, f_{a, b}) \geq (a', b', f_{a', b'})
\Leftrightarrow
a \geq a',\ b \geq b',\text{ and }
g_{b, b'} \circ f_{a, b} = f_{a', b'} \circ h_{a, a'}
$$
where $h_{a, a'} : X_a \to X_{a'}$ and $g_{b, b'} : Y_b \to Y_{b'}$
are the transition morphisms. The remarks above show that $I$
is directed and that the maps $I \to A$, $(a, b, f_{a, b}) \mapsto a$
and $I \to B$, $(a, b, f_{a, b})$ are cofinal. If for $i = (a, b, f_{a, b})$
we set $X_i = X_a$, $Y_i = Y_b$, and $f_i = f_{a, b}$, then we get
an inverse system of morphisms over $I$ and we have
$$
\lim_{i \in I} X_i = \lim_{a \in A} X_a = X
\quad\text{and}\quad
\lim_{i \in I} S_i = \lim_{b \in B} Y_b = Y
$$
by Categories, Lemma \ref{categories-lemma-initial} (recall that
limits over $I$ are really limits over the opposite category
associated to $I$ and hence cofinal turns into initial).
This finishes the proof.
\end{proof}

\begin{lemma}
\label{lemma-relative-approximation}
Let $S$ be a scheme. Let $f : X \to Y$ be a morphism of algebraic spaces
over $S$. Assume that
\begin{enumerate}
\item $X$ is quasi-compact and quasi-separated, and
\item $Y$ is quasi-separated.
\end{enumerate}
Then $X = \lim X_i$ is a limit of a directed inverse system of algebraic spaces
$X_i$ of finite presentation over $Y$ with affine transition morphisms
over $Y$.
\end{lemma}

\begin{proof}
Since $|f|(|X|)$ is quasi-compact we may replace $Y$ by a quasi-compact
open subspace whose set of points contains $|f|(|X|)$. Hence we may assume
$Y$ is quasi-compact as well.
By Lemma \ref{lemma-approximate-morphism} we can write
$(X \to Y) = \lim (X_i \to Y_i)$ for some directed inverse system
of morphisms of finite type schemes over $\mathbf{Z}$ with
affine transition morphisms. Since limits commute with limits
(Categories, Lemma \ref{categories-lemma-colimits-commute})
we have $X = \lim X_i \times_{Y_i} Y$. For $i \geq i'$
the transition morphism
$X_i \times_{Y_i} Y \to X_{i'} \times_{Y_{i'}} Y$ is affine as
the composition
$$
X_i \times_{Y_i} Y \to X_i \times_{Y_{i'}} Y \to X_{i'} \times_{Y_{i'}} Y
$$
where the first morphism is a closed immersion (by
Morphisms of Spaces, Lemma
\ref{spaces-morphisms-lemma-fibre-product-after-map})
and the second is a base change of an affine morphism
(Morphisms of Spaces, Lemma \ref{spaces-morphisms-lemma-base-change-affine})
and the composition of affine morphisms is affine
(Morphisms of Spaces, Lemma \ref{spaces-morphisms-lemma-composition-affine}).
The morphisms $f_i$ are of finite presentation
(Morphisms of Spaces, Lemmas
\ref{spaces-morphisms-lemma-noetherian-finite-type-finite-presentation} and
\ref{spaces-morphisms-lemma-finite-presentation-permanence})
and hence the base changes $X_i \times_{f_i, Y_i} Y \to Y$
are of finite presentation
(Morphisms of Spaces, Lemma
\ref{spaces-morphisms-lemma-base-change-finite-presentation}).
\end{proof}






\section{Finite type closed in finite presentation}
\label{section-finite-type-closed-in-finite-presentation}

\noindent
This section is the analogue of
Limits, Section \ref{limits-section-finite-type-closed-in-finite-presentation}.

\begin{lemma}
\label{lemma-affine-morphism-is-limit}
Let $S$ be a scheme. Let $f : X \to Y$ be an affine morphism of algebraic
spaces over $S$. If $Y$ quasi-compact and
quasi-separated, then $X$ is a directed limit $X = \lim X_i$
with each $X_i$ affine and of finite presentation over $Y$.
\end{lemma}

\begin{proof}
Consider the quasi-coherent $\mathcal{O}_Y$-module
$\mathcal{A} = f_*\mathcal{O}_X$. By
Lemma \ref{lemma-algebra-directed-colimit-finite-presentation}
we can write $\mathcal{A} = \colim \mathcal{A}_i$ as a directed
colimit of finitely presented
$\mathcal{O}_Y$-algebras $\mathcal{A}_i$.
Set $X_i = \underline{\Spec}_Y(\mathcal{A}_i)$, see
Morphisms of Spaces, Definition
\ref{spaces-morphisms-definition-relative-spec}.
By construction $X_i \to Y$ is affine and of finite presentation
and $X = \lim X_i$.
\end{proof}

\begin{lemma}
\label{lemma-integral-limit-finite-and-finite-presentation}
Let $S$ be a scheme. Let $f : X \to Y$ be an integral morphism of algebraic
spaces over $S$. Assume $Y$ quasi-compact and quasi-separated.
Then $X$ can be written as a directed limit $X = \lim X_i$
where $X_i$ are finite and of finite presentation over $Y$.
\end{lemma}

\begin{proof}
Consider the quasi-coherent $\mathcal{O}_Y$-module
$\mathcal{A} = f_*\mathcal{O}_X$. By
Lemma \ref{lemma-integral-algebra-directed-colimit-finite}
we can write $\mathcal{A} = \colim \mathcal{A}_i$ as a directed
colimit of finite and finitely presented $\mathcal{O}_Y$-algebras
$\mathcal{A}_i$.
Set $X_i = \underline{\Spec}_Y(\mathcal{A}_i)$, see
Morphisms of Spaces, Definition
\ref{spaces-morphisms-definition-relative-spec}.
By construction $X_i \to Y$ is finite and of finite presentation and
$X = \lim X_i$.
\end{proof}

\begin{lemma}
\label{lemma-finite-in-finite-and-finite-presentation}
Let $S$ be a scheme. Let $f : X \to Y$ be a finite morphism of algebraic
spaces over $S$. Assume $Y$ quasi-compact and quasi-separated.
Then $X$ can be written as a directed limit $X = \lim X_i$
where the transition maps are closed immersions and the objects
$X_i$ are finite and of finite presentation over $Y$.
\end{lemma}

\begin{proof}
Consider the finite quasi-coherent $\mathcal{O}_Y$-module
$\mathcal{A} = f_*\mathcal{O}_X$. By
Lemma \ref{lemma-finite-algebra-directed-colimit-finite-finitely-presented}
we can write $\mathcal{A} = \colim \mathcal{A}_i$ as a directed
colimit of finite and finitely presented $\mathcal{O}_Y$-algebras
$\mathcal{A}_i$ with surjective transition maps.
Set $X_i = \underline{\Spec}_Y(\mathcal{A}_i)$, see
Morphisms of Spaces, Definition
\ref{spaces-morphisms-definition-relative-spec}.
By construction $X_i \to Y$ is finite and of finite presentation,
the transition maps are closed immersions, and $X = \lim X_i$.
\end{proof}

\begin{lemma}
\label{lemma-closed-is-limit-closed-and-finite-presentation}
\begin{slogan}
Closed immersions of qcqs algebraic spaces can be approximated
by finitely presented closed immersions.
\end{slogan}
Let $S$ be a scheme. Let $f : X \to Y$ be a closed immersion of algebraic
spaces over $S$. Assume $Y$ quasi-compact and quasi-separated.
Then $X$ can be written as a directed limit $X = \lim X_i$
where the transition maps are closed immersions and the morphisms
$X_i \to Y$ are closed immersions of finite presentation.
\end{lemma}

\begin{proof}
Let $\mathcal{I} \subset \mathcal{O}_Y$ be the quasi-coherent sheaf
of ideals defining $X$ as a closed subspace of $Y$. By
Lemma \ref{lemma-directed-colimit-finite-type}
we can write $\mathcal{I} = \colim \mathcal{I}_i$ as the
filtered colimit of its finite type quasi-coherent submodules.
Let $X_i$ be the closed subspace of $X$ cut out by $\mathcal{I}_i$.
Then $X_i \to Y$ is a closed immersion of finite presentation,
and $X = \lim X_i$. Some details omitted.
\end{proof}

\begin{lemma}
\label{lemma-quasi-affine-closed-in-finite-presentation}
Let $S$ be a scheme. Let $f : X \to Y$ be a morphism of algebraic spaces
over $S$. Assume
\begin{enumerate}
\item $f$ is locally of finite type and quasi-affine, and
\item $Y$ is quasi-compact and quasi-separated.
\end{enumerate}
Then there exists a morphism of finite presentation
$f' : X' \to Y$ and a closed immersion $X \to X'$ over $Y$.
\end{lemma}

\begin{proof}
By Morphisms of Spaces, Lemma
\ref{spaces-morphisms-lemma-characterize-quasi-affine}
we can find a factorization $X \to Z \to Y$ where
$X \to Z$ is a quasi-compact open immersion and
$Z \to Y$ is affine. Write $Z = \lim Z_i$ with $Z_i$ affine and
of finite presentation over $Y$ (Lemma \ref{lemma-affine-morphism-is-limit}).
For some $0 \in I$ we can find a quasi-compact open $U_0 \subset Z_0$
such that $X$ is isomorphic to the inverse image of $U_0$ in $Z$
(Lemma \ref{lemma-descend-opens}). Let $U_i$ be the inverse image of
$U_0$ in $Z_i$, so $U = \lim U_i$. By
Lemma \ref{lemma-finite-type-eventually-closed}
we see that $X \to U_i$ is a closed immersion for some $i$ large enough.
Setting $X' = U_i$ finishes the proof.
\end{proof}

\begin{lemma}
\label{lemma-finite-type-closed-in-finite-presentation}
Let $S$ be a scheme. Let $f : X \to Y$ be a morphism of algebraic spaces
over $S$. Assume:
\begin{enumerate}
\item $f$ is of locally of finite type.
\item $X$ is quasi-compact and quasi-separated, and
\item $Y$ is quasi-compact and quasi-separated.
\end{enumerate}
Then there exists a morphism of finite presentation
$f' : X' \to Y$ and a closed immersion $X \to X'$ of
algebraic spaces over $Y$.
\end{lemma}

\begin{proof}
By Proposition \ref{proposition-approximate}
we can write $X = \lim_i X_i$ with $X_i$ quasi-separated of finite type over
$\mathbf{Z}$ and with transition morphisms $f_{ii'} : X_i \to X_{i'}$ affine.
Consider the commutative diagram
$$
\xymatrix{
X \ar[r] \ar[rd] & X_{i, Y} \ar[r] \ar[d] & X_i \ar[d] \\
& Y \ar[r] & \Spec(\mathbf{Z})
}
$$
Note that $X_i$ is of finite presentation over $\Spec(\mathbf{Z})$, see
Morphisms of Spaces, Lemma
\ref{spaces-morphisms-lemma-noetherian-finite-type-finite-presentation}.
Hence the base change $X_{i, Y} \to Y$ is of finite presentation by
Morphisms of Spaces, Lemma
\ref{spaces-morphisms-lemma-base-change-finite-presentation}.
Observe that $\lim X_{i, Y} = X \times Y$ and that $X \to X \times Y$ is a
monomorphism. By Lemma \ref{lemma-finite-type-eventually-closed}
we see that $X \to X_{i, Y}$ is a monomorphism for $i$ large enough.
Fix such an $i$. Note that $X \to X_{i, Y}$ is locally of finite type
(Morphisms of Spaces, Lemma
\ref{spaces-morphisms-lemma-permanence-finite-type})
and a monomorphism, hence separated and locally quasi-finite
(Morphisms of Spaces, Lemma
\ref{spaces-morphisms-lemma-monomorphism-loc-finite-type-loc-quasi-finite}).
Hence $X \to X_{i, Y}$ is representable.
Hence $X \to X_{i, Y}$ is quasi-affine because we can use the
principle Spaces, Lemma
\ref{spaces-lemma-representable-transformations-property-implication}
and the result for morphisms of schemes More on Morphisms, Lemma
\ref{more-morphisms-lemma-quasi-finite-separated-quasi-affine}.
Thus Lemma \ref{lemma-quasi-affine-closed-in-finite-presentation}
gives a factorization $X \to X' \to X_{i, Y}$
with $X \to X'$ a closed immersion and $X' \to X_{i, Y}$ of finite
presentation. Finally, $X' \to Y$ is of finite presentation as a
composition of morphisms of finite presentation
(Morphisms of Spaces, Lemma
\ref{spaces-morphisms-lemma-composition-finite-presentation}).
\end{proof}

\begin{proposition}
\label{proposition-separated-closed-in-finite-presentation}
Let $S$ be a scheme. $f : X \to Y$ be a morphism of algebraic spaces
over $S$. Assume
\begin{enumerate}
\item $f$ is of finite type and separated, and
\item $Y$ is quasi-compact and quasi-separated.
\end{enumerate}
Then there exists a separated morphism of finite presentation
$f' : X' \to Y$ and a closed immersion $X \to X'$ over $Y$.
\end{proposition}

\begin{proof}
By Lemma \ref{lemma-finite-type-closed-in-finite-presentation}
there is a closed immersion $X \to Z$ with $Z/Y$ of
finite presentation. Let $\mathcal{I} \subset \mathcal{O}_Z$
be the quasi-coherent sheaf of ideals defining $X$ as a closed
subscheme of $Y$. By
Lemma \ref{lemma-directed-colimit-finite-type}
we can write $\mathcal{I}$ as a directed colimit
$\mathcal{I} = \colim_{a \in A} \mathcal{I}_a$ of its
quasi-coherent sheaves of ideals of finite type.
Let $X_a \subset Z$ be the closed subspace defined by $\mathcal{I}_a$.
These form an inverse system indexed by $A$.
The transition morphisms $X_a \to X_{a'}$ are affine because
they are closed immersions. Each $X_a$ is quasi-compact and quasi-separated
since it is a closed subspace of $Z$ and $Z$ is quasi-compact and
quasi-separated by our assumptions.
We have $X = \lim_a X_a$ as follows directly from the
fact that $\mathcal{I} = \colim_{a \in A} \mathcal{I}_a$.
Each of the morphisms $X_a \to Z$ is of finite presentation, see
Morphisms, Lemma \ref{morphisms-lemma-closed-immersion-finite-presentation}.
Hence the morphisms $X_a \to Y$ are of finite presentation.
Thus it suffices to show that $X_a \to Y$ is separated for some
$a \in A$. This follows from Lemma \ref{lemma-eventually-separated}
as we have assumed that $X \to Y$ is separated.
\end{proof}






\section{Approximating proper morphisms}
\label{section-approximate-proper}

\begin{lemma}
\label{lemma-proper-limit-of-proper-finite-presentation}
Let $S$ be a scheme. Let $f : X \to Y$ be a proper morphism of algebraic
spaces over $S$ with $Y$ quasi-compact and quasi-separated. Then
$X = \lim X_i$ is a directed limit of algebraic spaces $X_i$
proper and of finite presentation over $Y$ and with transition
morphisms and morphisms $X \to X_i$ closed immersions.
\end{lemma}

\begin{proof}
By Proposition \ref{proposition-separated-closed-in-finite-presentation}
we can find a closed immersion $X \to X'$ with $X'$ separated and of
finite presentation over $Y$. By
Lemma \ref{lemma-closed-is-limit-closed-and-finite-presentation}
we can write $X = \lim X_i$ with $X_i \to X'$ a closed immersion of
finite presentation. We claim that for all $i$ large enough
the morphism $X_i \to Y$ is proper which finishes the proof.

\medskip\noindent
To prove this we may assume that $Y$ is an affine scheme, see
Morphisms of Spaces, Lemma \ref{spaces-morphisms-lemma-proper-local}.
Next, we use the weak version of Chow's lemma, see
Cohomology of Spaces, Lemma \ref{spaces-cohomology-lemma-weak-chow},
to find a diagram
$$
\xymatrix{
X' \ar[rd] & X'' \ar[d] \ar[l]^\pi \ar[r] & \mathbf{P}^n_Y \ar[dl] \\
& Y &
}
$$
where $X'' \to \mathbf{P}^n_Y$ is an immersion, and
$\pi : X'' \to X'$ is proper and surjective. Denote
$X'_i \subset X''$, resp.\ $\pi^{-1}(X)$ the scheme theoretic inverse image of
$X_i \subset X'$, resp.\ $X \subset X'$.
Then $\lim X'_i = \pi^{-1}(X)$. Since $\pi^{-1}(X) \to Y$ is proper
(Morphisms of Spaces, Lemmas \ref{spaces-morphisms-lemma-composition-proper}),
we see that $\pi^{-1}(X) \to \mathbf{P}^n_Y$ is a closed immersion
(Morphisms of Spaces, Lemmas
\ref{spaces-morphisms-lemma-universally-closed-permanence} and
\ref{spaces-morphisms-lemma-immersion-when-closed}).
Hence for $i$ large enough
we find that $X'_i \to \mathbf{P}^n_Y$ is a closed immersion by
Lemma \ref{lemma-eventually-closed-immersion}.
Thus $X'_i$ is proper over $Y$.
For such $i$ the morphism $X_i \to Y$ is proper by
Morphisms of Spaces, Lemma \ref{spaces-morphisms-lemma-image-proper-is-proper}.
\end{proof}

\begin{lemma}
\label{lemma-proper-limit-of-proper-finite-presentation-noetherian}
Let $f : X \to Y$ be a proper morphism of algebraic spaces over $\mathbf{Z}$
with $Y$ quasi-compact and quasi-separated. Then there exists a directed
set $I$, an inverse system $(f_i : X_i \to Y_i)$ of morphisms of algebraic
spaces over $I$, such that the transition morphisms $X_i \to X_{i'}$
and $Y_i \to Y_{i'}$ are affine, such that $f_i$ is proper and of
finite presentation, such that $Y_i$ is of finite presentation over
$\mathbf{Z}$, and such that $(X \to Y) = \lim (X_i \to Y_i)$.
\end{lemma}

\begin{proof}
By Lemma \ref{lemma-proper-limit-of-proper-finite-presentation}
we can write $X = \lim_{k \in K} X_k$ with $X_k \to Y$ proper and
of finite presentation. Next, by absolute Noetherian approximation
(Proposition \ref{proposition-approximate}) we can
write $Y = \lim_{j \in J} Y_j$ with $Y_j$ of finite presentation
over $\mathbf{Z}$.
For each $k$ there exists a $j$ and a morphism $X_{k, j} \to Y_j$
of finite presentation with $X_k \cong Y \times_{Y_j} X_{k, j}$
as algebraic spaces over $Y$, see
Lemma \ref{lemma-descend-finite-presentation}.
After increasing $j$ we may assume $X_{k, j} \to Y_j$ is proper, see
Lemma \ref{lemma-eventually-proper}. The set $I$ will be consist
of these pairs $(k, j)$ and the corresponding morphism is $X_{k, j} \to Y_j$.
For every $k' \geq k$ we can find a $j' \geq j$ and a morphism
$X_{j', k'} \to X_{j, k}$ over $Y_{j'} \to Y_j$ whose base change to $Y$
gives the morphism $X_{k'} \to X_k$ (follows again from
Lemma \ref{lemma-descend-finite-presentation}).
These morphisms form the transition morphisms of the system. Some details
omitted.
\end{proof}

\noindent
Recall the scheme theoretic support of
a finite type quasi-coherent module, see
Morphisms of Spaces, Definition
\ref{spaces-morphisms-definition-scheme-theoretic-support}.

\begin{lemma}
\label{lemma-eventually-proper-support}
Assumptions and notation as in Situation \ref{situation-descent-property}.
Let $\mathcal{F}_0$ be a quasi-coherent $\mathcal{O}_{X_0}$-module.
Denote $\mathcal{F}$ and $\mathcal{F}_i$ the pullbacks of
$\mathcal{F}_0$ to $X$ and $X_i$. Assume
\begin{enumerate}
\item $f_0$ is locally of finite type,
\item $\mathcal{F}_0$ is of finite type,
\item the scheme theoretic support of $\mathcal{F}$ is proper over $Y$.
\end{enumerate}
Then the scheme theoretic support of $\mathcal{F}_i$ is proper over $Y_i$
for some $i$.
\end{lemma}

\begin{proof}
We may replace $X_0$ by the scheme theoretic support of $\mathcal{F}_0$.
By Morphisms of Spaces, Lemma \ref{spaces-morphisms-lemma-support-finite-type}
this guarantees that $X_i$ is the support of $\mathcal{F}_i$ and $X$ is the
support of $\mathcal{F}$. Then, if $Z \subset X$ denotes the scheme
theoretic support of $\mathcal{F}$, we see that $Z \to X$ is a universal
homeomorphism. We conclude that $X \to Y$ is proper as this is true for
$Z \to Y$ by assumption, see
Morphisms, Lemma \ref{morphisms-lemma-image-proper-is-proper}.
By Lemma \ref{lemma-eventually-proper} we see that $X_i \to Y$ is proper
for some $i$. Then it follows that the scheme theoretic support $Z_i$ of
$\mathcal{F}_i$ is proper over $Y$ by
Morphisms of Spaces, Lemmas
\ref{spaces-morphisms-lemma-closed-immersion-proper} and
\ref{spaces-morphisms-lemma-composition-proper}.
\end{proof}








\section{Embedding into affine space}
\label{section-embedding}

\noindent
Some technical lemmas to be used in the proof of Chow's lemma later.

\begin{lemma}
\label{lemma-embedding-into-affine-over-ls-qs}
Let $S$ be a scheme. Let $f : U \to X$ be a morphism of algebraic
spaces over $S$. Assume $U$ is an affine scheme, $f$ is locally of
finite type, and $X$ quasi-separated and locally separated.
Then there exists an immersion $U \to \mathbf{A}^n_X$ over $X$.
\end{lemma}

\begin{proof}
Say $U = \Spec(A)$. Write $A = \colim A_i$ as a filtered colimit
of finite type $\mathbf{Z}$-subalgebras. For each $i$ the morphism
$U \to U_i = \Spec(A_i)$ induces a morphism
$$
U \longrightarrow X \times U_i
$$
over $X$. In the limit the morphism $U \to X \times U$ is an immersion
as $X$ is locally separated, see
Morphisms of Spaces, Lemma
\ref{spaces-morphisms-lemma-semi-diagonal}.
By Lemma \ref{lemma-finite-type-eventually-closed}
we see that $U \to X \times U_i$ is an immersion for some $i$.
Since $U_i$ is isomorphic to a closed subscheme of
$\mathbf{A}^n_{\mathbf{Z}}$ the lemma follows.
\end{proof}

\begin{remark}
\label{remark-cannot-embed-in-general}
We have seen in Examples, Section \ref{examples-section-embedding-affines}
that Lemma \ref{lemma-embedding-into-affine-over-ls-qs}
does not hold if we drop the assumption that $X$ be locally separated.
This raises the question: Does
Lemma \ref{lemma-embedding-into-affine-over-ls-qs}
hold if we drop the assumption that $X$ be quasi-separated?
If you know the answer, please email
\href{mailto:stacks.project@gmail.com}{stacks.project@gmail.com}.
\end{remark}

\begin{lemma}
\label{lemma-embedding-into-affine-over-qs}
Let $S$ be a scheme. Let $f : Y \to X$ be a morphism of algebraic
spaces over $S$. Assume $X$ Noetherian and $f$ of finite presentation.
Then there exists a dense open $V \subset Y$ and an immersion
$V \to \mathbf{A}^n_X$.
\end{lemma}

\begin{proof}
The assumptions imply that $Y$ is Noetherian
(Morphisms of Spaces, Lemma
\ref{spaces-morphisms-lemma-finite-presentation-noetherian}).
Then $Y$ is quasi-separated, hence has a dense open subscheme
(Properties of Spaces, Proposition
\ref{spaces-properties-proposition-locally-quasi-separated-open-dense-scheme}).
Thus we may assume that $Y$ is a Noetherian scheme.
By removing intersections of irreducible components of $Y$
(use Topology, Lemma \ref{topology-lemma-Noetherian} and
Properties, Lemma \ref{properties-lemma-Noetherian-topology})
we may assume that $Y$ is a disjoint union of irreducible
Noetherian schemes. Since there is an immersion
$$
\mathbf{A}^n_X \amalg \mathbf{A}^m_X
\longrightarrow
\mathbf{A}^{\max(n, m) + 1}_X
$$
(details omitted) we see that it suffices to prove the result in case
$Y$ is irreducible.

\medskip\noindent
Assume $Y$ is an irreducible scheme. Let $T \subset |X|$ be the closure of the
image of $f : Y \to X$. Note that since $|Y|$ and $|X|$ are sober topological
spaces (Properties of Spaces, Lemma
\ref{spaces-properties-lemma-quasi-separated-sober})
$T$ is irreducible with a unique generic point $\xi$ which is the
image of the generic point $\eta$ of $Y$.
Let $\mathcal{I} \subset X$ be a quasi-coherent sheaf of ideals
cutting out the reduced induced space structure on $T$
(Properties of Spaces, Definition
\ref{spaces-properties-definition-reduced-induced-space}).
Since $\mathcal{O}_{Y, \eta}$ is an Artinian local ring we see
that for some $n > 0$ we have $f^{-1}\mathcal{I}^n \mathcal{O}_{Y, \eta} = 0$.
As $f^{-1}\mathcal{I}\mathcal{O}_Y$ is a finite type quasi-coherent ideal
we conclude that $f^{-1}\mathcal{I}^n\mathcal{O}_V = 0$ for
some nonempty open $V \subset Y$. Let $Z \subset X$ be the closed subspace
cut out by $\mathcal{I}^n$. By construction $V \to Y \to X$ factors through
$Z$. Because $\mathbf{A}^n_Z \to \mathbf{A}^n_X$ is an immersion,
we may replace $X$ by $Z$ and $Y$ by $V$.
Hence we reach the situation where $Y$ and $X$ are irreducible and
$Y \to X$ maps the generic point of $Y$ onto the generic point of $X$.

\medskip\noindent
Assume $Y$ and $X$ are irreducible, $Y$ is a scheme,
and $Y \to X$ maps the generic point of
$Y$ onto the generic point of $X$. By Properties of Spaces, Proposition
\ref{spaces-properties-proposition-locally-quasi-separated-open-dense-scheme}
$X$ has a dense open subscheme $U \subset X$. Choose a nonempty affine
open $V \subset Y$ whose image in $X$ is contained in $U$. By
Morphisms, Lemma \ref{morphisms-lemma-quasi-affine-finite-type-over-S}
we may factor $V \to U$ as $V \to \mathbf{A}^n_U \to U$. Composing
with $\mathbf{A}^n_U \to \mathbf{A}^n_X$ we obtain the desired immersion.
\end{proof}








\section{Sections with support in a closed subset}
\label{section-sections-with-support-in-closed}

\noindent
This section is the analogue of
Properties, Section \ref{properties-section-sections-with-support-in-closed}.

\begin{lemma}
\label{lemma-quasi-coherent-finite-type-ideals}
Let $S$ be a scheme.
Let $X$ be a quasi-compact and quasi-separated algebraic space.
Let $U \subset X$ be an open subspace. The following are equivalent:
\begin{enumerate}
\item $U \to X$ is quasi-compact,
\item $U$ is quasi-compact, and
\item there exists a finite type quasi-coherent sheaf of ideals
$\mathcal{I} \subset \mathcal{O}_X$ such that
$|X| \setminus |U| = |V(\mathcal{I})|$.
\end{enumerate}
\end{lemma}

\begin{proof}
Let $W$ be an affine scheme and let $\varphi : W \to X$ be a surjective
\'etale morphism, see Properties of Spaces, Lemma
\ref{spaces-properties-lemma-quasi-compact-affine-cover}.
If (1) holds, then $\varphi^{-1}(U) \to W$ is quasi-compact, hence
$\varphi^{-1}(U)$ is quasi-compact, hence $U$ is quasi-compact
(as $|\varphi^{-1}(U)| \to |U|$ is surjective). If (2) holds, then
$\varphi^{-1}(U)$ is quasi-compact because $\varphi$ is quasi-compact
since $X$ is quasi-separated (Morphisms of Spaces,
Lemma \ref{spaces-morphisms-lemma-quasi-compact-quasi-separated-permanence}).
Hence $\varphi^{-1}(U) \to W$ is a quasi-compact morphism of schemes by
Properties, Lemma \ref{properties-lemma-quasi-coherent-finite-type-ideals}.
It follows that $U \to X$ is quasi-compact by
Morphisms of Spaces, Lemma \ref{spaces-morphisms-lemma-quasi-compact-local}.
Thus (1) and (2) are equivalent.

\medskip\noindent
Assume (1) and (2). By
Properties of Spaces, Lemma
\ref{spaces-properties-lemma-reduced-closed-subspace} there exists
a unique quasi-coherent sheaf of ideals $\mathcal{J}$ cutting
out the reduced induced closed subspace structure on $|X| \setminus |U|$.
Note that $\mathcal{J}|_U = \mathcal{O}_U$ which is an
$\mathcal{O}_U$-modules of finite type.
As $U$ is quasi-compact it follows from
Lemma \ref{lemma-directed-colimit-finite-type}
that there exists a quasi-coherent subsheaf
$\mathcal{I} \subset \mathcal{J}$ which is of finite type
and has the property that $\mathcal{I}|_U = \mathcal{J}|_U$.
Then $|X| \setminus |U| = |V(\mathcal{I})|$ and we obtain (3). Conversely,
if $\mathcal{I}$ is as in (3), then $\varphi^{-1}(U) \subset W$
is a quasi-compact open by the lemma for schemes
(Properties, Lemma \ref{properties-lemma-quasi-coherent-finite-type-ideals})
applied to $\varphi^{-1}\mathcal{I}$ on $W$.
Thus (2) holds.
\end{proof}

\begin{lemma}
\label{lemma-sections-annihilated-by-ideal}
Let $S$ be a scheme. Let $X$ be an algebraic space over $S$.
Let $\mathcal{I} \subset \mathcal{O}_X$ be a quasi-coherent sheaf of ideals.
Let $\mathcal{F}$ be a quasi-coherent $\mathcal{O}_X$-module.
Consider the sheaf of $\mathcal{O}_X$-modules $\mathcal{F}'$
which associates to every object $U$ of $X_\etale$ the module
$$
\mathcal{F}'(U)
=
\{s \in \mathcal{F}(U) \mid
\mathcal{I}s = 0\}
$$
Assume $\mathcal{I}$ is of finite type. Then
\begin{enumerate}
\item $\mathcal{F}'$ is a quasi-coherent sheaf of $\mathcal{O}_X$-modules,
\item for affine $U$ in $X_\etale$ we have
$\mathcal{F}'(U) = \{s \in \mathcal{F}(U) \mid \mathcal{I}(U)s = 0\}$, and
\item $\mathcal{F}'_x = \{s \in \mathcal{F}_x \mid \mathcal{I}_x s = 0\}$.
\end{enumerate}
\end{lemma}

\begin{proof}
It is clear that the rule defining $\mathcal{F}'$ gives a subsheaf
of $\mathcal{F}$. Hence we may work \'etale locally on $X$ to verify
the other statements. Thus the lemma reduces to the case of schemes
which is
Properties, Lemma \ref{properties-lemma-sections-annihilated-by-ideal}.
\end{proof}

\begin{definition}
\label{definition-subsheaf-sections-annihilated-by-ideal}
Let $S$ be a scheme. Let $X$ be an algebraic space over $S$.
Let $\mathcal{I} \subset \mathcal{O}_X$ be a quasi-coherent
sheaf of ideals of finite type.
Let $\mathcal{F}$ be a quasi-coherent $\mathcal{O}_X$-module.
The subsheaf $\mathcal{F}' \subset \mathcal{F}$ defined in
Lemma \ref{lemma-sections-annihilated-by-ideal} above is called
the {\it subsheaf of sections annihilated by $\mathcal{I}$}.
\end{definition}

\begin{lemma}
\label{lemma-push-sections-annihilated-by-ideal}
Let $S$ be a scheme.
Let $f : X \to Y$ be a quasi-compact and quasi-separated morphism
of algebraic spaces over $S$.
Let $\mathcal{I} \subset \mathcal{O}_Y$ be a quasi-coherent
sheaf of ideals of finite type. Let $\mathcal{F}$ be a quasi-coherent
$\mathcal{O}_X$-module. Let $\mathcal{F}' \subset \mathcal{F}$
be the subsheaf of sections annihilated by $f^{-1}\mathcal{I}\mathcal{O}_X$.
Then $f_*\mathcal{F}' \subset f_*\mathcal{F}$ is the subsheaf
of sections annihilated by $\mathcal{I}$.
\end{lemma}

\begin{proof}
Omitted. Hint: The assumption that $f$ is quasi-compact and
quasi-separated implies that $f_*\mathcal{F}$ is quasi-coherent
(Morphisms of Spaces, Lemma \ref{spaces-morphisms-lemma-pushforward})
so that Lemma \ref{lemma-sections-annihilated-by-ideal} applies
to $\mathcal{I}$ and $f_*\mathcal{F}$.
\end{proof}

\noindent
Next we come to the sheaf of sections supported in a closed subset.
Again this isn't always a quasi-coherent sheaf, but if the complement
of the closed is ``retrocompact'' in the given algebraic space, then
it is.

\begin{lemma}
\label{lemma-sections-supported-on-closed-subset}
Let $S$ be a scheme. Let $X$ be an algebraic space over $S$.
Let $T \subset |X|$ be a closed subset and let $U \subset X$ be
the open subspace such that $T \amalg |U| = |X|$.
Let $\mathcal{F}$ be a quasi-coherent $\mathcal{O}_X$-module.
Consider the sheaf of $\mathcal{O}_X$-modules $\mathcal{F}'$
which associates to every object $\varphi : W \to X$ of
$X_\etale$ the module
$$
\mathcal{F}'(W)
=
\{s \in \mathcal{F}(W) \mid
\text{the support of }s\text{ is contained in }|\varphi|^{-1}(T)\}
$$
If $U \to X$ is quasi-compact, then
\begin{enumerate}
\item for $W$ affine there exist a finitely generated
ideal $I \subset \mathcal{O}_X(W)$ such that $|\varphi|^{-1}(T) = V(I)$,
\item for $W$ and $I$ as in (1) we have
$\mathcal{F}'(W) = \{x \in \mathcal{F}(W) \mid
I^nx = 0 \text{ for some } n\}$,
\item $\mathcal{F}'$ is a quasi-coherent sheaf of $\mathcal{O}_X$-modules.
\end{enumerate}
\end{lemma}

\begin{proof}
It is clear that the rule defining $\mathcal{F}'$ gives a subsheaf
of $\mathcal{F}$. Hence we may work \'etale locally on $X$ to verify
the other statements. Thus the lemma reduces to the case of schemes
which is
Properties, Lemma \ref{properties-lemma-sections-supported-on-closed-subset}.
\end{proof}

\begin{definition}
\label{definition-subsheaf-sections-supported-on-closed}
Let $S$ be a scheme. Let $X$ be an algebraic space over $S$.
Let $T \subset |X|$ be a closed subset whose complement
corresponds to an open subspace $U \subset X$
with quasi-compact inclusion morphism $U \to X$.
Let $\mathcal{F}$ be a quasi-coherent $\mathcal{O}_X$-module.
The quasi-coherent subsheaf $\mathcal{F}' \subset \mathcal{F}$ defined in
Lemma \ref{lemma-sections-supported-on-closed-subset} above is called
the {\it subsheaf of sections supported on $T$}.
\end{definition}

\begin{lemma}
\label{lemma-push-sections-supported-on-closed-subset}
Let $S$ be a scheme.
Let $f : X \to Y$ be a quasi-compact and quasi-separated morphism
of algebraic spaces over $S$. Let $T \subset |Y|$ be a closed subset.
Assume $|Y| \setminus T$ corresponds to an open subspace $V \subset Y$
such that $V \to Y$ is quasi-compact. Let $\mathcal{F}$ be a quasi-coherent
$\mathcal{O}_X$-module. Let $\mathcal{F}' \subset \mathcal{F}$
be the subsheaf of sections supported on $|f|^{-1}T$.
Then $f_*\mathcal{F}' \subset f_*\mathcal{F}$ is the subsheaf
of sections supported on $T$.
\end{lemma}

\begin{proof}
Omitted. Hints: $|X| \setminus |f|^{-1}T$ is the support of the open subspace
$U = f^{-1}V \subset X$. Since $V \to Y$ is quasi-compact, so is
$U \to X$ (by base change). The assumption that $f$ is quasi-compact and
quasi-separated implies that $f_*\mathcal{F}$ is quasi-coherent.
Hence Lemma \ref{lemma-sections-supported-on-closed-subset}
applies to $T$ and $f_*\mathcal{F}$ as well as to
$|f|^{-1}T$ and $\mathcal{F}$. The equality of the given quasi-coherent
modules is immediate from the definitions.
\end{proof}















\section{Characterizing affine spaces}
\label{section-affine}

\noindent
This section is the analogue of Limits, Section \ref{limits-section-affine}.

\begin{lemma}
\label{lemma-affine}
Let $S$ be a scheme. Let $f : X \to Y$ be a morphism of algebraic spaces
over $S$. Assume that $f$ is surjective and finite, and assume that $X$
is affine. Then $Y$ is affine.
\end{lemma}

\begin{proof}
We may and do view $f : X \to Y$ as a morphism of algebraic spaces over
$\Spec(\mathbf{Z})$ (see
Spaces, Definition \ref{spaces-definition-base-change}).
Note that a finite morphism is affine and universally closed, see
Morphisms of Spaces, Lemma
\ref{spaces-morphisms-lemma-integral-universally-closed}.
By Morphisms of Spaces, Lemma
\ref{spaces-morphisms-lemma-image-universally-closed-separated}
we see that $Y$ is a separated algebraic space.
As $f$ is surjective and $X$ is quasi-compact we see that $Y$ is
quasi-compact.

\medskip\noindent
By Lemma \ref{lemma-finite-in-finite-and-finite-presentation}
we can write $X = \lim X_a$ with each $X_a \to Y$ finite and of
finite presentation. By
Lemma \ref{lemma-limit-is-affine}
we see that $X_a$ is affine for $a$ large enough.
Hence we may and do assume that $f : X \to Y$ is finite, surjective, and
of finite presentation.

\medskip\noindent
By Proposition \ref{proposition-approximate} we may write
$Y = \lim Y_i$ as a directed limit of algebraic
spaces of finite presentation over $\mathbf{Z}$.
By Lemma \ref{lemma-descend-finite-presentation} we can
find $0 \in I$ and a morphism $X_0 \to Y_0$ of finite presentation
such that $X_i = X_0 \times_{Y_0} Y_i$ for $i \geq 0$
and such that $X = \lim_i X_i$. By
Lemma \ref{lemma-descend-finite}
we see that $X_i \to Y_i$ is finite for $i$ large enough.
By Lemma \ref{lemma-descend-surjective}
we see that $X_i \to Y_i$ is surjective for $i$ large enough.
By Lemma \ref{lemma-limit-is-affine} we see that $X_i$ is
affine for $i$ large enough. Hence for $i$ large enough we can apply
Cohomology of Spaces, Lemma
\ref{spaces-cohomology-lemma-image-affine-finite-morphism-affine-Noetherian}
to conclude that $Y_i$ is affine. This implies that $Y$ is affine and
we conclude.
\end{proof}

\begin{proposition}
\label{proposition-affine}
Let $S$ be a scheme. Let $f : X \to Y$ be a morphism of algebraic spaces
over $S$. Assume that $X$ is affine and $f$ is surjective and universally
closed\footnote{An integral morphism is universally closed, see
Morphisms of Spaces, Lemma
\ref{spaces-morphisms-lemma-integral-universally-closed}.}. Then $Y$ is affine.
\end{proposition}

\begin{proof}
We may and do view $f : X \to Y$ as a morphism of algebraic spaces over
$\Spec(\mathbf{Z})$ (see
Spaces, Definition \ref{spaces-definition-base-change}).
By Morphisms of Spaces, Lemma
\ref{spaces-morphisms-lemma-image-universally-closed-separated}
we see that $Y$ is a separated algebraic space. Then by
Morphisms of Spaces, Lemma \ref{spaces-morphisms-lemma-affine-permanence}
we find that $f$ is affine. Whereupon by
Morphisms of Spaces, Lemma
\ref{spaces-morphisms-lemma-integral-universally-closed}
we see that $f$ is integral.

\medskip\noindent
By the preceding paragraph, we may assume $f : X \to Y$
is surjective and integral, $X$ is affine, and $Y$ is separated.
Since $f$ is surjective and $X$ is quasi-compact we also deduce that $Y$ is
quasi-compact.

\medskip\noindent
Consider the sheaf $\mathcal{A} = f_*\mathcal{O}_X$.
This is a quasi-coherent sheaf of $\mathcal{O}_Y$-algebras, see
Morphisms of Spaces, Lemma \ref{spaces-morphisms-lemma-pushforward}.
By Lemma \ref{lemma-colimit-finitely-presented}
we can write $\mathcal{A} = \colim_i \mathcal{F}_i$ as a filtered
colimit of finite type $\mathcal{O}_Y$-modules. Let
$\mathcal{A}_i \subset \mathcal{A}$ be the $\mathcal{O}_Y$-subalgebra
generated by $\mathcal{F}_i$. Since the map of algebras
$\mathcal{O}_Y \to \mathcal{A}$ is integral, we see that each $\mathcal{A}_i$
is a finite quasi-coherent $\mathcal{O}_Y$-algebra. Hence
$$
X_i = \underline{\Spec}_Y(\mathcal{A}_i) \longrightarrow Y
$$
is a finite morphism of algebraic spaces. Here 
$\underline{\Spec}$ is the construction of Morphisms of Spaces, Lemma
\ref{spaces-morphisms-lemma-affine-equivalence-algebras}.
It is clear
that $X = \lim_i X_i$. Hence by
Lemma \ref{lemma-limit-is-affine}
we see that for $i$ sufficiently large the scheme $X_i$ is affine.
Moreover, since $X \to Y$ factors through each $X_i$ we see that
$X_i \to Y$ is surjective. Hence we conclude that $Y$ is affine by
Lemma \ref{lemma-affine}.
\end{proof}

\noindent
The following corollary of the result above can be found in
\cite{CLO}.

\begin{lemma}
\label{lemma-reduction-scheme}
\begin{reference}
\cite[3.1.12]{CLO}
\end{reference}
Let $S$ be a scheme. Let $X$ be an algebraic space over $S$.
If $X_{red}$ is a scheme, then $X$ is a scheme.
\end{lemma}

\begin{proof}
Let $U' \subset X_{red}$ be an open affine subscheme.
Let $U \subset X$ be the open subspace corresponding to the open
$|U'| \subset |X_{red}| = |X|$. Then $U' \to U$ is surjective and
integral. Hence $U$ is affine by
Proposition \ref{proposition-affine}.
Thus every point is contained in an open subscheme of $X$, i.e.,
$X$ is a scheme.
\end{proof}

\begin{lemma}
\label{lemma-integral-universally-bijective-scheme}
Let $S$ be a scheme. Let $f : X \to Y$ be a morphism of algebraic spaces
over $S$. Assume $f$ is integral and induces a bijection $|X| \to |Y|$.
Then $X$ is a scheme if and only if $Y$ is a scheme.
\end{lemma}

\begin{proof}
An integral morphism is representable by definition, hence if $Y$
is a scheme, so is $X$. Conversely, assume that $X$ is a scheme.
Let $U \subset X$ be an affine open. An integral morphism is
closed and $|f|$ is bijective, hence $|f|(|U|) \subset |Y|$
is open as the complement of $|f|(|X| \setminus |U|)$. Let
$V \subset Y$ be the open subspace with $|V| = |f|(|U|)$, see
Properties of Spaces, Lemma \ref{spaces-properties-lemma-open-subspaces}.
Then $U \to V$ is integral and surjective, hence
$V$ is an affine scheme by Proposition \ref{proposition-affine}.
This concludes the proof.
\end{proof}

\begin{lemma}
\label{lemma-check-closed-infinitesimally}
Let $S$ be a scheme.
Let $f : X \to B$ and $B' \to B$ be morphisms of algebraic spaces over $S$.
Assume
\begin{enumerate}
\item $B' \to B$ is a closed immersion,
\item $|B'| \to |B|$ is bijective,
\item $X \times_B B' \to B'$ is a closed immersion, and
\item $X \to B$ is of finite type or $B' \to B$ is of finite presentation.
\end{enumerate}
Then $f : X \to B$ is a closed immersion.
\end{lemma}

\begin{proof}
Assumptions (1) and (2) imply that $B_{red} = B'_{red}$.
Set $X' = X \times_B B'$. Then $X' \to X$ is closed immersion
and $X'_{red} = X_{red}$. Let $U \to B$ be an \'etale morphism
with $U$ affine. Then $X' \times_B U \to X \times_B U$ is a
closed immersion of algebraic spaces inducing an isomorphism
on underlying reduced spaces. Since $X' \times_B U$ is a scheme
(as $B' \to B$ and $X' \to B'$ are representable) so is
$X \times_B U$ by Lemma \ref{lemma-reduction-scheme}.
Hence $X \to B$ is representable too. Thus we reduce to the
case of schemes, see
Morphisms, Lemma \ref{morphisms-lemma-check-closed-infinitesimally}.
\end{proof}










\section{Finite cover by a scheme}
\label{section-finite-cover}

\noindent
As an application of the limit results of this chapter, we prove that given
any quasi-compact and quasi-separated algebraic space $X$, there is a scheme
$Y$ and a surjective, finite morphism $Y \to X$. We will rely on the already
proven result that we can find a finite integral cover by a scheme, which
was proved in
Decent Spaces, Section \ref{decent-spaces-section-integral-cover}.

\begin{proposition}
\label{proposition-there-is-a-scheme-finite-over}
Let $S$ be a scheme. Let $X$ be a quasi-compact and quasi-separated
algebraic space over $S$.
\begin{enumerate}
\item There exists a surjective finite morphism $Y \to X$
of finite presentation where $Y$ is a scheme,
\item given a surjective \'etale morphism $U \to X$ we may choose
$Y \to X$ such that for every $y \in Y$ there is an open neighbourhood
$V \subset Y$ such that $V \to X$ factors through $U$.
\end{enumerate}
\end{proposition}

\begin{proof}
Part (1) is the special case of (2) with $U = X$.
Let $Y \to X$ be as in
Decent Spaces, Lemma \ref{decent-spaces-lemma-there-is-a-scheme-integral-over}.
Choose a finite affine open covering $Y = \bigcup V_j$ such that $V_j \to X$
factors through $U$. We can write $Y = \lim Y_i$ with
$Y_i \to X$ finite and of finite presentation, see
Lemma \ref{lemma-integral-limit-finite-and-finite-presentation}.
For large enough $i$ the algebraic space $Y_i$ is a scheme, see
Lemma \ref{lemma-limit-is-scheme}.
For large enough $i$ we can find affine opens $V_{i, j} \subset Y_i$
whose inverse image in $Y$ recovers $V_j$, see
Lemma \ref{lemma-descend-opens}.
For even larger $i$ the morphisms $V_j \to U$ over $X$ come
from morphisms $V_{i, j} \to U$ over $X$, see Proposition
\ref{proposition-characterize-locally-finite-presentation}.
This finishes the proof.
\end{proof}

\begin{lemma}
\label{lemma-integral-limit-finite-and-finite-presentation-refined}
Let $S$ be a scheme. Let $f : X \to Y$ be an integral morphism of algebraic
spaces over $S$. Assume $Y$ quasi-compact and quasi-separated.
Let $V \subset Y$ be a quasi-compact open subspace such that
$f^{-1}(V) \to V$ is finite and of finite presentation.
Then $X$ can be written as a directed limit $X = \lim X_i$
where $f_i : X_i \to Y$ are finite and of finite presentation
such that $f^{-1}(V) \to f_i^{-1}(V)$ is an isomorphism for all $i$.
\end{lemma}

\begin{proof}
This lemma is a slight refinement of Proposition
\ref{proposition-there-is-a-scheme-finite-over}.
Consider the integral quasi-coherent $\mathcal{O}_Y$-algebra
$\mathcal{A} = f_*\mathcal{O}_X$. In the next paragraph, we will
write $\mathcal{A} = \colim \mathcal{A}_i$ as a directed
colimit of finite and finitely presented $\mathcal{O}_Y$-algebras
$\mathcal{A}_i$ such that $\mathcal{A}_i|_V = \mathcal{A}|_V$.
Having done this we set $X_i = \underline{\Spec}_Y(\mathcal{A}_i)$, see
Morphisms of Spaces, Definition
\ref{spaces-morphisms-definition-relative-spec}.
By construction $X_i \to Y$ is finite and of finite presentation,
$X = \lim X_i$, and $f_i^{-1}(V) = f^{-1}(V)$.

\medskip\noindent
The proof of the assertion on algebras is similar to
the proof of part (2) of
Lemma \ref{lemma-integral-algebra-directed-colimit-finite}.
First, write $\mathcal{A} = \colim \mathcal{F}_i$
as a colimit of finitely presented $\mathcal{O}_Y$-modules using
Lemma \ref{lemma-colimit-finitely-presented}.
Since $\mathcal{A}|_V$ is a finite type $\mathcal{O}_V$-module
we may and do assume that $\mathcal{F}_i|_V \to \mathcal{A}|_V$
is surjective for all $i$.
For each $i$, let $\mathcal{J}_i$ be the kernel of the map
$$
\text{Sym}^*_{\mathcal{O}_X}(\mathcal{F}_i) \longrightarrow \mathcal{A}
$$
For $i' \geq i$ there is an induced map $\mathcal{J}_i \to \mathcal{J}_{i'}$.
We have $\mathcal{A} =
\colim \text{Sym}^*_{\mathcal{O}_X}(\mathcal{F}_i)/\mathcal{J}_i$.
Moreover, the quasi-coherent $\mathcal{O}_X$-algebras
$\text{Sym}^*_{\mathcal{O}_X}(\mathcal{F}_i)/\mathcal{J}_i$
are finite (as finite type quasi-coherent subalgebras of the integral
quasi-coherent $\mathcal{O}_Y$-algebra $\mathcal{A}$ over $\mathcal{O}_X$).
The restriction of $\text{Sym}^*_{\mathcal{O}_X}(\mathcal{F}_i)/\mathcal{J}_i$
to $V$ is $\mathcal{A}|_V$ by the surjectivity above.
Hence $\mathcal{J}_i|_V$ is finitely generated as an ideal sheaf of
$\text{Sym}^*_{\mathcal{O}_X}(\mathcal{F}_i)|_V$ due to the
fact that $\mathcal{A}|_V$ is finitely presented as an
$\mathcal{O}_Y$-algebra.
Write $\mathcal{J}_i = \colim \mathcal{E}_{ik}$
as a colimit of finitely presented $\mathcal{O}_X$-modules.
We may and do assume that $\mathcal{E}_{ik}|_V$
generates $\mathcal{J}_i|_V$ as a sheaf of ideal of
$\text{Sym}^*_{\mathcal{O}_X}(\mathcal{F}_i)|_V$ by the
statement on finite generation above.
Given $i' \geq i$ and $k$ there exists a $k'$ such that we
have a map $\mathcal{E}_{ik} \to \mathcal{E}_{i'k'}$
making
$$
\xymatrix{
\mathcal{J}_i \ar[r] & \mathcal{J}_{i'} \\
\mathcal{E}_{ik} \ar[u] \ar[r] & \mathcal{E}_{i'k'} \ar[u]
}
$$
commute. This follows from Cohomology of Spaces, Lemma
\ref{spaces-cohomology-lemma-finite-presentation-quasi-compact-colimit}.
This induces a map
$$
\mathcal{A}_{ik} =
\text{Sym}^*_{\mathcal{O}_X}(\mathcal{F}_i)/(\mathcal{E}_{ik})
\longrightarrow
\text{Sym}^*_{\mathcal{O}_X}(\mathcal{F}_{i'})/(\mathcal{E}_{i'k'}) =
\mathcal{A}_{i'k'}
$$
where $(\mathcal{E}_{ik})$ denotes the ideal generated by $\mathcal{E}_{ik}$.
The quasi-coherent $\mathcal{O}_X$-algebras $\mathcal{A}_{ki}$
are of finite presentation and finite for $k$ large enough
(see proof of
Lemma \ref{lemma-finite-algebra-directed-colimit-finite-finitely-presented}).
Moreover we have $\mathcal{A}_{ik}|_V = \mathcal{A}|_V$ by construction.
Finally, we have
$$
\colim \mathcal{A}_{ik} = \colim \mathcal{A}_i = \mathcal{A}
$$
Namely, the first equality was shown in the proof of
Lemma \ref{lemma-finite-algebra-directed-colimit-finite-finitely-presented}
and the second equality because $\mathcal{A}$ is the colimit of
the modules $\mathcal{F}_i$.
\end{proof}

\begin{lemma}
\label{lemma-there-is-a-scheme-finite-over-refined}
Let $S$ be a scheme. Let $X$ be a quasi-compact and quasi-separated
algebraic space over $S$ such that $|X|$ has finitely many irreducible
components.
\begin{enumerate}
\item There exists a surjective finite morphism $f : Y \to X$
of finite presentation where $Y$ is a scheme such that $f$
is finite \'etale over a quasi-compact dense open $U \subset X$,
\item given a surjective \'etale morphism $V \to X$ we may choose
$Y \to X$ such that for every $y \in Y$ there is an open neighbourhood
$W \subset Y$ such that $W \to X$ factors through $V$.
\end{enumerate}
\end{lemma}

\begin{proof}
Part (1) is the special case of (2) with $V = X$.

\medskip\noindent
Proof of (2). Let $\pi : Y \to X$ be as in Decent Spaces, Lemma
\ref{decent-spaces-lemma-there-is-a-scheme-integral-over-refined}
and let $U \subset X$ be a quasi-compact dense open such that
$\pi^{-1}(U) \to U$ is finite \'etale.
Choose a finite affine open covering $Y = \bigcup W_j$ such that
$W_j \to X$ factors through $V$. We can write $Y = \lim Y_i$ with
$\pi_i : Y_i \to X$ finite and of finite presentation such
that $\pi^{-1}(U) \to \pi_i^{-1}(U)$ is an isomorphism, see
Lemma \ref{lemma-integral-limit-finite-and-finite-presentation-refined}.
For large enough $i$ the algebraic space $Y_i$ is a scheme, see
Lemma \ref{lemma-limit-is-scheme}.
For large enough $i$ we can find affine opens $W_{i, j} \subset Y_i$
whose inverse image in $Y$ recovers $W_j$, see
Lemma \ref{lemma-descend-opens}.
For even larger $i$ the morphisms $W_j \to V$ over $X$ come
from morphisms $W_{i, j} \to U$ over $X$, see Proposition
\ref{proposition-characterize-locally-finite-presentation}.
This finishes the proof.
\end{proof}

\begin{lemma}
\label{lemma-there-is-a-scheme-finite-over-filtered}
Let $S$ be a scheme. Let $X$ be a quasi-compact and quasi-separated
algebraic space over $S$. There exists a $t \geq 0$ and closed
subspaces
$$
X \supset Z_0 \supset Z_1 \supset \ldots \supset Z_t = \emptyset
$$
such that $Z_i \to X$ is of finite presentation,
$Z_0 \subset X$ is a thickening, and for each $i = 0, \ldots t - 1$
there exists a scheme $Y_i$, a surjective, finite, and finitely
presented morphism $Y_i \to Z_i$ which is finite \'etale over
$Z_i \setminus Z_{i + 1}$.
\end{lemma}

\begin{proof}
We may view $X$ as an algebraic space over $\Spec(\mathbf{Z})$, see
Spaces, Definition \ref{spaces-definition-base-change} and
Properties of Spaces, Definition \ref{spaces-properties-definition-separated}.
Thus we may apply Proposition \ref{proposition-approximate}.
It follows that we can find an affine morphism $X \to X_0$
with $X_0$ of finite presentation over $\mathbf{Z}$.
If we can prove the lemma for $X_0$, then we can pull back
the stratification and the morphisms to $X$ and get the result for $X$;
some details omitted. This reduces us to the case discussed in the
next paragraph.

\medskip\noindent
Assume $X$ is of finite presentation over $\mathbf{Z}$.
Then $X$ is Noetherian and $|X|$ is a Noetherian topological
space (with finitely many irreducible components) of finite dimension.
Hence we may use induction on $\dim(|X|)$.
Any finite morphism towards $X$ is of finite presentation, so
we can ignore that requirement in the rest of the proof.
By Lemma \ref{lemma-there-is-a-scheme-finite-over-refined}
there exists a surjective finite morphism $Y \to X$ which is
finite \'etale over a dense open $U \subset X$.
Set $Z_0 = X$ and let $Z_1 \subset X$ be the reduced closed subspace
with $|Z_1| = |X| \setminus |U|$.
By induction we find an integer $t \geq 0$ and a filtration
$$
Z_1 \supset Z_{1, 0} \supset Z_{1, 1} \supset \ldots
\supset Z_{1, t} = \emptyset
$$
by closed subspaces, where $Z_{1, 0} \to Z_1$ is a thickening
and there exist finite surjective morphisms $Y_{1, i} \to Z_{1, i}$
which are finite \'etale over $Z_{1, i} \setminus Z_{1, i + 1}$.
Since $Z_1$ is reduced, we have $Z_1 = Z_{1, 0}$.
Hence we can set $Z_i = Z_{1, i - 1}$ and $Y_i = Y_{1, i - 1}$
for $i \geq 1$ and the lemma is proved.
\end{proof}











\section{Obtaining schemes}
\label{section-representable}

\noindent
A few more techniques to show an algebraic space is a scheme.
The first is that we can show there is a minimal closed subspace
which is not a scheme.

\begin{lemma}
\label{lemma-minimal-closed-subspace}
Let $S$ be a scheme. Let $X$ be a quasi-compact and quasi-separated
algebraic space over $S$. If $X$ is not a scheme, then there exists
a closed subspace $Z \subset X$ such that $Z$ is not a scheme, but
every proper closed subspace $Z' \subset Z$ is a scheme.
\end{lemma}

\begin{proof}
We prove this by Zorn's lemma. Let $\mathcal{Z}$ be the set
of closed subspaces $Z$ which are not schemes ordered by inclusion.
By assumption $\mathcal{Z}$ contains $X$, hence is nonempty.
If $Z_\alpha$ is a totally ordered subset of $\mathcal{Z}$, then
$Z = \bigcap Z_\alpha$ is in $\mathcal{Z}$. Namely,
$$
Z = \lim Z_\alpha
$$
and the transition morphisms are affine.
Thus we may apply Lemma \ref{lemma-limit-is-scheme} to see that if $Z$
were a scheme, then so would one of the $Z_\alpha$.
(This works even if $Z = \emptyset$, but note that by
Lemma \ref{lemma-limit-nonempty} this cannot happen.)
Thus $\mathcal{Z}$ has minimal elements by Zorn's lemma.
\end{proof}

\noindent
Now we can prove a little bit about these minimal non-schemes.

\begin{lemma}
\label{lemma-minimal-nonscheme}
Let $S$ be a scheme. Let $X$ be a quasi-compact and quasi-separated
algebraic space over $S$. Assume that every proper closed subspace
$Z \subset X$ is a scheme, but $X$ is not a scheme. Then $X$ is reduced
and irreducible.
\end{lemma}

\begin{proof}
We see that $X$ is reduced by Lemma \ref{lemma-reduction-scheme}.
Choose closed subsets $T_1 \subset |X|$ and $T_2 \subset |X|$ such that
$|X| = T_1 \cup T_2$. If $T_1$ and $T_2$ are proper closed subsets,
then the corresponding reduced induced closed subspaces $Z_1, Z_2 \subset X$
(Properties of Spaces, Definition
\ref{spaces-properties-definition-reduced-induced-space})
are schemes and so is $Z = Z_1 \times_X Z_2 = Z_1 \cap Z_2$ as a closed
subscheme of either $Z_1$ or $Z_2$. Observe that the coproduct
$Z_1 \amalg_Z Z_2$ exists in the category of schemes, see
More on Morphisms, Lemma
\ref{more-morphisms-lemma-pushout-along-closed-immersions}.
One way to proceed, is to show that $Z_1 \amalg_Z Z_2$ is isomorphic to $X$,
but we cannot use this here as the material on pushouts of algebraic
spaces comes later in the theory. Instead we will use
Lemma \ref{lemma-affine} to find an affine neighbourhood of every point.
Namely, let $x \in |X|$. If $x \not \in Z_1$, then $x$ has a neighbourhood
which is a scheme, namely, $X \setminus Z_1$. Similarly if $x \not \in Z_2$.
If $x \in Z = Z_1 \cap Z_2$, then we choose an affine open
$U \subset Z_1 \amalg_Z Z_2$ containing $z$. Then $U_1 = Z_1 \cap U$
and $U_2 = Z_2 \cap U$ are affine opens whose intersections with
$Z$ agree. Since $|Z_1| = T_1$ and $|Z_2| = T_2$ are closed subsets of
$|X|$ which intersect in $|Z|$, we find an open $W \subset |X|$
with $W \cap T_1 = |U_1|$ and $W \cap T_2 = |U_2|$. Let $W$ denote the
corresponding open subspace of $X$. Then $x \in |W|$ and the morphism
$U_1 \amalg U_2 \to W$ is a surjective finite morphism whose source
is an affine scheme. Thus $W$ is an affine scheme by
Lemma \ref{lemma-affine}.
\end{proof}

\noindent
A key point in the following lemma is that we only need to check
the condition in the images of points of $X$.

\begin{lemma}
\label{lemma-enough-local}
Let $f: X \to S$ be a quasi-compact and quasi-separated morphism from an
algebraic space to a scheme $S$. If for every $x \in |X|$ with image
$s = f(x) \in S$ the algebraic space $X \times_S \Spec(\mathcal{O}_{S,s})$
is a scheme, then $X$ is a scheme.
\end{lemma}

\begin{proof}
Let $x \in |X|$. It suffices to find an open neighbourhood $U$ of
$s = f(x)$ such that $X \times_S U$ is a scheme.
As $X \times_S \Spec(\mathcal{O}_{S, s})$ is a scheme, then, since
$\mathcal{O}_{S, s} = \colim \mathcal{O}_S(U)$ where the colimit is
over affine open neighbourhoods of $s$ in $S$ we see that
$$
X \times_S \Spec(\mathcal{O}_{S, s}) = \lim X \times_S U
$$
By Lemma \ref{lemma-limit-is-scheme} we see that $X \times_S U$
is a scheme for some $U$.
\end{proof}

\noindent
Instead of restricting to local rings as in Lemma \ref{lemma-enough-local},
we can restrict to closed subschemes of the base.

\begin{lemma}
\label{lemma-maximal-ideal}
Let $\varphi : X \to \Spec(A)$ be a quasi-compact and quasi-separated
morphism from an algebraic space to an affine scheme.
If $X$ is not a scheme, then there exists an ideal $I \subset A$
such that the base change $X_{A/I}$ is not a scheme, but
for every $I \subset I'$, $I \not = I'$ the base change
$X_{A/I'}$ is a scheme.
\end{lemma}

\begin{proof}
We prove this by Zorn's lemma. Let $\mathcal{I}$ be the set
of ideals $I$ such that $X_{A/I}$ is not a scheme. By
assumption $\mathcal{I}$ contains $(0)$. If $I_\alpha$ is
a chain of ideals in $\mathcal{I}$, then
$I = \bigcup I_\alpha$ is in $\mathcal{I}$. Namely,
$A/I = \colim A/I_\alpha$, hence
$$
X_{A/I} = \lim X_{A/I_\alpha}
$$
Thus we may apply Lemma \ref{lemma-limit-is-scheme} to see that if $X_{A/I}$
were a scheme, then so would be one of the $X_{A/I_\alpha}$.
Thus $\mathcal{I}$ has maximal elements by Zorn's lemma.
\end{proof}






\section{Glueing in closed fibres}
\label{section-change-over-closed-points}

\noindent
Applying our theory above to the spectrum of a local ring we obtain
a few pleasing glueing results for relative algebraic spaces.
We first prove a helper lemma (which will be vastly generalized
in Bootstrap, Section \ref{bootstrap-section-applications}).

\begin{lemma}
\label{lemma-relative-glueing}
Let $S = U \cup W$ be an open covering of a scheme. Then the functor
$$
FP_S \longrightarrow FP_U \times_{FP_{U \cap W}} FP_W
$$
given by base change is an equivalence where $FP_T$
is the category of algebraic spaces of finite presentation over
the scheme $T$.
\end{lemma}

\begin{proof}
First, since $S = U \cup W$ is a Zariski covering, we see that the
category of sheaves on $(\Sch/S)_{fppf}$ is equivalent to the category
of triples $(\mathcal{F}_U, \mathcal{F}_W, \varphi)$ where
$\mathcal{F}_U$ is a sheaf on $(\Sch/U)_{fppf}$,
$\mathcal{F}_W$ is a sheaf on $(\Sch/W)_{fppf}$, and
$$
\varphi :
\mathcal{F}_U|_{(\Sch/U \cap W)_{fppf}}
\longrightarrow
\mathcal{F}_W|_{(\Sch/U \cap W)_{fppf}}
$$
is an isomorphism. See Sites, Lemma \ref{sites-lemma-mapping-property-glue}
(note that no other gluing data are necessary because
$U \times_S U = U$, $W \times_S W = W$ and that the cocycle
condition is automatic for the same reason).
Now, if the sheaf $\mathcal{F}$ on $(\Sch/S)_{fppf}$
maps to $(\mathcal{F}_U, \mathcal{F}_W, \varphi)$
via this equivalence, then $\mathcal{F}$ is an algebraic space
if and only if $\mathcal{F}_U$ and $\mathcal{F}_W$ are algebraic spaces.
This follows immediately from
Algebraic Spaces, Lemma \ref{spaces-lemma-glueing-algebraic-spaces}
as $\mathcal{F}_U \to \mathcal{F}$ and $\mathcal{F}_W \to \mathcal{F}$
are representable by open immersions and cover $\mathcal{F}$.
Finally, in this case the algebraic space $\mathcal{F}$ is of finite
presentation over $S$ if and only if $\mathcal{F}_U$ is of finite presentation
over $U$ and $\mathcal{F}_W$ is of finite presentation over $W$
by Morphisms of Spaces, Lemmas
\ref{spaces-morphisms-lemma-quasi-compact-local},
\ref{spaces-morphisms-lemma-separated-local}, and
\ref{spaces-morphisms-lemma-finite-presentation-local}.
\end{proof}

\begin{lemma}
\label{lemma-glueing-near-closed-point}
Let $S$ be a scheme. Let $s \in S$ be a closed point such that
$U = S \setminus \{s\} \to S$ is quasi-compact. With
$V = \Spec(\mathcal{O}_{S, s}) \setminus \{s\}$ there is
an equivalence of categories
$$
FP_S \longrightarrow FP_U \times_{FP_V} FP_{\Spec(\mathcal{O}_{S, s})}
$$
where $FP_T$ is the category of algebraic spaces of finite presentation
over $T$.
\end{lemma}

\begin{proof}
Let $W \subset S$ be an open neighbourhood of $s$. The functor
$$
FP_S \to FP_U \times_{FP_{W \setminus \{s\}}} FP_W
$$
is an equivalence of categories by Lemma \ref{lemma-relative-glueing}.
We have $\mathcal{O}_{S, s} = \colim \mathcal{O}_W(W)$ where
$W$ runs over the affine open neighbourhoods of $s$.
Hence $\Spec(\mathcal{O}_{S, s}) = \lim W$ where $W$
runs over the affine open neighbourhoods of $s$.
Thus the category of algebraic spaces of finite presentation
over $\Spec(\mathcal{O}_{S, s})$ is the limit of the
category of algebraic spaces of finite presentation over
$W$ where $W$ runs over the affine open neighbourhoods
of $s$, see
Lemma \ref{lemma-descend-finite-presentation}.
For every affine open $s \in W$ we see that $U \cap W$
is quasi-compact as $U \to S$ is quasi-compact.
Hence $V = \lim W \cap U = \lim W \setminus \{s\}$ is a limit of
quasi-compact and quasi-separated schemes (see
Limits, Lemma \ref{limits-lemma-directed-inverse-system-has-limit}).
Thus also the category of algebraic spaces of finite presentation
over $V$ is the limit of the
categories of algebraic spaces of finite presentation over
$W \cap U$ where $W$ runs over the affine open neighbourhoods
of $s$. The lemma follows formally from a combination
of these results.
\end{proof}

\begin{lemma}
\label{lemma-glueing-near-point}
Let $S$ be a scheme. Let $U \subset S$ be a retrocompact open.
Let $s \in S$ be a point in the complement of $U$. With
$V = \Spec(\mathcal{O}_{S, s}) \cap U$ there is
an equivalence of categories
$$
\colim_{s \in U' \supset U\text{ open}} FP_{U'}
\longrightarrow
FP_U \times_{FP_V} FP_{\Spec(\mathcal{O}_{S, s})}
$$
where $FP_T$  is the category of algebraic spaces of finite presentation
over $T$.
\end{lemma}

\begin{proof}
Let $W \subset S$ be an open neighbourhood of $s$. By
Lemma \ref{lemma-relative-glueing} the functor
$$
FP_{U \cup W}
\longrightarrow
FP_U \times_{FP_{U \cap W}} FP_W
$$
is an equivalence of categories. We have
$\mathcal{O}_{S, s} = \colim \mathcal{O}_W(W)$ where
$W$ runs over the affine open neighbourhoods of $s$.
Hence $\Spec(\mathcal{O}_{S, s}) = \lim W$ where $W$
runs over the affine open neighbourhoods of $s$.
Thus the category of algebraic spaces of finite presentation
over $\Spec(\mathcal{O}_{S, s})$ is the limit of the
category of algebraic spaces of finite presentation over
$W$ where $W$ runs over the affine open neighbourhoods
of $s$, see
Lemma \ref{lemma-descend-finite-presentation}.
For every affine open $s \in W$ we see that $U \cap W$
is quasi-compact as $U \to S$ is quasi-compact.
Hence $V = \lim W \cap U$ is a limit of
quasi-compact and quasi-separated schemes (see
Limits, Lemma \ref{limits-lemma-directed-inverse-system-has-limit}).
Thus also the category of algebraic spaces of finite presentation
over $V$ is the limit of the
categories of algebraic spaces of finite presentation over
$W \cap U$ where $W$ runs over the affine open neighbourhoods
of $s$. The lemma follows formally from a combination
of these results.
\end{proof}

\begin{lemma}
\label{lemma-glueing-near-multiple-closed-points}
Let $S$ be a scheme. Let $s_1, \ldots, s_n \in S$ be
pairwise distinct closed points such that
$U = S \setminus \{s_1, \ldots, s_n\} \to S$ is quasi-compact. With
$S_i = \Spec(\mathcal{O}_{S, s_i})$ and $U_i = S_i \setminus \{s_i\}$
there is an equivalence of categories
$$
FP_S \longrightarrow
FP_U \times_{(FP_{U_1} \times \ldots \times FP_{U_n})}
(FP_{S_1} \times \ldots \times FP_{S_n})
$$
where $FP_T$ is the category of algebraic spaces of finite presentation
over $T$.
\end{lemma}

\begin{proof}
For $n = 1$ this is Lemma \ref{lemma-glueing-near-closed-point}.
For $n > 1$ the lemma can be proved in exactly the same way or it
can be deduced from it. For example, suppose that $f_i : X_i \to S_i$
are objects of $FP_{S_i}$ and $f : X \to U$ is an object
of $FP_U$ and we're given isomorphisms $X_i \times_{S_i} U_i = X \times_U U_i$.
By Lemma \ref{lemma-glueing-near-closed-point} we can find
a morphism $f' : X' \to U' = S \setminus \{s_1, \ldots, s_{n - 1}\}$
which is of finite presentation, which is isomorphic to
$X_i$ over $S_i$, which is isomorphic to $X$ over $U$, and
these isomorphisms are compatible with the given isomorphism
$X_i \times_{S_n} U_n = X \times_U U_n$.
Then we can apply induction to
$f_i : X_i \to S_i$, $i \leq n - 1$,
$f' : X' \to U'$, and the induced
isomorphisms $X_i \times_{S_i} U_i = X' \times_{U'} U_i$, $i \leq n - 1$.
This shows essential surjectivity. We omit the proof of
fully faithfulness.
\end{proof}







\section{Application to modifications}
\label{section-modifications-at-a-point}

\noindent
Using limits we can describe the category of modifications of
a decent algebraic space over a closed point in terms of the
henselian local ring.

\begin{lemma}
\label{lemma-excision-modifications}
Let $S$ be a scheme. Consider a separated \'etale morphism
$f : V \to W$ of algebraic spaces over $S$.
Assume there exists a
closed subspace $T \subset W$ such that $f^{-1}T \to T$ is
an isomorphism. Then, with $W^0 = W \setminus T$ and
$V^0 = f^{-1}W^0$ the base change functor
$$
\left\{
\begin{matrix}
g : X \to W\text{ morphism of algebraic spaces} \\
g^{-1}(W^0) \to W^0\text{ is an isomorphism}
\end{matrix}
\right\}
\longrightarrow
\left\{
\begin{matrix}
h : Y \to V\text{ morphism of algebraic spaces} \\
h^{-1}(V^0) \to V^0\text{ is an isomorphism}
\end{matrix}
\right\}
$$
is an equivalence of categories.
\end{lemma}

\begin{proof}
Since $V \to W$ is separated we see that
$V \times_W V = \Delta(V) \amalg U$ for some open and closed subspace
$U$ of $V \times_W V$. By the assumption that $f^{-1}T \to T$ is an
isomorphism we see that $U \times_W T = \emptyset$, i.e., the two
projections $U \to V$ maps into $V^0$.

\medskip\noindent
Given $h : Y \to V$ in the right hand category, consider the
contravariant functor $X$ on $(\Sch/S)_{fppf}$ defined by the rule
$$
X(T) = \{(w, y) \mid
w :  T \to W,\ y : T \times_{w, W} V \to Y\text{ morphism over }V\}
$$
Denote $g : X \to W$ the map sending $(w, y) \in X(T)$ to $w \in W(T)$.
Since $h^{-1}V^0 \to V^0$ is an isomorphism, we see that if
$w : T \to W$ maps into $W^0$, then there is a unique choice for $h$.
In other words $X \times_{g, W} W^0 = W^0$. On the other hand, consider
a $T$-valued point $(w, y, v)$ of $X \times_{g, W, f} V$.
Then $w = f \circ v$ and
$$
y : T \times_{f \circ v, W} V \longrightarrow V
$$
is a morphism over $V$. Consider the morphism
$$
T \times_{f \circ v, W} V \xrightarrow{(v, \text{id}_V)}
V \times_W V = V \amalg U
$$
The inverse image of $V$ is $T$ embedded via
$(\text{id}_T, v) : T \to T \times_{f \circ v, W} V$.
The composition $y' = y \circ (\text{id}_T, v) : T \to Y$
is a morphism with $v = h \circ y'$ which determines $y$ because the
restriction of $y$ to the other part is uniquely determined as
$U$ maps into $V^0$ by the second projection. It follows that
$X \times_{g, W, f} V \to Y$, $(w, y, v) \mapsto y'$ is an isomorphism.

\medskip\noindent
Thus if we can show that $X$ is an algebraic space, then we are done.
Since $V \to W$ is separated and \'etale it is representable by
Morphisms of Spaces, Lemma
\ref{spaces-morphisms-lemma-locally-quasi-finite-separated-representable}
(and Morphisms of Spaces, Lemma
\ref{spaces-morphisms-lemma-etale-locally-quasi-finite}).
Of course $W^0 \to W$ is representable and \'etale as it is an
open immersion. Thus
$$
W^0 \amalg Y = X \times_{g, W} W^0 \amalg X \times_{g, W, f} V
= X \times_{g, W} (W^0 \amalg V) \longrightarrow X
$$
is representable, surjective, and \'etale by Spaces, Lemmas
\ref{spaces-lemma-base-change-representable-transformations} and
\ref{spaces-lemma-base-change-representable-transformations-property}.
Thus $X$ is an algebraic
space by Spaces, Lemma
\ref{spaces-lemma-etale-locally-representable-by-space-gives-space}.
\end{proof}

\begin{lemma}
\label{lemma-excision-modifications-properties}
Notation and assumptions as in Lemma \ref{lemma-excision-modifications}.
Let $g : X \to W$ correspond to $h : Y \to V$ via the equivalence.
Then $g$ is quasi-compact, quasi-separated, separated, locally of finite
presentation, of finite presentation, locally of finite type, of finite type,
proper, integral, finite, and add more here if and only if
$h$ is so.
\end{lemma}

\begin{proof}
If $g$ is quasi-compact, quasi-separated, separated, locally of finite
presentation, of finite presentation, locally of finite type, of finite type,
proper, finite, so is $h$ as a base change of $g$ by
Morphisms of Spaces, Lemmas
\ref{spaces-morphisms-lemma-base-change-quasi-compact},
\ref{spaces-morphisms-lemma-base-change-separated},
\ref{spaces-morphisms-lemma-base-change-finite-presentation},
\ref{spaces-morphisms-lemma-base-change-finite-type},
\ref{spaces-morphisms-lemma-base-change-proper},
\ref{spaces-morphisms-lemma-base-change-integral}.
Conversely, let $P$ be a property of morphisms of algebraic
spaces which is \'etale local on the base and which holds for
the identity morphism of any algebraic space.
Since $\{W^0 \to W, V \to W\}$ is an \'etale
covering, to prove that $g$ has $P$ it suffices to show
that $h$ has $P$. Thus we conclude using
Morphisms of Spaces, Lemmas
\ref{spaces-morphisms-lemma-quasi-compact-local},
\ref{spaces-morphisms-lemma-separated-local},
\ref{spaces-morphisms-lemma-finite-presentation-local},
\ref{spaces-morphisms-lemma-finite-type-local},
\ref{spaces-morphisms-lemma-proper-local},
\ref{spaces-morphisms-lemma-integral-local}.
\end{proof}

\begin{lemma}
\label{lemma-modifications}
Let $S$ be a scheme. Let $X$ be a decent algebraic space over $S$.
Let $x \in |X|$ be a closed point such that $U = X \setminus \{x\} \to X$
is quasi-compact. With
$V = \Spec(\mathcal{O}_{X, x}^h) \setminus \{\mathfrak m_x^h\}$
the base change functor
$$
\left\{
\begin{matrix}
f : Y \to X\text{ of finite presentation} \\
f^{-1}(U) \to U\text{ is an isomorphism}
\end{matrix}
\right\}
\longrightarrow
\left\{
\begin{matrix}
g : Y \to \Spec(\mathcal{O}_{X, x}^h)\text{ of finite presentation} \\
g^{-1}(V) \to V\text{ is an isomorphism}
\end{matrix}
\right\}
$$
is an equivalence of categories.
\end{lemma}

\begin{proof}
Let $a : (W, w) \to (X, x)$ be an elementary \'etale neighbourhood of $x$
with $W$ affine as in
Decent Spaces, Lemma
\ref{decent-spaces-lemma-decent-space-elementary-etale-neighbourhood}.
Since $x$ is a closed point of $X$ and $w$ is the unique point of $W$
lying over $x$, we see that $w$ is a closed point of $W$. Since $a$
is \'etale and identifies residue fields at $x$ and $w$, it
follows that $a$ induces an isomorphism $a^{-1}x \to x$ (as closed
subspaces of $X$ and $W$). Thus we may apply
Lemma \ref{lemma-excision-modifications} and
\ref{lemma-excision-modifications-properties}
to reduce the problem to the case where $X$ is an affine scheme.

\medskip\noindent
Assume $X$ is an affine scheme. Recall that $\mathcal{O}_{X, x}^h$
is the colimit of $\Gamma(U, \mathcal{O}_U)$ over affine
elementary \'etale neighbourhoods $(U, u) \to (X, x)$.
Recall that the category of these neighbourhoods is
cofiltered, see Decent Spaces, Lemma
\ref{decent-spaces-lemma-elementary-etale-neighbourhoods} or
More on Morphisms, Lemma
\ref{more-morphisms-lemma-elementary-etale-neighbourhoods}.
Then $\Spec(\mathcal{O}_{X, x}^h) = \lim U$ and
$V = \lim U \setminus \{u\}$
(Lemma \ref{lemma-directed-inverse-system-has-limit})
where the limits are taken over the same category. Thus by
Lemma \ref{lemma-descend-finite-presentation}
The category on the right is the colimit of the categories
for the pairs $(U, u)$. And by the material in the first
paragraph, each of these categories is equivalent to the
category for the pair $(X, x)$. This finishes the proof.
\end{proof}






\section{Universally closed morphisms}
\label{section-universally-closed}

\noindent
In this section we discuss when a quasi-compact (but not necessarily
separated) morphism is universally closed. We first prove a lemma which
will allow us to check universal closedness after a base change
which is locally of finite presentation.

\begin{lemma}
\label{lemma-separate}
Let $S$ be a scheme. Let $f : X \to Y$ and $g : Z \to Y$ be
morphisms of algebraic spaces over $S$. Let $z \in |Z|$ and let
$T \subset |X \times_Y Z|$ be a closed subset
with $z \not \in \Im(T \to |Z|)$.
If $f$ is quasi-compact, then there exists
an \'etale neighbourhood $(V, v) \to (Z, z)$,
a commutative diagram
$$
\xymatrix{
V \ar[d] \ar[r]_a & Z' \ar[d]^b \\
Z \ar[r]^g & Y,
}
$$
and a closed subset $T' \subset |X \times_Y Z'|$ such that
\begin{enumerate}
\item the morphism $b : Z' \to Y$ is locally of finite presentation,
\item with $z' = a(v)$ we have $z' \not \in \Im(T' \to |Z'|)$, and
\item the inverse image of $T$ in $|X \times_Y V|$
maps into $T'$ via $|X \times_Y V| \to |X \times_Y Z'|$.
\end{enumerate}
Moreover, we may assume $V$ and $Z'$ are affine schemes and if $Z$
is a scheme we may assume $V$ is an affine open neighbourhood of $z$.
\end{lemma}

\begin{proof}
We will deduce this from the corresponding result for morphisms of schemes.
Let $y \in |Y|$ be the image of $z$. First we choose an affine \'etale
neighbourhood $(U, u) \to (Y, y)$ and then we choose an affine \'etale
neighbourhood $(V, v) \to (Z, z)$ such that the morphism $V \to Y$
factors through $U$. Then we may replace
\begin{enumerate}
\item $X \to Y$ by $X \times_Y U \to U$,
\item $Z \to Y$ by $V \to U$,
\item $z$ by $v$, and
\item $T$ by its inverse image in
$|(X \times_Y U) \times_U V| = |X \times_Y V|$.
\end{enumerate}
In fact, below we will show that after replacing $V$ by an affine
open neighbourhood of $v$ there will be a morphism $a : V \to Z'$ for
some $Z' \to U$ of finite presentation and a closed subset $T'$
of $|(X \times_Y U) \times_U Z'| = |X \times_Y Z'|$ such that
$T$ maps into $T'$ and $a(v) \not \in \Im(T' \to |Z'|)$.
Thus we may and do assume that $Z$ and $Y$ are affine schemes
with the proviso that we need to find a solution where $V$
is an open neighbourhood of $z$.

\medskip\noindent
Since $f$ is quasi-compact and $Y$ is affine, the algebraic space
$X$ is quasi-compact. Choose an affine scheme $W$ and a surjective
\'etale morphism $W \to X$. Let $T_W \subset |W \times_Y Z|$
be the inverse image of $T$. Then $z$ is not in the image of
$T_W$. By the schemes case (Limits, Lemma \ref{limits-lemma-separate})
we can find an open neighbourhood $V \subset Z$ of $z$
a commutative diagram of schemes
$$
\xymatrix{
V \ar[d] \ar[r]_a & Z' \ar[d]^b \\
Z \ar[r]^g & Y,
}
$$
and a closed subset $T' \subset |W \times_Y Z'|$ such that
\begin{enumerate}
\item the morphism $b : Z' \to Y$ is locally of finite presentation,
\item with $z' = a(z)$ we have $z' \not \in \Im(T' \to Z')$, and
\item $T_1 = T_W \cap |W \times_Y V|$ maps into $T'$ via
$|W \times_Y V| \to |W \times_Y Z'|$.
\end{enumerate}
The commutative diagram
$$
\xymatrix{
W \times_Y Z \ar[d] &
W \times_Y V \ar[l] \ar[rr]_{a_1} \ar[d]_c & &
W \times_Y Z' \ar[d]^q \\
X \times_Y Z &
X \times_Y V \ar[l] \ar[rr]^{a_2} & &
X \times_Y Z'
}
$$
has cartesian squares and the vertical maps are, surjective, \'etale
and a fortiori open. Looking at the left hand square we
see that $T_1 = T_W \cap |W \times_Y V|$ is the inverse image of
$T_2 = T \cap |X \times_Y V|$ by $c$. By Properties of Spaces, Lemma
\ref{spaces-properties-lemma-points-cartesian} we get
$a_1(T_1) = q^{-1}(a_2(T_2))$.
By Topology, Lemma \ref{topology-lemma-open-morphism-quotient-topology}
we get
$$
q^{-1}\left(\overline{a_2(T_2)}\right) =
\overline{q^{-1}(a_2(T_2))} =
\overline{a_1(T_1)} \subset T'
$$
As $q$ is surjective the image of $\overline{a_2(T_2)} \to |Z'|$
does not contain $z'$ since the same is true for $T'$.
Thus we can take the diagram with $Z', V, a, b$ above and the
closed subset $\overline{a_2(T_2)} \subset |X \times_Y Z'|$ as
a solution to the problem posed by the lemma.
\end{proof}

\begin{lemma}
\label{lemma-test-universally-closed}
Let $S$ be a scheme.
Let $f : X \to Y$ be a quasi-compact morphism of algebraic spaces over $S$.
The following are equivalent
\begin{enumerate}
\item $f$ is universally closed,
\item for every morphism $Z \to Y$ which is locally of finite presentation
the map $|X \times_Y Z| \to |Z|$ is closed, and
\item there exists a scheme $V$ and a surjective \'etale morphism $V \to Y$
such that $|\mathbf{A}^n \times (X \times_Y V)| \to |\mathbf{A}^n \times V|$
is closed for all $n \geq 0$.
\end{enumerate}
\end{lemma}

\begin{proof}
It is clear that (1) implies (2).
Suppose that $|X \times_Y Z| \to |Z|$ is not closed for some
morphism of algebraic spaces $Z \to Y$ over $S$. This means that there
exists some closed subset $T \subset |X \times_Y Z|$
such that $\Im(T \to |Z|)$ is not closed. Pick $z \in |Z|$
in the closure of the image of $T$ but not in the image.
Apply Lemma \ref{lemma-separate}.
We find an \'etale neighbourhood $(V, v) \to (Z, z)$, a commutative diagram
$$
\xymatrix{
V \ar[d] \ar[r]_a & Z' \ar[d]^b \\
Z \ar[r]^g & Y,
}
$$
and a closed subset $T' \subset |X \times_Y Z'|$ such that
\begin{enumerate}
\item the morphism $b : Z' \to Y$ is locally of finite presentation,
\item with $z' = a(v)$ we have $z' \not \in \Im(T' \to |Z'|)$, and
\item the inverse image of $T$ in $|X \times_Y V|$ maps into $T'$ via
$|X \times_Y V| \to |X \times_Y Z'|$.
\end{enumerate}
We claim that $z'$ is in the closure of $\Im(T' \to |Z'|)$
which implies that $|X \times_Y Z'| \to |Z'|$ is not closed.
The claim shows that (2) implies (1).
To see the claim is true we contemplate
following commutative diagram
$$
\xymatrix{
X \times_Y Z \ar[d] &
X \times_Y V \ar[l] \ar[d] \ar[r] &
X \times_Y Z' \ar[d] \\
Z & V \ar[l] \ar[r]^a & Z'
}
$$
Let $T_V \subset |X \times_Y V|$ be the inverse image of $T$.
By Properties of Spaces, Lemma \ref{spaces-properties-lemma-points-cartesian}
the image of $T_V$ in $|V|$ is the inverse image of the image
of $T$ in $|Z|$. Then since $z$ is in the closure of the image of
$T \to |Z|$ and since $|V| \to |Z|$ is open, we see that $v$ is in
the closure of the image of $T_V \to |V|$. Since the image of
$T_V$ in $|X \times_Y Z'|$ is contained in $|T'|$ it follows
immediately that $z' = a(v)$ is in the closure of the image of $T'$.

\medskip\noindent
It is clear that (1) implies (3). Let $V \to Y$ be as in (3).
If we can show that $X \times_Y V \to V$ is universally closed,
then $f$ is universally closed by
Morphisms of Spaces, Lemma
\ref{spaces-morphisms-lemma-universally-closed-local}.
Thus it suffices to show that $f : X \to Y$ satisfies (2)
if $f$ is a quasi-compact morphism of algebraic spaces,
$Y$ is a scheme, and $|\mathbf{A}^n \times X| \to |\mathbf{A}^n \times Y|$
is closed for all $n$. Let $Z \to Y$ be locally of finite presentation.
We have to show the map $|X \times_Y Z| \to |Z|$ is closed.
This question is \'etale local on $Z$ hence we may assume $Z$
is affine (some details omitted). Since $Y$ is a scheme, $Z$ is affine,
and $Z \to Y$ is locally of finite presentation we can find
an immersion $Z \to \mathbf{A}^n \times Y$, see
Morphisms, Lemma \ref{morphisms-lemma-quasi-affine-finite-type-over-S}.
Consider the cartesian diagram
$$
\vcenter{
\xymatrix{
X \times_Y Z \ar[d] \ar[r] & \mathbf{A}^n \times X \ar[d] \\
Z \ar[r] & \mathbf{A}^n \times Y
}
}
\quad
\begin{matrix}
\text{inducing the} \\
\text{cartesian square}
\end{matrix}
\quad
\vcenter{
\xymatrix{
|X \times_Y Z| \ar[d] \ar[r] & |\mathbf{A}^n \times X| \ar[d] \\
|Z| \ar[r] & |\mathbf{A}^n \times Y|
}
}
$$
of topological spaces whose horizontal arrows are homeomorphisms
onto locally closed subsets (Properties of Spaces, Lemma
\ref{spaces-properties-lemma-subspace-induced-topology}).
Thus every closed subset $T$
of $|X \times_Y Z|$ is the pullback of a closed subset $T'$ of
$|\mathbf{A}^n \times Y|$. Since the assumption is that the image
of $T'$ in $|\mathbf{A}^n \times X|$ is closed we conclude that
the image of $T$ in $|Z|$ is closed as desired.
\end{proof}

\begin{lemma}
\label{lemma-limited-base-change}
Let $S$ be a scheme. Let $f : X \to Y$ be a
morphism of algebraic spaces over $S$.
Assume $f$ separated and of finite type.
The following are equivalent
\begin{enumerate}
\item The morphism $f$ is proper.
\item For any morphism $Y \to Z$ which is locally of finite presentation
the map $|X \times_Y Z| \to |Z|$ is closed, and
\item there exists a scheme $V$ and a surjective \'etale morphism $V \to Y$
such that $|\mathbf{A}^n \times (X \times_Y V)| \to |\mathbf{A}^n \times V|$
is closed for all $n \geq 0$.
\end{enumerate}
\end{lemma}

\begin{proof}
In view of the fact that a proper morphism is the same thing as
a separated, finite type, and universally closed morphism, this
lemma is a special case of Lemma \ref{lemma-test-universally-closed}.
\end{proof}


\section{Noetherian valuative criterion}
\label{section-Noetherian-valuative-criterion}

\noindent
We have already proved some results in Cohomology of Spaces, Section
\ref{spaces-cohomology-section-Noetherian-valuative-criterion}.
The corresponding section for schemes is
Limits, Section \ref{limits-section-Noetherian-valuative-criterion}.

\medskip\noindent
Many of the results in this section can (and perhaps should)
be proved by appealing to the following lemma, although we have not
always done so.

\begin{lemma}
\label{lemma-reach-point-closure-Noetherian}
Let $S$ be a scheme. Let $f : X \to Y$ be a morphism of algebraic spaces
over $S$. Assume $f$ finite type and $Y$ locally Noetherian.
Let $y \in |Y|$ be a point in the closure of the image of $|f|$.
Then there exists a commutative diagram
$$
\xymatrix{
\Spec(K) \ar[r] \ar[d] & X \ar[d]^f \\
\Spec(A) \ar[r] & Y
}
$$
where $A$ is a discrete valuation ring and $K$ is its field of fractions
mapping the closed point of $\Spec(A)$ to $y$. Moreover, we can assume
that the point $x \in |X|$ corresponding to $\Spec(K) \to X$ is a
codimension $0$ point\footnote{See discussion in
Properties of Spaces, Section \ref{spaces-properties-section-generic-points}.}
and that $K$ is the residue field of a point
on a scheme \'etale over $X$.
\end{lemma}

\begin{proof}
Choose an affine scheme $V$, a point $v \in V$ and an \'etale morphism
$V \to Y$ mapping $v$ to $y$. The map $|V| \to |Y|$ is open and by
Properties of Spaces, Lemma \ref{spaces-properties-lemma-points-cartesian}
the image of $|X \times_Y V| \to |V|$ is the inverse image of the
image of $|f|$. We conclude that the point $v$ is in the closure of the
image of $|X \times_Y V| \to |V|$. If we prove the lemma for
$X \times_Y V \to V$ and the point $v$, then the lemma follows for
$f$ and $y$. In this way we reduce to the situation described in the
next paragraph.

\medskip\noindent
Assume we have $f : X \to Y$ and $y \in |Y|$ as in the lemma where
$Y$ is an affine scheme. Since $f$ is quasi-compact, we conclude that
$X$ is quasi-compact. Hence we can choose an affine scheme $W$ and
a surjective \'etale morphism $W \to X$. Then the image of
$|f|$ is the same as the image of $W \to Y$. In this way we reduce
to the case of schemes which is
Limits, Lemma \ref{limits-lemma-reach-point-closure-Noetherian}.
\end{proof}

\noindent
First we state the result concerning
separation. We will often use solid commutative diagrams of morphisms of
algebraic spaces over a base scheme $S$ having the following shape
\begin{equation}
\label{equation-valuative}
\vcenter{
\xymatrix{
\Spec(K) \ar[r] \ar[d] & X \ar[d] \\
\Spec(A) \ar[r] \ar@{-->}[ru] & Y
}
}
\end{equation}
with $A$ a valuation ring and $K$ its field of fractions.

\begin{lemma}
\label{lemma-Noetherian-dvr-valuative-separation}
Let $S$ be a scheme. Let $f : X \to Y$ be a morphism of algebraic spaces
over $S$. Assume $f$ is quasi-separated and locally of finite type and
$Y$ is locally Noetherian. The following are equivalent:
\begin{enumerate}
\item The morphism $f$ is separated.
\item For any diagram (\ref{equation-valuative}) there is at most
one dotted arrow.
\item For all diagrams (\ref{equation-valuative}) with $A$ a discrete
valuation ring there is at most one dotted arrow.
\item For all diagrams (\ref{equation-valuative}) where $A$ is a discrete
valuation ring and where the image of $\Spec(K) \to X$ is a point of
codimension $0$ on $X$ there is at most one dotted arrow.
\end{enumerate}
\end{lemma}

\begin{proof}
We have (1) $\Rightarrow$ (2) by
Morphisms of Spaces, Lemma
\ref{spaces-morphisms-lemma-separated-implies-valuative}.
The implications (2) $\Rightarrow$ (3) and (3) $\Rightarrow$ (4)
are immediate. It remains to show (4) implies (1).

\medskip\noindent
Assume (4). We have to show that the diagonal
$\Delta : X \to X \times_Y X$ is a closed
immersion. We already know $\Delta$ is representable, separated,
a monomorphism, and locally of finite type, see Morphisms of Spaces, Lemma
\ref{spaces-morphisms-lemma-properties-diagonal}.
Choose an affine scheme $U$ and an \'etale morphism
$U \to X \times_Y X$. Set $V = X \times_{\Delta, X \times_Y X} U$.
It suffices to show that $V \to U$ is a closed immersion
(Morphisms of Spaces, Lemma
\ref{spaces-morphisms-lemma-closed-immersion-local}).
Since $X \times_Y X$ is locally of finite type over $Y$ we see that
$U$ is Noetherian (use Morphisms of Spaces, Lemmas
\ref{spaces-morphisms-lemma-composition-finite-type},
\ref{spaces-morphisms-lemma-base-change-finite-type}, and
\ref{spaces-morphisms-lemma-locally-finite-type-locally-noetherian}).
Note that $V$ is a scheme as $\Delta$ is representable.
Also, $V$ is quasi-compact because $f$ is quasi-separated.
Hence $V \to U$ is separated and of finite type. Consider a commutative diagram
$$
\xymatrix{
\Spec(K) \ar[r] \ar[d] & V \ar[d] \\
\Spec(A) \ar[r] \ar@{-->}[ru] & U
}
$$
of morphisms of schemes where $A$ is a discrete valuation ring
with fraction field $K$ and where $K$ is the residue field of
a generic point of the Noetherian scheme $V$. Since $V \to X$
is \'etale (as a base change of the \'etale morphism $U \to X \times_Y X$)
we see that the image of $\Spec(K) \to V \to X$ is a point of
codimension $0$, see Properties of Spaces, Section
\ref{spaces-properties-section-dimension-local-ring}.
We can interpret the composition $\Spec(A) \to U \to X \times_Y X$
as a pair of morphisms $a, b : \Spec(A) \to X$ agreeing as morphisms
into $Y$ and equal when restricted to $\Spec(K)$ and that this
restriction maps to a point of codimension $0$. Hence our assumption
(4) guarantees $a = b$ and we find the dotted arrow in the diagram.
By Limits, Lemma \ref{limits-lemma-Noetherian-dvr-valuative-proper}
we conclude that $V \to U$ is proper. In other words, $\Delta$ is proper.
Since $\Delta$ is a monomorphism, we find that $\Delta$ is a
closed immersion (\'Etale Morphisms, Lemma
\ref{etale-lemma-characterize-closed-immersion}) as desired.
\end{proof}

\begin{lemma}
\label{lemma-Noetherian-dvr-valuative-proper}
Let $S$ be a scheme. Let $f : X \to Y$ be a morphism of algebraic spaces
over $S$. Assume $f$ is quasi-separated and of finite type and $Y$
is locally Noetherian. The following are equivalent:
\begin{enumerate}
\item $f$ is proper,
\item $f$ satisfies the valuative criterion, see
Morphisms of Spaces, Definition
\ref{spaces-morphisms-definition-valuative-criterion},
\item for any diagram (\ref{equation-valuative}) there exists exactly
one dotted arrow,
\item for all diagrams (\ref{equation-valuative}) with $A$ a discrete
valuation ring there exists exactly one dotted arrow, and
\item for all diagrams (\ref{equation-valuative}) where $A$ is a discrete
valuation ring and where the image of $\Spec(K) \to X$ is a point of
codimension $0$ on $X$ there exists exactly one dotted arrow\footnote{There is
a sharper formulation where in the existence part one only requires
the dotted arrow exists after an extension of discrete valuation rings.}.
\end{enumerate}
\end{lemma}

\begin{proof}
We have (1) $\Leftrightarrow$ (2) $\Leftrightarrow$ (3) by 
Morphisms of Spaces, Lemma \ref{spaces-morphisms-lemma-characterize-proper}.
It is clear that (3) $\Rightarrow$ (4) $\Rightarrow$ (5).
To finish the proof we will now show (5) implies (1).

\medskip\noindent
Assume (5). By Lemma \ref{lemma-Noetherian-dvr-valuative-separation}
we see that $f$ is separated. To finish the proof it suffices to show
that $f$ is universally closed. Let $V \to Y$ be an \'etale morphism
where $V$ is an affine scheme. It suffices to show that the base
change $V \times_Y X \to V$ is universally closed, see
Morphisms of Spaces, Lemma
\ref{spaces-morphisms-lemma-universally-closed-local}.
Let
$$
\xymatrix{
\Spec(K) \ar[r] \ar[d] & V \times_Y X \ar[d] \ar[r] & X \ar[d] \\
\Spec(A) \ar[r] \ar@{-->}[ru] \ar@{..>}[rru] & V \ar[r] & Y
}
$$
of algebraic spaces over $S$
be a commutative diagram where $A$ is a discrete valuation ring
with fraction field $K$ and where $\Spec(K) \to V \times_Y X$
maps to a point of codimension $0$ of the algebraic space
$V \times_Y X$. Since $V \times_Y X \to X$ is \'etale it follows
that the image of $\Spec(K) \to X$ is a point of codimension $0$
of $X$. Thus by (5) we obtain the longer of the two dotted
arrows fitting into the diagram. Then of course we obtain the
shorter one as well. It follows that our assumptions hold for
the morphism $V \times_Y X \to V$ and we reduce to the case discussed
in the next paragraph.

\medskip\noindent
Aassume $Y$ is a Noetherian affine scheme. In this case $X$ is a
separated Noetherian algebraic space (we already know $f$ is separated)
of finite type over $Y$. (In particular, the algebraic space $X$
has a dense open subspace which is a scheme by Properties of Spaces, Proposition
\ref{spaces-properties-proposition-locally-quasi-separated-open-dense-scheme}
although strictly speaking we will not need this.)
Choose a quasi-projective scheme $X'$ over $Y$ and a proper surjective
morphism $X' \to X$ as in the weak form of Chow's lemma
(Cohomology of Spaces, Lemma \ref{spaces-cohomology-lemma-weak-chow}).
We may replace $X'$ by the disjoint union of the irreducible components
which dominate an irreducible component of $X$; details omitted.
In particular, we may assume that generic points of the scheme $X'$ map to
points of codimension $0$ of $X$ (in this case these are exactly
the generic points of $X$). We claim that $X' \to Y$ is proper.
The claim implies $X$ is proper over $Y$ by
Morphisms of Spaces, Lemma \ref{spaces-morphisms-lemma-image-proper-is-proper}.
To prove this, according to
Limits, Lemma \ref{limits-lemma-Noetherian-dvr-valuative-proper}
it suffices to prove that in every solid commutative diagram
$$
\xymatrix{
\Spec(K) \ar[r] \ar[d] & X' \ar[r] & X \ar[d] \\
\Spec(A) \ar[rr] \ar@{-->}[ru]^a \ar@{-->}[rru]_b & & Y
}
$$
where $A$ is a dvr with fraction field $K$ and where $K$ is the residue
field of a generic point of $X'$ we can find the
dotted arrow $a$ (we already know uniqueness as $X'$ is separated).
By assumption (5) we can find the dotted arrow $b$.
Then the morphism $X' \times_{X, b} \Spec(A) \to \Spec(A)$
is a proper morphism of schemes and by the valuative criterion
for morphisms of schemes we can lift $b$ to the desired morphism $a$.
\end{proof}

\begin{lemma}
\label{lemma-check-universally-closed-Noetherian}
Let $S$ be a scheme. Let $f : X \to Y$ be a morphism of algebraic
spaces over $S$. Assume $Y$ is locally Noetherian and $f$ is
of finite type. Then the following are equivalent
\begin{enumerate}
\item $f$ is universally closed,
\item $f$ satisfies the existence part of the valuative criterion,
\item there exists a scheme $V$ and a surjective \'etale morphism
$V \to Y$ such that
$|\mathbf{A}^n \times X \times_Y V| \to |\mathbf{A}^n \times V|$ is closed
for all $n \geq 0$,
\item for all diagrams (\ref{equation-valuative}) with $A$ a discrete
valuation ring there there exists a finite separable extension $K'/K$
of fields, a discrete valuation ring $A' \subset K'$ dominating $A$, and
a morphism $\Spec(A') \to X$ such that the following diagram commutes
$$
\xymatrix{
\Spec(K') \ar[r] \ar[d] & \Spec(K) \ar[r] & X \ar[d] \\
\Spec(A') \ar[r] \ar[rru] & \Spec(A) \ar[r] & Y
}
$$
\item for all diagrams (\ref{equation-valuative}) with $A$ a discrete
valuation ring there there exists a field extension $K'/K$,
a valuation ring $A' \subset K'$ dominating $A$, and
a morphism $\Spec(A') \to X$ such that the following diagram commutes
$$
\xymatrix{
\Spec(K') \ar[r] \ar[d] & \Spec(K) \ar[r] & X \ar[d] \\
\Spec(A') \ar[r] \ar[rru] & \Spec(A) \ar[r] & Y
}
$$
\end{enumerate}
\end{lemma}

\begin{proof}
Parts (1), (2), and (3) are equivalent by
Lemma \ref{lemma-test-universally-closed} and
Morphisms of Spaces, Lemma
\ref{spaces-morphisms-lemma-quasi-compact-existence-universally-closed}.
These equivalent conditions imply part (4) as
Morphisms of Spaces, Lemma \ref{spaces-morphisms-lemma-finite-separable-enough}
tells us that we may always choose $K'/K$ finite separable in
the existence part of the valuative criterion and this
automatically forces $A'$ to be a discrete valuation ring by Krull-Akizuki
(Algebra, Lemma \ref{algebra-lemma-krull-akizuki}).
The implication (4) $\Rightarrow$ (5) is immediate.
In the rest of the proof we show that (5) implies (1).

\medskip\noindent
Assume (5). Chose an affine scheme $V$ and an \'etale morphism $V \to Y$.
It suffices to show that the base change of $f$ to $V$ is universally closed,
see Morphisms of Spaces, Lemma
\ref{spaces-morphisms-lemma-universally-closed-local}.
Exactly as in the proof of Lemma \ref{lemma-Noetherian-dvr-valuative-proper}
we see that assumption (5) is inherited by this base change; details
omitted. This reduces us to the case discussed in the next paragraph.

\medskip\noindent
Assume $Y$ is a Noetherian affine scheme and we have (5).
To prove that $f$ is universally closed it suffices to show that
$|X \times \mathbf{A}^n| \to |Y \times \mathbf{A}^n|$ is closed
for all $n$ (by the discussion above). Since assumption (5)
is inherited by the product morphism
$X \times \mathbf{A}^n \to Y \times \mathbf{A}^n$ (details omitted)
we reduce to proving that $|X| \to |Y|$ is closed.

\medskip\noindent
Assume $Y$ is a Noetherian affine scheme and we have (5).
Let $T \subset |X|$ be a closed subset. We have to show that
the image of $T$ in $|Y|$ is closed. We may replace $X$
by the reduced induced closed subspace structure on $T$; we
omit the verification that property (5) is preserved by
this replacement. Thus we reduce to proving that the image
of $|X| \to |Y|$ is closed.

\medskip\noindent
Let $y \in |Y|$ be a point in the closure of the image of
$|X| \to |Y|$. By Lemma \ref{lemma-reach-point-closure-Noetherian}
we may choose a commutative diagram
$$
\xymatrix{
\Spec(K) \ar[r] \ar[d] & X \ar[d]^f \\
\Spec(A) \ar[r] & Y
}
$$
where $A$ is a discrete valuation ring and $K$ is its field of fractions
mapping the closed point of $\Spec(A)$ to $y$. It follows immediately
from property (5) that $y$ is in the image of
$|X| \to |Y|$ and the proof is complete.
\end{proof}


















\section{Refined Noetherian valuative criteria}
\label{section-refined-valuative-criteria}

\noindent
This section is the analogue of
Limits, Section \ref{limits-section-refined-valuative-criteria}.
One usually does not have to consider all possible diagrams with
valuation rings when checking valuative criteria.

\begin{lemma}
\label{lemma-refined-valuative-criterion-proper}
Let $S$ be a scheme. Let $f : X \to Y$ and $h : U \to X$ be
morphisms of algebraic spaces over $S$. Assume that $Y$ is
locally Noetherian, that $f$ and $h$ are of finite type,
that $f$ is separated, and that the image of $|h| : |U| \to |X|$
is dense in $|X|$. If given any commutative solid diagram
$$
\xymatrix{
\Spec(K) \ar[r] \ar[d] & U \ar[r]^h & X \ar[d]^f \\
\Spec(A) \ar[rr] \ar@{-->}[rru] & & Y
}
$$
where $A$ is a discrete valuation ring with field of fractions $K$, there
exists a dotted arrow making the diagram commute, then $f$ is proper.
\end{lemma}

\begin{proof}
It suffices to prove that $f$ is universally closed.
Let $V \to Y$ be an \'etale morphism where $V$ is an affine scheme.
By Morphisms of Spaces, Lemma
\ref{spaces-morphisms-lemma-universally-closed-local}
it suffices to prove that the base change $X \times_Y V \to V$ is
universally closed. By Properties of Spaces, Lemma
\ref{spaces-properties-lemma-points-cartesian}
the image $I$ of $|U \times_Y V| \to |X \times_Y V|$
is the inverse image of the image of $|h|$. Since
$|X \times_Y V| \to |X|$ is open
(Properties of Spaces, Lemma \ref{spaces-properties-lemma-etale-open})
we conclude that $I$ is dense in $|X \times_Y V|$.
Therefore the assumptions of the lemma are
satisfied for the morphisms $U \times_Y V \to X \times_Y V \to V$.
Hence we may assume $Y$ is an affine scheme.

\medskip\noindent
Assume $Y$ is an affine scheme. Then $U$ is quasi-compact. Choose
an affine scheme and a surjective \'etale morphism $W \to U$.
Then we may and do replace $U$ by $W$ and assume that $U$ is affine.
By the weak version of Chow's lemma
(Cohomology of Spaces, Lemma \ref{spaces-cohomology-lemma-weak-chow})
we can choose a surjective proper morphism $X' \to X$
where $X'$ is a scheme. Then $U' = X' \times_X U$ is a scheme
and $U' \to X'$ is of finite type. We may replace $X'$ by
the scheme theoretic image of $h' : U' \to X'$ and hence $h'(U')$
is dense in $X'$. We claim that for every diagram
$$
\xymatrix{
\Spec(K) \ar[r] \ar[d] & U' \ar[r]^h & X' \ar[d]^{f'} \\
\Spec(A) \ar[rr] \ar@{-->}[rru] & & Y
}
$$
where $A$ is a discrete valuation ring with field of fractions $K$, there
exists a dotted arrow making the diagram commute. Namely, we first get an
arrow $\Spec(A) \to X$ by the assumption of the lemma and then we lift this
to an arrow $\Spec(A) \to X'$ using the valuative criterion for properness
(Morphisms of Spaces, Lemma \ref{spaces-morphisms-lemma-characterize-proper}).
The morphism $X' \to Y$ is separated
as a composition of a proper and a separated morphism.
Thus by the case of schemes the morphism $X' \to Y$ is proper
(Limits, Lemma
\ref{limits-lemma-refined-valuative-criterion-proper}).
By Morphisms of Spaces, Lemma
\ref{spaces-morphisms-lemma-image-proper-is-proper}
we conclude that $X \to Y$ is proper.
\end{proof}

\begin{lemma}
\label{lemma-refined-valuative-criterion-separated}
Let $S$ be a scheme. Let $f : X \to Y$ and $h : U \to X$ be
morphisms of algebraic spaces over $S$. Assume that $Y$ is
locally Noetherian, that $f$ is locally of finite type and quasi-separated,
that $h$ is of finite type, and that the image of $|h| : |U| \to |X|$
is dense in $|X|$.
If given any commutative solid diagram
$$
\xymatrix{
\Spec(K) \ar[r] \ar[d] & U \ar[r]^h & X \ar[d]^f \\
\Spec(A) \ar[rr] \ar@{-->}[rru] & & Y
}
$$
where $A$ is a discrete valuation ring with field of fractions $K$, there
exists at most one dotted arrow making the diagram commute, then $f$ is
separated.
\end{lemma}

\begin{proof}
We will apply Lemma \ref{lemma-refined-valuative-criterion-proper}
to the morphisms $U \to X$ and $\Delta : X \to X \times_Y X$.
We check the conditions. Observe that $\Delta$ is quasi-compact because
$f$ is quasi-separated. Of course $\Delta$ is locally of finite type and
separated (true for any diagonal morphism).
Finally, suppose given a commutative solid diagram
$$
\xymatrix{
\Spec(K) \ar[r] \ar[d] & U \ar[r]^h & X \ar[d]^\Delta \\
\Spec(A) \ar[rr]^{(a, b)} \ar@{-->}[rru] & & X \times_Y X
}
$$
where $A$ is a discrete valuation ring with field of fractions $K$.
Then $a$ and $b$ give two dotted arrows in the diagram of the lemma
and have to be equal. Hence as dotted arrow we can use $a = b$
which gives existence. This finishes the proof.
\end{proof}

\begin{lemma}
\label{lemma-refined-valuative-criterion-universally-closed}
Let $S$ be a scheme.
Let $f : X \to Y$ and $h : U \to X$ be morphisms of algebraic spaces over $S$.
Assume that $Y$ is locally Noetherian, that $f$ and $h$ are of finite type,
that $f$ is quasi-separated, and
that $h(U)$ is dense in $X$. If given any commutative solid diagram
$$
\xymatrix{
\Spec(K) \ar[r] \ar[d] & U \ar[r]^h & X \ar[d]^f \\
\Spec(A) \ar[rr] \ar@{-->}[rru] & & Y
}
$$
where $A$ is a discrete valuation ring with field of fractions $K$, there
exists a unique dotted arrow making the diagram commute, then $f$ is proper.
\end{lemma}

\begin{proof}
Combine Lemmas \ref{lemma-refined-valuative-criterion-separated} and
\ref{lemma-refined-valuative-criterion-proper}.
\end{proof}







\section{Descending finite type spaces}
\label{section-finite-type-quasi-separated}

\noindent
This section continues the theme of
Section \ref{section-finite-type-closed-in-finite-presentation}
in the spirit of the results discussed in
Section \ref{section-descending-relative}.
It is also the analogue of
Limits, Section \ref{limits-section-finite-type-quasi-separated}
for algebraic spaces.

\begin{situation}
\label{situation-limit-noetherian}
Let $S$ be a scheme, for example $\Spec(\mathbf{Z})$.
Let $B = \lim_{i \in I} B_i$ be the limit
of a directed inverse system of Noetherian spaces over $S$
with affine transition morphisms
$B_{i'} \to B_i$ for $i' \geq i$.
\end{situation}

\begin{lemma}
\label{lemma-good-diagram}
In Situation \ref{situation-limit-noetherian}.
Let $X \to B$ be a quasi-separated and finite type
morphism of algebraic spaces.
Then there exists an $i \in I$ and a diagram
\begin{equation}
\label{equation-good-diagram}
\vcenter{
\xymatrix{
X \ar[r] \ar[d] & W \ar[d] \\
B \ar[r] & B_i
}
}
\end{equation}
such that $W \to B_i$ is of finite type and such that
the induced morphism $X \to B \times_{B_i} W$ is a closed
immersion.
\end{lemma}

\begin{proof}
By Lemma \ref{lemma-finite-type-closed-in-finite-presentation}
we can find a closed immersion $X \to X'$
over $B$ where $X'$ is an algebraic space of finite presentation over $B$.
By Lemma \ref{lemma-descend-finite-presentation}
we can find an $i$ and a morphism of finite presentation
$X'_i \to B_i$ whose pull back is $X'$. Set $W = X'_i$.
\end{proof}

\begin{lemma}
\label{lemma-limit-from-good-diagram}
In Situation \ref{situation-limit-noetherian}.
Let $X \to B$ be a quasi-separated and finite type morphism
of algebraic spaces. Given $i \in I$ and a diagram
$$
\vcenter{
\xymatrix{
X \ar[r] \ar[d] & W \ar[d] \\
B \ar[r] & B_i
}
}
$$
as in (\ref{equation-good-diagram}) for $i' \geq i$ let
$X_{i'}$ be the scheme theoretic image of $X \to B_{i'} \times_{B_i} W$.
Then $X = \lim_{i' \geq i} X_{i'}$.
\end{lemma}

\begin{proof}
Since $X$ is quasi-compact and quasi-separated formation of the
scheme theoretic image of $X \to B_{i'} \times_{B_i} W$
commutes with \'etale localization
(Morphisms of Spaces, Lemma
\ref{spaces-morphisms-lemma-quasi-compact-scheme-theoretic-image}).
Hence we may and do assume $W$ is affine and maps into an affine
$U_i$ \'etale over $B_i$. Then
$$
B_{i'} \times_{B_i} W =
B_{i'} \times_{B_i} U_i \times_{U_i} W =
U_{i'} \times_{U_i} W
$$
where $U_{i'} = B_{i'} \times_{B_i} U_i$ is affine as the transition
morphisms are affine. Thus the lemma follows from the case of schemes
which is
Limits, Lemma \ref{limits-lemma-limit-from-good-diagram}.
\end{proof}

\begin{lemma}
\label{lemma-morphism-good-diagram}
In Situation \ref{situation-limit-noetherian}.
Let $f : X \to Y$ be a morphism of algebraic spaces quasi-separated
and of finite type over $B$. Let
$$
\vcenter{
\xymatrix{
X \ar[r] \ar[d] & W \ar[d] \\
B \ar[r] & B_{i_1}
}
}
\quad\text{and}\quad
\vcenter{
\xymatrix{
Y \ar[r] \ar[d] & V \ar[d] \\
B \ar[r] & B_{i_2}
}
}
$$
be diagrams as in (\ref{equation-good-diagram}). Let
$X = \lim_{i \geq i_1} X_i$ and
$Y = \lim_{i \geq i_2} Y_i$ be the corresponding
limit descriptions as in Lemma \ref{lemma-limit-from-good-diagram}.
Then there exists an $i_0 \geq \max(i_1, i_2)$ and a morphism
$$
(f_i)_{i \geq i_0} : (X_i)_{i \geq i_0} \to (Y_i)_{i \geq i_0}
$$
of inverse systems over $(B_i)_{i \geq i_0}$ such that
such that $f = \lim_{i \geq i_0} f_i$.
If $(g_i)_{i \geq i_0} : (X_i)_{i \geq i_0} \to (Y_i)_{i \geq i_0}$
is a second morphism of inverse systems over $(B_i)_{i \geq i_0}$ such that
such that $f = \lim_{i \geq i_0} g_i$
then $f_i = g_i$ for all $i \gg i_0$.
\end{lemma}

\begin{proof}
Since $V \to B_{i_2}$ is of finite presentation and
$X = \lim_{i \geq i_1} X_i$ we can appeal to Proposition
\ref{proposition-characterize-locally-finite-presentation}
as improved by Lemma \ref{lemma-better-characterize-relative-limit-preserving}
to find an $i_0 \geq \max(i_1, i_2)$ and a morphism $h : X_{i_0} \to V$
over $B_{i_2}$ such that $X \to X_{i_0} \to V$ is equal to $X \to Y \to V$.
For $i \geq i_0$ we get a commutative solid diagram
$$
\xymatrix{
X \ar[d] \ar[r] &
X_i \ar[r] \ar@{..>}[d] \ar@/_2pc/[dd] |!{[d];[ld]}\hole &
X_{i_0} \ar[d]^h \\
Y \ar[r] \ar[d] & Y_i \ar[r] \ar[d] & V \ar[d] \\
B \ar[r] & B_i \ar[r] & B_{i_0}
}
$$
Since $X \to X_i$ has scheme theoretically dense image
and since $Y_i$ is the scheme theoretic image of
$Y \to B_i \times_{B_{i_2}} V$
we find that the morphism $X_i \to B_i \times_{B_{i_2}} V$
induced by the diagram factors through $Y_i$
(Morphisms of Spaces, Lemma \ref{spaces-morphisms-lemma-factor-factor}).
This proves existence.

\medskip\noindent
Uniqueness. Let $E_i \to X_i$ be the equalizer of $f_i$ and $g_i$
for $i \geq i_0$. We have
$E_i = Y_i \times_{\Delta, Y_i \times_{B_i} Y_i, (f_i, g_i)} X_i$.
Hence $E_i \to X_i$ is a monomorphism of finite presentation as a base
change of the diagonal of $Y_i$ over $B_i$, see
Morphisms of Spaces, Lemmas \ref{spaces-morphisms-lemma-properties-diagonal} and
\ref{spaces-morphisms-lemma-diagonal-morphism-finite-type}.
Since $X_i$ is a closed subspace of $B_i \times_{B_{i_0}} X_{i_0}$
and similarly for $Y_i$ we see that
$$
E_i =
X_i \times_{(B_i \times_{B_{i_0}} X_{i_0})} (B_i \times_{B_{i_0}} E_{i_0}) =
X_i \times_{X_{i_0}} E_{i_0}
$$
Similarly, we have $X = X \times_{X_{i_0}} E_{i_0}$. Hence we conclude
that $E_i = X_i$ for $i$ large enough by Lemma \ref{lemma-descend-isomorphism}.
\end{proof}

\begin{remark}
\label{remark-finite-type-gives-well-defined-system}
In Situation \ref{situation-limit-noetherian}
Lemmas \ref{lemma-good-diagram}, \ref{lemma-limit-from-good-diagram}, and
\ref{lemma-morphism-good-diagram}
tell us that the category of algebraic spaces quasi-separated and
of finite type over $B$ is equivalent to certain types of
inverse systems of algebraic spaces over $(B_i)_{i \in I}$, namely
the ones produced by applying Lemma \ref{lemma-limit-from-good-diagram}
to a diagram of the form (\ref{equation-good-diagram}).
For example, given $X \to B$ finite type and quasi-separated
if we choose two different diagrams $X \to V_1 \to B_{i_1}$
and $X \to V_2 \to B_{i_2}$ as in (\ref{equation-good-diagram}), then
applying Lemma \ref{lemma-morphism-good-diagram} to $\text{id}_X$
(in two directions)
we see that the corresponding limit descriptions of
$X$ are canonically isomorphic (up to shrinking the
directed set $I$). And so on and so forth.
\end{remark}

\begin{lemma}
\label{lemma-morphism-good-diagram-flat}
Notation and assumptions as in Lemma \ref{lemma-morphism-good-diagram}.
If $f$ is flat and of finite presentation, then
there exists an $i_3 > i_0$ such that for $i \geq i_3$ we have
$f_i$ is flat, $X_i = Y_i \times_{Y_{i_3}} X_{i_3}$, and
$X = Y \times_{Y_{i_3}} X_{i_3}$.
\end{lemma}

\begin{proof}
By Lemma \ref{lemma-descend-finite-presentation}
we can choose an $i \geq i_2$ and a morphism
$U \to Y_i$ of finite presentation such that $X = Y \times_{Y_i} U$
(this is where we use that $f$ is of finite presentation).
After increasing $i$ we may assume that $U \to Y_i$ is flat, see
Lemma \ref{lemma-descend-flat}.
As discussed in Remark \ref{remark-finite-type-gives-well-defined-system}
we may and do replace the initial diagram used to define the system
$(X_i)_{i \geq i_1}$ by the system corresponding to
$X \to U \to B_i$. Thus $X_{i'}$ for $i' \geq i$ is defined as
the scheme theoretic image of $X \to B_{i'} \times_{B_i} U$.

\medskip\noindent
Because $U \to Y_i$ is flat (this is where we use that $f$ is flat),
because $X = Y \times_{Y_i} U$, and
because the scheme theoretic image of $Y \to Y_i$ is $Y_i$,
we see that the scheme theoretic image of $X \to U$ is $U$
(Morphisms of Spaces, Lemma
\ref{spaces-morphisms-lemma-flat-base-change-scheme-theoretic-image}).
Observe that $Y_{i'} \to B_{i'} \times_{B_i} Y_i$ is a closed
immersion for $i' \geq i$ by construction of the system of $Y_j$.
Then the same argument as above shows that the scheme theoretic image
of $X \to B_{i'} \times_{B_i} U$
is equal to the closed subspace $Y_{i'} \times_{Y_i} U$.
Thus we see that $X_{i'} = Y_{i'} \times_{Y_i} U$ for all $i' \geq i$
and hence the lemma holds with $i_3 = i$.
\end{proof}

\begin{lemma}
\label{lemma-morphism-good-diagram-smooth}
Notation and assumptions as in Lemma \ref{lemma-morphism-good-diagram}.
If $f$ is smooth, then there exists an $i_3 > i_0$ such that for
$i \geq i_3$ we have $f_i$ is smooth.
\end{lemma}

\begin{proof}
Combine Lemmas \ref{lemma-morphism-good-diagram-flat} and
\ref{lemma-descend-smooth}.
\end{proof}

\begin{lemma}
\label{lemma-morphism-good-diagram-proper}
Notation and assumptions as in Lemma \ref{lemma-morphism-good-diagram}.
If $f$ is proper, then there exists an $i_3 \geq i_0$ such that for
$i \geq i_3$ we have $f_i$ is proper.
\end{lemma}

\begin{proof}
By the discussion in 
Remark \ref{remark-finite-type-gives-well-defined-system}
the choice of $i_1$ and $W$ fitting into a diagram as in
(\ref{equation-good-diagram}) is immaterial for the truth of
the lemma. Thus we choose $W$ as follows.
First we choose a closed immersion $X \to X'$
with $X' \to Y$ proper and of finite presentation, see
Lemma \ref{lemma-proper-limit-of-proper-finite-presentation}.
Then we choose an $i_3 \geq i_2$ and a proper morphism $W \to Y_{i_3}$
such that $X' = Y \times_{Y_{i_3}} W$. This is possible because
$Y = \lim_{i \geq i_2} Y_i$ and
Lemmas \ref{lemma-relative-approximation} and \ref{lemma-eventually-proper}.
With this choice of $W$ it is immediate from the construction that
for $i \geq i_3$ the algebraic space $X_i$ is a closed subspace of
$Y_i \times_{Y_{i_3}} W \subset B_i \times_{B_{i_3}} W$
and hence proper over $Y_i$.
\end{proof}

\begin{lemma}
\label{lemma-good-diagram-fibre-product}
In Situation \ref{situation-limit-noetherian} suppose that we have a
cartesian diagram
$$
\xymatrix{
X^1 \ar[r]_p \ar[d]_q & X^3 \ar[d]^a \\
X^2 \ar[r]^b & X^4
}
$$
of algebraic spaces quasi-separated and of finite type over $B$.
For each $j = 1, 2, 3, 4$ choose $i_j \in I$ and a diagram
$$
\xymatrix{
X^j \ar[r] \ar[d] & W^j \ar[d] \\
B \ar[r] & B_{i_j}
}
$$
as in (\ref{equation-good-diagram}). Let
$X^j = \lim_{i \geq i_j} X^j_i$ be the corresponding limit descriptions
as in Lemma \ref{lemma-morphism-good-diagram}.
Let $(a_i)_{i \geq i_5}$, $(b_i)_{i \geq i_6}$, $(p_i)_{i \geq i_7}$, and
$(q_i)_{i \geq i_8}$ be the corresponding morphisms of inverse systems
contructed in Lemma \ref{lemma-morphism-good-diagram}. Then there exists an
$i_9 \geq \max(i_5, i_6, i_7, i_8)$ such that for $i \geq i_9$ we have
$a_i \circ p_i = b_i \circ q_i$ and such that
$$
(q_i, p_i) : X^1_i \longrightarrow X^2_i \times_{b_i, X^4_i, a_i} X^3_i
$$
is a closed immersion.
If $a$ and $b$ are flat and of finite presentation, then there exists an
$i_{10} \geq \max(i_5, i_6, i_7, i_8, i_9)$ such that for $i \geq i_{10}$
the last displayed morphism is an isomorphism.
\end{lemma}

\begin{proof}
According to the discussion in
Remark \ref{remark-finite-type-gives-well-defined-system}
the choice of $W^1$ fitting into a diagram as in
(\ref{equation-good-diagram}) is immaterial for the truth of
the lemma. Thus we may choose $W^1 = W^2 \times_{W^4} W^3$.
Then it is immediate from the construction of $X^1_i$ that 
$a_i \circ p_i = b_i \circ q_i$ and that
$$
(q_i, p_i) : X^1_i \longrightarrow X^2_i \times_{b_i, X^4_i, a_i} X^3_i
$$
is a closed immersion.

\medskip\noindent
If $a$ and $b$ are flat and of finite presentation, then so are
$p$ and $q$ as base changes of $a$ and $b$. Thus we can apply
Lemma \ref{lemma-morphism-good-diagram-flat}
to each of $a$, $b$, $p$, $q$, and $a \circ p = b \circ q$.
It follows that there exists an $i_9 \in I$ such that
$$
(q_i, p_i) : X^1_i \to X^2_i \times_{X^4_i} X^3_i
$$
is the base change of $(q_{i_9}, p_{i_9})$ by the morphism
by the morphism $X^4_i \to X^4_{i_9}$ for all $i \geq i_9$.
We conclude that $(q_i, p_i)$ is an isomorphism for all sufficiently
large $i$ by Lemma \ref{lemma-descend-isomorphism}.
\end{proof}












\begin{multicols}{2}[\section{Other chapters}]
\noindent
Preliminaries
\begin{enumerate}
\item \hyperref[introduction-section-phantom]{Introduction}
\item \hyperref[conventions-section-phantom]{Conventions}
\item \hyperref[sets-section-phantom]{Set Theory}
\item \hyperref[categories-section-phantom]{Categories}
\item \hyperref[topology-section-phantom]{Topology}
\item \hyperref[sheaves-section-phantom]{Sheaves on Spaces}
\item \hyperref[sites-section-phantom]{Sites and Sheaves}
\item \hyperref[stacks-section-phantom]{Stacks}
\item \hyperref[fields-section-phantom]{Fields}
\item \hyperref[algebra-section-phantom]{Commutative Algebra}
\item \hyperref[brauer-section-phantom]{Brauer Groups}
\item \hyperref[homology-section-phantom]{Homological Algebra}
\item \hyperref[derived-section-phantom]{Derived Categories}
\item \hyperref[simplicial-section-phantom]{Simplicial Methods}
\item \hyperref[more-algebra-section-phantom]{More on Algebra}
\item \hyperref[smoothing-section-phantom]{Smoothing Ring Maps}
\item \hyperref[modules-section-phantom]{Sheaves of Modules}
\item \hyperref[sites-modules-section-phantom]{Modules on Sites}
\item \hyperref[injectives-section-phantom]{Injectives}
\item \hyperref[cohomology-section-phantom]{Cohomology of Sheaves}
\item \hyperref[sites-cohomology-section-phantom]{Cohomology on Sites}
\item \hyperref[dga-section-phantom]{Differential Graded Algebra}
\item \hyperref[dpa-section-phantom]{Divided Power Algebra}
\item \hyperref[sdga-section-phantom]{Differential Graded Sheaves}
\item \hyperref[hypercovering-section-phantom]{Hypercoverings}
\end{enumerate}
Schemes
\begin{enumerate}
\setcounter{enumi}{25}
\item \hyperref[schemes-section-phantom]{Schemes}
\item \hyperref[constructions-section-phantom]{Constructions of Schemes}
\item \hyperref[properties-section-phantom]{Properties of Schemes}
\item \hyperref[morphisms-section-phantom]{Morphisms of Schemes}
\item \hyperref[coherent-section-phantom]{Cohomology of Schemes}
\item \hyperref[divisors-section-phantom]{Divisors}
\item \hyperref[limits-section-phantom]{Limits of Schemes}
\item \hyperref[varieties-section-phantom]{Varieties}
\item \hyperref[topologies-section-phantom]{Topologies on Schemes}
\item \hyperref[descent-section-phantom]{Descent}
\item \hyperref[perfect-section-phantom]{Derived Categories of Schemes}
\item \hyperref[more-morphisms-section-phantom]{More on Morphisms}
\item \hyperref[flat-section-phantom]{More on Flatness}
\item \hyperref[groupoids-section-phantom]{Groupoid Schemes}
\item \hyperref[more-groupoids-section-phantom]{More on Groupoid Schemes}
\item \hyperref[etale-section-phantom]{\'Etale Morphisms of Schemes}
\end{enumerate}
Topics in Scheme Theory
\begin{enumerate}
\setcounter{enumi}{41}
\item \hyperref[chow-section-phantom]{Chow Homology}
\item \hyperref[intersection-section-phantom]{Intersection Theory}
\item \hyperref[pic-section-phantom]{Picard Schemes of Curves}
\item \hyperref[weil-section-phantom]{Weil Cohomology Theories}
\item \hyperref[adequate-section-phantom]{Adequate Modules}
\item \hyperref[dualizing-section-phantom]{Dualizing Complexes}
\item \hyperref[duality-section-phantom]{Duality for Schemes}
\item \hyperref[discriminant-section-phantom]{Discriminants and Differents}
\item \hyperref[derham-section-phantom]{de Rham Cohomology}
\item \hyperref[local-cohomology-section-phantom]{Local Cohomology}
\item \hyperref[algebraization-section-phantom]{Algebraic and Formal Geometry}
\item \hyperref[curves-section-phantom]{Algebraic Curves}
\item \hyperref[resolve-section-phantom]{Resolution of Surfaces}
\item \hyperref[models-section-phantom]{Semistable Reduction}
\item \hyperref[functors-section-phantom]{Functors and Morphisms}
\item \hyperref[equiv-section-phantom]{Derived Categories of Varieties}
\item \hyperref[pione-section-phantom]{Fundamental Groups of Schemes}
\item \hyperref[etale-cohomology-section-phantom]{\'Etale Cohomology}
\item \hyperref[crystalline-section-phantom]{Crystalline Cohomology}
\item \hyperref[proetale-section-phantom]{Pro-\'etale Cohomology}
\item \hyperref[relative-cycles-section-phantom]{Relative Cycles}
\item \hyperref[more-etale-section-phantom]{More \'Etale Cohomology}
\item \hyperref[trace-section-phantom]{The Trace Formula}
\end{enumerate}
Algebraic Spaces
\begin{enumerate}
\setcounter{enumi}{64}
\item \hyperref[spaces-section-phantom]{Algebraic Spaces}
\item \hyperref[spaces-properties-section-phantom]{Properties of Algebraic Spaces}
\item \hyperref[spaces-morphisms-section-phantom]{Morphisms of Algebraic Spaces}
\item \hyperref[decent-spaces-section-phantom]{Decent Algebraic Spaces}
\item \hyperref[spaces-cohomology-section-phantom]{Cohomology of Algebraic Spaces}
\item \hyperref[spaces-limits-section-phantom]{Limits of Algebraic Spaces}
\item \hyperref[spaces-divisors-section-phantom]{Divisors on Algebraic Spaces}
\item \hyperref[spaces-over-fields-section-phantom]{Algebraic Spaces over Fields}
\item \hyperref[spaces-topologies-section-phantom]{Topologies on Algebraic Spaces}
\item \hyperref[spaces-descent-section-phantom]{Descent and Algebraic Spaces}
\item \hyperref[spaces-perfect-section-phantom]{Derived Categories of Spaces}
\item \hyperref[spaces-more-morphisms-section-phantom]{More on Morphisms of Spaces}
\item \hyperref[spaces-flat-section-phantom]{Flatness on Algebraic Spaces}
\item \hyperref[spaces-groupoids-section-phantom]{Groupoids in Algebraic Spaces}
\item \hyperref[spaces-more-groupoids-section-phantom]{More on Groupoids in Spaces}
\item \hyperref[bootstrap-section-phantom]{Bootstrap}
\item \hyperref[spaces-pushouts-section-phantom]{Pushouts of Algebraic Spaces}
\end{enumerate}
Topics in Geometry
\begin{enumerate}
\setcounter{enumi}{81}
\item \hyperref[spaces-chow-section-phantom]{Chow Groups of Spaces}
\item \hyperref[groupoids-quotients-section-phantom]{Quotients of Groupoids}
\item \hyperref[spaces-more-cohomology-section-phantom]{More on Cohomology of Spaces}
\item \hyperref[spaces-simplicial-section-phantom]{Simplicial Spaces}
\item \hyperref[spaces-duality-section-phantom]{Duality for Spaces}
\item \hyperref[formal-spaces-section-phantom]{Formal Algebraic Spaces}
\item \hyperref[restricted-section-phantom]{Algebraization of Formal Spaces}
\item \hyperref[spaces-resolve-section-phantom]{Resolution of Surfaces Revisited}
\end{enumerate}
Deformation Theory
\begin{enumerate}
\setcounter{enumi}{89}
\item \hyperref[formal-defos-section-phantom]{Formal Deformation Theory}
\item \hyperref[defos-section-phantom]{Deformation Theory}
\item \hyperref[cotangent-section-phantom]{The Cotangent Complex}
\item \hyperref[examples-defos-section-phantom]{Deformation Problems}
\end{enumerate}
Algebraic Stacks
\begin{enumerate}
\setcounter{enumi}{93}
\item \hyperref[algebraic-section-phantom]{Algebraic Stacks}
\item \hyperref[examples-stacks-section-phantom]{Examples of Stacks}
\item \hyperref[stacks-sheaves-section-phantom]{Sheaves on Algebraic Stacks}
\item \hyperref[criteria-section-phantom]{Criteria for Representability}
\item \hyperref[artin-section-phantom]{Artin's Axioms}
\item \hyperref[quot-section-phantom]{Quot and Hilbert Spaces}
\item \hyperref[stacks-properties-section-phantom]{Properties of Algebraic Stacks}
\item \hyperref[stacks-morphisms-section-phantom]{Morphisms of Algebraic Stacks}
\item \hyperref[stacks-limits-section-phantom]{Limits of Algebraic Stacks}
\item \hyperref[stacks-cohomology-section-phantom]{Cohomology of Algebraic Stacks}
\item \hyperref[stacks-perfect-section-phantom]{Derived Categories of Stacks}
\item \hyperref[stacks-introduction-section-phantom]{Introducing Algebraic Stacks}
\item \hyperref[stacks-more-morphisms-section-phantom]{More on Morphisms of Stacks}
\item \hyperref[stacks-geometry-section-phantom]{The Geometry of Stacks}
\end{enumerate}
Topics in Moduli Theory
\begin{enumerate}
\setcounter{enumi}{107}
\item \hyperref[moduli-section-phantom]{Moduli Stacks}
\item \hyperref[moduli-curves-section-phantom]{Moduli of Curves}
\end{enumerate}
Miscellany
\begin{enumerate}
\setcounter{enumi}{109}
\item \hyperref[examples-section-phantom]{Examples}
\item \hyperref[exercises-section-phantom]{Exercises}
\item \hyperref[guide-section-phantom]{Guide to Literature}
\item \hyperref[desirables-section-phantom]{Desirables}
\item \hyperref[coding-section-phantom]{Coding Style}
\item \hyperref[obsolete-section-phantom]{Obsolete}
\item \hyperref[fdl-section-phantom]{GNU Free Documentation License}
\item \hyperref[index-section-phantom]{Auto Generated Index}
\end{enumerate}
\end{multicols}


\bibliography{my}
\bibliographystyle{amsalpha}

\end{document}

